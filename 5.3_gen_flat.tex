\vspace{-2mm}
\subsection{Decision procedure for SNTs}\label{sec-gflat}
\vspace{-1mm}

We generalize the decision procedure for the case that the transition graphs of the SNTs are generalized lassos to the full class of SNTs.
We first define a \emph{generalized multi-lasso} as a sequence $\gmlasso= H_1 (C_{1,1},\dots,C_{1,n_1}) H_2 (C_{2,1},\dots,C_{2,n_2}) \dots H_r (C_{r,1},\dots, C_{r, n_r})$ s.t. (1) for each $s\in[r]$, $H_s=q_{s,1} \xrightarrow{(g_2,\eta_2)} q_{s,2} \dots q_{s,m_s-1} \xrightarrow{(g_{m_s},\eta_{m_s})} q_{s,m_s}$ is a generalized lasso, (2) for $1 \leq s< s' \leq r$, $H_s (C_{s,1},\dots,C_{s, n_s})$ and $H_{s'} (C_{s', 1},\dots,C_{s', n_{s'}})$ are state-disjoint, except the case that when $s'=s+1$, $q_{s, m_s}=q_{s',1}$, and (3) $q_{1,1}=q_0$.

Since the transition graph of $\Ss$ can be seen as a finite collection of generalized multi-lassos, in the following, we shall present the decision procedure by showing how to decide the non-zero output problem for generalized multi-lassos. 

We fix a generalized multi-lasso below and assume without loss of generality that $O(q_{r,m_r})=a_0+a_1 x_1 + \dots + a_k x_k + b_1 y_1  + \dots + b_l y_l$ and $O(q')$ is undefined for every other state $q'$ in $\gmlasso$.

\smallskip
\hspace{8mm} $\gmlasso= H_1 (C_{1,1},\dots,C_{1,n_1}) H_2 (C_{2,1},\dots,C_{2,n_2}) \dots H_r (C_{r,1},\dots, C_{r, n_r})$.

\subsubsection{Step I':} We do the same analysis as in Step I for the path $H_1\dots H_r$.
\hide{\smallskip\\
\framebox[\textwidth]{
	\begin{minipage}{0.95\textwidth}
		\noindent {\bf Step I$'$}. We do the same analysis as in Step I for the path $H_1\dots H_r$.
	\end{minipage}
}\smallskip}
\subsubsection{Step II':}
Let $s\in [1,r-1]$. In order to analyze the set of cycles $\Cc=\{C_{s,1},\dots,C_{s,n_{s}}\}$, next we show how to summarize effect of the path $H_{s+1}\dots H_r$ to $O(q_{s, m_{s}})$, which is shared by all those cycles in $\Cc$.
Suppose that $\eval{O(q_{r,m_r})}{\sumf^{(H_{s+1}\dots H_{r}, \initval)}}=a_0+a_1 \initval(x_1)+ \dots + a_k \initval(x_k) + b_1 \initval(y_1) + \dots + b_l \initval(y_l)+e$, where $e$ is a linear combination of the data variables that represent the data values introduced when traversing $H_{s+1}\dots H_r$. Since we already passed Step I, we know that $e$ should be identical to zero. 
%the constant atom is $a_{s,0}$, the coefficient of the $\initval(x_j)$-atom is $a_{s, j}$ for each $j \in [k]$, and the coefficient of the $\initval(y_{j'})$-atom is $b_{s, j'}$ for each $j' \in [l]$. 
Moreover, we know that $\initval(x_1)\dots\initval(x_k)$ and $\initval(y_1)\ldots \initval(y_l)$ are values of $x_1\dots x_k$ and $y_1 \dots y_l$ at the state $q_{s, m_{s}}$. Therefore, we can summarize the effect of $H_{s+1}\dots H_{r}$ to $q_{s, m_{s}}$ by letting
$O(q_{s, m_{s}}):=a_0+a_1 x_1 + \dots + a_k x_k + b_1 y_1 + \dots + b_l y_l$ and $O(q')$ is undefined for every other state $q'$.\smallskip\\
\framebox[\textwidth]{
	\begin{minipage}{0.95\textwidth}
\noindent {\bf Step II$'$}.  For each $s\in [r]$ and $s'\in [n_s]$, we check each cycle scheme $\schm = C^{\ell_1}_{s,s'} C_{i_{2}} \dots C_{i_{t}}$ such that $C_{i_{2}} \dots C_{i_{t}}\in \{C_{s, 1}, \dots, C_{s,n_s},\dots, C_{r,1}, \dots, C_{r,n_r}\}$ and $C_{i_{2}} \dots C_{i_{t}}$ are mutually distinct by performing an analysis of the expression $\eval{ O(q_{s, m_{s}})} {\sumf^{(\schm,\sumf^{(H_1 \dots H_{s}, \initval)} ) } }$, in a way similar to Step II. If the decision procedure does not return during the analysis, then go to Step III$'$.
	\end{minipage}
}\bigskip\\
Intuitively, in Step II$'$, during the analysis of $\eval{ O(q_{s, m_{s}})} {\sumf^{(\schm,\sumf^{(H_1 \dots H_{s}, \initval)} ) } }$, the effect of the paths $H_{s+1},  \dots,  H_r$ and the cycles $C_{i_{2}}, \dots, C_{i_{t}}$ to the atom coefficients containing cycle counter variables are expressions of the form $\cstl^{\circled{H_{s+1}}}_j \dots \cstl^{\circled{H_{r}}}_j  \cstl^{\circled{C_{i_{2}}}}_j \dots \cstl^{\circled{C_{i_{t}}}}_j $ for $j \in [l]$. Since $O(q_{s, m_{s}})$ has already taken into consideration the expressions $\cstl^{\circled{H_{s+1}}}_j \dots \cstl^{\circled{H_{r}}}_j$ for $j \in [l]$. In Step II$'$, we can do the analysis as if we have a generalized lasso where the handle is $H_1\dots H_s$ and the collection of cycles is $\{C_{s,1},\dots, C_{s,n_s}$, $\dots$, $C_{r,1},\dots, C_{r,n_r}\}$. 
%$\schm=C^{\ell_{s, 1}}_{i_{s,1}} \dots C^{\ell_{s, t_s}}_{i_{s, t_s}} C^{\ell_{s+1, 1}}_{i_{s+1, 1} } \dots C^{\ell_{s+1, t_{s+1}}}_{i_{s+1, t_{s+1}} } \dots C^{\ell_{r, 1}}_{i_{r,1}} \dots C^{\ell_{r, t_r}}_{i_{r, t_r}} $, where for each $s': s \le s' \le r$, $i_{s',1},\dots, i_{s', t_{s'}} \in [n_{s'}]$, 
%%%%%%%%%%%%%%%%%%%%%%%%%%%%%%%%%%%%%%%%%%%%%%%%%%%
%%%%%%%%%%%%%%%%%%%%%%%%%%%%%%%%%%%%%%%%%%%%%%%%%%%
%%%%%%%%%%%%%%%%%%%%%%%%%%%%%%%%%%%%%%%%%%%%%%%%%%%
\hide
{
At first, by using $O(q_m)$, we do the following computation, similarly to Step II: For each $i_1: 1 \le i_1 \le n$, if there are a cycle scheme $\schm$  
$HC_{i_1}^{\ell_1} C_{i_2}^{\ell_2} \dots C_{i_t}^{\ell_t}
$
or 
$HC_{i_1}^{\ell_1} C_{i_2}^{\ell_2} \dots C_{i_t}^{\ell_t} (C'_{i'_1})^{\ell'_1} (C'_{i'_2})^{\ell'_2} \dots (C'_{i'_{t'}})^{\ell'_{t'}}$,
and $j' \le k$ such that 
\begin{itemize}
\item $i_2,\dots,i_t \le n$ are mutually distinct, $\ell_2 = \dots = \ell_t = 1$, 
%
\item $i'_1,\dots,i'_{t'} \le n'$ are mutually distinct, $\ell'_2 = \dots = \ell'_{t'} = 1$, 
%
\item $\pi_{C_{i_1}}(j')=j'$, and $\mu_{\schm,(i_1,j')} \neq 0$ (recall that $\mu_{\schm,(i_1,j')}$ is obtained from the coefficient of $d^{(0)}_{\pi_H(j')-k}$ in  $\chi_\schm(O(q_m))$), 
\end{itemize}
then return $\ltrue$. 

Then by using $O(q'_{m'})$, we do the following: For each $i'_1: 1 \le i'_1 \le n'$, if there are a cycle scheme $\schm' =(C'_{i'_1})^{\ell'_1} (C'_{i'_2})^{\ell'_2} \dots (C'_{i'_{t'}})^{\ell'_{t'}}$, and $j' \le k$ such that
\begin{itemize}
\item $i'_2,\dots,i'_t \le n'$ are mutually distinct, $\ell'_2 = \dots = \ell'_t = 1$, 
%
\item $\pi_{C'_{i_1}}(j')=j'$, and $\mu_{\schm',(i'_1,j')} \neq 0$ (here $\mu_{\schm',(i_1,j')}$ is obtained from the coefficient of $d''_{j'}$ in  $\chi_{\schm'}(O(q'_{m'}))$, where $d''_1,\dots,d''_k$ denote the initial data values of $x_1,\dots,x_k$ respectively),
\end{itemize}
then return $\ltrue$. 

Similarly, we can apply an analysis for the constant coefficient to $\chi_\schm(O(q_m))$. 


If the decision procedure has not return yet, then go to Step III$'$. \qed
}
%%%%%%%%%%%%%%%%%%%%%%%%%%%%%%%%%%%%%%%%%%%%%%%%%%%
%%%%%%%%%%%%%%%%%%%%%%%%%%%%%%%%%%%%%%%%%%%%%%%%%%%
%%%%%%%%%%%%%%%%%%%%%%%%%%%%%%%%%%%%%%%%%%%%%%%%%%%
\subsubsection{Step III':}
After Step II$'$, if the decision procedure has not returned yet, then similar to Lemma~\ref{prop-bnd-domain-2}, the following hold.
\begin{itemize}
\item For each $s \in [r]$ and each path $\schm=H_1 \schm_1 H_2 \dots H_s \schm_s$ such that for each $s'\in [s]$, $\schm_{s'}$ is a cycle scheme over the collection of cycles $\{C_{s',1},\dots,C_{s',n_{s'}}\}$, it holds that the constant atom and all the coefficients of the non-constant atoms in ${\sumf^{(\schm,\sval_\bot)}}^-(y_j)$ are from a bounded domain $U$.
%
\item Moreover,  an abstraction of $\schm$, denoted by $\abs(\schm)$, can be defined, so that $\mathscr{A}$, which contains the set of $\abs(\schm)$ for the paths $\schm=H_1 \schm_1 H_2 \dots H_s \schm_s$ (where $s \in [r]$), can be computed effectively from 
$H_1, C_{1,1}, \dots, C_{1,n_1},H_2,\dots, H_r,C_{r,1},\dots, C_{r,n_r}$.
\end{itemize}
%Similarly to the generalized lassos, we can construct a finite state automaton $\Aa'$ from $\chi_H,\chi_{C_1},\dots,\chi_{C_n},\chi_{H'}, \chi_{C'_1},\dots,\chi_{C'_{n'}}$ to record the coefficients in the states and simulate the evolvement of these coefficients. The final states of $\Aa$ represent the coefficients obtained when reaching the state $q'_{m'}$ in $\Ss$. 
\framebox[\textwidth]{
	\begin{minipage}{0.95\textwidth}
\noindent {\bf Step III$'$}. We apply the same analysis to $\mathscr{A}$ as in Step III. If the procedure does not return during the analysis, return $\lfalse$.
	\end{minipage}
}\bigskip

\noindent {\it Complexity analysis of Step I$'$-III$'$}. The complexity of Step I$'$ is polynomial over the the maximum length of generalized multi-lassos in $\Ss$. The complexity of Step II$''$ is exponential over the maximum number of simple cycles in generalized multi-lassos. The complexity of Step III$'$ is still exponential over the number of data variables in $\Ss$.

