%!TEX root = main-cav.tex


\subsection{Decision procedure for generalized lassos}\label{sec-glasso}

%
%\yfc{newly added}
In this section, we present a decision procedure for SNTs whose transition graphs are generalized lassos. From Proposition~\ref{prop-sum-cycle}, we know that the coefficients containing the cycle counter variable $\ell$ in $\sumf^{(C^\ell,\initval)}(y_j)$ can be non-zero when $\cstl^{\circled{C}}_{j}=1$. The non-zero coefficients may propagate to the output expression.  In such a case, 
because the SNTs are ``transition-enabled'' (i.e. for any sequence of transitions, a corresponding run exists), %
%if according to the reachability graph, for any $c>0$, there exists a run traversing $C$ more than $c$ times, then 
intuitively, one can pick a run corresponding to a very large $\ell$ so that it dominates the value of the output expression and makes the output non-zero. 
In the decision procedure we are going to present, we first check if the handle of the generalized lasso produces a non-zero output in Step I.
We then check in Step II the coefficients containing $\ell$ in the output expression is non-zero. If this does not happen, then we show in Step III that the non-zero ouput problem of SNT can be reduced to a finite state reachability problem and thus can be easily decided.

Before presenting the decision procedure, we introduce some notations.
Let $e$ be an expression consisting of symbolic values $\initval(z)$ for $z\in X\cup Y$ and variables $\vard_1, \dots, \vard_{s}$ corresponding to the values of the input data word. More specifically, let $e:=\mu_0 + \mu_1 \initval(z_1) +\dots + \mu_{k+l} \initval(z_{k+l}) + \xi_1 \vard_1 + \dots + \xi_{s} \vard_{s}$,
such that $\mu_0,\mu_1,\dots,\mu_{k+l}, \xi_1,\dots,\xi_{s}$ are expressions containing only constants and cycle counter variables.
Then we call $\mu_0$ the \emph{constant atom}, $\mu_i \initval(z_i)$ the $\initval(z_i)$-atom for $i\in[k+l]$, and $\xi_j \vard_j$ the $\vard_j$-atom for $j\in[s]$ of the expression $e$. Moreover, $\mu_1, \dots, \mu_{k+l}, \xi_1,\dots, \xi_{s}$ are called the \emph{coefficients} and $\initval(z_1), \dots, \initval(z_{k+l}), \vard_1. \dots, \vard_{s}$ the \emph{subjects} of these atoms.
A non-constant atom is said to be \emph{nontrivial} if its coefficient is \emph{not} identical to zero.

In the rest of this subsection, we assume that the transition graph of $\Ss$ comprises a handle $H=q_0 \xrightarrow{(g_1,\eta_1)} q_1 \dots q_{m-1} \xrightarrow{(g_m,\eta_m)} q_{m}$, a collection of simple cycles $C_1,\dots,C_n$ such that $q_m$ is the unique state shared by each pair of distinct cycles from $\{C_1,\dots,C_n\}$, and a $\triangleright$-transition $q_m \xrightarrow{(\cur = \triangleright, \eta)} q_{m+1}$. Moreover, without loss of generality, we assume that $O(q_{m+1}) = a_0 + a_1 x_1 + \dots + a_k x_k + b_1 y_1 + \dots + b_l y_l$, and $O(q)$ is undefined for all the other states $q$. For convenience, we define $O(q_m)$ by replacing simultaneously $z$ with $\eta(z)$ in $O(q_{m+1})$,  for each $z \in \dom(\eta)$. Suppose $O(q_m) = a'_0 + a'_1 x_1 + \dots + a'_k x_k + b'_1 y_1 + \dots + b'_l y_l$. Then for the non-zero output problem, we can ignore the $\triangleright$-transition and use $O(q_m)$  directly.

A \emph{cycle scheme} $\schm$ is a path $C_{i_1}^{\ell_1} C_{i_2}^{\ell_2} \dots C_{i_t}^{\ell_t}$ such that $i_1,\dots,i_t \in [n]$, $\ell_1,\dots, \ell_t \ge 1$, and for each $j\in [t-1]$, $i_j \neq i_{j+1}$. Intuitively, $\schm$ is a path obtained by first iterating $C_{i_1}$ for $\ell_1$ times, then $C_{i_2}$ for $\ell_2$ times, and so on. From Proposition~\ref{prop-sum-cycle} and Corollary~\ref{cor-comp-two-paths}, a symbolic valuation $\sumf^{(\schm,\initval)}$ can be constructed 
to summarize the computation of $\Ss$ on $\schm$. 


\begin{lemma}\label{prop-cycle-schm}
Suppose $\schm=C_{i_1}^{\ell_1} C_{i_2}^{\ell_2} \dots C_{i_t}^{\ell_t}$ is a cycle scheme, and $\initval$ is a symbolic valuation representing the initial values of the control and data variables. 
For all $j' \in  I^{\circled{C_{i_{1}}}}_{pe}$, let $r_{j'}$ be the largest number $r \in [t]$ such that $j'\in\bigcap_{s\in[r]} I^{\circled{C_{i_{s}}}}_{pe}$, i.e., $x_{j'}$ remains persistent when traversing $C_{i_1}^{\ell_1} C_{i_2}^{\ell_2} \dots C_{i_{r_{j'}}}^{\ell_{r_{j'}}}$.
Then for each $j\in [l]$ and $j' \in  I^{\circled{C_{i_{1}}}}_{pe}$, the coefficient of the $\initval(x_{j'})$-atom in $\sumf^{(\schm,\initval)}(y_j)$ is 
\begin{center}
\resizebox{0.8\hsize}{!}{
$e+\sum\limits_{s_1\in[r_{j'}]}  
\left(1+\lambda^{\circled{C_{i_{s_1}}}}_{j} + \dots + (\lambda^{\circled{C_{i_{s_1}}}}_{j})^{\ell_{s_1}-1} \right) \csta^{\circled{C_{i_{s_1}}}}_{j,j'}\prod\limits_{{s_2}\in[{s_1}+1,t]}\left(\lambda^{\circled{C_{i_{s_2}}}}_{j}\right)^{\ell_{s_2}}$},
\end{center}
where (1) $e\!=\!0$ when $r_{j'}\!=\!t$ and (2) $e=(\lambda^{\circled{C_{i_s}}}_{j})^{\ell_s-1} \csta^{\circled{C_{i_{s}}}}_{j,j'} \prod\limits_{{s'}\in[s+1,t]}\left(\lambda^{\circled{C_{i_{s'}}}}_{j}\right)^{\ell_{s'}}$ with $s=r_{j'}+1$ when $r_{j'}<t$.\\
The constant atom of $\sumf^{(\schm,\initval)}(y_j)$ is 
\begin{center}
\resizebox{0.7\hsize}{!}{$
\sum\limits_{{s_1}\in[t]}
\left(1+\lambda^{\circled{C_{i_{s_1}}}}_{j} + \dots + (\lambda^{\circled{C_{i_{s_1}}}}_{j})^{\ell_{s_1}-1} \right)
\cste^{\circled{C_{i_{s_1}}}}_{j} 
\prod\limits_{{s_2}\in[{s_1}+1,t]}\left(\lambda^{\circled{C_{i_{s_2}}}}_{j}\right)^{\ell_{s_2}}$}
\end{center}
Moreover, for all $j\!\in\! [l]$, in $\sumf^{(\schm,\initval)}(y_j)$, only the constant atom and the coefficients of the $\initval(x_{j'})$-atoms with $j' \!\in\!I^{\circled{C_{i_{1}}}}_{pe}$ contain a subexpression of the form $ \mu_\schm \ell_1$ for some~$\mu_\schm\in \intnum$.
\end{lemma}
Notice that above, $\lambda^{\circled{C_{i_{s_1}}}}_j\in\{0,1\}$ for $j\in[l]$ and $s_1\in [t]$. Hence the value of $(1+\lambda^{\circled{C_{i_{s_1}}}}_{j} + \dots + (\lambda^{\circled{C_{i_{s_1}}}}_{j})^{\ell_{s_1}-1} )$ can only be $1$ or $\ell_{s_1}$ and $\left(\lambda^{\circled{C_{i_{s_2}}}}_{j}\right)^{\ell_{s_2}}\in\{0,1\}$.
Therefore, both the constant atom and the coefficient of the $\initval(x_{j'})$-atom with $j'\in I^{\circled{C_{i_{1}}}}_{pe}$ can be rewritten to the form of $c_0+c_1\ell_1+c_2\ell_2+\dots+c_t\ell_t$ for $c_0\ldots c_t\in \intnum$. Note that some of $c_0\ldots c_t$ might be zero.




%We are ready to present the decision procedure. By the ``well-defined'' and ``uniquely-valued'' constraints of normalized SNTs, without loss of generality, we assume that $I^{\circled{H}}_{tr}=[k]$, that is, after traversing $H$, the values of all control variables become defined.
%Under the assumption, 
\subsubsection{Step I:} 
We are ready to present the decision procedure. At first, we observe that  after traversing $H$ with the initial values of the variables given by some valuation $\sval_0$, for each $j' \in I^{\circled{H}}_{tr}$, the value of the control variable $x_{j'}$ becomes $\vard^{\circled{H}}_{\pi^{\circled{H}}_{tr}(j')}$,  more formally, $\sumf^{(H,\sval_0)}(x_{j'})=\vard^{\circled{H}}_{\pi^{\circled{H}}_{tr}(j')}$.

In Step I, we check if $\eval{O(q_m)}{\sumf^{(H,\sval_0)}}$ is not identical to zero.
This can be done by checking if the constant-atom or the coefficient of some non-constant atom of the output expression $\eval{O(q_m)}{\sumf^{(H,\sval_0)}}$ is not identical to zero.
%
\bigskip\\
\framebox[\textwidth]{
\begin{minipage}{0.95\textwidth}
\noindent {\bf Step I}. Decide whether $\eval{O(q_m)}{\sumf^{(H,\sval_0)}}$ is not identical to zero.
If the answer is yes, then the decision procedure terminates and returns the answer $\ltrue$. Otherwise, go to Step II.
\end{minipage}
}\medskip

\noindent{\it Complexity analysis of Step I}. Since $\sumf^{(H,\sval_0)}$ can be computed in polynomial time from $H$, it follows that Step I can be done in polynomial time.


\begin{example}\label{exmp-step-1}
Let $\Ss'_{\sf max}$ be the SNT in Example~\ref{exmp-sum}. Since $q_1 \xrightarrow{(\cur = \triangleright, \emptyset)} q_2$, we can define $O(q_1):= a'_0 +  a'_1 x_1 + b'_1 y_1 + b'_2 y_2 + b'_4y_3 = O(q_2) = y_1 - 2y_2 + y_3$.  The handle $H$ comprises exactly one transition. Thus $r^{\circled{H}}=1$ and 
\[\eval{O(q_1)}{\sumf^{(H, \initval_0)}} = \vard^{\circled{H}}_1 - 2 \vard^{\circled{H}}_1 + \vard^{\circled{H}}_1=0.\]
Therefore, the handle $H$ does not produce a non-zero output.
\end{example}

\subsubsection{Step II:}
The goal of Step II is to show that 
\begin{itemize}
\item either for each cycle scheme $\schm$, all subexpressions in $\eval{O(q_{m})}{\sumf^{(\schm,\sumf^{(H,\sval_0)})}}$ containing the cycle counter variables are identical to zero and hence can be ignored, 
\item or  there exists a cycle scheme $\schm$ such that $\eval{O(q_{m})}{\sumf^{(\schm,\sumf^{(H,\sval_0)})}}$ is not identical to zero. 
\end{itemize}
%
Let $\schm=C_{i_1}^{\ell_1} C_{i_2}^{\ell_2} \dots C_{i_t}^{\ell_t}$ be a cycle scheme. From Lemma~\ref{prop-cycle-schm}, for each $j'\in I^{\circled{C_{i_1}}}_{pe}$ and symbolic valuation $\sval$, the only subexpression containing $\ell_1$ in the coefficient of $\initval(x_{j'})$-atom of $\eval{O(q_{m})}{\sumf^{(\schm,\initval)}}$ is
\begin{center}
	\resizebox{0.7\hsize}{!}{$
\sum \limits_{1 \le j \le l} 
b'_j \left((\cstl^{\circled{C_{i_2}}}_{j})^{\ell_2} \dots (\cstl^{\circled{C_{i_t}}}_{j})^{\ell_t}\right) 
\left(1+\cstl^{\circled{C_{i_1}}}_{j} + \dots + (\cstl^{\circled{C_{i_1}}}_{j})^{\ell_1-1} \right) \csta^{\circled{C_{i_1}}}_{j,j'}.
\hspace{4mm} (\ast)
$}
\end{center}
Since $\cstl^{\circled{C_{i_1}}}_{j}, \cstl^{\circled{C_{i_2}}}_{j}, \dots, \cstl^{\circled{C_{i_t}}}_{j} \in \{0, 1\}$, the expression $(\ast)$  can be rewritten as  
 $\mu_{\schm, (i_1,j')} \ell_1 + \nu_{\schm, (i_1,j')}$ for some integer constants $\mu_{\schm, (i_1,j')}$ and $\nu_{\schm, (i_1,j')}$. 
 
The only subexpression containing $\ell_1$ in the constant atom of  $\eval{O(q_{m})}{\sumf^{(\schm,\initval)}}$ is
\begin{center}
	\resizebox{0.7\hsize}{!}{$
\sum \limits_{1 \le j \le l} b'_j
\begin{array}{l}
 \left((\lambda^{\circled{C_{i_2}}}_{j})^{\ell_2} \dots (\lambda^{\circled{C_{i_t}}}_{j})^{\ell_t}\right)
\left(1+\lambda^{\circled{C_{i_1}}}_{j} + \dots + (\lambda^{\circled{C_{i_1}}}_{j})^{\ell_1-1} \right) \cste^{\circled{C_{i_1}}}_{j}. \hspace{2mm} (\ast\ast)
\end{array}
$}
\end{center}
%
The expression $(\ast\ast)$ can be rewritten as $\mu_{\schm,(i_1,0)} \ell_1 + \nu_{\schm,(i_1,0)}$ for some integer constants $\mu_{\schm, (i_1,0)}$ and $\nu_{\schm, (i_1,0)}$. 

\hide{
We use $\sim_\initval$ to denote an equivalence relation over $[k]$, that is, over the set of indices of control variables. The intention of $\sim_\initval$ is as follows: $j'_1 \sim_\initval j'_2$ iff $\initval(x_{j'_1})=\initval(x_{j'_2})$, that is, the values of the control variables $x_{j'_1}$ and $x_{j'_2}$ assigned by $\initval$ are the same. 

Suppose $\initval$ is a symbolic valuation associated with an equivalence relation $\sim_\initval$.  If $\mu_{\schm,(i_1,0)}=0$, in addition, for each equivalence class $J$ of $\sim_\initval$ such that $J \cap I^{\circled{C_{i_1}}}_{pe} \neq \emptyset$, it holds $\sum \limits_{j' \in J \cap I^{\circled{C_{i_1}}}_{pe} } \mu_{\schm,(i_1,j')}=0$, then we can ignore all subexpressions containing the cycle counter variable $\ell_1$ in   $\eval{O(q_{m+1})}{\sumf^{(\schm,\initval)}}$, i.e., the subexpressions $\mu_{\schm,(i_1,0)}\ell_1$ and $\mu_{\schm,(i_1,j')}\ell_1$ for all $j' \in I^{\circled{C_{i_1}}}_{pe}$.
%

In the following, we compute inductively the set $\Ee$ of all the possible equivalence relations $\sim_\initval$ on the initial values of control variables for iterating the cycles $C_1,\dots, C_n$.
\begin{enumerate}
\item Initially, let $\Ee_0 = \{\sim_H\}$, where $\sim_H$ is the reflexive, symmetric and transitive closure of the set of tuples $(j_1,j_2)$ such that $\pi^{\circled{H}}(j_1) = \pi^{\circled{H}}({j_2})$.
%
\item For $i \ge 0$, $\Ee_{i+1}$ is the union of $\Ee_i$ and the set of $\sim'\ = \left(\sim \cap\ (I^{\circled{C_{i'}}}_{pe} \times I^{\circled{C_{i'}}}_{pe})\right) \cup \left\{(j_1,j_2) \in I^{\circled{C_{i'}}}_{tr} \times I^{\circled{C_{i'}}}_{tr} \mid \pi^{\circled{C_{i'}}}(j_1) = \pi^{\circled{C_{i'}}}(j_2) \right\}$, for some $\sim \in \Ee_i$ and some cycle $C_{i'}$ (where $i' \in [n]$).
%
\item If $\Ee_{i+1} = \Ee_i$, then let $\Ee = \Ee_i$ and stop the computation.
\end{enumerate}
}

\smallskip
We are ready to present Step II.
\bigskip\\
\framebox[\textwidth]{
	\begin{minipage}{0.95\textwidth}
		\noindent {\bf Step II}. For each $i_1 \in [n]$, check all cycle scheme $\schm=C_{i_1}^{\ell_1} C_{i_2} \dots C_{i_t}$ such that $i_2,\dots,i_t$ are mutually distinct. There are only finitely many this kind of cycle schemes. If 
		one of the following constraints is satisfied, then return $\ltrue$. 
		\begin{enumerate}
		\item There is an equivalence class $J$ of $\sim_{q_m}$ such that $ J \cap I^{\circled{C_{i_1}}}_{pe} \neq \emptyset$ and $\sum \limits_{j' \in  J \cap I^{\circled{C_{i_1}}}_{pe}} \mu_{\schm,(i_1,j')} \neq 0$. 
		\item $\mu_{\schm,(i_1,0)} \neq 0$.
		\end{enumerate}
		If the decision procedure has not returned yet, then go to Step III.
	\end{minipage}
}\bigskip

\noindent{\it Complexity analysis of Step II}. Since $i_1,\dots, i_t$ are mutually distinct, the number of cycle schemes $\schm = C_{i_1}^{\ell_1} C_{i_2} \dots C_{i_t}$ in Step II is exponential in the number of cycles in the generalized lasso. Once the cycle scheme $\schm$ is fixed, the two constraints in Step II can be decided in polynomial time. Therefore, the complexity of Step II is exponential in the number of simple cycles in the generalized lasso.



\begin{example}\label{exmp-step-2}
Let $\Ss'_{\sf max}$ be the SNT in Example~\ref{exmp-step-1}. We need consider the following cycle schemes, $C^{\ell_1}_1$, $C^{\ell_1}_2$, $C^{\ell_1}_1 C_2$, $C^{\ell_1}_2 C_1$, $C^{\ell_1}_1 C_2 C_1$, $C^{\ell_1}_2 C_1 C_2$. We use $\schm_1 = C^{\ell_1}_1$, $\schm_2 = C^{\ell_1}_1 C_2$, and $\schm_3 = C^{\ell_1}_1 C_2 C_1$ to illustrate Step II.
\begin{itemize}
\item In $\eval{O(q_1)}{\sumf^{(\schm_1, \initval)}}$,  the subexpression containing $\ell_1$ in the coefficient of the $\initval(x_1)$-atom is
\[
\begin{array} {l}
\sum \limits_{1 \le j \le 3} b'_j
\begin{array}{l}
\left(1+\lambda^{\circled{C_{1}}}_{j} + \dots + (\lambda^{\circled{C_{1}}}_{j})^{\ell_1-1} \right) \csta^{\circled{C_{1}}}_{j, 1}
\end{array}
\\
= 1 \times \ell_1 \times 4 +  (-2) \times \ell_1 \times 2 + 1 \times 1 \times 0 = 0.
\end{array}
\]
%
\item In $\eval{O(q_1)}{\sumf^{(\schm_2, \initval)}}$, the subexpression containing $\ell_1$ in the coefficient of the $\initval(x_1)$-atom is 
\[
\begin{array}{l}
\sum \limits_{1 \le j \le 3} b'_j
\begin{array}{l}
 \lambda^{\circled{C_{2}}}_{j}
\left(1+\lambda^{\circled{C_{1}}}_{j} + \dots + (\lambda^{\circled{C_{1}}}_{j})^{\ell_1-1} \right) \csta^{\circled{C_{1}}}_{j, 1}
\end{array}
\\
= 1 \times 1 \times \ell_1 \times 4 + (-2) \times 1 \times \ell_1 \times 2 + 1 \times 1 \times 1 \times 0 =0.
\end{array}
\]
%
\item In $\eval{O(q_1)}{\sumf^{(\schm_3, \initval)}}$, the subexpression containing $\ell_1$ in the coefficient of the $\initval(x_1)$-atom is 
\[
\begin{array}{l}
\sum \limits_{1 \le j \le 3} b_j
\begin{array}{l}
 (\lambda^{\circled{C_{2}}}_{j}\lambda^{\circled{C_{1}}}_{j})
\left(1+\lambda^{\circled{C_{1}}}_{j} + \dots + (\lambda^{\circled{C_{1}}}_{j})^{\ell_1-1} \right) \csta^{\circled{C_{1}}}_{j, 1}
\end{array}
\\
= 1 \times 1 \times \ell_1 \times 4 + (-2) \times 1 \times \ell_1 \times 2 + 1 \times 0 \times 1 \times 0 =0.
\end{array}
\]
\end{itemize}
From Example~\ref{exmp-sum}, we know that if $\schm=C^{\ell_1}_2, C^{\ell_1}_2 C_1$, or $C^{\ell_1}_2 C_1 C_2$, then for each $j =1, 2,3$, the coefficient of the $\initval(x_1)$-atom in $\sumf^{(\schm, \initval)}(y_j)$ does not contain the cycle counter variable $\ell_1$. Therefore, the coefficient of the $\initval(x_1)$-atom in $\eval{O(q_1)}{\sumf^{(\schm, \initval)}}$ does not contain $\ell_1$ as well. From this, we conclude that with $\Ss'_{\sf max}$ as the input, the decision procedure does not return in Step II and it will go to Step III.
\end{example}


\medskip
In the following, we present the arguments for the correctness of Step II. Suppose $\schm =C_{i_1}^{\ell_1} C_{i_2} \dots C_{i_t}$.
\begin{itemize}
\item If there exists an equivalence class $J$ of the equivalence relation $\sim_{q_m}$ such that $J \cap I^{\circled{C_{i_1}}}_{pe} \neq \emptyset$ and 
$\sum \limits_{j' \in  J \cap I^{\circled{C_{i_1}}}_{pe}} \mu_{\schm,(i_1,j')} \neq 0$. 
Then from $\schm=C_{i_1}^{\ell_1} C_{i_2} \dots C_{i_t}$, we can assign a sufficiently large value $s$ to $\ell_1$ so that the sum of the coefficients of the $\initval(x_{j'})$-atoms for $j' \in  J \cap I^{\circled{C_{i_1}}}_{pe}$ in $\eval{O(q_{m})}{\sumf^{(H\schm ,\sval_0)}}$, which is equal to the sum of the expression $\ell_1 \left(\sum \limits_{j' \in  J \cap I^{\circled{C_{i_1}}}_{pe}} \mu_{\schm,(i_1,j')}\right)$ and some constant,  becomes non-zero.  The guards and assignments in the path $H C_{i_1}^{s} C_{i_2} \dots C_{i_t}$ enforce a preorder over the subjects of those nontrivial non-constant atoms.
Pick one of the nontrivial non-constant atoms with a maximal subject w.r.t. the preorder. Since the subject is maximal, it can be assigned an arbitrarily large number so that the corresponding atom, or the sum of the $\initval(x_{j'})$-atoms for $j' \in J'$ (where $J'$ is an equivalence class of $\sim_{q_m}$), dominates $\eval{O(q_{m})}{\sumf^{(P\schm,\sval_0)}}$.  This is sufficient to make $\eval{O(q_{m})}{\sumf^{(H\schm,\sval_0)}}$ non-zero.
%
\item Otherwise, if $\eval{O(q_{m})}{\sumf^{(H\schm,\sval_0)}}$ contains some other nontrivial non-constant atoms, then we can apply a similar argument as above and conclude that $\eval{O(q_{m})}{\sumf^{(H\schm,\sval_0)}}$ can be made non-zero. 

\item On the other hand, if $\eval{O(q_{m})}{\sumf^{(H\schm,\sval_0)}}$ contains no nontrivial non-constant atoms, but $\mu_{\schm,(i_1,0)} \neq 0$, then we can let $\ell_1$ sufficiently large to make the expression $\eval{O(q_{m})}{\sumf^{(H\schm,\sval_0)}}$ non-zero. 
\end{itemize} 
Therefore, when Step II returns $\ltrue$, that is, at least one of the two conditions in Step II holds, then we are able to conclude that there \emph{must be}  an input data word $w$ and an initial valuation $\rho_0$ such that $ \Ss_{\rho_0}(w) \not \in \{\bot, 0\}$. 
%Similar arguments can be applied to $\ell_2\dots\ell_n$.

\smallskip

If Step II does not return $\ltrue$, we show below that for all cycle schemes $\schm_1=C_{i_1}^{\ell_1} C_{i_2}^{\ell_2} \dots C_{i_{t_1}}^{\ell_{t_1}}$ with $i_1,i_2,\dots,i_{t_1} \in [n]$, all the subexpressions containing cycle counter variables $\ell_1,\dots, \ell_{t_1}$ in $\eval{O(q_{m})}{\sumf^{(\schm_1, \sumf^{(H,\sval_0)})}}$ are identical to zero and hence can be removed. Let ${i'_2} \dots {i'_{t_2}}$ be the sequence obtained from $i_2 \dots i_{t_1}$ by keeping just one copy for each duplicated index therein.  
In Step II we already checked the cycle scheme $\schm_2=C_{i_1}^{\ell_1} C_{i'_2} \dots C_{i'_{t_2}}$. Step II guarantees that all the subexpressions containing $\ell_1$ in 
$\eval{O(q_{m})}{\sumf^{(\schm_2, \sumf^{(H,\sval_0)})}}$ are identical to zero and hence can be removed.
Because for all $j\in[l]$, $\cstl^{^{\circled{C_1}}}_j, \dots, \cstl^{^{\circled{C_n}}}_j \in \{0,1\}$,   $(\lambda^{\circled{C_{i_2}}}_{j})^{\ell_2} \dots (\lambda^{\circled{C_{i_{t_1}}}}_{j})^{\ell_{t_1}} = \lambda^{\circled{C_{i'_2}}}_{j} \dots \lambda^{\circled{C_{i'_{t_2}}}}_{j}$. We proved that the $(\ast)$ and $(\ast\ast)$ style expressions are equivalent in both $\eval{O(q_{m})}{\sumf^{(\schm_1, \sumf^{(H,\sval_0)}))}}$ and $\eval{O(q_{m})}{\sumf^{(\schm_2, \sumf^{(H,\sval_0)}))}}$.
Hence we can also remove all subexpressions containing $\ell_1$ from  $\eval{O(q_{m})}{\sumf^{(\schm_1, \sumf^{(H,\sval_0)}))}}$, without affecting its value.
Those subexpressions containing $\ell_2$ can also be removed by considering the cycle scheme $\schm_3=C_{i_2}^{\ell_2} C_{i''_3} \dots C_{i''_{t_3}}$ and applying a similar reasoning, where the sequence ${i''_3} \dots {i''_{t_3}}$ is obtained from ${i_3} \dots  i_{t_1}$, similarly to the construction of ${i'_2} \dots {i'_{t_2}}$ from $i_2 \dots i_{t_1}$. The same applies to all other cycle counter variables $\ell_3,\dots,\ell_{t_1}$.
For each $y_j \in Y$, we use the notation ${\sumf^{(\schm_1,\initval)}}^-(y_j)$ to denote the expression obtained by removing from the constant atom and coefficients of the non-constant atoms of $\sumf^{(\schm_1,\initval)}(y_j)$ all subexpressions containing the cycle counter variables. 

\begin{lemma}\label{prop-bnd-domain-1}
	Suppose that the decision procedure has not returned $\ltrue$ after Step~II. For each cycle scheme $\schm$, let $f=\eval{O(q_{m})}{\sumf^{(\schm, \sumf^{(H,\sval_0)})}}$ and $f'=\eval{O(q_{m})}{{\sumf^{(\schm, \sumf^{(H,\sval_0)})}}^-}$. For all valuations $\rho$, $\eval{f}{\rho}\neq 0$ iff $\eval{f'}{\rho} \neq 0$.
\end{lemma}



\subsubsection{Step III:} 

From Proposition~\ref{prop-sum-cycle},  we can observe that for each cycle $C_i$ (where $1 \le i \le n$),  the expression $\sumf^{(C^\ell_i, \sumf^{(H,\sval_0)})^-}(y_j)$ for $y_j \in Y$ is of one of the the following forms.
\begin{itemize}
\item If $\cstl^{\circled{C_i}}_{j}=0$, then $\sumf^{(C^\ell_i, \sumf^{(H,\sval_0)})^{-}}(y_j) = \sumf^{(C^\ell_i, \sumf^{(H,\sval_0)})}(y_j)$.
%\medskip\\
%\resizebox{0.9\hsize}{!}{$
%\sumf^{(C^\ell_i,\initval)}(y_j)   =  \cste^{\circled{C_i}}_{j} +  \sum \limits_{j' \in I^{\circled{C_i}}_{pe}} \csta^{\circled{C_i}}_{j,j'} \initval(x_{j'}) +
%\sum \limits_{j'  \in \rng(\pi^{\circled{C_i}})} \left(\sum \limits_{j'' \in (\pi^{\circled{C_i}})^{-1}(j')} \csta^{\circled{C_i}}_{j, j''} \right) \vard^{\circled{C_i , \ell  -  1}}_{j'} + \sum \limits_{j' \in [r^{\circled{C_i}}] }  
%\cstb^{\circled{C_i}}_{j, j'} \vard^{\circled{C_i , \ell}}_{j'}.$}

\item If $\cstl^{\circled{C_i}}_{j}=1$, then
%
\medskip\\
\resizebox{0.95\hsize}{!}{$
\begin{array}{l }
\sumf^{(C^\ell_i, \sumf^{(H,\sval_0)})^-}(y_j)  =    \sumf^{(H,\sval_0)}(y_j) +   
\sum \limits_{j' \in I^{\circled{C_i}}_{tr}} \csta^{\circled{C_i}}_{j,j'} \sumf^{(H,\sval_0)}(x_{j'})\ +   \\
\sum \limits_{j' \in \rng(\pi^{\circled{C_i}}_{tr})} \sum \limits_{s\in[\ell -1]}
\left(\cstb^{\circled{C_i}}_{j,j'} + \sum \limits_{j'' \in (\pi^{\circled{C_i}}_{tr})^{-1}(j')} \csta^{\circled{C_i}}_{j, j''}   \right) \vard^{\circled{C_i , s}}_{j'} + \sum \limits_{j' \in [r^{\circled{C_i}}]  \setminus \rng(\pi^{\circled{C_i}}_{tr}) }\sum \limits_{s\in[\ell -1]} 
\cstb^{\circled{C_i}}_{j,j'} \vard^{\circled{C_i , s}}_{j'} + \sum \limits_{j' \in [r^{\circled{C_i}}] }  
\cstb^{\circled{C_i}}_{j, j'} \vard^{\circled{C_i , \ell}}_{j'}.
\end{array}
$}
\end{itemize}
Note that if $\cstl^{\circled{C_i}}_{j}=1$, then the expressions containing the cycle counter variable $\ell$, e.g. $\ell \cste^{\circled{C_i}}_j$, are removed from $\sumf^{(C^\ell_i, \sumf^{(H,\sval_0)})^-}(y_j)$.

%%%%%%%%%%%%%%%%%%%%%%%%%%%%%%%%%%%%%%%%%%%%%%%%%%%
%%%%%%%%%%%%%%%%%%%%%%%%%%%%%%%%%%%%%%%%%%%%%%%%%%%
\hide{
For each cycle scheme $\schm$, let
%
\[
\small
\def\arraystretch{2}
\begin{array}{r l l}
\smallskip
\sumf^{(\schm,\sumf^{(H,\initval_0)})^-}(y_j)  & & \\
 & \hspace{-16mm} = & \hspace{-10mm}  \cste^{(\circled{\schm})^-}_{j} + \cstl^{\circled{\schm}}_j \sumf^{(H,\initval_0)}(y_j)  + \sum \limits_{j' \in [k]}  \csta^{(\circled{\schm})^-}_{j,j'}  \sumf^{(H,\initval_0)}(x_{j'}) + \sum \limits_{j' \in [r^{\circled{\schm}}]} \cstb^{\circled{\schm}}_{j,j'} \vard^{\circled{\schm}}_{j'}  \\
% expanded expression
	& \hspace{-16mm} = & \hspace{-10mm}  
	\left( \cste^{(\circled{\schm})^-}_{j}+ \cstl^{\circled{\schm}}_{j} \cste^{\circled{H}}_{j}\right)+ \left(\cstl^{\circled{\schm}}_{j} \cstl^{\circled{H}}_{j} \right) \initval_0(y_j)+  \sum \limits_{j' \in I^{\circled{H}}_{pe}} 
	\left(\csta^{(\circled{\schm})^-}_{j,j'} +\cstl^{\circled{\schm}}_{j} \csta^{\circled{H}}_{j,j'}\right) \initval_0(x_{j'}) \\
	%
	& &   \hspace{-16mm} + \sum \limits_{j' \in  I^{\circled{H}}_{tr}} 
	\left(\cstl^{\circled{\schm}}_{j} \csta^{\circled{H}}_{j,j'} \right) \initval_0(x_{j'}) + 	\sum \limits_{j' \in \rng(\pi^{\circled{H}})} \left(\cstl^{\circled{\schm}}_{j} \cstb^{\circled{H}}_{j,j'}+  \sum \limits_{j'' \in (\pi^{\circled{H}})^{-1}(j')} \csta^{(\circled{\schm})^-}_{j,j''} \right) \vard^{\circled{H}}_{j'} \\
	%
	& & 
\hspace{-16mm} +  \sum \limits_{j' \in [r^{\circled{H}}]\setminus \rng(\pi^{\circled{H}})} \left( \cstl^{\circled{\schm}}_{j} \cstb^{\circled{H}}_{j,j'} \right) \vard^{\circled{H}}_{j'} +
	\sum \limits_{j'\in[r^{\circled{\schm}}]} \cstb^{\circled{\schm}}_{j,j'} \vard^{\circled{\schm}}_{j'}.
	\end{array}
\]
 
Note that the coefficients of the $\vard^{\circled{\schm}}_1$-atom, $\dots$, and $\vard^{\circled{\schm}}_{r^{\circled{\schm}}}$-atom in $\sumf^{(\schm,\sumf^{(H,\initval_0)})^-}(y_j)$ are the same as those in $\sumf^{(\schm,\sumf^{(H,\initval_0)})}(y_j)$.
}
%%%%%%%%%%%%%%%%%%%%%%%%%%%%%%%%%%%%%%%%%%%%%%%%%%%
%%%%%%%%%%%%%%%%%%%%%%%%%%%%%%%%%%%%%%%%%%%%%%%%%%%


We define the abstraction of $\sumf^{(\schm,\sumf^{(H,\initval_0)})^-}$, denoted by $\abs(H\schm)$, as the union of the following three sets of tuples:
\begin{itemize}
\item the tuple for the constant atom: $\left\{\left(0, \left( \cste^{(\circled{\schm})^-}_{1}+ \cstl^{\circled{\schm}}_{1} \cste^{\circled{H}}_{1},\dots, \cste^{(\circled{\schm})^-}_{l}+ \cstl^{\circled{\schm}}_{l} \cste^{\circled{H}}_{l} \right) \right)\right\}$,
% 
\item tuples for the control variable atoms: $\{(j', (c_{j',1},\dots, c_{j', l})) \mid j' \in [k]\}$, where $c_{j', j}$ is the coefficient of the ${\sumf^{(\schm,\sumf^{(H,\sval_0)})}}(x_{j'})$-atom in ${\sumf^{(\schm,\sumf^{(H,\sval_0)})}}^-(y_{j})$ for $j\in[l]$,
% 
\item tuples for the other atoms: $\{(k+1, (c_1,\dots,c_l))\}$, where $(c_1,\dots,c_l)$ is the vector of coefficients of the $\vard'$-atom in ${(\sumf^{(\schm,\sumf^{(H,\sval_0)})}}^-(y_{j})$ for all $j \in [l]$ and $\vard'\not\in \{{\sumf^{(\schm,\sumf^{(H,\sval_0)})}}(x_{j'})\mid x_{j'}\in X\}$.
\end{itemize}
Note that in $\abs(H\schm)$, if $j_1, j_2 \in [k]$ and $j_1 \sim_{q_m} j_2$, then we have $\sumf^{(\schm,\sumf^{(H,\initval_0)})}(x_{j_1})=\sumf^{(\schm,\sumf^{(H,\initval_0)})}(x_{j_2})$. Therefore, $(c_{j_1, 1}, \dots, c_{j_1, l}) = (c_{j_2,1}, \dots, c_{j_2, l})$.

Let $\absset=\bigcup \{\abs(H\schm) \mid \schm \mbox{ a cycle scheme}\}$. Then $\absset$ can be constructed inductively as follows, until $\absset_{i+1}= \absset_i$. 
\begin{enumerate}
\item $\absset_0 = \{\abs(HC_1), \ldots \abs(HC_n)\}$,

\item For $i \ge 0$, $\absset_{i+1}$ is the union of $\absset_i$ and the set of $\Lambda'$ such that $\Lambda'$ is constructed from some $\Lambda \in \absset_i$ and some cycle $C_{i'}$ (where $ i' \in [n]$) as follows. At first we observe that $\sim_{q_m} \subseteq (I^{\circled{C_{i'}}}_{pe} \times I^{\circled{C_{i'}}}_{pe}) \cup (I^{\circled{C_{i'}}}_{tr} \times I^{\circled{C_{i'}}}_{tr})$. The argument is as follows: If $j' \in I^{\circled{C_{i'}}}_{tr}$ and $j'' \in I^{\circled{C_{i'}}}_{pe}$, then $x_{j'}$ is assigned a fresh value and $x_{j''}$ is assigned the initial value of some control variable, thus $x_{j'}$ and $x_{j''}$ are not equivalent w.r.t. $\sim_{q_m}$.
%\begin{itemize}
%\item $\sim' = \left(\sim \cap\ I^{\circled{C_{i'}}}_{pe} \times I^{\circled{C_{i'}}}_{pe} \right) \cup \left\{(j_1,j_2) \in I^{\circled{C_{i'}}}_{tr} \times I^{\circled{C_{i'}}}_{tr} \mid \pi^{\circled{C_{i'}}}(j_1) = \pi^{\circled{C_{i'}}}(j_2)\right\}$. 
%
%\item Moreover, $\Lambda'$ is obtained from $\Lambda$ and $C_{i'}$ as follows. 
\begin{itemize}
\item Suppose $(0, (c_1, \dots, c_l)) \in \Lambda$. Then $(0, (c'_1, \dots, c'_l)) \in \Lambda'$, where for each $j \in [l]$, if $\cstl^{\circled{C_{i'}}}_j = 0$, then $c'_j = \cste^{\circled{C_{i'}}}_{j}$, otherwise, $c'_j = c_j$ (in this case, the expression $\cste^{\circled{C_{i'}}}_{j}$ is removed).  
%
\item For each $j' \in [k]$, suppose $(j', (c_{j',1}, \dots, c_{j',l})) \in \Lambda$, do the following. 
\begin{itemize}
\item If $j' \in I^{\circled{C_{i'}}}_{pe}$, then $(j', (c'_{j', 1}, \dots, c'_{j', l})) \in \Lambda'$,  where for each $j \in [l]$, 
\begin{itemize}
\item if $\cstl^{\circled{C_{i'}}}_j = 0$, then $c'_{j', j}= \sum \limits_{j'' \sim_{q_m} j'} \csta^{\circled{C_{i'}}}_{j, j''}$, 
%
\item otherwise, $c'_{j', j} = c_{j',j} $ (in this case, the expressions $\csta^{\circled{C_{i'}}}_{j,j''}$ for $j'' \in  I^{\circled{C_{i'}}}_{pe}$ such that $j'' \sim_{q_m} j'$ are removed).
%+ \sum \limits_{j'' \sim_{q_m} j', j'' \in I^{\circled{C_{i'}}}_{tr}} \csta^{\circled{C_{i'}}}_{j, j''}
\end{itemize}
%
\item If $j' \in I^{\circled{C_{i'}}}_{tr}$, then 
$$(k+1, (c'_{j', 1}, \dots, c'_{j', l})), \left(j', \left(\cstb^{\circled{C_{i'}}}_{1, \pi^{\circled{C_{i'}}}_{tr}(j')}, \dots, \cstb^{\circled{C_{i'}}}_{l, \pi^{\circled{C_{i'}}}_{tr}(j')} \right) \right) \in \Lambda',$$ 
where for each $j \in [l]$, if $\cstl^{\circled{C_{i'}}}_j = 0$, then $c'_{j', j}= \sum \limits_{j'' \sim_{q_m} j'} \csta^{\circled{C_{i'}}}_{j, j''} $, otherwise, $c'_{j', j} = c_{j',j} +  \sum \limits_{j'' \sim_{q_m} j'} \csta^{\circled{C_{i'}}}_{j, j''}$. In this case, after going through $C_{i'}$, the control variable $x_{j'}$ stores a fresh value and the initial value of $x_{j'}$ is not stored in any control variable, thus the $(j',\dots)$-tuple is updated and a $(k+1,\dots)$-tuple is added for the initial value of $x_{j'}$.
%
\end{itemize}

\item For each tuple $(k+1, (c_1, \dots, c_l)) \in \Lambda$, we have $(k+1, (c'_1, \dots, c'_l)) \in \Lambda'$, where for each $j \in [l]$, if $\cstl^{\circled{C_{i'}}}_j = 0$, then $c'_{j}= 0$, otherwise, $c'_{j} = c_j$. In addition, for each $j' \in [r^{\circled{C_{i'}}}] \setminus \rng(\pi^{\circled{C_{i'}}}_{tr})$, $(k+1, (\cstb^{\circled{C_{i'}}}_{1, j'}, \dots, \cstb^{\circled{C_{i'}}}_{l, j'})) \in \Lambda'$. 
\end{itemize}
%\end{itemize}
\end{enumerate}


By a simple analysis on the inductive computation of $\absset$, we can show that all the tuples $(j', (c_1, \dots, c_l))$ occurring in $\absset$ satisfy that $c_1,\dots, c_l$ are from a bounded domain $U$, which is stated in the following lemma.

\begin{lemma}\label{prop-bnd-domain-2}
	Suppose that the decision procedure has not returned yet after Step II. 
	For all cycle scheme $\schm$ and $y_j \in Y$, the constant atom and the coefficients of all non-constant atoms in ${\sumf^{(\schm, \sumf^{(H,\initval_0)})}}^-(y_j)$ are from a finite set $U \subset \intnum$ comprising
%	\begin{enumerate}
%	\item 
the constant atom and the coefficients of the non-constant atoms in the expression $\sumf^{(C^{\ell_{i_1}}_{i_1}, \sumf^{(H, \initval_0)})^-}(y_j)$ or $\sumf^{(C_{i_1}C_{i_2}, \sumf^{(H, \initval_0)})^-}(y_j)$ for $i_1, i_2\in [n]$ such that $i_1 \neq i_2$ and $\ell_{i_1} \in \{1,2\}$.
%
\hide{
%
	\item the numbers  $2\sum \limits_{j'' \sim_{q_m} j'} \csta^{\circled{C_{i'}}}_{j,j''}$ for $i' \in [n]$ such that $\cstl^{\circled{C_{i'}}}_j = 0$, $i'' \in [n]$ such that $\cstl^{\circled{C_{i''}}}_j = 1$, and $j' \in I^{\circled{C_{i'}}}_{pe} \cap I^{\circled{C_{i''}}}_{tr}$, 
	%
	\item the numbers $\cstb^{\circled{H}}_{j,\pi^{\circled{H}}_{tr}(j')} + \sum \limits_{j'' \sim_{q_m} j'} \csta^{\circled{C_{i'}}}_{j,j''}$ for $i' \in [n]$ and $j' \in I^{\circled{H}}_{tr} \cap I^{\circled{C_{i'}}}_{tr}$, 

	\item the numbers $\sum \limits_{j'' \sim_{q_0} \pi^{\circled{H}}_{pe}(j')} \csta^{\circled{H}}_{j, j''} + \sum \limits_{ j''' \sim_{q_m} j' } \csta^{\circled{C_{i'}}}_{j,j'''}$ for $i' \in [n]$ and $j' \in I^{\circled{H}}_{pe} \cap I^{\circled{C_{i'}}}_{tr}$, 
	%
	\item the numbers $\cstb^{\circled{C_i}}_{j,\pi^{\circled{C_i}}(j')} + \sum \limits_{j'' \in J} \csta^{\circled{C_{i'_{j''}}}}_{j,j''}$ for $j' \in I^{\circled{C_i}}_{tr}$, $J \subseteq \{j'' \in I^{\circled{C_i}}_{tr} \mid  \pi^{\circled{C_i}}_{tr}(j'') = \pi^{\circled{C_i}}_{tr}(j')\}$, and $i'_{j''} \in [n]$  for each $j'' \in J$. 
}
%	\end{enumerate}
\end{lemma}

Note that although the set $U$ can be exhausted by the cycle schemes stated in Lemma~\ref{prop-bnd-domain-2}, the inductive computation of $\absset$ may not be.
%
%
%%%%%%%%%%%%%%%%%%%%%%%%%%%%%%%%%%%%%%%%%%%%%%%%%%%%%%
%%%%%%%%%%%%%%%%%%%%%%%%%%%%%%%%%%%%%%%%%%%%%%%%%%%%%%
\hide{
Lemma~\ref{prop-bnd-domain-2} follows from the observation that there are only the following two situations that the coefficients may get unbounded, 
\begin{itemize}
\item  $j' \in I^{\circled{C_{i'}}}_{pe}$, $(j', (c_{j',1}, \dots, c_{j',l})) \in \Lambda$, $(j', (c'_{j', 1}, \dots, c'_{j', l})) \in \Lambda'$, $\cstl^{\circled{C_{i'}}}_j = 1$, and $c'_{j', j} = c_{j',j} + \sum \limits_{j'' \sim_{q_m} j', j'' \in I^{\circled{C_{i'}}}_{tr}} \csta^{\circled{C_{i'}}}_{j, j''}$,
%
\item $j' \in I^{\circled{C_{i'}}}_{tr}$, $[j']_{\sim_{q_m}} \cap I^{\circled{C_{i'}}}_{pe} = \emptyset$, $(j', (c_{j',1}, \dots, c_{j',l})) \in \Lambda$, $(k+1, (c'_{j', 1}, \dots, c'_{j', l})) \in \Lambda'$, $\cstl^{\circled{C_{i'}}}_j = 1$, and $c'_{j', j} = c_{j',j} +  \sum \limits_{j'' \sim_{q_m} j'} \csta^{\circled{C_{i'}}}_{j, j''}$.
\end{itemize}
For the former situation above, if $[j']_{\sim_{q_m}} \cap I^{\circled{C_{i'}}}_{tr} \neq \emptyset$, then after going through the cycle $C_{i'}$, none of $j'' \in [j']_{\sim_{q_m}} \cap I^{\circled{C_{i'}}}_{tr}$ would be equivalent to $j'$. Therefore, after going through the cycle $C_{i'}$, the size of the equivalence class containing $j'$ decreases. From this, we deduce that this situation can only happen for at most $k-1$ times, while keeping the control variable $x_{j'}$ persistent. On the other hand, for each $j' \in I^{\circled{C_{i'}}}_{tr}$, the latter situation can only happen once. Therefore, the addition to $c_{j', j}$ with some constant can only happen for at most $k$ times. This justifies the 3rd and 4th item in Lemma~\ref{prop-bnd-domain-2}.
}
%
%
\bigskip\\
\framebox[\textwidth]{
	\begin{minipage}{0.95\textwidth}
		\noindent {\bf Step III.} We first construct the set $\absset$. Then for each $\Lambda \in \absset$, do the following.
		\begin{enumerate}
			\item If $(0,(c_{0,1},\dots,c_{0,l})) \in \Lambda$ such that $a'_0+b'_1 c_{0,1}+\dots + b'_l c_{0,l} \neq 0$, then return $\ltrue$.
			%
			\item If there is $j' \in [k]$ such that $(\sum \limits_{j'' \sim_{q_m} j'} a'_{j''}) + b'_1 c_{j',1} + \dots + b'_l c_{j',l} \neq 0$, then  return $\ltrue$, where $(j', (c_{j',1},\dots,c_{j',l})) \in \Lambda$.
			%
			\item If there is $(k+1,(c_1,\dots,c_l)) \in \Lambda$ such that $b'_1 c_1 + \dots + b'_l c_l \neq 0$, then return $\ltrue$. 
		\end{enumerate}
		If the decision procedure has not returned yet, return $\lfalse$.
	\end{minipage}
}\medskip\\

In order to reduce the size of $\absset$, we can restructure $\absset$ into a pair $\absset'=(\Xi, \Delta)$ as follows, without affecting the computation in Step III.
\begin{enumerate}
\item Initially, let $\Xi = \Delta := \emptyset$.
\item For each $\Lambda \in \absset$, do the following: Let $\Lambda' := \Lambda \setminus \{(k+1, (c_1, \dots, c_l)) \in \Lambda\}$. In addition, let $\Xi := \Xi \cup \{\Lambda'\}$ and $\Delta:= \Delta \cup \{(k+1, (c_1,\dots, c_l)) \in \Lambda\}$. 
\end{enumerate}

\noindent {\it Complexity analysis of Step III}. The size of the set $U$ is polynomial in the size of generalized lasso. Then size of $\Xi$ is at most exponential in $kl$ and the size of $\Delta$ is at most exponential in $l$. Therefore, the size of $\absset'$ is at most exponential in $kl$ and the computation of $\absset'$ takes exponential time in the worst case. The three conditions in Step III can be checked in time polynomial over the size of $\absset'$. In summary, the complexity of Step III is exponential in $kl$, the product of the number of control and data variables.

\begin{example}\label{exmp-step-3}
Let $\Ss'_{\sf max}$ be the SNT in Example~\ref{exmp-step-2}. Then
\[
\begin{array}{l c l}
\sumf^{(C^\ell_1, \initval)^{-}}(y_1) &= & \initval(y_1) + 0 \initval(x_1) +\ \ \vard^{\circled{C_1, 1}}_1 + \dots +\ \ \vard^{\circled{C_1, \ell}}_1,\\
%
\sumf^{(C^\ell_1, \initval)^-}(y_2) &=& \initval(y_2) + 0 \initval(x_1) + 2 \vard^{\circled{C_1, 1}}_1 + \dots + 2 \vard^{\circled{C_1, \ell}}_1,\\
%
\sumf^{(C^\ell_1, \initval)^-}(y_3) &=& \hspace{5.1cm} 3 \vard^{\circled{C_1, \ell}}_1,
\end{array}
\] 
and $\sumf^{(C^\ell_2, \initval)^-}(y_j) = \sumf^{(C^\ell_2, \initval)}(y_j)$ for each $j=1,2,3$.
The computation of $\absset$ starts with the set $\absset_0 = \{\abs(HC_1), \abs(HC_2)\}$. We illustrate how to compute $\abs(HC_1)$ and $\abs(HC_2)$. 
\begin{itemize}
\item Since none of $\sumf^{(H, \initval_0)}(y_j)$'s , $\sumf^{(C^\ell_1, \initval)^{-}}(y_j)$'s and $\sumf^{(C^\ell_2, \initval)^{-}}(y_j)$'s contains constant atoms, we know that $(0, (0,0,0))$ occurs in $\abs(HC_1)$ and  $\abs(HC_2)$. 
%
\item After going through $H$, each of $y_1,y_2,y_3$ is assigned the first data value, and after going through $HC_1$, $x_1$ holds the first data value. From $\sumf^{(C^\ell_1, \initval)^{-}}(y_j)$'s mentioned above, we know that each of $\sumf^{(C_1, \sumf^{(H, \initval_0)^{-}})}(y_1)$ and $\sumf^{(C_1, \sumf^{(H, \initval_0)^{-}})}(y_2)$ holds one copy of the first data value,  and $\sumf^{(C_1, \sumf^{(H, \initval_0)^{-}})}(y_3)$ contains no copies of the first data value. Therefore, $(1, (1,1,0))$ occurs in $\abs(HC_1)$. Similarly, since $x_1$ holds the second data value after going through $HC_2$, we have $(1, (3, 2, 1))$ occurs in $\abs(HC_2)$. 
%
\item In addition, by a simple calculation, we know that $\abs(HC_1)$ contains another tuple $(2,(1, 2, 3))$ and $\abs(HC_2)$ contains another tuple $(2, (2, 4, 6))$.
\end{itemize}
To summarize, $\abs(HC_1)=(\{(1,1)\}, \{(0, (0,0,0)), (1, (1, 1, 0)), (2, (1, 2, 3))\})$ and $\abs(HC_2) = (\{(1,1)\}, \{(0, (0, 0, 0)), (1, (3, 2,1)), (2, (2, 4, 6))\})$. Then starting from $\absset_0$, we compute $\absset_1 = \absset_0 \cup \{\absset(HC_{i_1}C_{i_2}) \mid i_1, i_2 = 1, 2\}$, and so on. We illustrate how to compute $\abs(HC_1 C_2)$ from $\abs(HC_1)$. 
%Suppose $\abs(HC_1C_2) = (\{(1,1)\}, \Lambda')$. 
\begin{itemize}
\item Since $(0, (0,0,0))$ occurs in $\abs(HC_1)$ and $\cstl^{\circled{C_2}}_1 = \cstl^{\circled{C_2}}_2 = \cstl^{\circled{C_2}}_3 = 1$, we know that $(0, (0,0,0)) \in \abs(HC_1C_2)$.

\item From $1 \in I^{\circled{C_2}}_{tr}$ and $\cstl^{\circled{C_2}}_1 = \cstl^{\circled{C_2}}_2 = \cstl^{\circled{C_2}}_3 = 1$, we have $(2, (1 + \csta^{\circled{C_2}}_{1, 1}, 1+\csta^{\circled{C_2}}_{2, 1}, 0 + \csta^{\circled{C_2}}_{3, 1})) = (2, (1+1, 1+3, 0 +5))= (2, (2, 4, 5)) \in \abs(HC_1C_2)$. In addition, we have $(1, (\cstb^{\circled{C_2}}_{1, 1}, \dots, \cstb^{\circled{C_2}}_{3, 1}))=(1, (3, 2, 1)) \in \abs(HC_1C_2)$.

\item Since $(2, (2, 4, 6))$ occurs in $\abs(HC_1)$ and $\cstl^{\circled{C_2}}_1 = \cstl^{\circled{C_2}}_2 = \cstl^{\circled{C_2}}_3 = 1$, we have $(2, (2, 4, 6)) \in \abs(HC_1C_2)$. Moreover, because $[r^{\circled{C_2}}] \setminus \rng(\pi^{\circled{C_2}})=\emptyset$, no other tuples are added into $\abs(HC_1C_2)$.
\end{itemize}
In summary, 
$$\abs(HC_1C_2)= \{(0,(0,0,0)), (1, (3,2,1)), (2, (2,4,5)), (2, (2, 4,6))\}.$$
From the fact that $(1, (1, 1, 0))$ occurs in $\abs(HC_1)$, we know that $a'_1 + b'_1 \times 1 + b'_2 \times 1 + b'_3 \times 0 = 0 + 1 \times 1 + (-2) \times 1 + 1 \times 0 = -1 \neq 0$. Therefore, Step III returns $\ltrue$ and a non-zero output can be produced by following the path $HC_1$.
\end{example}
