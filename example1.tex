\begin{figure}
	\centering
	\lstset{language=C,
		basicstyle=\ttfamily\scriptsize}
	\begin{tabular}{|c|c|c|}
		\hline
		\begin{minipage}[t]{0.39\textwidth}
		\vspace{-0.5cm}
			\begin{lstlisting}[mathescape=true]
max_abs {
 if $\cur$>0 then max:=$\cur$
 else max:= -$\cur$;
 $\nnext$;
 loop{
  if $\cur$>0 then
   if $\cur$>max then max:=$\cur$;
  else 
   if -$\cur$>max then max:=-$\cur$;;
  $\nnext$ };
 ret max;}
	\end{lstlisting}
		\end{minipage}&
		\begin{minipage}[t]{0.27\textwidth}
		\vspace{-0.5cm}
			\begin{lstlisting}[mathescape=true]
sum{
 sum:=$\cur$;$\nnext$;
 loop{sum+=$\cur$;$\nnext$};
 ret sum;}
			\end{lstlisting}
\hrule\vspace{0.1cm}%
			%(c)	
			\begin{lstlisting}[mathescape=true]
cnt{
 cnt:=0;$\nnext$;
 loop{cnt+=1;$\nnext$};
 ret cnt;}
			\end{lstlisting}			
		\end{minipage}&
		\begin{minipage}[t]{0.30\textwidth}
		\vspace{-0.5cm}			
			\begin{lstlisting}[mathescape=true]
2nd_largest {
 a:=$\cur$;b:=$\cur$;$\nnext$;
 if $\cur$>a then a:=$\cur$
 else b:=$\cur$;$\nnext$;
 loop{
  if $\cur$>a then 
   b:=a;a:=$\cur$
  else if $\cur$>b then 
   b:=$\cur$;;
  $\nnext$};
 ret b;}
			\end{lstlisting}		
		\end{minipage}\\
		\hline		
	\end{tabular}
	\caption{Examples of Reducers Performing Data Analytics Operations}
	\label{fig:examples}
\end{figure}
\vspace{-0.5cm}
