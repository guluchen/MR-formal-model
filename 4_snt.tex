%!TEX root = main-cav.tex


%%%%%%%%%%%%%%%%%%%%%%%%%%%%%%%%%%%%%%%%%%%%%%%%
%%%%%%%%%%%%streaming numerical transducer%%%%%%%%%%%%%%%%
%%%%%%%%%%%%%%%%%%%%%%%%%%%%%%%%%%%%%%%%%%%%%%%%

\section{Streaming Numerical Transducers}\label{sec-def-snt}

In this section, we introduce \emph{streaming numerical transducers} (SNTs), whose inputs are data words and outputs are integer values. Over a data word, a SNT scans the data word from left to right, records and computes information using control and data variables, and outputs a data value when it finishes reading the data word. We will use SNTs to decide the commutativity and equivalence problem of the Reduce programs defined in Section~\ref{sec-mr-prog}.


A SNT $\Ss$ is a tuple $(Q, X, Y, \delta, q_0, O)$ where 
\begin{itemize}
\item $Q$ is a finite set of states,
%
\item $X$ is a finite set of control variables to store data values that have been met,
%
\item $Y$ is a finite set of data variables to aggregate information for the output,
%
\item $\delta$ comprises the tuples $(q,  g, \eta, q')$, where $q,q'\in Q$, $g$ is a guard over $X^+$ (note that the variables from $Y$ do not occur in $g$), and $\eta$ is an assignment which is a partial function mapping $X \cup Y$  to $\Ee_{X^+ \cup Y}$ such that for each $x \in \dom(\eta) \cap X$, $\eta(x)=\cur$ or $\eta(x) = x'$ for some $x' \in X$.
%
\item $q_0 \in Q$ is the initial state,
%
\item $O$ is the output function, which is a partial function from $Q$ to $\Ee_{X \cup Y}$.%\zhilin{Here the variable $\cur$ should not be used.}
\end{itemize}
Moreover, we assume that the SNT $\Ss$ satisfies the following constraints.
\begin{description}
\item [Deterministic:] For each pair of distinct transitions originating from $q$, say $(q, g_1, \eta_1,q'_1)$ and $(q, g_2,\eta_2,q'_2)$, it holds that $g_1 \wedge g_2$ is unsatisfiable.
%
\item [Generalized flat:] Each SCC (strongly connected component) of the transition graph of $\Ss$ is either a state or a set of cycles sharing a unique state.
%
%\item[3 (Independently evolving).] For each $(q, g, \eta, q') \in \delta$ and each $y \in Y$, $\eta(y)$ contains no variables $y' \in Y$ such that $y' \neq y$.
%
\item[Independently evolving and copyless:] For each $(q, g, \eta, q') \in \delta$ and for each $y \in \dom(\eta)$, $\eta(y)=e$ or $\eta(y)=y+e$ for some expression $e$ over $X^+$.
%
\end{description}

We write $q \xrightarrow{(g,\eta)} q'$ to denote $(q,g,\eta,q') \in \delta$ for convenience. 
The semantics of a SNT $\Ss$  is defined as follows. A \emph{configuration} of $\Ss$ is a pair $(q,\rho)$, where $q \in Q$ and $\rho$ is a valuation of $X \cup Y$. 


The \emph{initial} configuration of $\Ss$ is $(q_0,\rho_0)$, where $\rho_0(z)=\bot$ for each $z \in X\cup Y$.
A sequence of configurations $(q_0,\rho_0)(q_1,\rho_1)\ldots(q_n,\rho_n)$ is
a \emph{run} of $\Ss$ over a data word $w=d_1 \dots d_n$ iff there exists a sequence of transitions $q_0 \xrightarrow{(g_1,\eta_1)} q_1 \xrightarrow{(g_2,\eta_2)} q_2 \dots q_{n-1} \xrightarrow{(g_n, \eta_n)} q_n$ such that for each $i: 1 \le i \le n$, $\rho_{i-1}[d_i/\cur]  \models g_i$, and $\rho_i$ is obtained from $\rho_{i-1}$ as follows,
\begin{itemize}
\item for each $x \in X$, if $ x \in \dom(\eta_i)$, then $\rho_i(x)=d_i$, otherwise, $\rho_i(x)=\rho_{i-1}(x)$,
%
\item for each $y \in Y$, if $y \in \dom(\eta_i)$, then $\rho_i(y)=(\rho_{i-1}[d_i/\cur])(\eta_i(y))$, otherwise, $\rho_i(y)=\rho_{i-1}(y)$,
\end{itemize}
We will call $(q_n,\rho_n)$ as the \emph{final configuration} of the run.
We would like to remark that for each data word $w$, there is at most one run of $\Ss$ over $w$, since $\Ss$ is deterministic. 
Over a data word $w = d_1 \dots d_n$, if there is a run of $\Ss$ over $w$ with the final configuration $(q_n,\rho_n)$, and $O(q_n)$ is defined, then the output of $\Ss$ over $w$, denoted by ${\Ss}(w)$, is $\eval{O(q_n)}{\rho_n}$. Otherwise, ${\Ss}(w)$ is $\bot$.

\begin{example}[SNT for max]
$\Ss_{\max}=(\{q_0,q_1\}, \{\maxv\}, \emptyset, \delta, q_0, O)$, such that $\delta = \{(q_0, \ltrue, \maxv:=\cur, q_1), (q_1, \maxv < \cur, \maxv:=\cur,q_1), (q_1, \maxv \ge \cur, \eta_\bot, q_1)\}$, and $O(q_1)=\maxv$, where $\maxv:=\cur$ denotes the function that assigns $\cur$ to $\maxv$ and $\eta_\bot$ denotes the assignment function of the empty domain.
\end{example}
\begin{example}[SNT for sum]
$\Ss_{\mathrm{sum}}=(\{q_0,q_1\}, \emptyset, \{\sumv\}, \delta, q_0, O)$ such that $\delta=\{(q_0, \ltrue, \sumv:=\cur, q_1), (q_1, \ltrue, \sumv:=\sumv + \cur, q_1)$, and $O(q_1)=\sumv$. 
\end{example}
\begin{proposition}\label{prop-mrprog-to-snt}
For each reducer program $p$, an equivalent SNT $\Ss$ (w.r.t. the commutativity and equivalence problems) can be constructed.
\end{proposition}

The main difference between $p$ and $\Ss$ is that several transitions in the semantic graph of $P$ correspond to one transition in $\Ss$. A reducer program moves to the next input only when a $\next$ statement is encountered while a SNT reads one input symbol in one transition. Based on this observation, we give a work-list algorithm translating a program $P$ to an SNT $\Ss$ in algorithm~\ref{fig:reducer2SNT} the appendix.


%\begin{example}[Example inspired by Pagerank]
%The following transducer sum all the data values, except the last position, then it outputs a concatenation of the sum and the last tuple: $(q_0, 1, true, sum:= sum + p_1, q_0)$, $(q_0, k, true, (x_i:=p_i)_{1 \le i \le k}, q_1)$, $O(q_1)=(sum, x_1,\dots, x_k)$.
%\end{example}



%We first compute a fixed point $\defval$ inductively as follows. 
%\begin{enumerate}
%\item Initially, let $\defval_0=\{(q_0,\emptyset)\}$.
%
%\item For each $i > 0$, compute $\defval_i$ from $\defval_{i-1}$ as follows,
%\begin{itemize}
%\item each element of $\defval_{i-1}$ is an element of $\defval_i$, 
%
%\item for each $(q,Z) \in \defval_{i-1}$ and each transition $(q,g,\eta,q') \in \delta$,  let $Z' = Z \cup (X \cap \dom(\eta)) \cup \{y \in Y \cap \dom(\eta) \mid \vars(\eta(y)) \subseteq Z \cup \{\cur\}\}$, put $(q',Z')$ into $\defval_i$.
%\end{itemize}
%
%\item If $\defval_{i-1}=\defval_i$, then the computation stops, otherwise, let $i:=i+1$ and the computation continues.
%\end{enumerate}




We focus on three decision problems of SNTs:
\begin{description}
\item[Commutativity:] Given a SNT $\Ss$, decide whether $\Ss$ is commutative, that is, whether for each data word $w$ and each permutation $w'$ of $w$, $\Ss(w)=\Ss(w')$.
%
\item[Equivalence:] Given two SNTs $\Ss_1,\Ss_2$, decide whether $\Ss_1$ and $\Ss_2$ are equivalent, that is, whether over each data word $w$, $\Ss_1(w)=\Ss_2(w)$.
%
\item[Non-zero output:] Given a SNT $\Ss$, decide whether $\Ss$ has a non-zero output, that is, whether there is a data word $w$ such that $\Ss(w)\notin \{\bot, 0\}$. 
\end{description}
The following propositions state that the commutativity problem can be reduced to the equivalence problem, which in turn can be reduced to the non-zero output problem.

\begin{proposition}\label{prop-snt-cmm-to-eqv}
The commutativity problem of SNTs is reduced to the equivalence problem of SNTs in exponential time. 
\end{proposition}

\begin{proposition}\label{prop-snt-eqv-to-nzero}
From SNT $\Ss_1$ and $\Ss_2$, a SNT $\Ss_3$ can be constructed in polynomial time such that $\Ss_1$ and $\Ss_2$ are inequivalent iff there is a data word $w$ such that the output of $\Ss_3(w) \not\in \{\bot,0\}$. 
\end{proposition}



We \emph{normalize} the SNTs in order to facilitate the presentation of the decision procedure.
Suppose $\Ss=(Q,X,Y,\delta,q_0,O)$ is a SNT. Let $c_{min}$ and $c_{max}$ denote the minimum resp. maximum integer constant occurring in the guards of the transitions of $\Ss$. If no integer constant occurs in the guards, then $c_{min}$ and $c_{max}$ are undefined.

A SNT $\Ss=(Q,X,Y,\delta,q_0,O)$ is said to be normalized if the following constraints are satisfied.
\begin{itemize}
	\item Well-defined: For each run $(q_0,\rho_0),\dots,(q_n,\rho_n)$ of $\Ss$, if $O(q_n)$ is defined, then $\rho_n(z)\neq\bot$ for all  $z\in \vars(O(q_n))$.
	\item Unique-valued: For each $(q,g,\eta,q') \in \delta$, if $\eta(x)=\cur$ for some $x \in X$, then the guard $g$ implies $\bigwedge_{x \in X} \cur \neq x$.  Intuitively, when the current data value $\cur$ is stored into some control variable, it is required that $\cur$ is distinct from all the data values that have already been stored in the control variables.
	%
	\item Constant-partitioned: If $c_{min}$ and $c_{max}$ are defined. For each $(q, g, \eta, q') \in \delta$, the guard $g$ implies one of the followings: $\cur < c_{min}$, $\cur = c$ for $c_{min} \le c \le c_{max}$, $\cur > c_{max}$. 
\end{itemize}


\begin{proposition}\label{prop-snt-norm}
	From each SNT, an equivalent normalized SNT can be constructed in time exponential to the number of control variables. 
\end{proposition}
The construction can be found in the appendix. The idea behind is simple. To ensure the constructed SNT is well-defined, we record in the state the set of variables with defined values and change the value output function $O(q)$ to undefined if some undefined variables occur in $O(q)$. To ensure the result is unique-valued, we record in states the equivalence relation between control variables and modify the assignments accordingly. To ensure the result is 