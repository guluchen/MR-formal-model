\documentclass[runningheads,a4paper]{llncs}

\usepackage{latexsym}
\usepackage{setspace}
\usepackage{cancel}
\usepackage{listings}
\usepackage{graphicx}
\usepackage{appendix}
\usepackage{amssymb}
\usepackage{stmaryrd}
\usepackage{amsmath}
\usepackage{leftidx}
\usepackage{mathtools}
\usepackage[linesnumbered,noend]{algorithm2e}
\usepackage{paralist}
\usepackage{color}
\usepackage{mathrsfs}
\usepackage{tikz}
\usetikzlibrary{shapes}
\usepackage{times}
\newcommand{\hide}[1]{}

\newcommand{\eval}[2]{\llbracket#1\rrbracket_{#2}}
\newcommand\cur{\mathsf{cur}}
\newcommand\dom{\mathsf{dom}}
\newcommand\rng{\mathsf{rng}}

\newcommand{\yfc}[1]{\color{blue} {YF: #1 :FY} \color{black}}
\newcommand{\zhilin}[1]{\color{cyan} {ZL: #1 :LZ} \color{black}}

\title{The~Commutativity~Problem~of~the~MapReduce Framework: A Transducer-based Approach}
\titlerunning{Commutativity of MapReduce: A Transducer-based Approach}
\author{}
\institute{}

%\author{Yu-Fang Chen, Lei Song, Zhilin Wu}

\begin{document}

\maketitle

\section{SNT}
Let $X^+$ be the set $X\cup \{ \cur \}$.
We use $B$ and $U$ to denote the set of all uninterpreted binary and unary functions, respectively. For the set of binary functions, we use subscripted $A$ and $C$ to denote the set of associative and commutative binary functions, respectively. Notice that we have $B_{AC}\subset B_A \subset B$ and $B_{AC}\subset B_C \subset B$.
Let $E(X, B, U)$ be the set of expressions over variables $X$, $B$, and $U$.
A guard over $X^+$ is a predicate defined by $g:= p(x_1,\ldots,x_n) \mid \neg g \mid g\wedge g$, where $x_1,\ldots,x_n\in X^+$ and $p$ is an uninterpreted predicate over $x_1,\ldots,x_n$. 

A SNT $S$ is a tuple $(Q, X, \delta, q_0, O)$ where $Q$ is a finite set of states, $X$ is a finite set of variables, $\delta$ is the set of transitions, $q_0 \in Q$ is the initial state, $O$ is the output function, which is a partial function from $Q$ to $E(X,B, U)$
The set of transitions $\delta$ comprises the tuples $(q, g, \eta, q')$, where $q,q'\in Q$, $g$ is a guard over $X^+$ or , $\eta$ is an assignment which is a partial function mapping $X$  to $E(X,B, U)$. We write $q \xrightarrow{(g,\eta)} q'$ to denote $(q,g,\eta,q') \in \delta$ for convenience. 

Moreover, we assume that an SNT $S$ satisfies the following constraints. (1) \emph{Deterministic:} For each pair of distinct transitions originating from $q$, say $(q, g_1, \eta_1,q'_1)$ and $(q, g_2,\eta_2,q'_2)$, it holds that $g_1 \wedge g_2$ is unsatisfiable and (2) \emph{Generalized flat:} Each SCC (strongly connected component) of the transition graph of $S$ is either a single state or a set of simple cycles $\{C_1,\dots, C_n\}$ which contains a state $q$ such that for each $i,j: 1 \le i < j \le n$, $q$ is the \emph{only} state shared by $C_i$.

The semantics of an SNT $S$  is defined as follows. A \emph{configuration} of $S$ is a pair $(q,\rho)$, where $q \in Q$ and $\rho$ is a valuation of $X$. The \emph{initial} configuration of $S$ is $(q_0,\rho_0)$, where $\rho_0(z)=z_0$ for each $z \in X$.
A sequence of configurations $(q_0,\rho_0)(q_1,\rho_1)\ldots(q_n,\rho_n)$ is
a \emph{run} of $S$ over a data word $w=d_1 \dots d_n$ iff there exists a path (sequence of transitions) $q_0 \xrightarrow{(g_1,\eta_1)} q_1 \xrightarrow{(g_2,\eta_2)} q_2 \dots q_{n-1} \xrightarrow{(g_n, \eta_n)} q_n$ such that for each $i \in [n]$, $\rho_{i-1}[d_i/\cur] \models g_i$, and $\rho_i$ is obtained from $\rho_{i-1}$ as follows: For each $x \in X$, if $x \in \dom(\eta_i)$, then $\rho_i(x)=\eval{\eta_i(x)}{\rho_{i-1}[d_i/\cur]}$, otherwise, $\rho_i(x)=\rho_{i-1}(x)$.
We call $(q_n,\rho_n)$ the \emph{final configuration} of the run. %We say that $(q_i,\rho_i)$ is \emph{reachable} from $(q_0,\rho_0)$, for $i \in [n]$.

We would like to remark that for each data word $w$, there is at most one run of $S$ over $w$, since $S$ is deterministic. 
Over a data word $w = d_1 \dots d_n$, if there is a run of $S$ over $w$ with the final configuration $(q_n,\rho_n)$, and $O(q_n)$ is defined, then the output of $S$ over $w$, denoted by ${S}(w)$, is $\eval{O(q_n)}{\rho_n}$. Otherwise, ${S}(w)$ is $\bot$.

We focus on three decision problems of SNTs: (1) \emph{Commutativity}: Given an SNT $S$, decide whether $S$ is commutative, that is, whether for each data word $w$ and each permutation $w'$ of $w$, $S(w)=S(w')$. (2) \emph{Equivalence}: Given two SNTs $S_1,S_2$, decide whether $S_1$ and $S_2$ are equivalent, that is, whether over each data word $w$, $S_1(w)=S_2(w)$.
\bibliographystyle{abbrv}
\bibliography{data}


\end{document}