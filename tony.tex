\documentclass[runningheads,a4paper]{llncs}

\usepackage{latexsym}
\usepackage{setspace}
\usepackage{cancel}
\usepackage{listings}
\usepackage{graphicx}
\usepackage{appendix}
\usepackage{amssymb}
%\usepackage{dsfont}
\usepackage{amsmath}
%\usepackage{amsthm}
%\usepackage{cancel}
%\usepackage{verbatim}
%\usepackage{chngpage}
%\usepackage{fullpage}

\usepackage{color}

\usepackage{mathrsfs}

%\newtheorem{definition}{Definition}
%\newtheorem{theorem}{Theorem}
%\newtheorem{proposition}[theorem]{Proposition}
%\newtheorem{corollary}[theorem]{Corollary}
%\newtheorem{example}{Example}
%\newtheorem{question}{Open Question}
%\newtheorem{lemma}[theorem]{Lemma}
%\newtheorem{Algo}{Algorithm}
%\newtheorem{remark}[theorem]{Remark}

\def\arr#1{\stackrel{#1}{\longrightarrow}}

\def\Aa{{\mathscr{A} }}

\def\Bb{{\mathscr{B} }}

\def\Cc{{\mathcal{C} }}

\def\Dd{{\mathbb{D} }}

\def\Ee{{\mathcal{E} }}

\def\Ff{{\mathcal{F} }}

\def\Zz{{\mathcal{Z} }}

\def\Nn{{\mathbb{N} }}

\def\Ss{{\mathcal{S} }}

\def\schm{{\mathfrak{s} }}

\def\Tt{{\mathcal{T} }}

\def\Ii{{\mathbb{Z} }}

\def\Jj{{\mathcal{J}}}

\def\Vv{{\mathcal{V}}}

\def\Rr{{\mathcal{R} }}

\def\Ll{{\mathcal{L}}}


\def\treeset{{\mathscr{T}}}

\def\contextset{{\mathcal{C}}}

\def\theory{{\mathcal{L}}}

\def\termset{{\mathcal{T}}}

\def\formulaset{{\mathcal{F}}}

\newcommand\univ{\mathsf{Univ}}

\newcommand\op{\mathfrak{o}}

\newcommand\dv{\mathtt{x}}

\newcommand\ydv{\mathtt{y}}

\newcommand\cv{\mathtt{z}}

\newcommand\thla{\mathcal{LIA}}

\newcommand\thdif{\mathcal{DIF}}

\newcommand\thord{\mathcal{ORD}}

\newcommand\thset{\mathcal{SET}}

\newcommand\thmset{\mathcal{MUS}}

\newcommand\natnum{{\mathbb{N} }}

\newcommand\intnum{{\mathbb{Z} }}

\newcommand{\hide}[1]{}
\newcommand{\yfc}[1]{\color{blue} {YF: #1 :FY} \color{black}}
\newcommand{\zhilin}[1]{\color{cyan} {ZL: #1 :LZ} \color{black}}
\newcommand{\lei}[1]{\color{green} {LE: #1 :EL} \color{black}}
\newcommand{\SDSIT}{SDSIT}
\newcommand{\Name}{Streaming data string to integer transducer}
\newcommand{\name}{streaming data string to integer transducer}
%\def\Ss{{$\mathcal{A}$\ }}

\title{The Formal Model}
\author{}
\institute{}

%\author{Yu-Fang Chen, Lei Song, Zhilin Wu}

\begin{document}

\maketitle


For a mapping $\pi$, let $dom(\pi)$ and $rng(\pi)$ denote the domain and range of $\pi$.
Let $\Ii$ denote the set of integers. Let $X$ denote a finite set of variables ranging over $\Ii$ and $x,y$ are variables in $X$. We assume that $X$ contains a special variable $cur$, which is used to denote the current data value. Then a guard over $X$ is a formula defined by the rules, $g::= x\ o\ c \mid x\ o\ x \mid g \wedge g \wedge \neg g$, where $o \in \{=,\neq,<, >, \le, \ge\}$. Let $Y$ be another set of variables over $\Ii$. An arithmetic expression over $Y$ is defined by the rules, $e::= y \mid c \mid e + e \mid e-e$, where $y \in Y$ and $c$ is a constant over $\Ii$. For an expression $e$, let $vars(e)$ denote the set of variables occurring in $e$. Let $\Ee_Y$ denote the set of arithmetic expressions over $Y$. An assignment $\eta$ for $X \cup Y$ is a partial function from $(X \setminus \{cur\}) \cup Y$ to $\Ee_{X \cup Y}$ (the set of arithmetic expressions over $X \cup Y$) such that for each $x \in dom(\eta) \cap X$, $\eta(x)\in X$. Note that the assignments to the special variable $cur$ are disallowed.
A data word is a sequences of data values, that is, sequences $d_1\dots d_n$, where $d_i \in \Ii$ for each $i$.

A streaming numerical transducer (SNT) $\Ss$ is a tuple $(Q, X, Y, \delta, q_0, O)$ such that 
\begin{itemize}
\item $Q$ is a finite set of states,
%\item $K \in \Nn$ is the maximum number of data values in each position, 
\item $X$ is a finite set of control variables which is used to store data values that have been met,
\item $Y$ is a finite set of data variables, which is used to aggregate information for the output,
\item $\delta$ comprises the tuples $(q,  g, \eta, q')$, where $q,q'\in Q$, $g$ is a guard over $X$ (note that the variables from $Y$ do not occur in $g$), $\eta$ is an assignment for $X \cup Y$, 

\item $q_0 \in Q$ is the initial state,
\item $O$ is the output function, which is a partial function from $Q$ to $\Ee_{(X \setminus \{cur\}) \cup Y}$.
\end{itemize}

We restrict our attention to SNTs $\Ss$ to deterministic ones. That is, for every pair of distinct tuples $(q, g_1, \eta_1,q'_1), (q, g_2,\eta_2,q'_2) \in \delta$, it holds that $g_1$ and $g_2$ are mutually exclusive, that is, $g_1 \wedge g_2$ is unsatisfiable.

The semantics of a SNT $\Ss$  is defined as follows: A configuration of $\Ss$ is a pair $(q,\beta)$, where $q \in Q$ and $\beta$ is a valuation of $X \cup Y$, that is, a partial function from $X \cup Y$ to $\Ii$. When reading a data word $w=d_1 \dots d_n$, $\Ss$ runs over $w$ from left to right. Let $(q, \beta)$ be the configuration of $\Ss$ reached after running over $w$, then the output of $\Ss$ is $\beta(O(q))$, if $O(q)$ is defined and $vars(O(q)) \subseteq dom(\beta)$, and the output is undefined otherwise.

We focus on the following decision problems of SNT.
\begin{description}
\item[(Commutativity).] Given a SNT $\Ss$, decide whether $\Ss$ is commutative, that is, whether for each data word $w$ and each permutation $w'$ of $w$, the output of $\Ss$ over $w$ is equal to that of $\Ss$ over $w'$.
%
\item[(Non-zero output reachability).] Given a SNT $\Ss$, decide whether $\Ss$ has a non-zero output, that is, whether there is an input $w$ such that the output of $\Ss$ on $w$ is non-zero. 
\end{description}





\end{document}
%%%%%%%%%%%%%%%%%%%%%%%%%%%%%%%%%%%%%%%%%%%%%%%%%%%%%
%%%%%%%%%%%%%%%%%%%%%%%%%%%%%%%%%%%%%%%%%%%%%%%%%%%%%%%%%%%



