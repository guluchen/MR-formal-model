%!TEX root = main-cav.tex

\section{Preliminaries}
%
Let $\intnum$,  $\intnum^+$ be the set of integers, positive integers, respectively and $\bot \not \in \intnum$ be  the undefined value. By convention, it is assumed that $\bot + \bot = \bot$, $\bot + n = \bot$ for each $n \in \intnum$, $0 \times \bot = 0$, and $n \times \bot = \bot$ for each $n \in \intnum^+$.
In this paper, we assume that all variables range over $\intnum \cup \{\bot\}$.  

For a function $\pi$, let $\dom(\pi)$ and $\rng(\pi)$ denote the \emph{domain}  and \emph{range} of $\pi$, respectively. 
An \emph{expression} $e$ over the set of variables $Z$ is defined by the following rules, $e\equiv  c \mid  c z \mid (e + e) \mid (e - e)$, where $z \in Z$ and $c\in \intnum$.  As a result of the commutativity of $+$, without loss of generality, we assume that all expressions $e$ in this paper are of the form $c_1 z_1 + \dots c_n z_n$, where $c_1,\dots,c_n \in \intnum$ and $z_1,\dots,z_n \in Z$. % \zhilin{I changed a bit the definition of expressions here} 

For an expression $e$, let $\vars(e)$ denote the set of variables occurring in $e$. Let $\Ee_Z$ denote the set of all expressions over the set of variables $Z$. Let $e$ be an expression and $\eta$ be a partial function from $\vars(e)$ to expressions. Then we use $\sub{e}{\eta}$ to denote the expression obtained from $e$ by replacing each variable $z \in \vars(e)\cap\dom(\eta)$ with $\eta(z)$. A \emph{valuation} $\rho$ of $Z$ is a function from $Z$ to $\intnum\cup\{\bot\}$. The value of $e$ under $\rho$, denoted by $\eval{e}{\rho}$, is defined as the value obtained by replacing each variable $z \in \vars(e)$ in $e$ with $\rho(z)$.
A \emph{symbolic valuation} $\sval$ of $Z$ is a function which maps each variable $z \in Z$ to an expression (possibly over a different set of variables). 
For a function $\rho$, we define the function $\rho[d/x]$ such that $\rho[d/x](x)=d$ and $\rho[d/x](y)=\rho(y)$ for $y\neq x$. 

In this paper, we use $X$ and $Y$ to denote the sets of \emph{control variables} and \emph{data variables}, respectively. We use the variable $\cur \notin X\cup Y$ to store the data value that is currently being processed in the input list and use $X^+$ to denote the set $X\cup \{\cur\}$.
A \emph{guard} is a formula defined by the following rules, $g::= \ltrue \mid \cur\odot c \mid \cur\odot x \mid g \wedge g$, where $\odot \in \{=,\neq,<, >, \le, \ge\}$, $x \in X^+$, and $c\in \intnum$. 
Let $\rho$ be a valuation and $g$ be a guard. Then $\rho$ satisfies $g$, denoted by $\rho \models g$, iff for each variable $z \in \vars(g)$, $\rho(z) \neq \bot$, and $g$ is evaluated to $\ltrue$ under $\rho$. 

Let $\interval{n}$ be the set $\{ 1, 2, \ldots, n \}$. Let $\interval{a,b}$ be the set $\{ a, a+1, \ldots, b \}$ when $b\geq a$ and $\emptyset$ otherwise. A \emph{permutation} on
$\interval{n}$ is a one-to-one and onto mapping from $\interval{n}$ to
$\interval{n}$. The set of
permutations on $\interval{n}$ is denoted by $S_n$.
\hide{Define the two functions $\pi_2$ and $\pi_n$ as follows : $\pi_2(1)=2,\pi_2(2)=1,\pi_2(k)=k$, for $k>2$ and
$\pi_n(1)=2,\pi_n(2)=3,\ldots,\pi_n(n)=1, \pi_n(k)=k$, for $k>n$.

\begin{theorem}[\cite{algebra}]
	For every permutation $\sigma \in S_n$, $\sigma$ is equal to a
	composition of $\tau_2$ and $\tau_n$.
	\label{theorem:symmetric-basis}
\end{theorem}
}
A \emph{list $w$} is a sequence of integer values $d_1\ldots d_n$ such that $d_i \in \intnum$ for each $i$.
We use $\head(w)$, $\tail(w)$, and $|w|$ to denote the \emph{head}, \emph{tail}, and \emph{length} of $w$, respectively.
We use $\emptyset$ to denote an empty list.
Given two lists $w,w'$, we use $w.w'$ to denote their concatenation.
Under the context of a transducer, we use the alternative name \emph{data word} to refer to a list.
Given a list $l=d_1\ldots d_n$ and $\sigma \in S_n$, we lift $\sigma$ to list by defining $\sigma(l)=d_{\pi(1)} \ldots d_{\pi(n)}$ and call $\sigma(l)$ a permutation of $l$.