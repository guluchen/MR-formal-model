%!TEX root = main-cav.tex

\begin{appendix}

\section{Formal Semantics of the Programming Language}
\begin{figure}
	\hspace{-0.4cm}
	\scalebox{0.9}{
		\begin{tabular}{|l|l|}
			\hline
			Transitions&
			Side Condition\\
			\hline
			$(y := e;p, l, \rho) \longrightarrow (p, l, \rho')$&
			$\rho'(z) =\rho(z)$ for $z\neq y$, $\rho'(y) = \eval{\rho}{e}$\\
			
			$(y \addeq e;p, l, \rho) \longrightarrow (p, l, \rho')$&
			$\rho'(z) =\rho(z)$ for $z\neq y$, $\rho'(y) = \eval{y+e}{\rho}$\\
			
			$(\ite{g}{s_1}{s_2};p, l, \rho) \longrightarrow (s_1;p, l, \rho)$&
			$\eval{g}{\rho} =\ltrue$\\
			
			$(\ite{g}{s_1}{s_2};p, l, \rho) \longrightarrow (s_2;p, l, \rho)$& $\eval{g}{\rho} =\lfalse$\\
			
			$(\nnext;p, l, \rho) \longrightarrow (p, \tail(l), \rho')$&
			$\rho'(z) =\rho(z)$ for $z\neq \cur$, $\rho'(\cur) = \head(l)$, $l\neq \emptyset$\\
			
			$(x':=x;p, l, \rho) \longrightarrow (p, l, \rho')$&
			$\rho'(z) =\rho(z)$ for $z\neq x'$, $\rho'(x') = \rho(x)$\\
			
			$(\loopL{s;}\mbox{ret }r, l, \rho) \longrightarrow (s;\loopL{s;}\mbox{ret }r, l, \rho)$& \\
			
			$(\loopL{s;}\mbox{ret }r, \emptyset, \rho) \longrightarrow (\mbox{ret }r,  \emptyset, \rho)$& 	\\	
			\hline
			
		\end{tabular}
	}
	\caption{The Semantics of the Programming Language}
	\label{fig:semantics}
\end{figure}

Formally, the semantics of a program $p_0$ in the programming language can be defined as a graph in Figure~\ref{fig:semantics}. Each node in the graph is a triple $(p,l,\rho)$, where $p$ is a program, $l$ is the input list, and $\rho$ is an valuation over $X^+\cup Y$ such that $\rho(\cur)=\head(l)$. 
Let $\rho_l$ be an assignment that $\rho_l(\cur)=\head(l)$ and $\rho_l(v)=\bot$ for $v\neq \cur$.
The set of initial nodes is $\{(p_0,l_0,\rho_{l_0})\mid l_0\mbox{ is an input list}\}$.
We use $p(l)$ to denote the \emph{output} of a program $p$ with respect to a list $l$. Then $p(l) =v$ if there exists a path from the initial node $(p, l, \rho_l)$ to some return node $(\mbox{ret }r,  \emptyset, \rho_r)$ and $
{r}{\rho_r}=v$. Otherwise, $p(l)=\bot$. Since the program is deterministic, i.e., each input list corresponds at most one output, the function $p$ is well-defined.

\section{Proofs in Section~\ref{sec:def-snt}}

\noindent {\bf Proposition~\ref{prop-snt-cmm-to-eqv}}. 
\emph{The commutativity problem of SNTs is reduced to the equivalence problem of SNTs in exponential time}.

\begin{proof}
Suppose that $\Ss=(Q, X, Y, \delta, q_0, O)$ is a SNT. Without loss of generality, we assume that the output of $\Ss$ is defined only for data words of length at least two. We will construct two SNTs $\Ss_1$ and $\Ss_2$ so that $\Ss$ is commutative iff $\Ss$ is equivalent to both $\Ss_1$ and $\Ss_2$.
\begin{itemize}
\item The intuition of $\Ss_1$ is that over a data word $w=d_1 d_2 d_3 \dots d_n$ with $n\ge 2$, $\Ss_1$ simulates the run of $\Ss$ over $d_2 d_1 d_3 \dots d_n$, that is, the data word obtained from $w$ by swapping the first two data values.
%
\item The intuition of $\Ss_2$ is that over a data word $w=d_1 d_2 d_3 \dots d_n$ with $n\ge 2$, $\Ss_1$ simulates the run of $\Ss$ over $d_2 d_3 \dots d_n d_1$, that is, the data word obtained from $w$ by moving first data value to the end. 
\end{itemize}
The correctness of this reduction follows from Proposition 1 in \cite{CHSW15}.

\smallskip

\noindent {\it The construction of $\Ss_1$}.

Intuitively, over a data word $w=d_1d_2 d_3 \dots d_n$, we introduce an additional control variable $x'$ to store $d_1$, then simulates the run of $\Ss$ over $d_2 d_1 d_3 \dots d_n$ as follows: When reading $d_2$ in $w$, the data variables are updated properly by letting $x'$ to represent $d_1$ and $\cur$ to represent $d_2$.

Without loss of generality, we assume that for each pair of transitions $q_0 \xrightarrow{(g_1,\eta_1)} q_1 \xrightarrow{(g_2,\eta_2)} q_2$ in $\Ss$, the following constraints are satisfied,
\begin{itemize}
\item $g_1$ does not contain any variable from $X$,
%
\item for each variable $x \in X$ such that $x$ occurs in $g_2$, it holds that $x \in \dom(\eta_1)$,
%
\item after these two transitions, the values of all the variables from $\dom(\eta_1) \cup \dom(\eta_2)$ are defined, more specifically, for each $y \in Y \cap \dom(\eta_2)$ and each $z \in \vars(\eta_2(y))$, it holds that $z \in \dom(\eta_1)$.
\end{itemize}

Let $q'_{0},q'_{1} \not \in Q$ and $x' \not \in X$. Then $\Ss_1 = (Q \cup \{q'_{0},q'_1\}, X \cup \{x'\}, Y, \delta_1, q'_{0}, O_1)$ such that 
\begin{itemize}
\item $O_1(q'_0)$ and $O_1(q'_1)$ are undefined, and for each $q \in Q$, $O_1(q)=O(q)$,
%
\item $\delta_1$ is constructed from $\delta$ as follows,
\begin{itemize}
\item each element of $\delta$ is an element of $\delta_1$,
%
\item for each pair of transitions $q_0 \xrightarrow{(g_1,\eta_1)} q_1 \xrightarrow{(g_2,\eta_2)} q_2$ in $\Ss$, we add the transitions $(q_0, \ltrue, \eta'_1, q'_1)$ and $(q'_1, g', \eta'_2, q_2)$ into $\delta_1$, where $\eta'_1,g',\eta'_2$ are defined in the following. Suppose $X \cap \dom(\eta_1)=\{x_1,\dots,x_k\}$, for each $y \in Y \cap \dom(\eta_1)$, $\eta_1(y)=a_{1,y} + b_{1,y}\cur$, and for each $y \in Y \cap \dom(\eta_2)$, 
\[\eta_2(y)=a_{2,y} + b_{2,y} \cur + c_{y,0} y + c_{y,1} x_1 + \dots + c_{y,k} x_k,\] 
or 
\[
\eta_2(y)=a_{2,y} + b_{2,y} \cur + c_{y,1} x_1 + \dots + c_{y,k} x_k.
\]
Then $\eta'_1, g', \eta'_2$ are defined as follows.
\begin{itemize}
\item $\eta'_1(x')=\cur$, for each $x \in X \cap \dom(\eta_2)$, $\eta'_1(x)=\cur$, and for all the other variables $z$ from $X \cup Y$, $\eta'_1(z)$ is undefined.
%
\item $g' = g_1 \wedge g'_2$, where $g'_2$ is obtained from $g_2$ by replacing $\cur$ with $x'$, and each $x \in X$ with $\cur$.
%
\item For each $x \in X$, if $x \in \dom(\eta_2)$, then $\eta'_2(x)$ is undefined, otherwise, if $x \in \dom(\eta_1)$, then $\eta'_2(x)=\cur$, otherwise, $\eta'_2(x)$ is undefined.
%
\item For each $y \in Y$, if $y \in \dom(\eta_2)$, then 
\[
\begin{array}{l c l}
\eta'_2(y) & = & a_{2,y} + b_{2,y} x' + c_{y,0} (a_{1,y} + b_{1,y}\cur) + c_{y,1} \cur + \dots + c_{y,k} \cur \\
& = & (a_{2,y} + c_{y,0} a_{1,y}) + b_{2,y} x' + (c_{y,0} b_{1,y}  + c_{y,1} + \dots + c_{y,k} )\cur,
\end{array}
\]
or 
\[
\begin{array}{l c l}
\eta'_2(y) & = & a_{2,y} + b_{2,y} x' + c_{y,1} \cur + \dots + c_{y,k} \cur \\
& = & a_{2,y} + b_{2,y} x' + (c_{y,1} + \dots + c_{y,k} )\cur.
\end{array}
\]
%
Otherwise, if $y \in \dom(\eta_1)$, then $\eta'_2(y)= a_{1,y} + b_{1,y} \cur$. Otherwise, $\eta'_2(y)$ is undefined.
\end{itemize}
\end{itemize}
\end{itemize}
It is easy to see that the size of $\Ss_1$ is polynomial with respect to the size of $\Ss$.

\smallskip

\noindent {\it The construction of $\Ss_2$}.

Intuitively, over a data word $w=d_1\dots d_n$, we introduce an additional control variable $x'$ to store $d_1$, then simulates the run of $\Ss$ over $d_2\dots d_n d_1$: When reaching the end of $w$, $\Ss_2$ outputs immediately by using $x'$ to represent $d_1$ and simulating the last transition of $\Ss$ over $d_2 \dots d_n d_1$. In order to simulate \emph{deterministically} the last transition of $\Ss$ over $d_2 \dots d_n d_1$ when reading the end of $w$ (since SNTs are required to be deterministic), we need record in the states of $\Ss_2$ the relationship between $d_1$ and all the values stored in the control variables. This implies an exponential blow-up of the size of $\Ss_2$ with respect to $\Ss$.

Let $c_{\max}$ and $c_{\min}$ denote the maximum resp. minimum constant occurring the guards of the transitions of $\Ss$.

Suppose $q'_{0} \not \in Q$ and $x' \not \in X$. Then $\Ss_2 = (Q', Y, \delta_2, q'_{0}, O_2)$, where $O',\delta_2,O_2$ are defined as follows. 
\begin{itemize}
\item $Q' = \{q_0\} \cup \left(\left([c_{\min}, c_{\max}] \cup \{-\infty,+\infty\}\right) \times X^{\{=, <, >,\bot\}} \right)$, where $[c_{\min},c_{\max}]= \{c \in \intnum \mid c_{\min} \le c \le c_{\max} \}$.
%
\item $\delta_2$ is defined as follows, 
\begin{itemize}
\item for each $c \in [c_{\min}, c_{\max}] \cup \{-\infty,+\infty\}$, $\delta_2$ contains $(q'_0,\ltrue,\eta, (q_0,(c, o_0)))$, where $\dom(\eta)=\{x'\}$, $\eta(x')=\cur$, and $o_0(x) = \bot$ for each $x \in X$,
%
\item for each $(q,g,\eta,q') \in \delta$ and $(q,(c,o)) \in Q'$ such that the guard $g \wedge \bigwedge \limits_{x \in X, o(x) \neq \bot} x'\ o(x)\ x$ is satisfiable, then $\delta_2$ contains the following three transitions, 
$((q,(c,o)), g \wedge \cur = x', \eta, (q',(c,o'_1)))$, $((q,(c,o)), g \wedge \cur< x', \eta, (q',(c,o'_2)))$, and $((q,(c,o)), g \wedge \cur > x', \eta, (q',(c,o'_3)))$, 
where 
for each $x \in X$, if $x \in \dom(\eta)$, then $o'_1(x) = \ =$, $o'_2(x)=\ >$, and $o'_3(x) =\ <$, otherwise, $o'_1(x) = o'_2(x) = o'_3(x) = o(x)$.
\end{itemize}
%
\item $O_2$ is defined as follows: For each $(q,(c,o)) \in Q'$  such that there is $(q,g,\eta,q') \in \delta$ satisfying that $\left(\bigwedge \limits_{x \in X, o(x) \neq \bot} \cur\ o(x)\ x \right) \models g$, and $O(q')$ is defined, suppose 
\[O(q')=a+b_1 x_1 + \dots + b_k x_k + c_1 y_1 + \dots + c_l y_l,\]
where $x_1,\dots,x_k$ (resp. $y_1,\dots,y_l$) are pairwise distinct variables from $X$ (resp. $Y$),
then let
\[O_2((q,(c,o)))=a+b_1 \eta'(x_1) + \dots + b_k \eta'(x_k) + c_1 \eta'(y_1) + \dots + c_l \eta'(y_l),\]
where for each $z \in \dom(\eta)$, $\eta'(z)=\eta(z)$, and for all the other variables $z' \in X \cup Y$, $\eta'(z')=z'$.  \\
Note that $O_2$ is well-defined since for each $(q,(c,o)) \in Q'$, there is a unique $(q,g,\eta,q') \in \delta$ satisfying the aforementioned constraint.
\end{itemize}
%
Note that $\Ss_1$ and $\Ss_2$ constructed above preserve the generalized flatness of $\Ss$.
\qed
\end{proof}


\noindent {\bf Proposition \ref{prop-snt-eqv-to-nzero}}.
\emph{From SNT $\Ss_1$ and $\Ss_2$, a SNT $\Ss_3$ can be constructed in polynomial time such that $\Ss_1$ and $\Ss_2$ are  inequivalent iff there is a data word $w$ such that the output of $\Ss_3$ over $w$ is nonzero.}

\begin{proof}
Let $\Ss_1 = (Q_1,X_1,Y_1,\delta_1,q_{1,0}, O_1)$ and  $\Ss_2 = (Q_2,X_2,Y_2,\delta_2,q_{2,0}, O_2)$ be two SNTs. Without loss of generality, we assume that $Q_1 \cap Q_2 = \emptyset$, $X_1 \cap X_2 = \emptyset$, and $Y_1 \cap Y_2 = \emptyset$. 

Intuitively, we construct $\Ss$ as the product of $\Ss_1$ and $\Ss_2$. Specifically, $\Ss=(Q_1 \times Q_2, X_1 \cup X_2, Y_1 \cup Y_2, \delta, (q_{1,0},q_{2,0}), O)$, where
\begin{itemize}
\item $\delta$ comprises $((q_1,q_2), g_1 \wedge g_2, \eta_1 \cup \eta_2, (q'_1,q'_2))$ such that $(q_1,g_1,\eta_1,q'_1) \in \delta_1$ and $(q_2,g_2,\eta_2,q'_2) \in \delta_2$,
%
\item for each $(q_1,q_2) \in Q_1 \times Q_2$, 
\begin{itemize}
\item if $O_1(q_1)$ is defined and $O_2(q_2)$ is undefined or vice versa, then $O((q_1,q_2))=1$, 
%
\item otherwise, if both $O_1(q_1)$ and $O_2(q_2)$ are defined, then $O((q_1,q_2))=O_1(q_1) - O_2(q_2)$, 
%
\item otherwise (both $O_1(q_1)$ and $O_2(q_2)$ are undefined), $O((q_1,q_2))$ is undefined. 
\end{itemize}
\end{itemize}
From the aforementioned construction and the assumption that $\Ss$ is well-defined, it is easy to see that $\Ss_1$ and $\Ss_2$ are  inequivalent iff there is a data word $w$ such that the output of $\Ss$ over $w$ is non-zero.\qed
\end{proof}



\section{Translate a Reducer Program to a SNT}
\begin{algorithm}[H]
	%  \SetAlgoLine
	\KwData{A reducer program $p$}
	$Q=\{q_0\}, \delta=\emptyset, O=\emptyset$, $\mathsf{toState}(p) =q_0$, $\mathsf{toVisit}=\{(\mathsf{toState}(p),p,\ltrue,\emptyset)\}$\;
	\While{$\mathsf{toVisit}\neq \emptyset$}{
		remove $(q,p,g,\eta)$ from $\mathsf{toVisit}$\;
		\Switch{$p$}{
			\lCase{$y := e;p'$,$y \addeq e;p'$,$x'=x;p'$: }{add $(q,p',g,\eta[e/y])$, $(q,p',g,\eta[(y+e)/y])$, $(q,p',g,\eta[x'/x])$ to $\mathsf{toVisit}$, respectively}
			\lCase{$\ite{g'}{s_1}{s_2};p'$: }{add both $(q,s_1;p',g\wedge g',\eta)$ and $(q,s_2;p',g\wedge \neg g',\eta)$ to $\mathsf{toVisit}$}
			\lCase{$\loopL{s;}\mbox{ret }r$: }{add both $(q,s;\loopL{s;}\mbox{ret }r,g,\eta)$ and $(q,\mbox{ret }r, g,\eta)$ to $\mathsf{toVisit}$}
			\lCase{$\nnext;p'$: }{\label{alg:next}
				\uIf{$\mathsf{toState}(p') \not\in Q$}{add $(\mathsf{toState}(p'),p',\ltrue,\emptyset)$ to $\mathsf{toVisit}$ and add $\mathsf{toState}(p')$ to $Q$}
				add $(q, \mathsf{toState}(p'),g,\eta)$ to $\delta$
			}
			\lCase{$\mbox{ret }r: $}{\label{alg:output}
				add a fresh state $q_r$ to $Q$, 
				add $(q, q_r,g,\eta)$ to $\delta$, and $O:=O[r/q_r]$}
		}
	}
	\Return $(Q,X,Y,\delta, \mathsf{toState}(p),O)$\;
	
	\caption{Translate a Reducer Program to a SNT}
	\label{fig:reducer2SNT}
\end{algorithm}
We use a tuple $(q,p,g,\eta)$ to store intermediate results of the translation, where $q$ is the source SNT state, $p$ is a reducer program, $g$ is a guard, and $\eta$ is an assignment.
The algorithm begins with the tuple $(p,p,\ltrue,\emptyset)$. The algorithm add a transition to SNT only when a $\nnext$ statement is encountered (line~\ref{alg:next}). When a $\mbox{ret }r$ statement is encountered, the algorithm adds a fresh state $q_r$ to the SNT and extends the output function to $O[r/q_r]$ (line~\ref{alg:output}).

The SNT returned from Algorithm~\ref{fig:reducer2SNT} is not yet generalized flat. It might have cycles sharing more than one states. All the cycles coming from the loop and branches inside the loop. There must be at least one state $s$ shared by all cycles. Therefore, we can make it generalized flat by duplicating all shared stated other than $s$ so all cycles will have their own copy of the shared states other than $s$.  



\section{Proofs in Section~\ref{sec:dec-snt}}


\noindent {\bf Proposition~\ref{prop-snt-norm}}.
{\it From each SNT, an equivalent normalized SNT can be constructed in exponential time.} 


\newcommand{\tog}[1]{\mathsf{toGuard(#1)}}
\newcommand{\toec}[1]{\mathsf{toEqClass(#1)}}
\begin{proof}
Given a SNT $\Ss=(Q, X, Y, \delta, q_0, O)$, we show that a normalized ${\Ss}'=(Q', X, Y, \delta', q'_0, O')$ such that ${\Ss}(w) = {\Ss}'(w)$ can be constructed.

We use $X^c$ to denote the set $X^+\cup  \interval{c_{min},c_{max}}$.
%The set of states $Q'=Q\times 2^X \times (X\times X^c) \rightarrow \{=, <, >,  \bot\}$. 
Each state in $Q'$ is a triple $(q, Z, M)$, where $q\in Q$, $Z\in 2^X$ is a set of defined variables, $M: (X\times X^c) \rightarrow \{\hat{=}, \hat{<}, \hat{>}\}$ is a partial function recording the relation between control variables $X$ and elements in $X^+$. Specifically, $M$ is a function from a pair $(x_1, x_2) \in X\times X^c$ to their relationship in $\{\hat{=}, \hat{<}, \hat{>}\}$.  The function $M$ can be translated to a guard with the same meaning as follows $\tog{M}=\bigwedge_{M(x_1,x_2)= \hat{=}} x_1 = x_fv2 \wedge \bigwedge_{M(x_1,x_2)= \hat{>}} x_1 > x_2 \wedge \bigwedge_{M(x_1,x_2)= \hat{<}} x_1 < x_2$. 

We say that $M$ is \emph{consistent} if there is no inconsistency when we saturate it with all indirect relations between elements in $X^c$. E.g., if $M(x_1,x_2)=\hat{>}$, $M(x_2,x_3)=\hat{>}$, and we will have the indirection relation $x_1>x_3$. If we also have $M(x_1,x_2)=\hat{<}$, then we detect an inconsistency. If $M(x_1,x_2)=\bot$ then the saturation procedure set $M(x_1,x_2)=\hat{>}$ because now the relation between is no longer unclear. From a consistent and saturated function $M$, one can induce an equivalence relation between elements in $X^c$, which partition $X^c$ into finite number of classes. We definite the \emph{representative element} of an equivalence class in $M$ as the minimal element w.r.t. the following order $c_{min}<\ldots<c_{max}<x_1<\ldots<x_{|X|}<\cur$. By abusing the notation, we also use $M$ to denote a function maps an element in $x\in X$ to the representative element in $X^c$ of the equivalence class $x$ belongs to.


Let $(q,q',g,\eta)$ be a transition in $\delta$. 
$(q,Z,M) \xrightarrow{(g',\eta')} (q',Z', M') \in \delta$ iff
\begin{itemize}
\item $\tog{M}\wedge g \neq \lfalse$
\item $M'$ is obtained by the following steps: 
  \begin{itemize}
    \item if $\tog{M}\wedge g$ implies $x_1>x_2$, $x_1<x_2$, $x_1=x_2$ for $x_1\in X, x_2\in X^c$, set $M'(x_1,x_2)=\hat{>}$, $M'(x_1,x_2)=\hat{<}$, $M'(x_1,x_2)=\hat{=}$, respectively.
	\item clean all rows and columns related to $\dom(\eta)$, $M':=M'[\bot/(x_1,x_2)]$ for all $x_1\in X\wedge x_2 \in \dom(\eta)$ or $x_1 \in \dom(\eta) \wedge x_2\in X^c$ for $x_1\neq x_2$ 
	\item handle the equivalence related caused by $\eta$, for all $x_1\in X,x_2\in X$ such that $\eta(x_1)=x_2$ execute $M':=M'[\hat{=}/(x_1,x_2)][\hat{=}/(x_2,x_1)]$ and for all $x\in X$ such that $\eta(x)=\cur$ execute $M':=M'[\hat{=}/(x_1,\cur)]$.
	\item saturate $M'$ with all possible indirect relations. 
  \end{itemize}
\item $\eta'(M'(x))= \eval{\eta(x)}{M}$ if $x \in X$ and  $\eta'(y)= \eval{\eta(y)}{M}$ if $y\in Y$.
\item $g'$ is obtained from $g\wedge g^c \wedge \bigwedge_{x \in X} \cur \neq x$ by replacing all occurrences of $x\in\dom(M)$ with $M(x)$, where $g^c\in \{\cur < c_{min}$, $\cur = c$ for $c_{min} \le c \le c_{max}$, $\cur > c_{max}\}\}$. 
\item $Z'= Z\cup (\dom(\eta)\cap X)$.
\end{itemize}
%
The initial state $q_0 = (q_0, \emptyset, M_0)$, where $M_0(x,x) = \hat{=}$ for all $x\in X$.
The output function $O'((q,Z,M))=\bot$ if $\vars(O(q)) \setminus Z\neq \emptyset$, i.e. there are some undefined variables in $O(q)$. Otherwise $O'((q,Z,M))=\eval{O(q)}{M'}$.
\qed
\end{proof}


\noindent {\bf Proposition~\ref{prop-sum-path}}.
{\it The values of the control and data variables after traversing the path $P$ are specified by a function $\chi$ satisfying the following conditions.
\begin{itemize}
\item There is an injective mapping $\pi: \{1,\dots,k\} \rightarrow \{1,\dots, k+r\}$ such that for each $x_j \in X$, if $\pi(j) \le k$, then $\pi(j)=j$ and $\chi(x_j)=d^{(0)}_{j}$, otherwise, $\chi(x_j)=d^{(1)}_{\pi(j)-k}$.
% 
\item For each $y_j \in Y$, $\chi(y_j) = \alpha_{j,0} + \alpha_{j,1} o_j + \beta_{j,1} d^{(0)}_1 + \dots + \beta_{j,k} d^{(0)}_k + \gamma_{j,1} d^{(1)}_1 +\dots + \gamma_{j,r} d^{(1)}_{r}$ for some constants $\alpha_{j,0},\alpha_{j,1}, \beta_{j,1},\dots,\beta_{j,k}, \gamma_{j,1},\dots,\gamma_{j,r}$ such that $\alpha_{j,1} \in \{0,+1,-1\}$ (as a result of the ``independently evolving and copyless'' constraint).
\end{itemize}
}

\begin{proof}
and the $n$ data values met when traversing the path are represented by $d_{k+1},\dots,d_{k+n}$ (these data values may repeat). Then the guards and the assignments in the path induce an equivalence relation $\sim$ on $\{1,\dots, k+n\}$ so that  $i \sim j$ iff it can be inferred from the guards and assignments that $d_i = d_j$. Since $\Ss$ is normalized, we know that for each pair of indices $i,j: 1 \le i < j \le k+n$ such that $i \sim j$, it holds that $j \ge k+1$. Let $I_1,\dots, I_{k+r}$ be an enumeration of the equivalence classes of $\sim$ on $\{1,\dots, k+n\}$ such that $\min(I_1) < \dots < \min(I_{k+r})$. Then for each $j: 1 \le j \le k$, $\min(I_j)=j$.



We show by an induction that for each $i: 1 \le i \le n$ and each variable $x_j \in X$ (resp. $y_j \in Y$), an expression $e_{i,x_j}$ (resp. $e_{i,y_j}$) can be constructed to describe the value of $x_j$(resp. $y_j$) after going through the first $i$ transitions of the path. 
%
\begin{itemize}
\item Let $\theta_0$ be an assignment such that for each $x_j \in X$ (resp. $y_j \in Y$), $\theta_0(x_j)=d^{(0)}_j$ (resp. $\theta_0(y_j)=o_j$), moreover, $\theta_0(\cur)=d^{(1)}_1$.
%
%\item For each $x_j \in X$, if $x_j \in \dom(\eta_1)$, then $e_{1,x_j}=d^{(1)}_1$, otherwise, $e_{1,x_j}=d^{(0)}_j$. For each $y_j \in Y$, if $y_j \in \dom(\eta_1)$, then $e_{1,y_j} = \theta_0(\eta_{1}(y_j))$,
%otherwise, $e_{1,y_j}=o_j$. 
%
\item Let $i: 1 \le i \le n$. 
\begin{itemize}
\item For each $x_j \in X$, if $x_j \in \dom(\eta_i)$, then $e_{i,x_j}=\theta_{i-1}(\cur)$, otherwise, $e_{i,x_j}=\theta_{i-1}(x_j)$.
%
\item For each $y_j \in Y$, if $y_j \in \dom(\eta_i)$, then $e_{i,y_j} = \theta_{i-1}(\eta_i(y_j))$, otherwise, $e_{i,y_j}=\theta_{i-1}(y_j)$.
%
\item For each $x_{j} \in X$ (resp. $y_j \in Y$), $\theta_i(x_{j})=e_{i,x_{j}}$ (resp. $\theta_i(y_{j})=e_{i, y_{j}}$). If $i < n$, then $\theta_i(\cur)=d^{(1)}_{s}$, where $1\le s \le r$ and $k+i + 1 \in I_s$, otherwise, $\theta_i(\cur)=\bot$.
\end{itemize}
\end{itemize}
Then the function $\chi$ can be defined as the restriction of $\theta_n$ to $X \cup Y$.
\qed
\end{proof}


\noindent {\bf Proposition~\ref{prop-sum-cycle}}.
{\it Suppose $P=C^{\ell}$ such that $\ell \ge 2$. Then the function $\chi^{(C)}_{\ell}$ to summarize the computation of $\Ss$ on $P$ is defined as follows,
\[
\begin{array}{l c l}
\chi^{(C)}_{\ell}(y_j)  & = & (\alpha^{(C)}_{j,0} + \alpha^{(C)}_{j,1} \alpha^{(C)}_{j,0}+ \dots +(\alpha^{(C)}_{j,1})^{\ell-1} \alpha^{(C)}_{j,0}) + (\alpha^{(C)}_{j,1})^\ell o_j + \\
& & \sum \limits_{j' \le k, \pi_C(j')=j'} (\beta^{(C)}_{j,j'}+\alpha^{(C)}_{j,1}\beta^{(C)}_{j,j'} + \dots +(\alpha^{(C)}_{j,1})^{\ell-1}  \beta^{(C)}_{j,j'}) d^{(0)}_{j'} + \\
%
& & \sum \limits_{j'\le k, \pi_C(j') \neq j'} ((\alpha^{(C)}_{j,1})^{\ell-1} \beta^{(C)}_{j,j'}) d^{(0)}_{j'} +  \\
%
& & \sum \limits_{j' \le r_C, j'+k \in \rng(\pi_C)} ( (\alpha^{(C)}_{j,1})^{\ell-2} \beta^{(C)}_{j, \pi_C^{-1}(j'+k)} +(\alpha^{(C)}_{j,1})^{\ell-1}\gamma^{(C)}_{j,j'}) d^{(C,1)}_{j'} + \\
%
& & \sum \limits_{j' \le r_C,  j'+k \not \in \rng(\pi_C)} ((\alpha^{(C)}_{j,1})^{\ell-1} \gamma^{(C)}_{j,j'}) d^{(C,1)}_{j'} + \dots + \\
%
& & \sum \limits_{j' \le r_C, j'+k \in \rng(\pi_C)} (\beta^{(C)}_{j, \pi_C^{-1}(j'+k)}+\alpha^{(C)}_{j,1}\gamma^{(C)}_{j,j'}) d^{(C,\ell-1)}_{j'} + \\
%
& & \sum \limits_{j' \le r_C,  j'+k \not \in \rng(\pi_C)} (\alpha^{(C)}_{j,1} \gamma^{(C)}_{j,j'}) d^{(C,\ell-1)}_{j'} + \gamma^{(C)}_{j, 1} d^{(C,\ell)}_{1} + \dots + \gamma^{(C)}_{j,r_C} d^{(C,\ell)}_{r_C},
\end{array} 
\]
where $d^{(C,2)}_{1},\dots, d^{(C,2)}_{r_C},\dots, d^{(C,\ell)}_{1},\dots,d^{(C,\ell)}_{r_C}$
 are the data values introduced when traversing $C$ for the second time, $\dots$, and for $\ell$ times.
}
%
\begin{proof}
Let $d^{(C,2)}_{1},\dots,d^{(C,2)}_{r_C}$ be the data values introduced when traversing the cycle for the second time. Then from Corollary~\ref{cor-comp-two-paths}, we know that $\chi^{(C)}_2 = \chi_{C} \circ \chi_C$ is defined as follows: For each $y_j \in Y$,
\[
\begin{array}{l c l}
\smallskip
\chi^{(C)}_2(y_j) & =  & (\alpha^{(C)}_{j,0}+\alpha^{(C)}_{j,1} \alpha^{(C)}_{j,0})+ (\alpha^{(C)}_{j,1})^2 o_j + \\
%
\smallskip
& & \sum \limits_{j' \le k, \pi_C(j')=j'} (\beta^{(C)}_{j,j'}+\alpha^{(C)}_{j,1} \beta^{(C)}_{j,j'}) d^{(0)}_{j'} + \sum \limits_{j' \le k, \pi_C(j') \neq j'} (\alpha^{(C)}_{j,1} \beta^{(C)}_{j,j'}) d^{(0)}_{j'}  \\
%
& & + \sum \limits_{j' \le r_C, j' + k \in \rng(\pi_C)} (\beta^{(C)}_{j,\pi_C^{-1}(j'+k)}+\alpha^{(C)}_{j,1} \gamma^{(C)}_{j,j'}) d^{(C,1)}_{j'} +\\
%
& & \sum \limits_{j' \le r_C, j' + k \not \in \rng(\pi_C)} (\alpha^{(C)}_{j,1} \gamma^{(C)}_{j,j'}) d^{(C,1)}_{j'} + 
 \gamma^{(C)}_{j,1} d^{(C,2)}_1 +\dots + \gamma^{(C)}_{j,r_C} d^{(C,2)}_{r_C}.
\end{array}
\] 
For $\ell > 3$, let $d^{(C,\ell)}_{1},\dots,d^{(C,\ell)}_{r_C}$ be the data values introduced when traversing the cycle for the $\ell$-th time. Then from Corollary~\ref{cor-comp-two-paths}, the fact $\chi^{(C)}_\ell = \chi_C \circ \chi^{(C)}_{\ell-1}$, and the induction hypothesis, we can show that $\chi^{(C)}_\ell$ is of the desired form.
\qed
\end{proof}


\smallskip

\noindent {\bf Proposition~\ref{prop-cycle-schm}}.
{\it 
Suppose $\schm=C_{i_1}^{\ell_1} C_{i_2}^{\ell_2} \dots C_{i_t}^{\ell_t}$ is a cycle scheme, and $\initval$ is a symbolic valuation representing the initial values of the control and data variables. Then for each $j: 1 \le j \le l$ and each $j' \in I^{(C_{i_1})}_{pe}$, the $(\initval(x_{j'}))$-atom of $\sumf^{(\schm,\initval)}(y_j)$ is 
\[\left((\lambda^{(C_{i_2})}_{j})^{\ell_2} \dots (\lambda^{(C_{i_t})}_{j})^{\ell_t}\right) \\
\left(1+\lambda^{(C_{i_1})}_{j} + \dots + (\lambda^{(C_{i_1})}_{j})^{\ell_1-1} \right) \csta^{(C_{i_1})}_{j,j'} \initval(x_{j'}),\]
and the constant atom of $\sumf^{(\schm,\initval)}(y_j)$ is
\[
\begin{array}{l c l}
\left((\lambda^{(C_{i_2})}_{j})^{\ell_2} \dots (\lambda^{(C_{i_t})}_{j})^{\ell_t}\right)
\left(1+\lambda^{(C_{i_1})}_{j} + \dots + (\lambda^{(C_{i_1})}_{j})^{\ell_1-1} \right) \cste^{(C_{i_1})}_{j} + \\
\left((\lambda^{(C_{i_3})}_{j})^{\ell_3} \dots (\lambda^{(C_{i_t})}_{j})^{\ell_t}\right)
\left(1+\lambda^{(C_{i_2})}_{j} + \dots + (\lambda^{(C_{i_2})}_{j})^{\ell_2-1} \right) \cste^{(C_{i_2})}_{j} +\\
 \dots + 
\left(1+\lambda^{(C_{i_t})}_{j} + \dots + (\lambda^{(C_{i_t})}_{j})^{\ell_t-1} \right) \cste^{(C_{i_t})}_{j}.
\end{array}
\]
Moreover, for each $j: 1 \le j \le l$, it holds that each \emph{nontrivial} atom of $\sumf^{(\schm,\initval)}(y_j)$ such that its coefficient is of the form $c\ell_1$ for some $c \in \intnum$, is in fact the constant atom of $\sumf^{(\schm,\initval)}(y_j)$ or an $(\initval(x_{j'}))$-atom of $\sumf^{(\schm,\initval)}(y_j)$ with $j' \in I^{(C_{i_1})}_{pe}$.
}


\begin{proposition}
Let $\schm=HC_{i_1}^{\ell_1} C_{i_2}^{\ell_2} \dots C_{i_t}^{\ell_t}$ be a cycle scheme. Then for each $j'$ such that $j' \le k$ and $\pi_{C_{i_1}}(j')=j'$, $\chi_{\schm}(O(q_m))$ contains the following expression,
\[
\left(\sum \limits_{1 \le j \le l} 
\begin{array}{l}
b_j \left((\alpha^{(C_{i_2})}_{j,1})^{\ell_2} \dots (\alpha^{(C_{i_t})}_{j,1})^{\ell_t}\right) \\
\left(1+\alpha^{(C_{i_1})}_{j,1} + \dots + (\alpha^{(C_{i_1})}_{j,1})^{\ell_1-1} \right) \beta^{(C_{i_1})}_{j,j'}
\end{array}
\right) d^{(0)}_{\pi_H(j')-k}. 
\]
Moreover, the constant coefficient of $\chi_\schm(O(q_m))$ contains the following expression,
\[
\sum \limits_{1 \le j \le l} 
%\begin{array}{l}
b_j \left((\alpha^{(C_{i_2})}_{j,1})^{\ell_2} \dots (\alpha^{(C_{i_t})}_{j,1})^{\ell_t}\right) 
%\\
\left(1+\alpha^{(C_{i_1})}_{j,1} + \dots + (\alpha^{(C_{i_1})}_{j,1})^{\ell_1-1} \right) \alpha^{(C_{i_1})}_{j,0}.
%\end{array}
\]
\end{proposition}

\begin{proof}
At first,  let us check $\chi^{(C_{i_1})}_{\ell_1}(O(q_m))$.
\[
\begin{array}{l c l}
\chi^{(C_{i_1})}_{\ell_1}(O(q_m)) & = & a_0 + a_1 \chi^{(C_{i_1})}_{\ell_1}(x_1) + \dots a_k \chi^{(C_{i_1})}_{\ell_1}(x_k) + \\
& & b_1 \chi^{(C_{i_1})}_{\ell_1}(y_1) + \dots + b_l \chi^{(C_{i_1})}_{\ell_1}(y_l).
\end{array}
\] 

Then $\chi^{(C_{i_1})}_{\ell_1}(O(q_m))$ is a linear combination of the variables $d^{(0)}_1,\dots, d^{(0)}_{r_H}$ and $d^{(C_{i_1},1)}_1,\dots, d^{(C_{i_1},1)}_{r_{C_{i_1}}}, \dots, d^{(C_{i_1},\ell_1)}_1,\dots, d^{(C_{i_1},\ell_1)}_{r_{C_{i_1}}}$.

For each $j'$ such that $j' \le k$ and $\pi_{C_{i_1}}(j')=j'$, the coefficient of $d^{(0)}_{\pi_H(j')-k}$ in $\chi^{(C_{i_1})}_{\ell_1}(O(q_m))$ is 

\[a_{j'} + \sum \limits_{1 \le j \le l} b_j \left(1+\alpha^{(C_{i_1})}_{j,1} + \dots + (\alpha^{(C_{i_1})}_{j,1})^{\ell_1-1} \right) \beta^{(C_{i_1})}_{j,j'}.\]
%
%
For each $j: 1 \le j \le l$, $\chi_{\schm}(y_j)$ contains the following expression 
\[\left((\alpha^{(C_{i_2})}_{j,1})^{\ell_2} \dots (\alpha^{(C_{i_t})}_{j,1})^{\ell_t}\right)\left(1+\alpha^{(C_{i_1})}_{j,1} + \dots + (\alpha^{(C_{i_1})}_{j,1})^{\ell_1-1} \right) \beta^{(C_{i_1})}_{j,j'} d^{(0)}_{\pi_H(j')-k}.\]

Since 
\[
\chi_{\schm}(O(q_m)) = a_0 + a_1 \chi_{\schm}(x_1) + \dots a_k \chi_{\schm}(x_k) + b_1 \chi_{\schm}(y_1) + \dots + b_l \chi_{\schm}(y_l),
\] 
it is not hard to see that $\chi_\schm(O(q_m))$ contains the expression,
\[
\left(\sum \limits_{1 \le j \le l} 
\begin{array}{l}
b_j \left((\alpha^{(C_{i_2})}_{j,1})^{\ell_2} \dots (\alpha^{(C_{i_t})}_{j,1})^{\ell_t}\right) \\
\left(1+\alpha^{(C_{i_1})}_{j,1} + \dots + (\alpha^{(C_{i_1})}_{j,1})^{\ell_1-1} \right) \beta^{(C_{i_1})}_{j,j'}
\end{array}
\right) d^{(0)}_{\pi_H(j')-k}. 
\]
The argument for the constant coefficient is similar.
\qed
\end{proof}



%%%%%%%%%%%%%%%%%%%%%%%%%%%%%%%%%%%%%%%%%%%%%%%%%%%%%%%
%%%%%%%%%%%%%%%%%%the proof of the claim%%%%%%%%%%%%%%%%%%%%%%%%
%%%%%%%%%%%%%%%%%%%%%%%%%%%%%%%%%%%%%%%%%%%%%%%%%%%%%%%

\noindent {\bf Proposition~\ref{prop-bnd-domain}}. 
{\it 
Suppose that the decision procedure has not returned yet after Step II. For each cycle scheme $\schm=C_{i_1}^{\ell_1} C_{i_2}^{\ell_2} \dots C_{i_t}^{\ell_t}$ and $y_j \in Y$, let $(\sumf^{(\schm,\initval)})'(y_j)$ denote the expression obtained by removing from the coefficients of the atoms of $\sumf^{(\schm,\initval)}(y_j)$ all the expressions of the form $c\ \ell_1$, $\dots$, or $c\ \ell_t$ (where $c$ is an integer constant).  Then the following three facts hold.
\begin{enumerate}
\item There is a valuation $\rho$ such that $\eval{\eval{O(q_m)}{\sumf^{(\schm, \sumf^{(H,\initval_0)})}}}{\rho}\neq 0$ iff there is a valuation $\rho$ such that $\eval{\eval{O(q_m)}{(\sumf^{(\schm, \sumf^{(H,\initval_0)})})'}}{\rho} \neq 0$.
%
\item There is a finite subset of $\intnum$, say $K$, such that for every cycle scheme $\schm$ and $y_j \in Y$, the constant atom and all the coefficients of atoms in $(\sumf^{(\schm, \sumf^{(H,\initval_0)})})'(y_j)$ are from $K$. 
%
\item For each cycle scheme $\schm$, an abstraction of $(\sumf^{(\schm, \sumf^{(H,\initval_0)})})'$, denoted by $\abs(\schm)$, can be defined such that $\abs(\schm) \subseteq \{0,\dots, k+1\} \times K^l$ and for each $j: 0 \le j \le k$,  $\abs(\schm) \cap (\{j\} \times K^l)$ is a singleton. Let $\Kk=\{\abs(\schm) \mid \schm \mbox{ a cycle scheme}\}$. Then $\Kk$ can be constructed effectively from $\sumf^{(H,\initval)}$, $\sumf^{(C_1,\initval)}$, $\dots$, and $\sumf^{(C_n,\initval)}$.
\end{enumerate}
}

\begin{proof}
Let $\schm=C_{i_1}^{\ell_1} C_{i_2}^{\ell_2} \dots C_{i_t}^{\ell_t}$ be a cycle scheme. If the decision procedure has not returned yet after Step II, then all these expressions of the form $c\ \ell_1,\dots, c\ \ell_t$ in $\sumf^{(\schm, \sumf^{(H,\initval_0)})}(y_j)$ for $y_j \in Y$ can be ignored, since our only concern is whether the output $\eval{O(q_m)}{\sumf^{(\schm, \sumf^{(H,\initval_0)})}}$ is non-zero. More specifically, the arguments are as follows: From the fact that  $\mu_{\schm,(i_1,0)}=0$ and $\mu_{\schm,(i_1,j')}=0$ for each $j' \in I^{(C_{i_1})}_{pe}$, we know that in $\eval{O(q_m)}{\sumf^{(\schm, \sumf^{(H,\initval_0)})}}$, the  expressions of the form $c\ \ell_1$ from different $\sumf^{(\schm, \sumf^{(H,\initval_0)})}(y_j)$ for $y_j \in Y$ will cancel each other. Similarly, let $\schm' = C_{i_2}^{\ell_2} \dots C_{i_t}^{\ell_t}$, then in $\eval{O(q_m)}{\sumf^{(\schm', \sumf^{(H,\initval_0)})}}$, the  expressions of the form $c\ \ell_2$ from different $\sumf^{(\schm', \sumf^{(H,\initval_0)})}(y_j)$ for $y_j \in Y$ will cancel each other. Thus, all the expressions of the form $c\ \ell_2$ in $\sumf^{(\schm', \sumf^{(H,\initval_0)})}(y_j)$ for $y_j \in Y$ can be ignored. From this, we deduce that in $\eval{O(q_m)}{\sumf^{(\schm, \sumf^{(H,\initval_0)})}}$, all the expressions of the form $c\ \ell_2$ from different $\sumf^{(\schm, \sumf^{(H,\initval_0)})}(y_j)$ for $y_j \in Y$ will cancel each other. Therefore, all the expressions of the form $c\ \ell_2$ in $\sumf^{(\schm, \sumf^{(H,\initval_0)})}(y_j)$ for $y_j \in Y$ can be ignored. We can apply the same arguments for the expressions of the form $c\ \ell_3,\dots,c\ \ell_t$.

\medskip

We prove the second fact next. 

Let $K$ be the subset of $\intnum$ comprising the following numbers,
\begin{itemize}
\item the constant atom and the coefficients of the non-constant atoms in the expression $(\sumf^{(C^{\ell_i}_{i}, \sumf^{(H,\initval_0)})})'(y_j)$, where $1 \le i \le n$ and $\ell_i \ge 1$ (note that if $\ell_i \ge 2$, then the exact value of $\ell_i$ is indifferent to those coefficients),

%(note that the exact value of $\ell_i$ is indifferent to these numbers),
%
\item  the numbers $\csta^{(C_{i_2})}_{j,j'} + \cstb^{(C_{i_1})}_{j,\pi^{(C_{i_1})}(j')}$, where $1 \le i_1,i_2 \le n$,  $1 \le j \le l$, and $j' \in I^{(C_{i_1})}_{tr} \cap I^{(C_{i_2})}_{tr}$,
%
\item  the numbers $\csta^{(C_{i_1})}_{j, j'} + \csta^{(C_{i_2})}_{j,j'}$, where $1 \le i_1,i_2 \le n$, $1 \le j \le l$, and $j' \in [k]$. 
%$j' \in I^{(C_{i_1})}_{tr}$, and $\cstl^{(C_{i_2})}_j = 0$.
\end{itemize}

For each cycle scheme $\schm=C^{\ell_1}_{i_1} \dots C^{\ell_t}_{i_t}$ and each $y_j \in Y$, we know that $(\sumf^{(\schm,\sumf^{(H,\initval_0)})})'(y_j)$ is a linear combination of $\vard^{(H)}_1$, $\dots$, $\vard^{(H)}_{r^{(H)}}$, $\vard^{(C_{i_1},1)}_{1,1}$, $\dots$, $\vard^{(C_{i_1},\ell_1)}_{1,r^{(C_{i_1})}}$, $\dots$, $\vard^{(C_{i_t},1)}_{t,1}$, $\dots$, $\vard^{(C_{i_t},\ell_t)}_{t, r^{(C_{i_t})}}$. 

%Similarly, $(\sumf^{(\schm_1,\sumf^{(H,\initval_0)})})'(y_j)$ is a linear combination of $\vard^{(H)}_1$, $\dots$, $\vard^{(H)}_{r^{(H)}}$, $\vard^{(C_{i_1},1)}_{1,1}$, $\dots$, $\vard^{(C_{i_1},\ell_1)}_{1,r^{(C_{i_1})}}$, $\dots$, $\vard^{(C_{i_{t-1}},1)}_{t-1,1}$, $\dots$, $\vard^{(C_{i_{t-1}},\ell_{t-1})}_{t-1, r^{(C_{i_{t-1}})}}$.

Suppose for each $y_j \in Y$,
\[
\begin{array}{l cl }
(\sumf^{(\schm,\sumf^{(H,\initval_0)})})'(y_j) &:= & (\cste^{(\schm)}_{j})'  + (\csta^{(\schm)}_{j,1})' \vard^{(H)}_1 + \dots + (\csta^{(\schm)}_{j,r^{(H)}})' \vard^{(H)}_{r^{(H)}} + \\
& & (\cstb^{(\schm,1)}_{1,1})' \vard^{(C_{i_1},1)}_{1,1}  + \dots + (\cstb^{(\schm,\ell_1)}_{1,r^{(C_{i_1})}})' \vard^{(C_{i_1},\ell_1)}_{1,r^{(C_{i_1})}}  +  \\
& & \dots + \\
& & (\cstb^{(\schm,1)}_{t,1})' \vard^{(C_{i_{t}},1)}_{t,1} + \dots + (\cstb^{(\schm,\ell_{t})}_{t,r^{(C_{i_{t}})}})' \vard^{(C_{i_{t}},\ell_{t})}_{t, r^{(C_{i_{t}})}}.
\end{array}
\]



In the following, we show by induction on $t$ that for each cycle scheme $\schm=C^{\ell_1}_{i_1} \dots C^{\ell_t}_{i_t}$ and $y_j \in Y$, the following results hold.
\begin{enumerate}
\item The constant atom and all the coefficients of the non-constant atoms in $(\sumf^{(\schm, \sumf^{(H,\initval_0)})})'(y_j)$ are from $K$.
%
\item For each $\vard^{(H)}_{j'}$ such that $(\sumf^{(\schm,\sumf^{(H,\initval_0)})})'(x_{j''})=\vard^{(H)}_{j'}$ for some $j''  \in [k]$, the following fact holds: if there is $s \in [t]$ such that $\cstl^{(C_{i_s})}_j =0$, let $s_0$ be the maximum $s$ satisfying the constraint, then $(\csta^{(\schm)}_{j,j'})'=\csta^{(C_{i_{s_0}})}_{j, j''}$, otherwise, $(\csta^{(\schm)}_{j,j'})'= \beta^{(H)}_{j,j'}$.
%
\item For each $s \in [t]$, $i \in [\ell_s]$, and $j' \in [r^{(C_{i_s})}]$ such that $(\sumf^{(\schm,\sumf^{(H,\initval_0)})})'(x_{j''})=\vard^{(\schm, i)}_{s, j'}$ for some $j''  \in [k]$, it holds that $i = \ell_s$, $j'' \in I^{(C_{i_s})}_{tr}$, $j'' \in I^{(C_{i_{s'}})}_{pe}$ for each $s': s < s' \le t$, and the following fact holds: if there is $s': s < s' \le t$ such that $\cstl^{(C_{i_{s'}})}_j =0$, let $s'_0$ be the maximum $s'$ satisfying the constraint, then $(\cstb^{(\schm, i)}_{j,j'})'=\csta^{(C_{i_{s'_0}})}_{j, j''}$, otherwise, $(\cstb^{(\schm, i)}_{j,j'})'= \cstb^{(C_{i_s})}_{j,j'}$. 
\end{enumerate}



Induction base: $t=1$. 
\begin{itemize}
\item The first result: Follow from the definition of $K$. 

\item The second result: If $\cstl^{(C_{i_1})}_j = 0$, then $(\csta^{(\schm)}_{j,j'})'=\csta^{(C_{i_{1}})}_{j, j''}$,  otherwise, $(\csta^{(\schm)}_{j,j'})'=0$, since the expression $\csta^{(C_{i_1})}_{j,j''} \ell_1$ is removed from the coefficient of the $\sumf^{(H,\initval_0)}(x_{j''})$-atom in $\sumf^{(C^{\ell_1}_{i_1},\sumf^{(H,\initval_0)})}(y_j)$, i.e. the $\vard^{(H)}_{j'}$-atom in $\sumf^{(C^{\ell_1}_{i_1},\sumf^{(H,\initval_0)})}(y_j)$. 

\item The third result: Suppose that $(\sumf^{(C^{\ell_1}_{i_1},\sumf^{(H,\initval_0)})})'(x_{j''})=\vard^{(C_{i_1}, i)}_{1, j'}$ for $i \in [\ell_1]$, $j' \in [r^{(C_{i_1})}]$, and $j'' \in [k]$. Then $j'' \in I^{(C_{i_1})}_{tr}$, otherwise, $\vard^{(C_{i_1}, i)}_{1, j'}$ would not be assigned to $x_{j''}$. From this, we deduce that $i = \ell_s$. Moreover, $(\csta^{(C^{\ell_1}_{i_1})}_{j,j'})'= \cstb^{(C_{i_1})}_{j,j'}$. 
\end{itemize}

Induction step: Suppose $t \ge 2$ and $\schm=C^{\ell_1}_{i_1} \dots C^{\ell_t}_{i_t}$.

Let $\schm_1= C^{\ell_1}_{i_1} \dots C^{\ell_{t-1}}_{i_{t-1}}$.  Then for each $y_j \in Y$, 
\[(\sumf^{(\schm,\sumf^{(H,\initval_0)})})'(y_j)=(\sumf^{(C^{\ell_t}_{i_t},\ (\sumf^{(\schm_1, \sumf^{(H,\initval_0)})})')})'(y_j).\] 

By the induction hypothesis, the three results hold for $(\sumf^{(\schm_1, \sumf^{(H,\initval_0)})})'$.

%Suppose for each $s: 1 \le s \le t$, the data values introduced when traversing $C_{i_s}^{\ell_s}$ in $\schm$ are represented by the variables $\vard^{(C_{i_s},1)}_{s,1}$, $\dots$, $\vard^{(C_{i_s},1)}_{s,r^{(C_{i_s})}}$, $\dots$, $\vard^{(C_{i_s},\ell_s)}_{s,1}$, $\dots$, $\vard^{(C_{i_s},\ell_s)}_{s,r^{(C_{i_s})}}$. 


We illustrate the arguments for the case $\cstl^{(C_{i_t})}_{j} = 1$. The case $\cstl^{(C_{i_t})}_{j} = 0$ is simpler and can be discussed similarly. Suppose $y_j \in Y$.  In the following, we check that the constant atom and the coefficients of all the non-constant atoms of $(\sumf^{(\schm,\sumf^{(H,\initval_0)})})'(y_j)$ belong to $K$.  
%For brevity, we abbreviate an(resp. the) atom of $(\sumf^{(\schm,\sumf^{(H,\initval_0)})})'(y_j)$ as an (resp. the) atom below.
\begin{itemize}
\item $(\cste^{(\schm)}_{j})' = 0 + (\cstl^{(C_{i_t})}_{j})^{\ell_t} (\cste^{(\schm_1)}_j)' = (\cste^{(\schm_1)}_j)' \in K$ (here $\cste^{(C_{i_t})}_{j} \ell_t$ is removed).
%
%\item the coefficient of $o_j$ is $1$,
%
\item For each $j' \in [r^{(H)}]$ such that there does not exist $j'' \in [k]$ satisfying that $(\sumf^{(\schm_1,\sumf^{(H,\initval_0)})})'(x_{j''})=\vard^{(H)}_{j'}$, $(\csta^{(\schm)}_{j, j'})' = (\cstl^{(C_{i_t})}_{j})^{\ell_t} (\csta^{(\schm_1)}_{j, j'})' = (\csta^{(\schm_1)}_{j, j'})' \in K$.
%
\item For each $j' \in [r^{(H)}]$ such that $(\sumf^{(\schm_1,\sumf^{(H,\initval_0)})})'(x_{j''})=\vard^{(H)}_{j'}$ for some $j''  \in I^{(C_{i_t})}_{pe}$, $(\csta^{(\schm)}_{j, j'})' = 0 + (\cstl^{(C_{i_t})}_{j})^{\ell_t} (\csta^{(\schm_1)}_{j, j'})'= (\csta^{(\schm_1)}_{j, j'})' \in K$ (here $\csta^{(C_{i_t})}_{j,j'} \ell_t$ is removed). In this case, we have $(\sumf^{(\schm,\sumf^{(H,\initval_0)})})'(x_{j''})=\vard^{(H)}_{j'}$. From the induction hypothesis, it is easy to see that the second result holds for $\schm$, $y_j$.
%
\item For each $j' \in [r^{(H)}]$ such that $(\sumf^{(\schm_1,\sumf^{(H,\initval_0)})})'(x_{j''})=\vard^{(H)}_{j'}$ for some $j''  \in I^{(C_{i_t})}_{tr}$, $(\csta^{(\schm)}_{j, j'})' =(\cstl^{(C_{i_t})}_{j})^{\ell_t-1} \csta^{(C_{i_t})}_{j,j''} + (\cstl^{(C_{i_t})}_{j})^{\ell_t} (\csta^{(\schm_1)}_{j, j'})' = \csta^{(C_{i_t})}_{j,j''} + (\csta^{(\schm_1)}_{j, j'})' $. From the induction hypothesis, we know that either $(\csta^{(\schm_1)}_{j, j'})' = \csta^{(C_{i_{s_0}})}_{j,j''}$ if there is $s$ such that $\cstl^{(C_{i_s})}_j =0$,  or otherwise, $(\csta^{(\schm_1)}_{j, j'})'=\cstb^{(H)}_{j, j'}$. Therefore, $(\csta^{(\schm)}_{j, j'})' =\csta^{(C_{i_t})}_{j,j''} +  \csta^{(C_{i_{s_0}})}_{j,j''}$ or $\csta^{(C_{i_t})}_{j,j''}+ \cstb^{(H)}_{j, j'}$. We conclude that $(\csta^{(\schm)}_{j, j'})' \in K$.
%
\item For each $s \in [t-1]$, $i \in [\ell_s]$, and $j' \in [r^{(C_{i_s})}]$ such that there does not exist $j'' \in [k]$ satisfying that $(\sumf^{(\schm_1,\sumf^{(H,\initval_0)})})'(x_{j''}) = \vard^{(C_{i_s}, i)}_{s, j'}$, it holds that $(\cstb^{(\schm,i)}_{s,j'})' = (\cstl^{(C_{i_t})}_{j})^{\ell_t}  (\cstb^{(\schm_1, i)}_{s, j'})'  =  (\cstb^{(\schm_1, i)}_{s, j'})'  \in K$. 
%
\item For each $s \in [t-1]$, $i \in [\ell_s]$, and $j' \in [r^{(C_{i_s})}]$ such that $(\sumf^{(\schm_1,\sumf^{(H,\initval_0)})})'(x_{j''}) = \vard^{(C_{i_s}, i)}_{s, j'}$ for some $j'' \in I^{(C_{i_t})}_{pe}$, $(\cstb^{(\schm,i)}_{s,j'})' =(\cstl^{(C_{i_t})}_{j})^{\ell_t}  (\cstb^{(\schm_1, i)}_{s, j'})'+ 0  =  (\cstb^{(\schm_1, i)}_{s, j'})' \in K$ (here $\csta^{(C_{i_t})}_{j, j''} \ell_t$ is removed). Moreover, from the induction hypothesis, the third condition holds for $\schm$ and $y_j$.
%
\item For each $s \in [t-1]$, $i \in [\ell_s]$, and $j' \in [r^{(C_{i_s})}]$ such that $(\sumf^{(\schm_1,\sumf^{(H,\initval_0)})})'(x_{j''}) = \vard^{(C_{i_s}, i)}_{s, j'}$ for some $j'' \in I^{(C_{i_t})}_{tr}$, it holds that $i = \ell_s$, $j'' \in I^{(C_{i_s})}_{tr}$, and for each $s': s < s' \le t$, $j'' \in I^{(C_{i_{s'}})}_{pe}$. Then $(\cstb^{(\schm,i)}_{s,j'})' =(\cstl^{(C_{i_t})}_{j})^{\ell_t}  (\cstb^{(\schm_1, i)}_{s, j'})' + \csta^{(C_{i_t})}_{j, j''}  =  (\cstb^{(\schm_1, i)}_{s, j'})' + \csta^{(C_{i_t})}_{j, j''}$. From the induction hypothesis,  if there is $s': s < s' \le t-1$ such that $\cstl^{(C_{i_{s'}})}_j =0$, let $s'_0$ be the maximum $s'$ satisfying the constraint, then $(\cstb^{(\schm_1, i)}_{j,j'})'=\csta^{(C_{i_{s'_0}})}_{j, j''}$, otherwise, $(\cstb^{(\schm_1, i)}_{j,j'})'= \cstb^{(C_{i_s})}_{j,j'}$. Therefore, $(\cstb^{(\schm,i)}_{s,j'})' = \csta^{(C_{i_{s'_0}})}_{j, j''} + \csta^{(C_{i_t})}_{j, j''}$ or $\cstb^{(C_{i_s})}_{j,j'} + \csta^{(C_{i_t})}_{j, j''}$, thus belongs to $K$. In this case, $(\sumf^{(\schm,\sumf^{(H,\initval_0)})})'(x_{j''})=\vard^{(C_{i_t},\ell_t)}_{t, \pi^{(C_{i_t})}(j'')}$ and $(\cstb^{(\schm,\ell_t)}_{t, \pi^{(C_{i_t})}(j'')})' = \cstb^{(C_{i_t})}_{j, \pi^{(C_{i_t})}(j'')}$. So the third condition holds for $\schm$ and $y_j$.
%
\item The coefficients of all the other atoms are those of  the $\vard^{(C_{i_t}, i)}_{j, j'}$-atoms for $i: 1 \le i \le \ell_t$ in $(\sumf^{(C^{\ell_t}_{i_t},\sumf^{(H,\initval_0)})})'(y_j)$.
\end{itemize} 

\smallskip

Finally, we consider the third fact.

Let $\schm=C^{\ell_1}_{i_1} \dots C^{\ell_t}_{i_t}$ be a cycle scheme. An abstraction of $(\sumf^{(\schm,\sumf^{(H,\initval_0)})})'$, denoted by $\abs(\schm)$, can be computed from $\schm$ by the following procedure.
\begin{enumerate}
\item Let $\abs(\schm):=\{(0, ( (\cste^{(\schm)}_{1})',\dots, (\cste^{(\schm)}_l)'))\}$.
%
\item For each $j \in [k]$ and $j' \in [l]$, let $\cstg_{j, j'}$ be the coefficient of the $((\sumf^{(\schm,\sumf^{(H,\initval_0)})})'(x_j))$-atom in $(\sumf^{(\schm,\sumf^{(H,\initval_0)})})'(y_{j'})$. Let $\abs(\schm): = \abs(\schm) \cup \{(j, (\cstg_{j,1},\dots, \cstg_{j,l})) \mid j \in [k]\}$.
%
\item For each data variable $\vard'$ such that $\vard' \neq (\sumf^{(\schm,\sumf^{(H,\initval_0)})})'(x_j)$ for each $j \in [k]$, let $\abs(\schm):=\abs(\schm) \cup \{(k+1, (c_1,\dots,c_l))\}$, where $(c_1,\dots,c_l) \in K^l$ is the tuple of the coefficients of the $\vard'$-atom in $((\sumf^{(\schm,\sumf^{(H,\initval_0)})})'(y_{j'})$ for $j' \in [l]$. 
\end{enumerate}
Intuitively, $\abs(\schm)$, the abstraction of  $(\sumf^{(\schm,\sumf^{(H,\initval_0)})})'$,  is the set comprising the tuple of the constant atoms, the tuples of the coefficients of the atoms corresponding to the control variables, and the tuples of the coefficients of the atoms corresponding to the other data variables.

From the proof that all the constant atoms and the coefficients of the non-constant atoms in $(\sumf^{(\schm,\sumf^{(H,\initval_0)})})'(y_j)$ for $y_j \in Y$ are from a bounded domain $K$, it is not hard to see that $\Kk = \{\abs(\schm) \mid \schm \mbox{ a cycle scheme}\}$ can be computed effectively from $\sumf^{(H,\initval)}$, $\sumf^{(C_1,\initval)}$, $\dots$, and $\sumf^{(C_n,\initval)}$.\qed
\end{proof}
%%%%%%%%%%%%%%%%%%%%%%%%%%%%%%%%%%%%%%%%%%%%%%%%%%%%%%%
%%%%%%%%%%%%%%%%%%the proof of the claim%%%%%%%%%%%%%%%%%%%%%%%%
%%%%%%%%%%%%%%%%%%%%%%%%%%%%%%%%%%%%%%%%%%%%%%%%%%%%%%%


\subsection{Comparisons with constants in guards}

We now consider the situation that the guards contain the comparisons with constants, that is, the atomic formulae of the form $cur\ o\ c$, where $c$ is an integer constant. 

We illustrate the argument for generalized lassos and adapt the decision procedure Step I-III to Step I$''$-III$''$ below.

Let $q_0 \dots q_m C_1 \dots C_n$ be a generalized lasso and $O(q_m)=a_0 + a_1 x_1 + \dots + a_k x_k + b_1 y_1 + \dots + b_l y_l$.

\smallskip

\noindent {\bf Step I$''$}. Let
\[
\begin{array}{l c l }
\chi_H(O(q_m)) & = & a_0 + a_1 \chi_H(x_1) + \dots + a_k \chi_H(x_k) + b_1 \chi_H(y_1) + \dots b_l \chi_H(y_l) \\
& = & (a_0+b_1 \alpha^{(H)}_{1,0} + \dots + b_l \alpha^{(H)}_{l,0}) + \\
& & \sum \limits_{j \le r_H, j+k \in \rng(\pi_H)} (a_{\pi_H^{-1}(j+k)}+ b_1 \gamma^{(H)}_{j,1}+\dots + b_l \gamma^{(H)}_{j,l}) d^{(0)}_j +
\\
& & \sum \limits_{j \le r_H, j +k \not \in \rng(\pi_H)}  (b_1 \gamma^{(H)}_{j,1}+\dots + b_l \gamma^{(H)}_{j,l}) d^{(0)}_j .
\end{array}
\]
From the guards and assignments of the transitions in $H$, we know that some of $d^{(0)}_1,\dots,d^{(0)}_{r_H}$ are integer constants between $c_{\min}$ and $c_{\max}$. Let $J_H \subseteq \{1,\dots, r_H\}$ denote the indices $j: 1 \le j \le r_H$ such that $d^{(0)}_j = c^{(H)}_j$ for some $c_{\min}\le c^{(H)}_j \le c_{\max}$. Since $\Ss$ is assumed to be normalized, we know that for $j \not \in J_H$,  $d^{(0)}_j < c_{\min}$ or $d^{(0)}_j > c_{\max}$.

For each $j \not \in J_H$, decide whether the coefficient of $d^{(0)}_j$ in $\chi_H(O(q_m))$ is nonzero. If the answer is yes, then return $\ltrue$. 

If the algorithm has not return yet, we substitute the data variables $d^{(0)}_j$ for $j \in J_H$ with the integer constants. Then $\chi_H(O(q_m))$ becomes an integer constant $c_H$. If $c_H \neq 0$, then return $\ltrue$.  Otherwise, go to Step II$''$. \qed

\medskip

For each $i_1: 1 \le i_1 \le n$, let $J_{C_{i_1}}$ denote the indices $j': 1 \le j' \le l$ such that there is $c^{C_{i_1}}_{j'}$ satisfying that $c_{\min} \le c^{C_{i_1}}_{j'} \le c_{\max}$, and for each $i: 1 \le i \le \ell_1$, $d^{(C_{i_1},i)}_{j'}=c^{C_{i_1}}_{j'}$ in $\chi^{(C_{i_1})}_{\ell_1}$. 

For a cycle scheme $\schm=HC_{i_1}^{\ell_1} C_{i_2}^{\ell_2} \dots C_{i_t}^{\ell_t}$, let $c_{\schm,i_1}$ denote the sum of all the following expressions (intuitively, $c_{\schm,i_1}$ is the constant coefficient of $\chi_\schm(O(q_m))$ involving the terms $ c\ \ell_1$ for some integer constant $c$, when taking into consideration the fact that some introduced data values are in fact integer constants between $c_{\min}$ and $c_{\max}$),
\[
\sum \limits_{1 \le j \le l} 
%\begin{array}{l}
b_j \left((\alpha^{(C_{i_2})}_{j,1})^{\ell_2} \dots (\alpha^{(C_{i_t})}_{j,1})^{\ell_t}\right) 
%\\
\left(1+\alpha^{(C_{i_1})}_{j,1} + \dots + (\alpha^{(C_{i_1})}_{j,1})^{\ell_1-1} \right) \alpha^{(C_{i_1})}_{j,0},
%\end{array}
\]
the expressions 
\[
\left(\sum \limits_{1 \le j \le l} 
\begin{array}{l}
b_j \left((\alpha^{(C_{i_2})}_{j,1})^{\ell_2} \dots (\alpha^{(C_{i_t})}_{j,1})^{\ell_t}\right) \\
\left(1+\alpha^{(C_{i_1})}_{j,1} + \dots + (\alpha^{(C_{i_1})}_{j,1})^{\ell_1-1} \right) \beta^{(C_{i_1})}_{j,j'}
\end{array}
\right) c^{(H)}_{\pi_H(j')-k}
\]
such that $j' \le k$, $\pi_H(j')-k \in J_H$ (Recall that $d^{(0)}_{\pi_H(j')-k}=c^{(H)}_{\pi_H(j')-k}$), \\
the expressions
\[
\left(\sum \limits_{1 \le j \le l} 
\begin{array}{l}
b_j \left((\alpha^{(C_{i_2})}_{j,1})^{\ell_2} \dots (\alpha^{(C_{i_t})}_{j,1})^{\ell_t}\right) \\
\left((\alpha^{(C_{i_1})}_{j,1})^{\ell_1-i-1} \beta^{(C_{i_1})}_{j,\pi^{-1}_{C_{i_1}}(j'+k)} + (\alpha^{(C_{i_1})}_{j,1})^{\ell_1-i} \gamma^{(C_{i_1})}_{j,j'}\right)
\end{array}
\right) c^{(C_{i_1})}_{j'},
\]
such that $1 \le i < \ell_1$, $j' \in J_{C_{i_1}}$, and $j'+k \in \rng(\pi_{C_{i_1}})$,\\
the expressions
\[
\left(\sum \limits_{1 \le j \le l} 
%\begin{array}{l}
b_j \left((\alpha^{(C_{i_2})}_{j,1})^{\ell_2} \dots (\alpha^{(C_{i_t})}_{j,1})^{\ell_t}\right)  \left( (\alpha^{(C_{i_1})}_{j,1})^{\ell_1-i} \gamma^{(C_{i_1})}_{j,j'}\right)
%\end{array}
\right) c^{(C_{i_1})}_{j'},
\]
such that $i: 1 \le i \le \ell_1$, $j' \in J_{C_{i_1}}$, and $j'+k \not \in \rng(\pi_{C_{i_1}})$.

Then $c_{\schm,i_1}$ can be rewritten as $\mu_{\schm, (i_1,0)} \ell_1 + \nu_{\schm, (i_1,0)}$ for constants $\mu_{\schm, (i_1,0)},\nu_{\schm, (i_1,0)}$.

\medskip

\noindent {\bf Step II$''$}. For each $i_1: 1 \le i_1 \le n$, if there are a cycle scheme $\schm=HC_{i_1}^{\ell_1} C_{i_2}^{\ell_2} \dots C_{i_t}^{\ell_t}$  and $j' \le k$ such that 
\begin{itemize}
\item $i_2,\dots,i_t$ are mutually distinct, $\ell_2 = \dots = \ell_t = 1$, 
%
\item $\pi_H(j')-k \not \in J_H$, $\pi_{C_{i_1}}(j')=j'$, and $\mu_{\schm,(i_1,j')} \neq 0$, 
\end{itemize}
then return $\ltrue$.

If there is a cycle scheme $\schm=HC_{i_1}^{\ell_1} C_{i_2}^{\ell_2} \dots C_{i_t}^{\ell_t}$ such that 
$i_2,\dots,i_t$ are mutually distinct, $\ell_2 = \dots = \ell_t = 1$, 
%
and $\mu_{\schm,(i_1,0)} \neq 0$, 
%
then return $\ltrue$. 

If the decision procedure has not return yet, then go to Step III$''$. \qed

\medskip

If after Step II, the algorithm has not return yet, then after removing the terms involving $c\ \ell_1,\dots, c\ \ell_t$, a finite state automaton $\Aa$ can be constructed to simulate the evolvement of the coefficients of the non-constant data variables, and the sum of the constant coefficients and all the terms involving the data variables which are integer constants. 

\medskip



\noindent {\bf Step III$''$}. For each final state $\chi$ of $\Aa$, do the following.
\begin{itemize}
\item Check whether the coefficient of some non-constant data variable in $\chi(O(q_m))$ is nonzero, if the answer is yes, then return $\ltrue$.
%
\item Otherwise, check whether the number obtained from $\chi(O(q_m))$ by replacing the remaining data variables with the corresponding integer constants, is nonzero. If the answer is yes, then return $\ltrue$. 
\end{itemize}
Finally, return $\lfalse$ and the algorithm terminates.

\end{appendix}

%%%%%%%%%%%%%%%%%%%%%%%%%%%%%%%%%%%%%%%%%%%%%%%%%%%%%
%%%%%%%%%%%%%%%%%%%%%%%%%%%%%%%%%%%%%%%%%%%%%%%%%%%%%%%%%%%



