\section{Case Studies}
\label{sec:cases}

\begin{figure}
	\centering
	\lstset{language=C,
		basicstyle=\ttfamily\scriptsize}
	\begin{tabular}{|c|c|c|}
		\hline
		\begin{minipage}[t]{0.2\textwidth}
		\vspace{-0.5cm}
			\begin{lstlisting}[mathescape=true]
int avg() {
 sum:=$\cur$;
 cnt:=0;$\nnext$;
 loop{
  sum+=$\cur$;
  cnt+=1;
  $\nnext$;};
 ret sum/cnt;}
			\end{lstlisting}
		\end{minipage}&
		\begin{minipage}[t]{0.4\textwidth}
		\vspace{-0.5cm}
\begin{lstlisting}[mathescape=true]
int MAD() {
 sum:=$\cur$;cnt:=0;$\nnext$;
 loop{sum+=$\cur$;cnt+=1;$\nnext$;};
 avg:= sum/cnt;mad:=0;$\init$;
 loop{
  if($\cur$<avg){mad=mad+(avg-$\cur$);}
  else{mad=mad+($\cur$-avg);};$\nnext$;};
 ret mad/cnt;}
\end{lstlisting}
		\end{minipage}&
		\begin{minipage}[t]{0.4\textwidth}
		\vspace{-0.5cm}
			\begin{lstlisting}[mathescape=true]
int SD() {
 sum:=$\cur$;cnt:=0;$\nnext$;
 loop{sum+=$\cur$;cnt+=1;$\nnext$;};
 avg:= sum/cnt;sd:=0;$\init$;
 loop{
  sd+=($\cur$-avg)*($\cur$-avg);$\nnext$;
 };
 ret SQRT(sd/cnt);}
			\end{lstlisting}
		\end{minipage}\\
		\hline		
	\end{tabular}
	\caption{More Challenging Examples of Reducers Performing Data Analytics Operations}
	\label{fig:examples2}
\end{figure}
%\vspace{-0.5cm}
	
In this section we discuss more challenging examples. 
For cases with multiplication, division, or other more complicated functions at the return point, e.g., the \texttt{avg} case in Figure~\ref{fig:examples2}(a), we can model them as an \emph{uninterpreted $k$-ary function} and verify all $k$ parameters of the uninterpreted functions remain the same no matter how the input is permuted, e.g., the \texttt{avg} program always produces the same \texttt{sum} and \texttt{cnt} for all permutation of the same input data word. This is a \emph{sound} but \emph{incomplete} procedure for verifying programs of this type. Nevertheless, it is not often that a  practical program for data analytics produces, e.g., $2q/2r$ from some input and $q/r$ for its permutation. Hence this procedure is often enough for proving commutativity for real world programs.

The case \texttt{MAD} (Mean Absolute Deviation) in Figure~\ref{fig:examples2}(b) is a bit more involved. Beside the division operator $/$ that also occurs in the \texttt{avg} example, it uses a new iterator operation $\init$, which reset $\cur$ to the head of the input data word. The strategy to verify this program is to divide the task into two parts: (1) ensure that the value of \texttt{avg} is independent of the order of the input, (2) treat \texttt{avg} as a symbolic variable and then check if the 2nd half of the program (the part after $\nnext$) is commutative. The latter requires an extension to the SNT decision problems. Given a data word $w$, we say a data word $w'$ is a $k$-permutation of $w$ iff there exists $\sigma_k\in S_{|w|}$ such that $\sigma_k(i)=i$ for $1\leq i\leq k$ and $w'=\sigma_k(w)$. A \emph{$k$-commutativity problem} of a SNT $\Ss$ asks whether for each data word $w$ and its $k$-permutation $w'$, $\Ss(w)=\Ss(w')$. The problem can be reduce to SNT equivalence problems in a similar way to the standard commutativity problem.

\begin{proposition}\label{prop-snt-kcmm-to-eqv}
	The $k$-commutativity problem of SNTs is reduced to the equivalence problem of SNTs in exponential time. 
\end{proposition}

To verify the 2nd task of the MAD example, we model \texttt{avg} as a control variable and assign the first value in the input data word to it. Then we can reduce the commutative problem of the 2nd part to the $1$-commutative problem of the corresponding SNT. We handle the division at the end of the problem in the same way as we did for the \texttt{avg} program.

The case \texttt{SD} (Standard Deviation) in Figure~\ref{fig:examples2}(c) is more difficult to handle. The main difficulty comes from the use of multiplication in the middle of the program (instead of at the return point). In order to have a sound procedure to verify this kind of program, we will need the support of uninterpreted $k$-ary function also in the SNT transitions. However, this is not a trivial extension and we consider it as a future work.


