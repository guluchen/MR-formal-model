%!TEX root = main-cav.tex
	
\section{Conclusion}
\label{sec:conclusion}

%From the analysis of the commutativity of reducers in \cite{XZZ+14}, the commutativity of a reducer in a sequential composition of map-reduce jobs may depend on some implicit data properties guaranteed by the preceding map-reduce jobs. Therefore, to analyze the commutativity of a reducer in a sequential composition of map-reduce jobs, we may need model both mappers and reducers and do a backward analysis.

The contribution of the paper is twofold. We propose a verifiable programming language for reducers. Although it is still far away from a practical programming language, we believe some ideas behind our language (e.g., the separation of control variables and data variables) would be valuable for the design of a practical reducer language. On the other hand, we propose the model of streaming numerical transducers, a transducer model over infinite alphabets. To our best knowledge, this is the first automata model over infinite alphabets that allows linear arithmetics over the input values and the integer variables. Although we required that the transition graphs of SNTs are generalized flat,  SNTs with such kind of transition graphs turn out to be quite powerful, since they are capable of simulating reducer programs without nested loops, which is a typical scenario of reducer programs in practice.


% , parameterized counter automata, integer VASS

%restate the contribution of our work. (1) hint for verifiable reducer/list manipulating program language (2) SNT is a first autoamta with infinite alphabet supporting presburgh arithemetic over input and variables.