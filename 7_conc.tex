%!TEX root = main-cav.tex
	
\section{Conclusion}
\label{sec:conclusion}

%From the analysis of the commutativity of reducers in \cite{XZZ+14}, the commutativity of a reducer in a sequential composition of map-reduce jobs may depend on some implicit data properties guaranteed by the preceding map-reduce jobs. Therefore, to analyze the commutativity of a reducer in a sequential composition of map-reduce jobs, we may need model both mappers and reducers and do a backward analysis.

The contribution of the paper is two folds. We propose a verifiable programming language for reducers. Although it is still far away from a practical programming language, we believe some idea behind our language (the separation of control and data variable) would be valuable for the design of a practical reducer language. One the other hand, we propose the SNT model, a transducer model over infinite alphabet. We believe this is the first model over infinite alphabet that allows Presburgh arithmetics over variables and values from the unbounded input tape. Although we have some constraints over the structure of SNTs, we show that it can simulate at least the simple language for reducers. Besides the reducer programs, we  believe SNTs can be applied to the verification of list manipulating programs.


% , parameterized counter automata, integer VASS

%restate the contribution of our work. (1) hint for verifiable reducer/list manipulating program language (2) SNT is a first autoamta with infinite alphabet supporting presburgh arithemetic over input and variables.