\documentclass[runningheads,a4paper]{llncs}

\usepackage{latexsym}
\usepackage{setspace}
\usepackage{cancel}
\usepackage{listings}
\usepackage{graphicx}
\usepackage{appendix}
\usepackage{amssymb}
\usepackage{stmaryrd}
\usepackage{amsmath}
\usepackage{leftidx}
\usepackage{mathtools}
\usepackage[linesnumbered,noend]{algorithm2e}
\usepackage{paralist}


%\usepackage{cancel}
%\usepackage{verbatim}
%\usepackage{chngpage}
%\usepackage{fullpage}

\usepackage{color}

\usepackage{mathrsfs}

%\newtheorem{definition}{Definition}
%\newtheorem{theorem}{Theorem}
%\newtheorem{proposition}[theorem]{Proposition}
%\newtheorem{corollary}[theorem]{Corollary}
%\newtheorem{example}{Example}
%\newtheorem{question}{Open Question}
%\newtheorem{lemma}[theorem]{Lemma}
%\newtheorem{Algo}{Algorithm}
%\newtheorem{remark}[theorem]{Remark}

\def\arr#1{\stackrel{#1}{\longrightarrow}}

\def\Aa{{\mathcal{A} }}

\def\Bb{{\mathscr{B} }}

\def\Cc{{\mathcal{C} }}

\def\Dd{{\mathbb{D} }}

\def\Ee{{\mathcal{E} }}

\def\Ff{{\mathcal{F} }}

\def\Zz{{\mathcal{Z} }}

\def\Nn{{\mathbb{N} }}

\def\Ss{{\mathcal{S} }}

\def\schm{{\mathfrak{s} }}

\def\Tt{{\mathcal{T} }}

\def\Ii{{\mathbb{Z} }}

\def\Jj{{\mathcal{J}}}

\def\Vv{{\mathcal{V}}}

\def\Rr{{\mathcal{R} }}

\def\Ll{{\mathcal{L}}}


\def\treeset{{\mathscr{T}}}

\def\contextset{{\mathcal{C}}}

\def\theory{{\mathcal{L}}}

\def\termset{{\mathcal{T}}}

\def\formulaset{{\mathcal{F}}}

\newcommand\univ{\mathsf{Univ}}

\newcommand\op{\mathfrak{o}}

\newcommand\dv{\mathtt{x}}

\newcommand\ydv{\mathtt{y}}

\newcommand\cv{\mathtt{z}}

\newcommand\thla{\mathcal{LIA}}

\newcommand\thdif{\mathcal{DIF}}

\newcommand\thord{\mathcal{ORD}}

\newcommand\thset{\mathcal{SET}}

\newcommand\thmset{\mathcal{MUS}}

\newcommand\natnum{{\mathbb{N} }}

\newcommand\intnum{{\mathbb{Z} }}

\newcommand\cur{\mathsf{cur}}
\newcommand\next{\mathsf{next}}
\newcommand\head{\mathsf{hd}}
\newcommand\tail{\mathsf{tl}}
\newcommand\init{\mathsf{init}}

\newcommand{\loopL}[1]{\mbox{loop\{} #1\mbox{\}}}
\newcommand{\ite}[3]{\mbox{if } #1 \mbox{ then } #2\mbox{ else }#3 }



\newcommand\vars{\mathsf{vars}}

\newcommand\intvars{\mathcal{X}}

\newcommand\dom{\mathsf{dom}}

\newcommand\rng{\mathsf{rng}}

\newcommand\ltrue{\mathsf{true}}

\newcommand\lfalse{\mathsf{false}}

\newcommand\avg{\mathrm{avg}}

\newcommand\maxv{\mathsf{max}}

\newcommand\sumv{\mathsf{sum}}

\newcommand\cntv{\mathsf{cnt}}

\newcommand\addeq{+\!\!=}

\newcommand\defval{\mathsf{DEF}}

\newcommand{\sub}[2]{\mathsf{sub}_{#2}(#1)}
\newcommand{\eval}[2]{\llbracket#1\rrbracket_{#2}}

%%%%%%%%%%%%%%%%%%%%%%%%%%%%%%%%%%%%%%%%%%%%
%%%%%%%%%%%The macros introduced by Zhilin%%%%%%%%%%%%%
%%%%%%%%%%%%%%%%%%%%%%%%%%%%%%%%%%%%%%%%%%%%
\newcommand{\sval}{\Omega}
\newcommand{\sumf}{\Theta}
\newcommand{\initval}{\sval}

% The macros for the data variables.
\newcommand{\vard}{\mathfrak{d}}
\newcommand{\vare}{\mathfrak{e}}
\newcommand{\varf}{\mathfrak{f}}
\newcommand{\varo}{\mathfrak{o}}
\newcommand{\varx}{\mathfrak{x}}
\newcommand{\vary}{\mathfrak{y}}

% The macros for the data variables.
\newcommand{\csta}{\alpha}
\newcommand{\cstb}{\beta}
\newcommand{\cstc}{\gamma}
\newcommand{\cste}{\varepsilon}
\newcommand{\cstl}{\lambda}
\newcommand{\cstm}{\mu}
\newcommand{\cstn}{\nu}
%%%%%%%%%%%%%%%%%%%%%%%%%%%%%%%%
%%%%%%%%%%%%%%%%%%%%%%%%%%%%%%%%
%%%%%%%%%%%%%%%%%%%%%%%%%%%%%%%%

\newcommand{\hide}[1]{}
\newcommand{\yfc}[1]{\color{blue} {YF: #1 :FY} \color{black}}
\newcommand{\zhilin}[1]{\color{cyan} {ZL: #1 :LZ} \color{black}}
\newcommand{\lei}[1]{\color{green} {LE: #1 :EL} \color{black}}
\newcommand{\SDSIT}{SDSIT}
\newcommand{\Name}{Streaming data string to integer transducer}
\newcommand{\name}{streaming data string to integer transducer}
\newcommand{\interval}[1]{[#1]}
%\def\Ss{{$\mathcal{A}$\ }}

\title{The~Commutativity~Problem~of~the~Map-Reduce Framework: A Transducer-based Approach}
\titlerunning{Commutativity of MapReduce: A Transducer-based Approach}
\author{}
\institute{}

%\author{Yu-Fang Chen, Lei Song, Zhilin Wu}

\begin{document}

\maketitle

\begin{abstract}

MapReduce is a popular programming model for data parallel computation. 
In MapReduce, the \emph{reducer} produces an output from a list of inputs. Due the scheduling policy and the settings of machines, the input may arrive the reducers with different orders. The \emph{commutative problem} of reducers asks if the output of a reducer independent of the order of its inputs. The problem is in general undecidable due to Rice's theorem and thus is seemingly uninteresting. However, the MapReduce model is usually used for data analytics and thus requires very simple data and control flow. 
By exploiting the simplicity, we propose a simple programming language for reducers where the commutative problem can be decide by a reduction to the equivalence problem of \emph{streaming numerical transducers} (SNTs). 
%We show that the language is expressive enough for common data analytics operations.
\end{abstract}

%!TEX root = main-cav.tex

\section{Introduction}
%What is Map-Reduce
MapReduce is a  popular framework for data parallel computations. It has been adopted in various cloud computing frameworks such as Hadoop~\cite{Hadoop} and Spark~\cite{Spark}. In a typical MapReduce program, a \emph{mapper} reads from data sources and outputs a list of key-value pairs. 
The load balance mechanism of the Map-Reduce framework reorganizes the key-value pairs $(k, v_1), (k,v_2)\ldots(k,v_n)$ with the same key $k$ to a pair $(k,l)$, where $l$ is a list of values $v_1,v_2,\ldots,v_n$, and sends $(k,l)$ to a \emph{reducer}. The reducer then iterates through the list and outputs a key-value pair.

To be more concrete, taking the ``word-counting'' MapReduce program as an example. It counts the occurrences of each word in a set of documents. The mappers read the documents and output for each document a list in the form of $(word_1, count_1)$, $(word_2, count_2)$, $\ldots$ , $(word_n, count_n)$, where $count_k$ is the number of occurrences of $word_k$ in the document being processed. These lists will be reorganized into the form of $(word_1, list_1), (word_2,list_2), \ldots, (word_n,list_n)$ and sent to the reducers, where $list_k$ is a list of integers recording the number of occurrences of $word_k$ in the set of documents. Note that the \emph{order} of the integers in the lists can differ in different executions due to the scheduling policy. This results in the \emph{commutativity problem}.

%The communitativity problem and its importance
We say that a reducer is \emph{commutative} if its output is independent of the order of its inputs. The commutativity problem asks if a reducer is commutative. A study from Microsoft~\cite{XZZ+14} reports that 58\% of the 507 reducers submitted to their MapReduce platform are non-commutative, which may lead to very tricky and hard-to-find bugs.
As an evidence, those reducers already went through serious code review, testing, and experiments with real data for more than three months. Still, among them 5 reducers containing very subtle bugs caused by non-commutativity (confirmed by the programmers). 
%Moreover, having the commutativity property makes reproducing a bug found by program testing easier.

%The reason for studying syntatical restrictions 
The reducer commutativity problem in general is undecidable. However, in practice, MapReduce programs are usually used for data analytics and have very simple control structures. Many of them just iterate through the input list and compute the output with very simple operations. We want to study if the commutativity problem of real-world reducers are decidable. It has been shown in~\cite{CHSW15} that even with a simple programming language where  the only loop structure allowed is to go over the input list once, the commutativity problem is already undecidable. Under scrutiny, we found that the language is still too expressive for typical data analytics programs. For example, it allows arbitrary multiplications of variables and input values, which is a key element in the undecidability proof.

%The tacas 2015 undecidability result
%A simple programming language of reducers over integers has been considered in~\cite{CHSW15}, where the only loop structure allowed is an iteration over the input list and it is not allowed to reset the iterator to the list head. They show that the commutative problem of programs written by such a simple language is undecidable by a reduction from the satisfiability problem of Diophantine equations. Under scrutiny, we found that the language is still too expressive for typical data analytics programs. For example, it allows arbitrary multiplications of variables and input values, which is a key element in the undecidability proof. 
%our languge

\smallskip

\noindent {\it Contributions}.
By observing the behavioral patterns of reducer programs for data analytics, we first design a programming language for reducers to characterize the essential features of them. %However, we found that even only with the essential parts of the language, the commutativity problem is still undecidable. 
We found that the commutativity problem becomes decidable if we partition variables into \emph{control variables} and \emph{data variables}. Control variables can occur in transition guards, but can only store values directly from the input list (e.g., it is not allowed to store the sum of two input values in a control variable). On the other hand, data variables are used to aggregate some information for outputs (e.g. sum of the values from the input list), but cannot be used in transition guards. This distinction is inspired by the streaming transducer model~\cite{RP11}, which, we believe, provides good insights for reducer programming language design in the MapReduce framework. Moreover, we assume that there are no nested loops in the language for reducers, which is a typical situation for MapReduce programs in practice.

%%%SNTs
We then introduce a formalism called \emph{streaming numerical transducers (SNT)} and obtain a decision procedure for the commutativity problem of the aforementioned language for reducers.
Similar to the language for reducers, SNTs distinguish between control variables and data variables. Although conceptually SNTs are similar to streaming transducers over data words introduced in \cite{RP11}, they are intrinsically different in the following sense: The outputs of SNTs are integers and the integer variables therein are manipulated by linear arithmetic operations. On the other hand, the outputs of streaming transducers are data words, and the data word variables are manipulated by concatenation operations. SNTs in this paper are assumed to be \emph{generalized flat}, which generalizes the ``flat'' automata (c.f. e.g. \cite{LS06}) in the sense that each nontrivial strongly connected component (SCC) of the transition graph is a collection of cycles, instead of one single cycle. Generalized flat transition graphs are sufficient to capture the transition structures of the programs in the aforementioned language for reducers.

The decision procedure for the commutativity problem is obtained by reducing to the equivalence problem of SNTs, which is further reduced to the non-zero output problem. The non-zero output problem asks whether given an SNT, there exists some input data word $w$ such that the output of the SNT on $w$ is defined and non-zero.  For the non-zero output problem of SNTs, we apply a nontrivial combinatorial analysis of the evolvement of the integer variables during the runs of SNTs (Section~\ref{sec-sum}). The key idea of the decision procedure is that generally speaking, if only the non-zero output problem is concerned, the different cycles in the SCCs can be dealt with \emph{independently} (Section~\ref{sec-glasso} and \ref{sec-gflat}). 
%
As a further evidence of the usefulness of SNTs for MapReduce programs, we demonstrate that SNTs can be composed to model and analyze the reducer programs that read the input list multiple times (Section~\ref{sec:cases}). 

As a novel formalism over infinite alphabets, the model of SNTs is interesting in its own right: On the one hand, SNTs are expressive in the sense that they include linear arithmetic operations on integer variables, while at the same time admit rather general transition graphs, that is, generalized flat transition graphs. On the other hand, despite this strong expressibility, it turns out that the commutativity problem, the equivalence problem, and the non-zero output problem of SNTs are still decidable.  
%The model of SNTs and the decision procedure are interesting in their own right.

\smallskip

\noindent {\it Related work}.
SNTs can be seen as generalizations of classical register automata~\cite{KF94,NSV04} where registers correspond to the control variables in our terminology. Although register automata can have very general transition graphs beyond the generalized flat ones, they have neither data variables nor arithmetic operations on these variables. More recently,  register automata with polynomial updates on the registers were investigated and it was shown that if there is only one register, then the reachability problem is PSPACE-complete \cite{FGH13}.
%
%As far as we know, this is the \emph{first} automata model over infinite alphabet allowing quantifier-free Presburgh arithmetics over variables and values from an unbounded input tape. Moreover, the model can be applied to the verification of other interesting classes of programs, e.g., programs with unbounded list as the input.
%As far as we know, the SNT model  is different from all other automata models. 
%
The model of symbolic transducers~\cite{VHL+12} also has data words as inputs. Symbolic transducers can put guards on the input value in one position of data words, but are unable to compare and aggregate multiple input values in different positions. The model of cost register automata~\cite{ADD+13} includes arithmetic operations on the registers which store the costs associated with runs. Nevertheless, the inputs of cost register automata are words on finite alphabets, instead of data words. There is also plenty of work on counter automata (see \cite{Iba78,CJ98,LS06}, to cite a few), where counters can be seen as integer variables in our terminology. Nevertheless, the inputs of counter automata are words on finite alphabets.

The rest of the paper is organized as follows. Section~\ref{sec:preliminaries} fixes the notations of this paper. Section~\ref{sec:language} describes our design of the programming language for reducers. Section~\ref{sec:def-snt} presents the definition of SNTs. Section~\ref{sec:dec-snt} describes the decision procedure of SNTs. Sec~\ref{sec:cases} demonstrates how to use our framework to verify the commutativity property of the more challenging data analytics programs. We conclude this paper in Section~\ref{sec:conclusion}. 

%!TEX root = main-cav.tex

\section{Preliminaries}
\label{sec:preliminaries}
Let $\intnum$,  $\intnum^+$ be the set of integers, positive integers, respectively and $\bot \not \in \intnum$ be  the undefined value. By convention, it is assumed that $\bot + \bot = \bot$, $\bot + n = \bot$ for each $n \in \intnum$, $0 \times \bot = 0$, and $n \times \bot = \bot$ for each $n \in \intnum^+$.
In this paper, we assume that all variables range over $\intnum \cup \{\bot\}$.  

For a function $\pi$, let $\dom(\pi)$ and $\rng(\pi)$ denote the \emph{domain}  and \emph{range} of $\pi$, respectively. We use $\emptyset$ to denote an empty function. An \emph{expression} $e$ over the set of variables $Z$ is defined by the following rules, $e\equiv  c \mid  c z \mid (e + e) \mid (e - e)$, where $z \in Z$ and $c\in \intnum$.  As a result of the commutativity of $+$, without loss of generality, we assume that all expressions $e$ in this paper are of the form $c_0 + c_1 z_1 + \dots c_n z_n$, where $c_0, c_1,\dots,c_n \in \intnum$ and $z_1,\dots,z_n \in Z$. 

% \zhilin{I changed a bit the definition of expressions here} 

For an expression $e$, let $\vars(e)$ denote the set of variables occurring in $e$. Let $\Ee_Z$ denote the set of all expressions over the set of variables $Z$. 
%Let $e$ be an expression and $\eta$ be a partial function from $\vars(e)$ to expressions. Then we use $\sub{e}{\eta}$ to denote the expression obtained from $e$ by replacing each variable $z \in \vars(e)\cap\dom(\eta)$ with $\eta(z)$. 
A \emph{valuation} $\rho$ of $Z$ is a function from $Z$ to $\intnum\cup\{\bot\}$. A \emph{symbolic valuation} $\sval$ of $Z$ is a function which maps each variable $z \in Z$ to an expression (possibly over a different set of variables). The value of $e$ under a (symbolic) valuation $\rho$, denoted by $\eval{e}{\rho}$, is defined recursively in the standard way. For example, let $\rho$ be a symbolic valuation the maps $x$ to $x+y$ and $y$ to $3y$, then $\eval{2x+3y}{\rho}=2\eval{x}{\rho}+3\eval{y}{\rho}=2(x+y)+3(3y)=2x+11y$.
%Note that a valuation can be seen as a special symbolic valuation where the image of each variable is a constant.
For a function $\rho$, define the function $\rho[d/x]$ such that $\rho[d/x](x)=d$ and $\rho[d/x](y)=\rho(y)$ for $y\neq x$. 

In this paper, we use $X$ and $Y$ to denote the sets of \emph{control variables} and \emph{data variables}, respectively. We use the variable $\cur \notin X\cup Y$ to store the data value that is currently being processed in the input list and use $X^+$ to denote the set $X\cup \{\cur\}$.
A \emph{guard} is a formula defined by the following rules, $g::= \ltrue \mid \cur\odot c \mid \cur\odot x \mid g \wedge g$, where $\odot \in \{=,\neq,<, >, \le, \ge\}$, $x \in X$, and $c\in \intnum$. 
Let $\rho$ be a valuation and $g$ be a guard. Then $\rho$ satisfies $g$, denoted by $\rho \models g$, iff for each variable $z \in \vars(g)$, $\rho(z) \neq \bot$, and $g$ is evaluated to $\ltrue$ under $\rho$. 

Let $\interval{n}$ be the set $\{ 1, 2, \ldots, n \}$. Let $\interval{a,b}$ be the set $\{ a, a+1, \ldots, b \}$ when $b\geq a$ and $\emptyset$ otherwise. A \emph{permutation} on
$\interval{n}$ is a one-to-one and onto mapping from $\interval{n}$ to
$\interval{n}$. The set of
permutations on $\interval{n}$ is denoted by $S_n$.

A \emph{data word $w$} is a sequence of integer values $d_1\ldots d_n$ such that $d_i \in \intnum$ for each $i$.
We use $\head(w)$, $\tail(w)$, and $|w|$ to denote the \emph{head}, \emph{tail}, and \emph{length} of $w$, respectively.
We use $\emptyset$ to denote an empty data word.
Given two data words $w,w'$, we use $w.w'$ to denote their concatenation.
Given $\sigma \in S_n$, we lift $\sigma$ to data words by defining $\sigma(w)=d_{\sigma(1)} \ldots d_{\sigma(n)}$, for each data word $w=d_1\ldots d_n$. We call $\sigma(w)$ as a permutation of $w$.

%!TEX root = main-cav.tex
 
\section{Language For Integer Reducers}\label{sec-mr-prog}
\label{sec:language}
We discuss the rationale behind the design of the programming language for reducers such that the commutativity problem is decidable. The language intends to support the following typical behavior pattern of reducers: A reducer program iterates through the input data word once, aggregates intermediate information into variables, and produces an output when it stops. 
%We focus on a language that only allows integer data type and hence the input is a data word. It is allowed to iterate though the input data word only once. 
%
Later in Section~\ref{sec:cases}, we will show an extension that allows resetting the iterators so that an input data word can be traversed multiple times.
%
\vspace{-2mm}
\begin{figure}
	\centering
	\begin{tabular}{rcl}
        $ s \in Statements$&$::=$&$y := e;\mid y \addeq e; \mid x:=\cur;\mid s\ s\mid \nnext; \mid \ite{g}{s}{s}$\\
		$ p\in Programs$&$::=$&$\loopL{s\ \nnext;}\mbox{ret }r; \mid s\ \nnext;p$		
	\end{tabular}
	\caption{A Simple Programming Language for Reducers. Here $x\in X$ are control variables, $y\in Y$ are data variables, $x' \in X^+$, $e\in \Ee_{X^+}$ are expressions, and $r$ is an expression in $\Ee_{X \cup Y}$. The square brackets mean that the else branch is optional. }
	\label{fig:language}
%		\vspace{-0.5cm}
\end{figure}

\vspace{-4mm}

More concretely, we focus on the programming language in Fig.~\ref{fig:language}. The language includes the usual features of program languages, variable assignments, sequential compositions, and conditional branchings. It also includes a statement $\nnext;$ which is used to advance the data word iterator. The $\loopL{s\ \nnext;}$ statement repeatedly executes the loop body $s\ \nnext;$ until reaching the end of input data word.
The novel feature of the language is that we partition the variables into two sets: \emph{control variables} $X$ and \emph{data variables} $Y$.
The variables from $X$ are used for guiding the control flow and the variables from $Y$ are used for storing aggregated intermediate data values.
The variables from $X$ can store only either initial values of variables in $X$ or values  occurring in the input data word. They can occur both in guards $g$ or arithmetic expressions $e$.
On the other hand, the variables from $Y$ can aggregate the results obtained from arithmetic expressions $e$, but cannot occur in guards $g$ or arithmetic expressions $e$. The initial values of variables can be arbitrary.
Given a program $p$, a data word $w$, and a valuation $\rho_0$, we use $p_{\rho_0}(w)$ to denote the output of $p$ on $w$, with the initial values of variables given by $\rho_0$. The formal semantics of the language can be found in the appendix. 

In this paper, we assume that the reducer programs $p$ satisfy that \emph{all the guards between two consecutive $\nnext$ statements are mono-typed}, more specifically,
for each execution path in the control flow graph of $p$ and each pair of consecutive $\nnext$ statements, either all the guards $g$ of the branching statements between them are of type 1, or all the guards between them are of type 2 (cf. Section~\ref{sec:preliminaries} for the definition of guards).
In addition, to simplify the presentation, we assume that the reducer programs $p$ are \emph{transition-enabled} in the following sense, 
%
for each execution path in the control flow graph of $p$, there is an input $w$ and initial valuations of variables $\rho_0$ so that the run of $p$ over $w$ and $\rho_0$ follows the execution path.


Note that we do not allow multiplications in the language, so the reduction from the Diophantine equations in \cite{CHSW15} no longer works. Even though, if we do not distinguish the control and data variables, we can show easily that commutativity problem for this language is still undecidable, by a reduction from the reachability problem of Petri nets with inhibitor arcs~\cite{Min71,Rei08}.
%Intuitively, integer variables are used for remembering the number of tokens in each place. Transitions between places can be simulated by the loop body: The updates of the tokens of the places are simulated by variable assignments.
%The inhibitor arcs are simulated by the guards on the integer variables in the branching statements (recall that we are discussing the version that we do not distinguishing the control and data variables). The non-determinism in the transitions of Petri nets is resolved by the guards on the current data values, e.g. $\cur = 2$. 
The reachability problem of Petri nets with inhibitor arcs is reduced to the reachability problem of the reducer programs, which is in turn easily reduced to the commutativity problem of reducer programs.

Notice that in the programming language, we only allow additions ($\addeq$) or assignments ($:=$) of a new value computed from an expression over $X^+$ to data variables. 
In Fig.~\ref{fig:examples} we demonstrate a few examples performing data analytics operations. Observe that all of them follow the same behavioral pattern: The program iterates through the input data word and aggregates some intermediate information into some variables. The operations used for the aggregation are usually rather simple: either a new value is added to the variable (e.g. \texttt{sum} and \texttt{cnt} in Fig.~\ref{fig:examples}) storing the aggregated information, or a new value is assigned to the variable (e.g. \texttt{max} in Fig.~\ref{fig:examples}). Actually, the similar behavioral pattern occurs in all programs we have investigated.
Still, one may argue that allowing only additions and subtractions is too restrictive for data analytics. 
In Section~\ref{sec:cases}, we will discuss the extensions of the language to support more challenging examples, such as \emph{Mean Absolute Deviation} and \emph{Standard Deviation}.

\begin{figure}
	\centering
	\lstset{language=C,
		basicstyle=\ttfamily\scriptsize}
	\begin{tabular}{|c|c|c|}
		\hline
		\begin{minipage}[t]{0.39\textwidth}
			(a)
			\begin{lstlisting}[mathescape=true]
max_abs {
 if $\cur$>0 then max:=$\cur$
 else max:= -$\cur$;
 $\next$;
 loop{
  if $\cur$>0 then
   if $\cur$>max then max:=$\cur$
  else 
   if -$\cur$>max then max:=-$\cur$;;
  $\next$;  
 } 
 ret max;
}
	\end{lstlisting}
		\end{minipage}&
		\begin{minipage}[t]{0.27\textwidth}
			(b)
			\begin{lstlisting}[mathescape=true]
sum{
 sum:=$\cur$;$\next$;
 loop{sum+=$\cur$;$\next$;};
 ret sum;
}
			\end{lstlisting}
\hrule\vspace{0.1cm}			
			(c)
			\begin{lstlisting}[mathescape=true]
cnt{
 cnt:=0;$\next$;
 loop{cnt+=1;$\next$;};
 ret cnt;
}
			\end{lstlisting}			
		\end{minipage}&
		\begin{minipage}[t]{0.30\textwidth}
			(d) 
			\begin{lstlisting}[mathescape=true]
2nd_largest {
 a:=$\cur$;b:=$\cur$;$\next$;
 if $\cur$>a then a:=$\cur$
 else b:=$\cur$;
 $\next$;
 loop{
  if $\cur$>a then 
   b:=a;a:=$\cur$
  else if $\cur$>b then 
   b:=$\cur$;;
  $\next$;
 }
 ret b;
}
			\end{lstlisting}		
		\end{minipage}\\
		\hline		
	\end{tabular}
	\caption{Examples of Reducers Performing Data Analytics Operations}
	\label{fig:examples}
\end{figure}

%The observation is actually not a big surprise. The \emph{reduce} operation in MapReduce has a tight connection with the \emph{fold} operation in functional languages, which aggregates the values in an input data word using a \emph{binary function}. We argue that our language is sufficient to describe the \emph{fold} operations in functional languages involving only additions and subtractions: The intermediate results of the fold operations can be stored in some data variables. The binary functions in the fold operations can be simulated by ``adds/assigns a value to data variables'' in our language. However, our language is far more flexible than the fold operations in the sense that we have the full control of what to do when iterating through an input data word (similar to Hadoop~\cite{Hadoop}).
 

We focus on the following problems of reducer programs: (1) \emph{Commutativity}: given a program $p$, decide whether for each data word $w$ and its permutation $w'$, it holds that $p_{\rho_0}(w) = p_{\rho_0}(w')$ for all initial valuations $\rho_0$. (2) \emph{Equivalence:} given two programs $p$ and $p'$, decide whether for each data word $w$ and each initial valuation $\rho_0$, it holds that $p_{\rho_0}(w)=p'_{\rho_0}(w)$.







%!TEX root = main-cav.tex


%%%%%%%%%%%%%%%%%%%%%%%%%%%%%%%%%%%%%%%%%%%%%%%%
%%%%%%%%%%%%streaming numerical transducer%%%%%%%%%%%%%%%%
%%%%%%%%%%%%%%%%%%%%%%%%%%%%%%%%%%%%%%%%%%%%%%%%

\section{Streaming Numerical Transducers}\label{sec:def-snt}

In this section, we introduce \emph{streaming numerical transducers} (SNTs), whose inputs are data words and outputs are integer values. A SNT scans a data word from left to right, records and aggregates some information using control and data variables, and outputs an integer value when it finishes reading the data word. We will use SNTs to decide the commutativity and equivalence problem of the reducer programs defined in Section~\ref{sec-mr-prog}.


A SNT $\Ss$ is a tuple $(Q, X, Y, \delta, q_0, O)$ where $Q$ is a finite set of states, $X$ is a finite set of control variables to store data values that have been met, $Y$ is a finite set of data variables to aggregate information for the output, $\delta$ is the set of transitions, $q_0 \in Q$ is the initial state, $O$ is the output function, which is a partial function from $Q$ to $\Ee_{X \cup Y}$.%\zhilin{Here the variable $\cur$ should not be used.}
The set of transitions $\delta$ comprises the tuples $(q,  g, \eta, q')$, where $q,q'\in Q$, $g$ is a guard over $X^+$ (defined in Section~\ref{sec:preliminaries}), and $\eta$ is an assignment which is a partial function mapping $X \cup Y$  to $\Ee_{X^+ \cup Y}$ such that for each $x \in \dom(\eta) \cap X$, $\eta(x)=\cur$ or $\eta(x) = x'$ for some $x' \in X$. We write $q \xrightarrow{(g,\eta)} q'$ to denote $(q,g,\eta,q') \in \delta$ for convenience. 

Moreover, we assume that the SNT $\Ss$ satisfies the following constraints. (1) \emph{Deterministic:} For each pair of distinct transitions originating from $q$, say $(q, g_1, \eta_1,q'_1)$ and $(q, g_2,\eta_2,q'_2)$, it holds that $g_1 \wedge g_2$ is unsatisfiable. (2) \emph{Generalized flat:} Each SCC (strongly connected component) of the transition graph of $\Ss$ is either a single state or a set of simple cycles $\{C_1,\dots, C_n\}$ which contains a state $q$ such that for each $i,j: 1 \le i < j \le n$, $q$ is the \emph{only} state shared by $C_i$ and $C_j$. (3) \emph{Independently evolving and copyless:} For each $(q, g, \eta, q') \in \delta$ and for each $y \in \dom(\eta)$, $\eta(y)=e$ or $\eta(y)=y+e$ for some expression $e$ over $X^+$.

The semantics of a SNT $\Ss$  is defined as follows. A \emph{configuration} of $\Ss$ is a pair $(q,\rho)$, where $q \in Q$ and $\rho$ is a valuation of $X \cup Y$. The \emph{initial} configuration of $\Ss$ is $(q_0,\rho_0)$, where $\rho_0(z)=\bot$ for each $z \in X\cup Y$.
A sequence of configurations $(q_0,\rho_0)(q_1,\rho_1)\ldots(q_n,\rho_n)$ is
a \emph{run} of $\Ss$ over a data word $w=d_1 \dots d_n$ iff there exists a path (sequence of transitions) $q_0 \xrightarrow{(g_1,\eta_1)} q_1 \xrightarrow{(g_2,\eta_2)} q_2 \dots q_{n-1} \xrightarrow{(g_n, \eta_n)} q_n$ such that for each $i: 1 \le i \le n$, $\rho_{i-1}[d_i/\cur] \models g_i$, and $\rho_i$ is obtained from $\rho_{i-1}$ as follows: (1) For each $x \in X$, if $\eta_i(x)=\cur$, then $\rho_i(x)=d_i$, otherwise, if $\eta_i(x)=x' \in X$, then $\rho_i(x)=\rho_{i-1}(x')$, otherwise, $\rho_i(x)=\rho_{i-1}(x)$. (2) For each $y \in Y$, if $y \in \dom(\eta_i)$, then $\rho_i(y)=\eval{\eta_i(y)}{\rho_{i-1}[d_i/\cur]}$, otherwise, $\rho_i(y)=\rho_{i-1}(y)$.

We call $(q_n,\rho_n)$ the \emph{final configuration} of the run. We say that $(q_i,\rho_i)$ is \emph{reachable} from $(q_0,\rho_0)$, for $i \in [n]$.
We would like to remark that for each data word $w$, there is at most one run of $\Ss$ over $w$, since $\Ss$ is deterministic. 
Over a data word $w = d_1 \dots d_n$, if there is a run of $\Ss$ over $w$ with the final configuration $(q_n,\rho_n)$, and $O(q_n)$ is defined, then the output of $\Ss$ over $w$, denoted by ${\Ss}(w)$, is $\eval{O(q_n)}{\rho_n}$. Otherwise, ${\Ss}(w)$ is $\bot$.

\begin{example}[SNT for max]
The SNT $\Ss_{\max}$ for computing the maximum value of an input data word is defined as $(\{q_0,q_1\}, \{\maxv\}, \emptyset, \delta, q_0, O)$ such that $\delta = \{(q_0, \ltrue, \maxv:=\cur, q_1), (q_1, \maxv < \cur, \maxv:=\cur,q_1), (q_1, \maxv \ge \cur, \emptyset, q_1)\}$ and $O(q_1)=\maxv$, where $\maxv:=\cur$ denotes the assignment mapping $\maxv$ to $\cur$.
\end{example}
%\begin{example}[SNT for sum]
%$\Ss_{\mathrm{sum}}=(\{q_0,q_1\}, \emptyset, \{\sumv\}, \delta, q_0, O)$ such that %$\delta=\{(q_0, \ltrue, \sumv:=\cur, q_1), (q_1, \ltrue, \sumv:=\sumv + \cur, q_1)$, and %$O(q_1)=\sumv$. 
%\end{example
\begin{proposition}\label{prop-mrprog-to-snt}
For each reducer program $p$, an equivalent SNT $\Ss$ can be constructed.
\end{proposition}
The main difference between $p$ and $\Ss$ is that several statements in the control flow of $p$ correspond to one transition of $\Ss$. A reducer program moves to the next value of an input data word only when a $\nnext$ statement is executed while an SNT advances the iterator in each transition. Based on this observation, we can have a work-list algorithm to translate a program $p$ into an SNT $\Ss$ (see Algorithm~\ref{fig:reducer2SNT} in the appendix).
In fact, we can show that if we add guards at the entry points of loops and allow multiple loops in the reducer language, then the language becomes expressively equivalent to SNTs.

%\begin{example}[Example inspired by Pagerank]
%The following transducer sum all the data values, except the last position, then it outputs a concatenation of the sum and the last tuple: $(q_0, 1, true, sum:= sum + p_1, q_0)$, $(q_0, k, true, (x_i:=p_i)_{1 \le i \le k}, q_1)$, $O(q_1)=(sum, x_1,\dots, x_k)$.
%\end{example}



%We first compute a fixed point $\defval$ inductively as follows. 
%\begin{enumerate}
%\item Initially, let $\defval_0=\{(q_0,\emptyset)\}$.
%
%\item For each $i > 0$, compute $\defval_i$ from $\defval_{i-1}$ as follows,
%\begin{itemize}
%\item each element of $\defval_{i-1}$ is an element of $\defval_i$, 
%
%\item for each $(q,Z) \in \defval_{i-1}$ and each transition $(q,g,\eta,q') \in \delta$,  let $Z' = Z \cup (X \cap \dom(\eta)) \cup \{y \in Y \cap \dom(\eta) \mid \vars(\eta(y)) \subseteq Z \cup \{\cur\}\}$, put $(q',Z')$ into $\defval_i$.
%\end{itemize}
%
%\item If $\defval_{i-1}=\defval_i$, then the computation stops, otherwise, let $i:=i+1$ and the computation continues.
%\end{enumerate}




We focus on three decision problems of SNTs: (1) \emph{Commutativity}: Given a SNT $\Ss$, decide whether $\Ss$ is commutative, that is, whether for each data word $w$ and each permutation $w'$ of $w$, $\Ss(w)=\Ss(w')$. (2) \emph{Equivalence}: Given two SNTs $\Ss_1,\Ss_2$, decide whether $\Ss_1$ and $\Ss_2$ are equivalent, that is, whether over each data word $w$, $\Ss_1(w)=\Ss_2(w)$. (3) \emph{Non-zero output}: Given a SNT $\Ss$, decide whether $\Ss$ has a non-zero output, that is, whether there is a data word $w$ such that $\Ss(w)\notin \{\bot, 0\}$. 

It turns out that the commutativity problem can be reduced to the equivalence problem, whichcan be further reduced to the non-zero output problem.

\begin{proposition}\label{prop-snt-cmm-to-eqv}
The commutativity problem of SNTs is reduced to the equivalence problem of SNTs in exponential time. 
\end{proposition}
\begin{proposition}\label{prop-snt-eqv-to-nzero}
From SNT $\Ss_1$ and $\Ss_2$, a SNT $\Ss_3$ can be constructed in polynomial time such that there is a data word $w$ that $\Ss_1(w) \neq \Ss_2(w)$ iff $\Ss_3(w) \not\in \{\bot,0\}$. 
\end{proposition}


We normalize SNTs in order to simplify the presentation of the decision procedure later.
Suppose $\Ss=(Q,X,Y,\delta,q_0,O)$ is an SNT. Let $c_{min}$ and $c_{max}$ denote the minimum resp. maximum integer constant occurring in the guards of the transitions in $\delta$. If no integer constant occurs in the guards, then $c_{min}=c_{max}=0$.

An SNT $\Ss=(Q,X,Y,\delta,q_0,O)$ is said to be \emph{normalized} if the following constraints are satisfied:
(1) \emph{Well-defined}: For each run $(q_0,\rho_0) \dots (q_n,\rho_n)$ of $\Ss$ with $q_0 \xrightarrow{(g_1,\eta_1)} q_1 \dots q_{n-1} \xrightarrow{(g_n,\eta_n)} q_n$ as the transitions, and each $i \in [n]$, it holds that $\rho_{i}(z) \neq \bot$ for all $z \in \dom(\eta_i)$, 
%more formally, it holds that $\vars(\eta_i(z)) \subseteq \{z' \mid \rho_i(z') \neq \bot\} \cup \{\cur\}$, 
%
moreover, if $O(q_n)$ is defined, then $\rho_n(z)\neq\bot$ for all  $z\in \vars(O(q_n))$. (2) \emph{Uniquely-valued}: For each $(q,g,\eta,q') \in \delta$, if $\eta(x)=\cur$ for some $x \in X$, then the guard $g$ implies $\bigwedge_{x \in X} \cur \neq x$.  Intuitively, when the current data value $\cur$ is stored into some control variable, it is required that $\cur$ is distinct from all the data values that have already been stored in the control variables. (3) \emph{State-dominating}: For each state $q \in Q$, and every pair of valuations $\rho,\rho'$ such that $(q,\rho)$ and $(q,\rho')$ are reachable from the initial configuration $(q_0,\rho_0)$, it holds that $\rho,\rho'$ are equivalent in the following sense: For each guard $g \in \{x_i < x_j \mid 1 \le i, j \le k\} \cup \{x_i = c \mid 1 \le i \le k, c_{min} \le c \le c_{max} \} \cup \{x_i < c_{min},x_i > c_{max} \mid 1 \le i \le k\}$, $\rho \models g$ iff $\rho' \models g$.

% $(q, g, \eta, q') \in \delta$, the guard $g$ implies one of the followings: $\cur < c_{min}$, $\cur = c$ for $c_{min} \le c \le c_{max}$, or $\cur > c_{max}$. 
%
%(3) \emph{Constant-partitioned}: For each $(q, g, \eta, q') \in \delta$, the guard $g$ implies one of the followings: $\cur < c_{min}$, $\cur = c$ for $c_{min} \le c \le c_{max}$, or $\cur > c_{max}$. 
\begin{proposition}\label{prop-snt-norm}
	From each SNT, an equivalent normalized SNT can be constructed in  exponential time w.r.t. the number of control variables. 
\end{proposition}
%
The construction can be found in the appendix. The idea of the construction is simple. To ensure the constructed SNT is well-defined, we record in the states the set of variables whose values are defined, and change the transitions and the output function accordingly
%by removing $q$ from the domain of $O$ if $O(q)$ contains some variable whose value is undefined (this information can be discovered from the states). 
%
To ensure the ``uniquely-valued'' and ``state-dominating'' constraints, we record in the states the equivalence relation and order relation between the control variables, as well as their relation with the constants from $[c_{min}, c_{max}]$, remove the duplicated values from control variables, and enforce that the guards in the transitions conform to the these relations recorded in the states.

% and modify the transitions accordingly. To ensure the ``state-dominating'' constraint, we replace every transition $(q, g, \eta, q')$ with the following set of transitions 
%$\{(q, g\wedge g', \eta, q') \mid g' \equiv \cur < c_{min}$, $\cur = c$ for $c_{min} \le c \le c_{max}$, \mbox{ or } $\cur > c_{max}\}$.


%!TEX root = main-cav.tex

\section{Decision procedure for the non-zero output problem}\label{sec:dec-snt}
%
We prove our main result, Theorem~\ref{thm:correctness}, by presenting a decision procedure for the non-zero output problem of SNTs. We fix a normalized SNT $\Ss = (Q,X,Y,\delta,q_0,O)$ such that $X=\{ x_1,\dots, x_k\}$ and $Y = \{y_1,\dots,y_l\}$. Due to space constraint, we only present a simplified version where the transition guards are constant-free and leave the procedure for the general case in the full version.
We define summaries of the computations of $\Ss$ on paths and cycles in Section~\ref{sec-sum}. We then present a decision procedure for the case that the transition graph of $\Ss$ is a \emph{generalized lasso} in Section~\ref{sec-glasso}. The transition graph of $\Ss$ is said to be a generalized lasso if it comprises a handle $H=q_0 q_1 \dots q_m$ and a collection of simple cycles $C_1,\dots,C_n$ such that $q_m$ is the unique state shared by each pair of distinct cycles from $\{C_1,\dots,C_n\}$. We will generalize the procedure to full SNTs in Section~\ref{sec-gflat}.

\begin{theorem}\label{thm:correctness}
The non-zero output problem of normalized SNTs can be decided in time exponential over the number of data variables and the maximum number of simple cycles in nontrivial SCCs of transition graphs.
\end{theorem}
\vspace{-2mm}

\begin{corollary}\label{cor:snt-dec-proc}
The commutativity problem of SNTs (resp. reducer programs) can be decided in time exponential over the numbers of control and data variables, and exponential over the maximum number of simple cycles in nontrivial SCCs (resp. doubly exponential over the number of branching statements in reducer programs). 
%On the other hand, the commutativity problem of reducer programs can be decided in time exponential over the number of control variables and the number of data variables, but .
\end{corollary}

\begin{remark}
Though the decision procedure for the commutativity problem of reducer programs has a complexity exponential over the number of control resp. data variables, and doubly exponential over the number of branching statements, we believe that the decision procedure could still be implemented to automatically analyze the programs in practice, in which these numbers are usually small. 
\end{remark}

\vspace{-4mm}
\subsection{Summarization of the computations on paths and cycles}\label{sec-sum}
\vspace{-1mm}

Suppose $P=p_0 \xrightarrow{(g_1,\eta_1)} p_1 \dots p_{n-1} \xrightarrow{(g_n,\eta_n)} p_{n}$ is a path of $\Ss$. We assume that the initial values of the control and data variables are represented by a symbolic valuation $\sval$ over $X \cup Y$. When $P$ is traversed in a run of $\Ss$ over a data word $w$,  the data value in a position of $w$ may have to be (un)equal to the initial value of some control variable or some other data value in $w$ that have been met before (enforced by the guards and assignments in $P$). Let $\sim$ denote the equivalence relation on $[n]$ induced by $P$ such that $i \sim j$ iff the guards and assignments on $P$ enforce that the data value in the $i$-th position of $w$ must be equal to that in the $j$-th position of $w$. Assuming that there are $r^{\circled{P}}$ equivalence classes of $\sim$, we use the variables $\vard^{\circled{P}}_1,\vard^{\circled{P}}_2,\dots, \vard^{\circled{P}}_{r^{\circled{P}}}$ to denote the data values met when traversing $P$, one for each equivalence class. Note here we use the superscript ${\circled{P}}$ to denote the fact that $r^{\circled{P}}$ (resp. $\vard^{\circled{P}}_1$, $\dots$) is associated with the path $P$.

\begin{proposition}\label{prop-sum-path}
Suppose that $P$ is a path and the initial values of $X \cup Y$ are represented by a symbolic valuation $\initval$. Then the values of $X \cup Y$ after traversing the path $P$ are specified by a symbolic valuation $\sumf^{(P,\initval)}$ satisfying the following conditions.
\begin{itemize}
\item The set of indices of $X$, i.e., $[k]$, is partitioned into $I^{\circled{P}}_{pe}$ and $I^{\circled{P}}_{tr}$, the indices of \emph{persistent} and \emph{transient} control variables, respectively. A control variable is persistent if its value has not been changed while traversing $P$, otherwise, it is transient.
\item For each $x_j\in X$ such that $j\in I^{\circled{P}}_{pe}$, $\sumf^{(P,\initval)}(x_j)=\sval(x_j)$.
%
\item  For each $x_j\in X$ such that $j\in I^{\circled{P}}_{tr}$,
$\sumf^{(P,\initval)}(x_j)=\vard^{\circled{P}}_{\pi^{\circled{P}}(j)}$, where $\pi^{\circled{P}}: I^{\circled{P}}_{tr} \rightarrow [r^{\circled{P}}]$ is an injective mapping from the index of a transient control variable to the index of the data value assigned to it.
% 
\item For each $y_j \in Y$, 
$
 \sumf^{(P,\initval)}(y_j)  =
 \cste^{\circled{P}}_{j} + 
 \cstl^{\circled{P}}_j \initval(y_j)  + 
  \sum\limits_{j'\in [k]}\csta^{\circled{P}}_{j,j'}\initval(x_{j'}) +
  \sum\limits_{j''\in [r^{\circled{P}}]}\cstb^{\circled{P}}_{j,j''} \vard^{\circled{P}}_{j''}$,
\hide{
\item For each $y_j \in Y$, 
\[
\small
\begin{array}{l}
\smallskip
\sumf^{(P,\initval)}(y_j)  = \\
\hspace{2mm} \cste^{\circled{P}}_{j} + \cstl^{\circled{P}}_j \initval(y_j)  + \csta^{\circled{P}}_{j,1} \initval(x_1) + \dots + \csta^{\circled{P}}_{j,k} \initval(x_k) +  \cstb^{\circled{P}}_{j,1} \vard^{\circled{P}}_1 +\dots + \cstb^{\circled{P}}_{j,r^{\circled{P}}} \vard^{\circled{P}}_{r^{\circled{P}}},
\end{array}
\]} 
where $\cste^{\circled{P}}_j,\cstl^{\circled{P}}_j, \csta^{\circled{P}}_{j,1},\dots,\csta^{\circled{P}}_{j,k}, \cstb^{\circled{P}}_{j,1},\dots,\cstb^{\circled{P}}_{j,r^{\circled{P}}}$ are integer constants such that $\cstl^{\circled{P}}_{j} \in \{0,1\}$ (as a result of the ``independently evolving and copyless'' constraint).  It can happen that $\cstl^{\circled{P}}_j =0$,  which means that $\initval(y_j)$ is irrelevant to $\sumf^{(P,\initval)}(y_j)$. Similarly for $\csta^{\circled{P}}_{j,1}=0$, and so on.
\end{itemize}
\end{proposition}
In Proposition~\ref{prop-sum-path}, the sets $I^{\circled{P}}_{pe}$, $I^{\circled{P}}_{tr}$, the mapping $\pi^{\circled{P}}$, and the constants $\cste^{\circled{P}}_j,\cstl^{\circled{P}}_j, \dots, \cstb^{\circled{P}}_{j,r^{\circled{P}}}$ only depend on $P$ and are independent of $\initval$. Due to the uniquely-valued constraint of normalized SNTs, $\pi^{\circled{P}}$ is injective, and the inverse function of $\pi^{\circled{P}}$, denoted $(\pi^{\circled{P}})^{-1}$, exists.

As a corollary of Proposition~\ref{prop-sum-path}, the following result demonstrates how to summarize the computations of $\Ss$ on the composition of two paths.

\begin{corollary}\label{cor-comp-two-paths}
Suppose that $P_1$ and $P_2$ are two paths in $\Ss$ such that the last state of $P_1$ is the first state of $P_2$. Moreover, let $\sumf^{(P_1, \initval)}$ (resp. $\sumf^{(P_2, \initval)}$) be the symbolic valuation summarizing the computation of $\Ss$ on $P_1$ (resp. $P_2$). Then the symbolic valuation summarizing the computation of $\Ss$ on $P_1 P_2$ is $\sumf^{(P_2,\ \sumf^{(P_1,\initval)})}$.
\end{corollary}

In order to get a better understanding of the relation between $\sumf^{(P_2,\ \sumf^{(P_1,\initval)})}$ and $(\sumf^{(P_1, \initval)},\sumf^{(P_2, \initval)})$, in the following, for each $y_j \in Y$, we obtain a more explicit form of the expression $\sumf^{(P_2,\ \sumf^{(P_1,\initval)})}(y_j)$, by unfolding therein the expression $\sumf^{(P_1,\initval)}$\medskip.
\resizebox{\hsize}{!}{
	$\begin{array}{rl}
	\medskip
	\sumf^{(P_2,\ \sumf^{(P_1,\initval)})}(y_j) = & 
	\left(\cste^{\circled{P_2}}_{j}+
	\cstl^{\circled{P_2}}_{j} \cste^{\circled{P_1}}_{j}\right)+ \left(\cstl^{\circled{P_2}}_{j} \cstl^{\circled{P_1}}_{j} \right) \initval(y_j)+ \sum \limits_{j' \in I^{\circled{P_1}}_{pe}} 
	\left(\csta^{\circled{P_2}}_{j,j'} +\cstl^{\circled{P_2}}_{j} \csta^{\circled{P_1}}_{j,j'}\right) \initval(x_{j'}) +\\
	\medskip
	& 
	\sum \limits_{j' \in  I^{\circled{P_1}}_{tr}} 
	\left(\cstl^{\circled{P_2}}_{j} \csta^{\circled{P_1}}_{j,j'} \right) \initval(x_{j'}) +
	\sum \limits_{j' \in \rng(\pi^{\circled{P_1}})} \left( \csta^{\circled{P_2}}_{j,(\pi^{\circled{P_1}})^{-1}(j')}+\cstl^{\circled{P_2}}_{j} \cstb^{\circled{P_1}}_{j,j'} \right) \vard^{\circled{P_1}}_{j'} + 
	 \\
	%
	\smallskip
	& 
	\sum \limits_{j' \in [r^{\circled{P_1}}]\setminus \rng(\pi^{\circled{P_1}})} \left( \cstl^{\circled{P_2}}_{j} \cstb^{\circled{P_1}}_{j,j'} \right) \vard^{\circled{P_1}}_{j'} +
	
	\sum \limits_{j'\in[r^{\circled{P_2}}]} \cstb^{\circled{P_2}}_{j,j'} \vard^{\circled{P_2}}_{j'}.
	\end{array}$
}\medskip\\
In the equation, $j'\in  I^{\circled{P_1}}_{pe}$ implies that $x_{j'}$ remains unchanged when traversing $P_1$, which means the initial value of $x_{j'}$ before traversing $P_2$ is still $\initval(x_{j'})$ and therefore we have the item $ (\csta^{\circled{P_2}}_{j,j'}) \initval(x_{j'})$. When $j' \in \rng(\pi^{\circled{P_1}})$, the initial value of $x_{(\pi^{\circled{P_1}})^{-1}(j')}$ before traversing $P_2$ is $\vard^{\circled{P_1}}_{j'}$ and therefore we have the item $( \csta^{\circled{P_2}}_{j,(\pi^{\circled{P_1}})^{-1}(j')}) \vard^{\circled{P_1}}_{j'}$.
For all $j'\in [k] = I^{\circled{P_1}}_{pe} \cup I^{\circled{P_1}}_{tr}$, we have the item $(\cstl^{\circled{P_2}}_{j} \csta^{\circled{P_1}}_{j,j'}) \initval(x_{j'})$, i.e. the coefficient of $\initval(x_{j'})$ in $\sumf^{(P_1, \initval)}$ multiplied by $\cstl^{\circled{P_2}}_{j}$. Moreover, for all $j'\in [r^{\circled{P_1}}] = \rng(\pi^{\circled{P_1}}) \cup ([r^{\circled{P_1}}] \setminus \rng(\pi^{\circled{P_1}}))$, we have 
the item $( \cstl^{\circled{P_2}}_{j} \cstb^{\circled{P_1}}_{j,j'}) \vard^{\circled{P_1}}_{j'}$, i.e. the coefficient of $\vard^{\circled{P_1}}_{j'}$ in $\sumf^{(P_1, \initval)}$ multiplied by $\cstl^{\circled{P_2}}_{j}$.

In the following, by utilizing Proposition~\ref{prop-sum-path} and Corollary~\ref{cor-comp-two-paths}, for each path $C^{\ell}$ which is obtained by iterating a cycle $C$ for $\ell$ times, we illustrate how $\sumf^{(C^\ell,\initval)}$ is related to $\sumf^{(C, \initval)}$ and $\ell$. For convenience, we call $\ell$ a \emph{loop counter variable}.

\begin{proposition}\label{prop-sum-cycle}
Suppose that $C$ is a cycle and $P=C^{\ell}$ such that $\ell \ge 2$. Then the symbolic valuation $\sumf^{(C^\ell,\initval)}$ to summarize the computation of $\Ss$ on $P$ is as follows,\medskip\\
\resizebox{\hsize}{!}{
$\begin{array}{l c l}
\sumf^{(C^\ell,\initval)}(y_j)  & = & 
\left(1 + \cstl^{\circled{C}}_{j} + \dots +(\cstl^{\circled{C}}_{j})^{\ell - 1} \right)\cste^{\circled{C}}_{j} + (\cstl^{\circled{C}}_{j})^\ell \initval(y_j) + \smallskip\\
%
& & \sum \limits_{j' \in I^{\circled{C}}_{pe}} \left(1+\cstl^{\circled{C}}_{j} + \dots +(\cstl^{\circled{C}}_{j})^{\ell - 1} \right) \csta^{\circled{C}}_{j,j'}\initval(x_{j'}) +  \sum \limits_{j' \in I^{\circled{C}}_{tr}}  (\cstl^{\circled{C}}_{j})^{\ell - 1} \csta^{\circled{C}}_{j,j'} \initval(x_{j'}) +  \\
%
& & \sum \limits_{j' \in \rng(\pi^{\circled{C}})} \sum \limits_{s\in[\ell -1]}
\left(  \csta^{\circled{C}}_{j, (\pi^{\circled{C}})^{-1}(j')} +(\cstl^{\circled{C}}_{j})\cstb^{\circled{C}}_{j,j'} \right)
(\cstl^{\circled{C}}_{j})^{\ell-s-1}
\vard^{\circled{C , s}}_{j'} +\\
%
& & \sum \limits_{j' \in [r^{\circled{C}}] \setminus \rng(\pi^{\circled{C}})}\sum \limits_{s\in[\ell -1]} \left((\cstl^{\circled{C}}_{j})^{\ell - s} \cstb^{\circled{C}}_{j,j'} \right) \vard^{\circled{C , s}}_{j'} + 
\sum \limits_{j' \in [r^{\circled{C}}] }  
 \cstb^{\circled{C}}_{j, j'} \vard^{\circled{C , \ell}}_{j'},
\end{array} 
$}\medskip\\
where the variables $\vard^{\circled{C , s}}_{1},\dots, \vard^{\circled{C ,s}}_{r^{\circled{C}}}$ for $s\in [\ell]$
 represent the data values introduced when traversing $C$ for the $s$-th time.
\end{proposition}

From Proposition~\ref{prop-sum-cycle} and the fact that $\cstl_{j} \in \{0, 1\}$, we have the following observation.
\begin{itemize}
\item If $\cstl^{\circled{C}}_{j}=0$, then\medskip\\
\resizebox{0.9\hsize}{!}{$
\sumf^{(C^\ell,\initval)}(y_j)   =  \cste^{\circled{C}}_{j} +  \sum \limits_{j' \in I^{\circled{C}}_{pe}} \csta^{\circled{C}}_{j,j'} \initval(x_{j'}) +
\sum \limits_{j'  \in \rng(\pi^{\circled{C}})} \csta^{\circled{C}}_{j, (\pi^{\circled{C}})^{-1}(j')}\ \vard^{\circled{C , \ell  -  1}}_{j'} + \sum \limits_{j' \in [r^{\circled{C}}] }  
\cstb^{\circled{C}}_{j, j'} \vard^{\circled{C , \ell}}_{j'}.$}


\item If $\cstl^{\circled{C}}_{j}=1$, then\medskip\\
\resizebox{0.95\hsize}{!}{$
\begin{array}{l }
\sumf^{(C^\ell,\initval)}(y_j)  =    \ell \cste^{\circled{C}}_{j}  + \initval(y_j) +   \sum  \limits_{j' \in I^{\circled{C}}_{pe}} \ell \csta^{\circled{C}}_{j,j'}  \initval(x_{j'}) + 
\sum \limits_{j' \in I^{\circled{C}}_{tr}} \csta^{\circled{C}}_{j,j'} \initval(x_{j'}) +  \smallskip\\
\sum \limits_{j' \in \rng(\pi^{\circled{C}})} \sum \limits_{s\in[\ell -1]}
\left(\csta^{\circled{C}}_{j, (\pi^{\circled{C}})^{-1}(j')} + \cstb^{\circled{C}}_{j,j'} \right) \vard^{\circled{C , s}}_{j'} + \sum \limits_{j' \in [r^{\circled{C}}]  \setminus \rng(\pi^{\circled{C}}) }\sum \limits_{s\in[\ell -1]} 
\cstb^{\circled{C}}_{j,j'} \vard^{\circled{C , s}}_{j'} + \sum \limits_{j' \in [r^{\circled{C}}] }  
\cstb^{\circled{C}}_{j, j'} \vard^{\circled{C , \ell}}_{j'}.
\end{array}
$}
%



\hide{
\item If $\alpha^{\circled{C}}_{j,1}=-1$ and $\ell$ is even, then
\[
\begin{array}{l c l}
\chi^{\circled{C}}_{\ell}(y_j)  & = &  o_j + \sum \limits_{j'\le k, \pi_C(j') \neq j'} (-\beta^{\circled{C}}_{j,j'}) d^{(0)}_{j'} +  \\
%
& & \sum \limits_{j' \le r_C ,  j'+k \in \rng(\pi_C)} ( \beta^{\circled{C}}_{j, \pi_C^{-1}(j'+k)} - \gamma^{\circled{C}}_{j,j'}) d^{\circled{C , 1}}_{j'} + \\
%
& & \sum \limits_{j' \le r_C ,   j'+k \not \in \rng(\pi_C)} (-\gamma^{\circled{C}}_{j,j'}) d^{\circled{C , 1}}_{j'} + \dots + \\
%
& & \sum \limits_{j' \le r_C ,  j'+k \in \rng(\pi_C)} (\beta^{\circled{C}}_{j, \pi_C^{-1}(j'+k)}-\gamma^{\circled{C}}_{j,j'}) d^{\circled{C , \ell  -  1}}_{j'} + \\
%
& & \sum \limits_{j' \le r_C ,   j'+k \not \in \rng(\pi_C)} (-\gamma^{\circled{C}}_{j,j'}) d^{\circled{C , \ell  -  1}}_{j'} + \gamma^{\circled{C}}_{j, 1} d^{\circled{C , \ell}}_{1} + \dots + \gamma^{\circled{C}}_{j,r_C} d^{\circled{C , \ell}}_{r_C}.
\end{array} 
\]
\item If $\alpha_{j,1}=-1$ and $\ell$ is odd, then
\[
\begin{array}{l c l}
\smallskip
\chi^{\circled{C}}_{\ell}(y_j)  & = &  \alpha^{\circled{C}}_{j,0} - o_j + \sum \limits_{j' \le k, \pi_C(j')=j'} \beta^{\circled{C}}_{j,j'} d^{(0)}_{j'} +  \sum \limits_{j'\le k, \pi_C(j') \neq j'}  \beta^{\circled{C}}_{j,j'} d^{(0)}_{j'} +  \\
%
& & \sum \limits_{j' \le r_C ,  j'+k \in \rng(\pi_C)} ( -\beta^{\circled{C}}_{j, \pi_C^{-1}(j'+k)} +\gamma^{\circled{C}}_{j,j'}) d^{\circled{C , 1}}_{j'} + \\
%
& & \sum \limits_{j' \le r_C ,   j'+k \not \in \rng(\pi_C)} \gamma^{\circled{C}}_{j,j'} d^{\circled{C , 1}}_{j'} + \dots + \\
%
& & \sum \limits_{j' \le r_C ,  j'+k \in \rng(\pi_C)} (\beta^{\circled{C}}_{j, \pi_C^{-1}(j'+k)}-\gamma^{\circled{C}}_{j,j'}) d^{\circled{C , \ell  -  1}}_{j'} + \\
%
& & \sum \limits_{j' \le r_C ,   j'+k \not \in \rng(\pi_C)} (-\gamma^{\circled{C}}_{j,j'}) d^{\circled{C , \ell  -  1}}_{j'} + \gamma^{\circled{C}}_{j, 1} d^{\circled{C , \ell}}_{1} + \dots + \gamma^{\circled{C}}_{j,r_C} d^{\circled{C , \ell}}_{r_C}.
\end{array} 
\]
}
\end{itemize}

%
%From the analysis above, we observe that in $\chi^{\circled{C}}_\ell(y_j)$, 
%\begin{itemize}
%\item the constant coefficient is either $\alpha^{\circled{C}}_{j,0}$, or $\alpha^{\circled{C}}_{j,0} \ell$, 
%
%\item the coefficient of $o_j$ is $0$, or $1$, 
%
%\item for each data value $d^{(0)}_{j'}$, the coefficient of $d^{(0)}_{j'}$ is either $\beta^{\circled{C}}_{j,j'}$, or $0$, or $\beta^{\circled{C}}_{j,j'} \ell$,
%
%\item for each data value $d^{(C , i)}_{j'}$ with $i \ge 1$, the coefficient of $d^{(C , i)}_{j'}$ is either $0$, or $\beta^{\circled{C}}_{j, \pi_C^{-1}(j'+k)}$, or $\beta^{\circled{C}}_{j, \pi_C^{-1}(j'+k)}+\gamma^{\circled{C}}_{j,j'}$, or $\gamma^{\circled{C}}_{j,j'}$.
%\end{itemize}

\vspace{-0.5cm}
\subsection{Decision procedure for generalized lassos}\label{sec-glasso}
\vspace{-0.2cm}
%
\yfc{newly added}
Observe the pattern of the computation summary in Proposition~\ref{prop-sum-cycle}.
The coefficients of the loop counter variable $\ell$ can be none-zero when $\cstl^{\circled{C}}_{j}=1$. The none-zero coefficients may propagate to the output expression.  In such a case, due to the ``transition-enablement'' constraint of a normalized SNT (for any sequence of transitions, one can find a corresponding run), one can pick a run corresponding to a very large $\ell$ such that it dominates the value of the output expression. Therefore the output of the SNT can be made none-zero. 
In the decision procedure we are going to present, we first check if the handle of the generalized lasso produces none-zero output in Step I.
We then check if the coefficient of $\ell$ is none-zero in the output expression in Step II. If the coefficient of $\ell$ is zero in the output expression, we show in Step III that the testing of output none-zeroness can be reduced to a finite state reachability problem and thus can be easily decided.

Before presenting the decision procedure for generalized lassos, we introduce some notations.
Let $e$ be an expression consists of symbolic values $\initval(z)$ for $z\in X\cup Y$ and data variables $\vard_1, \dots, \vard_{s_2}$. More specifically, let $e:=\mu_0 + \mu_1 \initval(z_1) +\dots + \mu_{s_1} \initval(z_{s_1}) + \xi_1 \vard_1 + \dots + \xi_{s_2} \vard_{s_2}$,
such that $\mu_0,\mu_1,\dots,\mu_{s_1}, \xi_1,\dots,\xi_{s_2}$ are expressions containing only constants and loop counter variables.
Then we call $\mu_0$ as the \emph{constant atom}, $\mu_i \initval(z_i)$ the $\initval(z_i)$-atom for $i\in[s_1]$, and $\xi_j \vard_j$ the $\vard_j$-atom for $j\in[s_2]$ of the expression $e$. Moreover, $\mu_1, \dots, \mu_{s_1}, \xi_1,\dots, \xi_{s_2}$ are called the \emph{coefficients} of these atoms. A non-constant atom is said to be \emph{nontrivial} if its coefficient is \emph{not} identical to zero.

In the rest of this subsection, we assume that the transition graph of $\Ss$ comprises a handle $H=q_0 q_1 \dots q_m$ and a collection of simple cycles $C_1,\dots,C_n$ such that $q_m$ is the unique state shared by each pair of distinct cycles from $\{C_1,\dots,C_n\}$. Moreover, without loss of generality, we assume that $O(q_m) = a_0 + a_1 x_1 + \dots + a_k x_k + b_1 y_1 + \dots + b_l y_l$, and $O(q)$ is undefined for all the other states $q$.

A \emph{cycle scheme} $\schm$ is a path $C_{i_1}^{\ell_1} C_{i_2}^{\ell_2} \dots C_{i_t}^{\ell_t}$ such that $i_1,\dots,i_t \in [n]$, $\ell_1,\dots, \ell_t \ge 1$, and for each $j\in [t-1]$, $i_j \neq i_{j+1}$. Intuitively, $\schm$ is a path obtained by iterating $C_{i_1}$ for $\ell_1$ times, $C_{i_2}$ for $\ell_2$ times, and so on. From Proposition~\ref{prop-sum-cycle} and Corollary~\ref{cor-comp-two-paths}, a symbolic valuation $\sumf^{(\schm,\initval)}$ can be constructed 
to summarize the computation of $\Ss$ on $\schm$. 


\begin{lemma}\label{prop-cycle-schm}
Suppose $\schm=C_{i_1}^{\ell_1} C_{i_2}^{\ell_2} \dots C_{i_t}^{\ell_t}$ is a cycle scheme, and $\initval$ is a symbolic valuation representing the initial values of the control and data variables. 
For all $j' \in  I^{\circled{C_{i_{1}}}}_{pe}$, let $r_{j'}$ be the largest number $r \in [t]$ such that $j'\in\bigcap_{s\in[r]} I^{\circled{C_{i_{s}}}}_{pe}$, i.e., $x_{j'}$ remains persistent when traversing $C_{i_1}^{\ell_1} C_{i_2}^{\ell_2} \dots C_{i_{r_{j'}}}^{\ell_{r_{j'}}}$.
Then for each $j\in [l]$ and $j' \in  I^{\circled{C_{i_{1}}}}_{pe}$, the coefficient of the $\initval(x_{j'})$-atom in $\sumf^{(\schm,\initval)}(y_j)$ is 
\begin{center}
\resizebox{0.8\hsize}{!}{
$e+\sum\limits_{s_1\in[r_{j'}]}  
\left(1+\lambda^{\circled{C_{i_{s_1}}}}_{j} + \dots + (\lambda^{\circled{C_{i_{s_1}}}}_{j})^{\ell_{s_1}-1} \right) \csta^{\circled{C_{i_{s_1}}}}_{j,j'}\prod\limits_{{s_2}\in[{s_1}+1,t]}\left(\lambda^{\circled{C_{i_{s_2}}}}_{j}\right)^{\ell_{s_2}}$},
\end{center}
where (1) $e\!=\!0$ when $r_{j'}\!=\!t$ and (2) $e=(\lambda^{\circled{C_{i_s}}}_{j})^{\ell_s-1} \csta^{\circled{C_{i_{s}}}}_{j,j'}$ with $s=r_{j'}+1$ when $r_{j'}<t$.\\
The constant atom of $\sumf^{(\schm,\initval)}(y_j)$ is 
\begin{center}
\resizebox{0.7\hsize}{!}{$
\sum\limits_{{s_1}\in[t]}
\left(1+\lambda^{\circled{C_{i_{s_1}}}}_{j} + \dots + (\lambda^{\circled{C_{i_{s_1}}}}_{j})^{\ell_{s_1}-1} \right)
\cste^{\circled{C_{i_{s_1}}}}_{j} 
\prod\limits_{{s_2}\in[{s_1}+1,t]}\left(\lambda^{\circled{C_{i_{s_2}}}}_{j}\right)^{\ell_{s_2}}$}
\end{center}
Moreover, for all $j\!\in\! [l]$, in $\sumf^{(\schm,\initval)}(y_j)$, only the constant atom and the coefficients of the $\initval(x_{j'})$-atoms with $j' \!\in\!I^{\circled{C_{i_{1}}}}_{pe}$ contain a subexpression of the form $ \mu_\schm \ell_1$ for some~$\mu_\schm\in \intnum$.
\end{lemma}
Notice that above, $\lambda^{\circled{C_{i_{s_1}}}}_j\in\{0,1\}$ for $j\in[l]$ and $s_1\in [t]$. Hence the value of $(1+\lambda^{\circled{C_{i_{s_1}}}}_{j} + \dots + (\lambda^{\circled{C_{i_{s_1}}}}_{j})^{\ell_{s_1}-1} )$ can only be $1$ or $\ell_{s_1}$ and $\left(\lambda^{\circled{C_{i_{s_2}}}}_{j}\right)^{\ell_{s_2}}\in\{0,1\}$.
Hence both the constant atom and the coefficient of the $\initval(x_{j'})$-atom with $j'\in I^{\circled{C_{i_{1}}}}_{pe}$ can be rewritten to the form of $c_0+c_1\ell_1+c_2\ell_2+\dots+c_t\ell_t$ for $c_0\ldots c_t\in \intnum$. Note that some of $c_0\ldots c_t$ might be zero.




We are ready to present the decision procedure. By the ``well-defined'' and ``uniquely-valued'' constraints of normalized SNTs, without loss of generality, we assume that $I^{\circled{H}}_{tr}=[k]$, that is, after traversing $H$, the values of all control variables become defined.
Under the assumption, for all $j' \in [k]$, the symbolic value of $\sumf^{(H,\sval_\bot)}(x_{j'})=\vard^{\circled{H}}_{\pi^{\circled{H}}(j')}$\smallskip\\
\framebox[\textwidth]{
\begin{minipage}{0.95\textwidth}
\noindent {\bf Step I}. Decide whether $\eval{O(q_m)}{\sumf^{(H,\sval_\bot)}}$ is not identical to zero.
This can be done by checking if the constant-atom or the coefficient of some non-constant atom of $\eval{O(q_m)}{\sumf^{(H,\sval_\bot)}}$ is not identical to zero.
If the answer is yes, then the decision procedure terminates and returns the answer $\ltrue$. Otherwise, go to Step II.
\end{minipage}
}\bigskip

The goal of Step II is either showing that in $f=\eval{O(q_m)}{\sumf^{(\schm,\sumf^{(H,\sval_\bot)})}}$, all subexpressions containing the cycle counter variables are identical to zero and hence can be ignored or showing that $f$ is not identical to zero. Let $\schm=C_{i_1}^{\ell_1} C_{i_2}^{\ell_2} \dots C_{i_t}^{\ell_t}$ be a cycle scheme. From Lemma~\ref{prop-cycle-schm}, for each $j'\in I^{\circled{C_{i_1}}}_{pe}$ and symbolic valuation $\sval$, the only subexpression containing $\ell_1$ in the coefficient of $\initval(x_{j'})$-atom of $\eval{O(q_m)}{\sumf^{(\schm,\initval)}}$ is
\begin{center}
	\resizebox{0.7\hsize}{!}{$
\sum \limits_{1 \le j \le l} 
b_j \left((\cstl^{\circled{C_{i_2}}}_{j})^{\ell_2} \dots (\cstl^{\circled{C_{i_t}}}_{j})^{\ell_t}\right) 
\left(1+\cstl^{\circled{C_{i_1}}}_{j} + \dots + (\cstl^{\circled{C_{i_1}}}_{j})^{\ell_1-1} \right) \csta^{\circled{C_{i_1}}}_{j,j'}.
\hspace{4mm} (\ast)
$}
\end{center}
Since $\cstl^{\circled{C_{i_1}}}_{j}, \cstl^{\circled{C_{i_2}}}_{j}, \dots, \cstl^{\circled{C_{i_t}}}_{j} \in \{0, 1\}$, the expression $(\ast)$  can be rewritten as  
 $\mu_{\schm, (i_1,j')} \ell_1 + \nu_{\schm, (i_1,j')}$ for some integer constants $\mu_{\schm, (i_1,j')}$ and $\nu_{\schm, (i_1,j')}$. 
 
The only subexpression containing $\ell_1$ in the constant atom of  $\eval{O(q_m)}{\sumf^{(\schm,\initval)}}$ is
\begin{center}
	\resizebox{0.7\hsize}{!}{$
\sum \limits_{1 \le j \le l} b_j
\begin{array}{l}
 \left((\lambda^{\circled{C_{i_2}}}_{j})^{\ell_2} \dots (\lambda^{\circled{C_{i_t}}}_{j})^{\ell_t}\right)
\left(1+\lambda^{\circled{C_{i_1}}}_{j} + \dots + (\lambda^{\circled{C_{i_1}}}_{j})^{\ell_1-1} \right) \cste^{\circled{C_{i_1}}}_{j}. \hspace{2mm} (\ast\ast)
\end{array}
$}
\end{center}
%
The expression $(\ast\ast)$ can be rewritten as $\mu_{\schm,(i_1,0)} \ell_1 + \nu_{\schm,(i_1,0)}$ for some integer constants $\mu_{\schm, (i_1,0)}$ and $\nu_{\schm, (i_1,0)}$. If $\mu_{\schm,(i_1,0)}=\mu_{\schm,(i_1,j')}=0$ for all $j' \in I^{\circled{C_{i_1}}}_{pe}$, then we can ignore all subexpressions containing the cycle counter variable $\ell_1$ in   $\eval{O(q_m)}{\sumf^{(\schm,\initval)}}$, i.e., the subexpressions $\mu_{\schm,(i_1,0)}\ell_1$ and $\mu_{\schm,(i_1,j')}\ell_1$ for all $j' \in I^{\circled{C_{i_1}}}_{pe}$.\smallskip\\
\framebox[\textwidth]{
	\begin{minipage}{0.95\textwidth}
		\noindent {\bf Step II}. For each $i_1 \in [n]$, check all cycle scheme $\schm=C_{i_1}^{\ell_1} C_{i_2} \dots C_{i_t}$ such that $i_2,\dots,i_t$ are mutually distinct. There are only finitely many this kind of cycle schemes. If 
		one of the following constraints is satisfied, then return $\ltrue$. \\(1) There is $j' \in  I^{\circled{C_{i_1}}}_{pe}$ such that $\mu_{\schm,(i_1,j')} \neq 0$. (2) $\mu_{\schm,(i_1,0)} \neq 0$.
		%
		If the decision procedure has not returned yet, then go to Step III.
	\end{minipage}
}\smallskip\\
If there exists $j' \in I^{\circled{C_{i_1}}}_{pe}$ such that $\mu_{\schm,(i_1,j')} \neq 0$, then we let $\vard^{\circled{H}}_{\pi^{\circled{H}}(j')} \neq 0$ and $\ell_1$ be arbitrarily large, so that the coefficient of the  $\vard^{\circled{H}}_{\pi^{\circled{H}}(j')}$-atom in $\eval{O(q_m)}{\sumf^{(\schm,\sumf^{(H,\sval_\bot)})}}$, which includes the expression $\mu_{\schm, (i_1,j')} \ell_1 + \nu_{\schm, (i_1,j')}$, dominates $\eval{O(q_m)}{\sumf^{(\schm,\sumf^{(H,\sval_\bot)})}}$. This is sufficient to make $\eval{O(q_m)}{\sumf^{(\schm,\sumf^{(H,\sval_\bot)})}}$ non-zero. Similarly, if $\mu_{\schm,(i_1,0)} \neq 0$, then we can let $\ell_1$ arbitrarily large to make the expression $\eval{O(q_m)}{\sumf^{(\schm,\sumf^{(H,\sval_\bot)})}}$ non-zero.
Similar arguments can be applied for $\ell_2\dots\ell_n$.

If Step II does not return $\ltrue$, we show below that for all cycle schemes $\schm_1=C_{i_1}^{\ell_1} C_{i_2}^{\ell_2} \dots C_{i_{s_1}}^{\ell_{s_1}}$ with $i_1,i_2,\dots,i_{s_1} \in [n]$, all subexpressions containing cycle counter variables in $\eval{O(q_m)}{\sumf^{(\schm,\initval)}}$ are identical to zero and hence can be removed. Let ${i'_2} \dots {i'_{s_2}}$ be the sequence obtained from $i_2 \dots i_{s,1}$ by keeping jus tone copy for each duplicated index therein.  
In Step II we already checked a cycle scheme $\schm_2=C_{i_1}^{\ell_1} C_{i'_2} \dots C_{i'_{s_2}}$. Step II guarantees that all subexpressions containing $\ell_1$ in 
$\eval{O(q_m)}{\sumf^{(\schm_2,\initval)}}$ are identical to zero and hence can be removed.
Because for all $j\in[l]$, $\cstl^{^{\circled{C_1}}}_j, \dots, \cstl^{^{\circled{C_n}}}_j \in \{0,1\}$,   $(\lambda^{\circled{C_{i_2}}}_{j})^{\ell_2} \dots (\lambda^{\circled{C_{i_{s_1}}}}_{j})^{\ell_{s_1}} = \lambda^{\circled{C_{i'_2}}}_{j} \dots \lambda^{\circled{C_{i'_{s_2}}}}_{j}$. We proved that the $(\ast)$ and $(\ast\ast)$ style expressions are equivalent in both $\schm_1$ and $\schm_2$.
Hence we can also remove all subexpressions containing $\ell_1$ from  $\eval{O(q_m)}{\sumf^{(\schm_1,\initval)}}$, without affecting its value.
Those subexpressions containing $\ell_2$ can also be removed by considering the cycle scheme $\schm_3=C_{i_2}^{\ell_2} C_{i''_3} \dots C_{i''_{s_3}}$ and applying a similar reasoning, where the sequence ${i''_3} \dots {i''_{s_3}}$ is obtained from ${i_3} \dots  i_{s_1}$, similarly to the construction of ${i'_2} \dots {i'_{s_2}}$ from $i_2 \dots i_{s,1}$. The same applies to all other cycle counter variables $\ell_3,\dots,\ell_{s_1}$.
We use the notation ${\sumf^{(\schm,\initval)}}^-(y_j)$ to denote the expression obtained by removing from the constant atom and coefficients of the non-constant atoms of $\sumf^{(\schm,\initval)}(y_j)$ all subexpressions containing cycle counter variables, for all $y_j \in Y$. 

\begin{lemma}\label{prop-bnd-domain-1}
	Suppose that the decision procedure has not returned $\ltrue$ after Step~II. For each cycle scheme $\schm$, let $f=\eval{O(q_m)}{\sumf^{(\schm, \sumf^{(H,\sval_\bot)})}}$ and $f'=\eval{O(q_m)}{{\sumf^{(\schm, \sumf^{(H,\sval_\bot)})}}^-}$. For all valuation $\rho$, $\eval{f}{\rho}\neq 0$ iff $\eval{f'}{\rho} \neq 0$.
\end{lemma}





\begin{lemma}\label{prop-bnd-domain-2}
Suppose that the decision procedure has not returned yet after Step II. 
For all cycle scheme $\schm$ and $y_j \in Y$, the constant atom and the coefficients of all non-constant atoms in ${\sumf^{(\schm, \sumf^{(H,\initval_\bot)})}}^-(y_j)$ are from a finite set $U \subset \intnum$ comprising \\ (1)
the constant atom and the coefficients of the non-constant atoms in the expression ${\sumf^{(C^{\ell_i}_{i}, \sumf^{(H,\initval_\bot)})}}^-(y_j)$ for $i\in [n]$ and $\ell_i \in \{1,2\}$,\smallskip\\(2) the numbers $\csta^{\circled{C_{s_2}}}_{j,j'} + \cstb^{\circled{C_{s_1}}}_{j,\pi^{\circled{C_{s_1}}}(j')}$ and $\csta^{\circled{C_{s_1}}}_{j, j''} + \csta^{\circled{C_{s_2}}}_{j,j''}$, where  $s_1,s_2 \in [n], j\in[l],j' \in I^{\circled{C_{s_1}}}_{tr} \cap I^{\circled{C_{s_2}}}_{tr},  j'' \in [k]$. 

\end{lemma}

For each cycle scheme $\schm$, an abstraction of ${\sumf^{(\schm, \sumf^{(H,\initval_\bot)})}}^-$, denoted by $\abs(\schm)$,  is the union of the following three sets:
(1)~constant atom: $\{(0, ( {\cste^{(\schm)}_{1}}^-,\dots, {\cste^{(\schm)}_l}^-))\}$. 
(2)~control variable atom: $\{(j', (c_{j',1},\dots, c_{j', l})) \mid j' \in [k]\}$, where $c_{j', j}$ is the coefficient of the ${\sumf^{(\schm,\sumf^{(H,\sval_\bot)})}}^-(x_{j'})$-atom in ${\sumf^{(\schm,\sumf^{(H,\sval_\bot)})}}^-(y_{j})$ for $j\in[l]$. (3)~data variable atom: $\{(k+1, (c_1,\dots,c_l))\}$, where $(c_1,\dots,c_l) \in U^l$ is the coefficients of the $\vard'$-atom in ${(\sumf^{(\schm,\sumf^{(H,\sval_\bot)})}}^-(y_{j})$ for all $j \in [l]$ and $\vard'\not\in \{{\sumf^{(\schm,\sumf^{(H,\sval_\bot)})}}^-(x_{j'})\mid x_{j'}\in X\}$.
Let $\mathscr{A}=\bigcup \{\abs(\schm) \mid \schm \mbox{ a cycle scheme}\}$. Then $\mathscr{A}$ can be constructed as follows. We first compute $\abs(HC_1), \ldots \abs(HC_n)$ and then compute the next abstract elements from them w.r.t. $C_1\ldots C_n$ until reached a fixed point.\medskip\\
\framebox[\textwidth]{
	\begin{minipage}{0.95\textwidth}
		\noindent {\bf Step III} We first construct the set $\mathscr{A}$ and then. 
		\begin{enumerate}
			\item Check whether there is $(0,(c_{0,1},\dots,c_{0,l})) \in \mathscr{A}$ such that $a_0+b_1 c_{0,1}+\dots + b_l c_{0,l} \neq 0$. If the answer is yes, then return $\ltrue$.
			%
			\item Check whether there are $j' \in [k]$ and $(j', (c_{j',1},\dots,c_{j',l})) \in \mathscr{A}$ such that $a_{j'} + b_1 c_{j',1} + \dots + b_l c_{j',l} \neq 0$. If the answer is yes, then return $\ltrue$. 
			%
			\item Check whether there is $(k+1,(c_1,\dots,c_l)) \in \mathscr{A}$ such that $b_1 c_1 + \dots + b_l c_l \neq 0$. If the answer is yes, then return $\ltrue$. 
		\end{enumerate}
		If the decision procedure has not returned yet, return $\lfalse$.
	\end{minipage}
}\smallskip\\

\vspace{-0.5cm}
\subsection{Decision procedure for SNTs without Constants}\label{sec-gflat}
\vspace{-0.3cm}

We generalize the decision procedure for the case that the transition graphs of the SNTs are generalized lassos to the full class of SNTs.
We first define a \emph{generalized multi-lasso} as a sequence $\gmlasso= H_1 (C_{1,1},\dots,C_{1,n_1}) H_2 (C_{2,1},\dots,C_{2,n_2}) \dots H_r (C_{r,1},\dots, C_{r, n_r})$ s.t. (1) for each $s\in[r]$, $H_s = q_{s,1} \dots q_{s, m_s}$ and $H_s (C_{s,1},\dots,C_{s, n_s})$ is a generalized lasso, (2) for $1 \leq s< s' \leq r$, $H_s (C_{s,1},\dots,C_{s, n_s})$ and $H_{s'} (C_{s', 1},\dots,C_{s', n_{s'}})$ are state-disjoint, except the case that when $s'=s+1$, $q_{s, m_s}=q_{s',1}$, and (3) $q_{1,1}=q_0$.

Since the transition graph of $\Ss$ can be seen as a collection of generalized multi-lassos, in the following, we shall present the decision procedure by showing how to decide the non-zero output problem for generalized multi-lassos. 

We fix a generalized multi-lasso

\smallskip
\hspace{8mm} $\gmlasso= H_1 (C_{1,1},\dots,C_{1,n_1}) H_2 (C_{2,1},\dots,C_{2,n_2}) \dots H_r (C_{r,1},\dots, C_{r, n_r})$.

\smallskip
\noindent Without loss of generality, we assume that $O(q_{r,m_r})=a_0+a_1 x_1 + \dots + a_k x_k + b_1 y_1  + \dots + b_l y_l$ and $O(q')$ is undefined for every other state $q'$ in $\gmlasso$.\smallskip\\
\framebox[\textwidth]{
	\begin{minipage}{0.95\textwidth}
		\noindent {\bf Step I$'$}. We do the same analysis as in Step I for the path $H_1\dots H_r$.
	\end{minipage}
}\smallskip

Let $s\in [2,r]$. In order to analyze the set of cycles $\Cc=\{C_{s-1,1},\dots,C_{s-1,n_{s-1}}\}$, below we show how to summarize effect of the path $H_s\dots H_r$ to $O(q_{s-1, m_{s-1}})$, which is shared by all those cycles in $\Cc$.
Suppose that $\eval{O(q_{r,m_r})}{\sumf^{(H_s\dots H_{r}, \initval)}}=a_{s, 0}+a_{s, 1} \initval(x_1)+ \dots + a_{s, k} \initval(x_k) + b_{s,1} \initval(y_1) + \dots + b_{s, l} \initval(y_l)+e$, where $e$ is a linear combination of the data variables that represent the data values introduced when traversing $H_s\dots H_r$. 
%the constant atom is $a_{s,0}$, the coefficient of the $\initval(x_j)$-atom is $a_{s, j}$ for each $j \in [k]$, and the coefficient of the $\initval(y_{j'})$-atom is $b_{s, j'}$ for each $j' \in [l]$. 
Then we change the output function and let
$O(q_{s-1, m_{s-1}}):=a_{s, 0}+a_{s, 1} x_1 + \dots + a_{s, k} x_k + b_{s,1} y_1 + \dots + b_{s, l} y_l$.\smallskip\\
\framebox[\textwidth]{
	\begin{minipage}{0.95\textwidth}
\noindent {\bf Step II$'$}.  For each $s\in [r]$ and $s'\in [n_s]$, we check each cycle scheme $\schm = C^{\ell_1}_{s,s'} C_{i_{2}} \dots C_{i_{t}}$ such that $C_{i_{2}} \dots C_{i_{t}}\in \{C_{s, 1}, \dots, C_{s,n_s},\dots, C_{r,1}, \dots, C_{r,n_r}\}$ and $C_{i_{2}} \dots C_{i_{t}}$ are mutually distinct by performing an analysis of the expression $\eval{ O(q_{s, m_{s}})} {\sumf^{(\schm,\sumf^{(H_1 \dots H_{s}, \initval)} ) } }$, in a way similar to Step II. If the decision procedure does not return during the analysis, then go to Step III$'$.
	\end{minipage}
}\smallskip

Intuitively, in Step II$'$, during the analysis of $\eval{ O(q_{s, m_{s}})} {\sumf^{(\schm,\sumf^{(H_1 \dots H_{s}, \initval)} ) } }$, the effect of the paths $H_{s+1},  \dots,  H_r$ and the cycles $C_{i_{2}}, \dots, C_{i_{t}}$ is described by the expressions $\cstl^{\circled{H_{s+1}}}_j \dots \cstl^{\circled{H_{r}}}_j  \cstl^{\circled{C_{i_{2}}}}_j \dots \cstl^{\circled{C_{i_{t}}}}_j $ for $j \in [l]$. Since $O(q_{s, m_{s}})$ has already taken into consideration the expressions $\cstl^{\circled{H_{s+1}}}_j \dots \cstl^{\circled{H_{r}}}_j$ for $j \in [l]$, conceptually, in Step II$'$, we can do the analysis as if we have a generalized lasso where the handle is $H_1\dots H_s$ and the collection of cycles is $\{C_{s,1},\dots, C_{s,n_s}$, $\dots$, $C_{r,1},\dots, C_{r,n_r}\}$. 
%$\schm=C^{\ell_{s, 1}}_{i_{s,1}} \dots C^{\ell_{s, t_s}}_{i_{s, t_s}} C^{\ell_{s+1, 1}}_{i_{s+1, 1} } \dots C^{\ell_{s+1, t_{s+1}}}_{i_{s+1, t_{s+1}} } \dots C^{\ell_{r, 1}}_{i_{r,1}} \dots C^{\ell_{r, t_r}}_{i_{r, t_r}} $, where for each $s': s \le s' \le r$, $i_{s',1},\dots, i_{s', t_{s'}} \in [n_{s'}]$, 
%%%%%%%%%%%%%%%%%%%%%%%%%%%%%%%%%%%%%%%%%%%%%%%%%%%
%%%%%%%%%%%%%%%%%%%%%%%%%%%%%%%%%%%%%%%%%%%%%%%%%%%
%%%%%%%%%%%%%%%%%%%%%%%%%%%%%%%%%%%%%%%%%%%%%%%%%%%
\hide
{
At first, by using $O(q_m)$, we do the following computation, similarly to Step II: For each $i_1: 1 \le i_1 \le n$, if there are a cycle scheme $\schm$  
$HC_{i_1}^{\ell_1} C_{i_2}^{\ell_2} \dots C_{i_t}^{\ell_t}
$
or 
$HC_{i_1}^{\ell_1} C_{i_2}^{\ell_2} \dots C_{i_t}^{\ell_t} (C'_{i'_1})^{\ell'_1} (C'_{i'_2})^{\ell'_2} \dots (C'_{i'_{t'}})^{\ell'_{t'}}$,
and $j' \le k$ such that 
\begin{itemize}
\item $i_2,\dots,i_t \le n$ are mutually distinct, $\ell_2 = \dots = \ell_t = 1$, 
%
\item $i'_1,\dots,i'_{t'} \le n'$ are mutually distinct, $\ell'_2 = \dots = \ell'_{t'} = 1$, 
%
\item $\pi_{C_{i_1}}(j')=j'$, and $\mu_{\schm,(i_1,j')} \neq 0$ (recall that $\mu_{\schm,(i_1,j')}$ is obtained from the coefficient of $d^{(0)}_{\pi_H(j')-k}$ in  $\chi_\schm(O(q_m))$), 
\end{itemize}
then return $\ltrue$. 

Then by using $O(q'_{m'})$, we do the following: For each $i'_1: 1 \le i'_1 \le n'$, if there are a cycle scheme $\schm' =(C'_{i'_1})^{\ell'_1} (C'_{i'_2})^{\ell'_2} \dots (C'_{i'_{t'}})^{\ell'_{t'}}$, and $j' \le k$ such that
\begin{itemize}
\item $i'_2,\dots,i'_t \le n'$ are mutually distinct, $\ell'_2 = \dots = \ell'_t = 1$, 
%
\item $\pi_{C'_{i_1}}(j')=j'$, and $\mu_{\schm',(i'_1,j')} \neq 0$ (here $\mu_{\schm',(i_1,j')}$ is obtained from the coefficient of $d''_{j'}$ in  $\chi_{\schm'}(O(q'_{m'}))$, where $d''_1,\dots,d''_k$ denote the initial data values of $x_1,\dots,x_k$ respectively),
\end{itemize}
then return $\ltrue$. 

Similarly, we can apply an analysis for the constant coefficient to $\chi_\schm(O(q_m))$. 


If the decision procedure has not return yet, then go to Step III$'$. \qed
}
%%%%%%%%%%%%%%%%%%%%%%%%%%%%%%%%%%%%%%%%%%%%%%%%%%%
%%%%%%%%%%%%%%%%%%%%%%%%%%%%%%%%%%%%%%%%%%%%%%%%%%%
%%%%%%%%%%%%%%%%%%%%%%%%%%%%%%%%%%%%%%%%%%%%%%%%%%%
After Step II$'$, if the decision procedure has not returned yet, then similar to Lemma~\ref{prop-bnd-domain-2}, the following hold.
\begin{itemize}
\item For each $s \in [r]$ and each path $\schm=H_1 \schm_1 H_2 \dots H_s \schm_s$ such that for each $s'\in [s]$, $\schm_{s'}$ is a cycle scheme over the collection of cycles $\{C_{s',1},\dots,C_{s',n_{s'}}\}$, it holds that the constant atom and all the coefficients of the non-constant atoms in ${\sumf^{(\schm,\sval_\bot)}}^-(y_j)$ are from a bounded domain $U$.
%
\item Moreover,  an abstraction of $\schm$, denoted by $\abs(\schm)$, can be defined, so that $\mathscr{A}$, which contains the set of $\abs(\schm)$ for the paths $\schm=H_1 \schm_1 H_2 \dots H_s \schm_s$ (where $s \in [r]$), can be computed effectively from 
$H_1, C_{1,1}, \dots, C_{1,n_1},H_2,\dots, H_r,C_{r,1},\dots, C_{r,n_r}$.
\end{itemize}
%Similarly to the generalized lassos, we can construct a finite state automaton $\Aa'$ from $\chi_H,\chi_{C_1},\dots,\chi_{C_n},\chi_{H'}, \chi_{C'_1},\dots,\chi_{C'_{n'}}$ to record the coefficients in the states and simulate the evolvement of these coefficients. The final states of $\Aa$ represent the coefficients obtained when reaching the state $q'_{m'}$ in $\Ss$. 
\framebox[\textwidth]{
	\begin{minipage}{0.95\textwidth}
\noindent {\bf Step III$'$}. We apply the same analysis to $\mathscr{A}$ as in Step III. If the procedure does not return during the analysis, return $\lfalse$.
	\end{minipage}
}

\subsection{Decision procedure for SNTs}

We now consider the situation that the guards of the transitions in $\Ss$ may contain constants, that is, the atomic formulae of the form $cur \odot c$ for integer constants $c$. 

We illustrate the arguments for generalized lassos and adapt the decision procedure Step I-III to Step I$''$-III$''$ below. The arguments for the SNTs whose transition graphs are not necessarily generalized lassos are similar.

Suppose $H (C_1, \dots, C_n)$ is a generalized lasso, $H=q_0 \dots q_m$, and $O(q_m)=a_0 + a_1 x_1 + \dots + a_k x_k + b_1 y_1 + \dots + b_l y_l$.

From the guards and assignments of the transitions in $H$, we know that some of $\vard^{\circled{H}}_1,\dots, \vard^{\circled{H}}_{r^{\circled{H}}}$ have to be integer constants in $[c_{min}, c_{max}]$. Let $J^{\circled{H}} \subseteq [r^{\circled{H}}]$ denote the set of indices $j \in [r^{\circled{H}}]$ such that $\vard^{\circled{H}}_j$ has to be in $[c_{min}, c_{max}]$. Let $c^{\circled{H}}_j \in [c_{min}, c_{max}]$ denote this constant corresponding to $\vard^{\circled{H}}_j$. Since $\Ss$ is assumed to be normalized, we know that for each $j \not \in J^{\circled{H}}$,  $\vard^{\circled{H}}_j < c_{min}$ or $\vard^{\circled{H}}_j > c_{max}$.
%
%
%
%
\smallskip\\
\framebox[\textwidth]{
	\begin{minipage}{0.95\textwidth}
		\noindent {\bf Step I$''$}. For each $j \in [r^{\circled{H}}] \setminus J^{\circled{H}}$, decide whether the coefficient of the $\vard^{\circled{H}}_j$-atom in $\eval{O(q_m)}{\sumf^{(H,\initval_\bot)}}$ is nonzero. If the answer is yes, then return $\ltrue$. If the decision procedure has not returned yet, substitute each data variable $\vard^{\circled{H}}_j$ for $j \in J^{\circled{H}}$ with $c^{\circled{H}}_j$ in $\eval{O(q_m)}{\sumf^{(H,\initval_\bot)}}$. Then $\eval{O(q_m)}{\sumf^{(H,\initval_\bot)}}$ becomes an integer constant $c^{\circled{H}}$. If $c^{\circled{H}} \neq 0$, then return $\ltrue$.  Otherwise, go to Step II$''$.
	\end{minipage}
}\bigskip

For each $i_1 \in [n]$, let $J^{\circled{q_m}}$ denote the set of indices $j' \in [k]$ such that $\initval(x_{j'})$ has to be in $[c_{min}, c_{max}]$, which is enforced by the state $q_m$. Recall that the SNT $\Ss$ is state-dominating. Therefore, from the state $q_m$, we know which control variable should have a value in $[c_{min},c_{max}]$. For each $j' \in J^{\circled{q_m}}$, let $c^{\circled{q_m}}_{j'}$ denote this constant corresponding to $j'$. 

Let $\schm=C_{i_1}^{\ell_1} C^{\ell_2}_{i_2} \dots C^{\ell_t}_{i_t}$ be a cycle scheme. 
For each $i_1 \in [n]$, let $J^{\circled{C_{i_1}}}$ denote the set of indices $j'' \in [r^{\circled{C_{i_1}}}]$ such that there is $c^{\circled{C_{i_1}}}_{j''} \in [c_{min}, c_{max}]$ satisfying that for each $i \in [\ell_1]$, $\vard^{\circled{C_{i_1}, i}}_{1, j''}$ has to be equal to $c^{\circled{C_{i_1}}}_{j''}$, as a result of the guards and assignments on $C_{i_1}$. (Recall that the SNT $\Ss$ is required to be normalized). Note that the definition of $J^{\circled{C_{i_1}}}$ is independent of the choices of $\schm$.

Suppose $\schm=C_{i_1}^{\ell_1} C^{\ell_2}_{i_2} \dots C^{\ell_t}_{i_t}$ is a cycle scheme. Then for each $j' \in  I^{\circled{C_{i_1}}}_{pe} \setminus J^{\circled{q_m}}$, we can obtain $\mu_{\schm, (i_1,j')} \ell_1 + \nu_{\schm, (i_1,j')}$ as in Step II of Section~\ref{sec-glasso}. 

We then define $\mu'_{\schm,(i_1,0)} \ell_1 + \nu'_{\schm,(i_1,0)}$ as follows. Because $\initval(x_{j'})$ is a constant for each $j' \in J^{\circled{q_m}}$, and for each $j \in [r^{\circled{C_{i_1}}}]$ and $i \in [\ell_1]$, $\vard^{\circled{C_{i_1}, i}}_{1, j}$ is a constant, we should take as a group the constant atom of  $\eval{O(q_m)}{\sumf^{(\schm,\initval)}}$, the $\initval(x_{j'})$-atoms for $j' \in J^{\circled{q_m}}$, and the $\vard^{\circled{C_{i_1}, i}}_{1, j}$-atoms for $j \in [r^{\circled{C_{i_1}}}]$ and $i \in [\ell_1]$.
%
Recall that the constant atom of $\eval{O(q_m)}{\sumf^{(\schm,\initval)}}$ contains the following  subexpression,
\begin{center}
	\vspace{-0.2cm}
	\resizebox{0.8\hsize}{!}{$
		e_0:= \sum \limits_{1 \le j \le l} b_j
		\begin{array}{l}
		\left((\lambda^{\circled{C_{i_2}}}_{j})^{\ell_2} \dots (\lambda^{\circled{C_{i_t}}}_{j})^{\ell_t}\right)
		\left(1+\lambda^{\circled{C_{i_1}}}_{j} + \dots + (\lambda^{\circled{C_{i_1}}}_{j})^{\ell_1-1} \right) \cste^{\circled{C_{i_1}}}_{j}. 
		\end{array}
		$}
\end{center}
Moreover, for each $j' \in  J^{\circled{q_m}} \cap I^{\circled{C_{i_1}}}_{pe}$, the coefficient of the $\initval(x_{j'})$-atom of $\eval{O(q_m)}{\sumf^{(\schm,\initval)}}$ contains the following subexpression, 
\begin{center}
	\resizebox{0.8\hsize}{!}{$
		e_{j'} := \sum \limits_{1 \le j \le l} 
		%\begin{array}{l}
		b_j \left((\cstl^{\circled{C_{i_2}}}_{j})^{\ell_2} \dots (\cstl^{\circled{C_{i_t}}}_{j})^{\ell_t}\right) 
		\left(1+\cstl^{\circled{C_{i_1}}}_{j} + \dots + (\cstl^{\circled{C_{i_1}}}_{j})^{\ell_1-1} \right) \csta^{\circled{C_{i_1}}}_{j,j'}.
		%\end{array} \hspace{1cm}  
		$}
\end{center}

For each $j'' \in J^{\circled{C_{i_1}}} \cap \rng(\pi^{\circled{C_{i_1}}})$ and $i \in [\ell_1]$, the coefficient of the $\vard^{\circled{C_{i_1}, i}}_{1, j''}$-atom of $\eval{O(q_m)}{\sumf^{(\schm,\initval)}}$ contains the following subexpression, 
\begin{center}
	\resizebox{0.95\hsize}{!}{$
		f_{j'',i} := \sum \limits_{1 \le j \le l} 
		%\begin{array}{l}
		b_j \left((\cstl^{\circled{C_{i_2}}}_{j})^{\ell_2} \dots (\cstl^{\circled{C_{i_t}}}_{j})^{\ell_t}\right) 
		\left((\cstl^{\circled{C_{i_1}}}_{j})^{\ell_1-i-1} \csta^{\circled{C_{i_1}}}_{j, (\pi^{\circled{C_{i_1}}})^{-1}(j'') } + (\cstl^{\circled{C_{i_1}}}_{j})^{\ell_1-i} \cstb^{\circled{C_{i_1}}}_{j,j'}\right).
		%\end{array}
		$}
\end{center}

For each $j'' \in J^{\circled{C_{i_1}}} \setminus \rng(\pi^{\circled{C_{i_1}}})$ and $i \in [\ell_1]$, the coefficient of the $\vard^{\circled{C_{i_1}, i}}_{1, j''}$-atom of $\eval{O(q_m)}{\sumf^{(\schm,\initval)}}$ contains the following subexpression, 
\begin{center}
	\resizebox{0.75\hsize}{!}{$
		f_{j'',i} := \sum \limits_{1 \le j \le l} 
		%\begin{array}{l}
		b_j \left((\cstl^{\circled{C_{i_2}}}_{j})^{\ell_2} \dots (\cstl^{\circled{C_{i_t}}}_{j})^{\ell_t}\right) 
		\left( (\cstl^{\circled{C_{i_1}}}_{j})^{\ell_1-i} \cstb^{\circled{C_{i_1}}}_{j,j'}\right).
		%\end{array}
		$}
\end{center}

Then we consider the expression 
\[e_0 + \sum \limits_{j' \in J^{\circled{q_m}} \cap I^{\circled{C_{i_1}}}_{pe}} (e_{j'} c^{\circled{q_m}}_{j'}) + \sum \limits_{j'' \in J^{\circled{C_{i_1}}} } \sum \limits_{i \in [\ell_1]} (f_{j'',i}\ c^{\circled{C_{i_1}}}_{j''}),\] 
which can be rewritten as $\mu'_{\schm,(i_1,0)} \ell_1 + \nu'_{\schm,(i_1,0)}$ for some integer constants $\mu'_{\schm, (i_1,0)}$ and $\nu'_{\schm, (i_1,0)}$. 
%If $\mu'_{\schm,(i_1,0)}=\mu_{\schm,(i_1,j')}=0$ for all $j' \in J^{\circled{q_m}} \cap I^{\circled{C_{i_1}}}_{pe}$, then we can ignore all subexpressions containing the cycle counter variable $\ell_1$ in   $\eval{O(q_m)}{\sumf^{(\schm,\initval)}}$, i.e., the subexpressions $\mu'_{\schm,(i_1,0)}\ell_1$ and $\mu_{\schm,(i_1,j')}\ell_1$ for all $j' \in J^{\circled{q_m}} \cap  I^{\circled{C_{i_1}}}_{pe}$.
%
\medskip\\
\framebox[\textwidth]{
	\begin{minipage}{0.95\textwidth}
		\noindent {\bf Step II$''$}. For each $i_1 \in [n]$, check all cycle scheme $\schm=C_{i_1}^{\ell_1} C_{i_2} \dots C_{i_t}$ such that $i_2,\dots,i_t$ are mutually distinct. There are only finitely many this kind of cycle schemes. If 
		one of the following constraints is satisfied, then return $\ltrue$. (1) There is $j' \in  I^{\circled{C_{i_1}}}_{pe} \setminus J^{\circled{q_m}}$ such that $\mu_{\schm,(i_1,j')} \neq 0$. (2) $\mu'_{\schm,(i_1,0)} \neq 0$.
		%
		If the decision procedure has not returned yet, then go to Step III$''$.
	\end{minipage}
}\smallskip\\

If after Step II$''$, the algorithm has not return yet, then similarly to Section~\ref{sec-glasso}, we can construct a finite set $U'' \subset \intnum$, which acts as a bounded domain for the constant atoms and the coefficients of non-constant atoms after removing all the subexpressions related to the cycle counting variables $\ell_1,\dots,\ell_t$. Moreover, we can construct the set $\mathscr{A''}$ of abstractions of cycle schemes.
%
\medskip\\
\framebox[\textwidth]{
	\begin{minipage}{0.95\textwidth}
		\noindent {\bf Step III$''$}. Similar to Step III, with $U,\mathscr{A}$ replaced by $U''$ and $\mathscr{A''}$.
	\end{minipage}
}\smallskip\\

\section{Case Studies}
\label{sec_cases}
\begin{figure}
	\centering
	\lstset{language=C,
		basicstyle=\ttfamily\scriptsize}
	\begin{tabular}{|c|c|c|}
		\hline
		\begin{minipage}[t]{0.2\textwidth}
		\vspace{-0.5cm}
			\begin{lstlisting}[mathescape=true]
int avg() {
 sum:=$\cur$;
 cnt:=0;$\nnext$;
 loop{
  sum+=$\cur$;
  cnt+=1;
  $\nnext$;}
 ret sum/cnt;}
			\end{lstlisting}
		\end{minipage}&
		\begin{minipage}[t]{0.4\textwidth}
		\vspace{-0.5cm}
\begin{lstlisting}[mathescape=true]
int MAD() {
 sum:=$\cur$;cnt:=0;$\nnext$;
 loop{sum+=$\cur$;cnt+=1;$\nnext$;}
 avg:= sum/cnt;mad:=0;$\init$;
 loop{
  if($\cur$<avg){mad+=avg-$\cur$;}
  else{mad+=$\cur$-avg;}$\nnext$;}
 ret mad/cnt;}
\end{lstlisting}
		\end{minipage}&
		\begin{minipage}[t]{0.4\textwidth}
		\vspace{-0.5cm}
			\begin{lstlisting}[mathescape=true]
int SD() {
 sum:=$\cur$;cnt:=0;$\nnext$;
 loop{sum+=$\cur$;cnt+=1;$\nnext$;}
 avg:= sum/cnt;sd:=0;$\init$;
 loop{
  sd+=($\cur$-avg)*($\cur$-avg);$\nnext$;
 }
 ret SQRT(sd/cnt);}
			\end{lstlisting}
		\end{minipage}\\
		\hline		
	\end{tabular}
	\caption{More Challenging Examples of Reducers Performing Data Analytics Operations}
	\label{fig:examples2}
\vspace{-0.5cm}
\end{figure}

	
	
	
For cases with multiplication and division at the return point, e.g., \texttt{avg} in Figure~\ref{fig:examples}(b), we can model them as an \emph{uninterpreted $k$-ary function}  (modeled as a tuple in our language) and still have a sound procedure for the decision problems. 

%!TEX root = main-cav.tex
	
\section{Conclusion}
\label{sec:conclusion}

%From the analysis of the commutativity of reducers in \cite{XZZ+14}, the commutativity of a reducer in a sequential composition of map-reduce jobs may depend on some implicit data properties guaranteed by the preceding map-reduce jobs. Therefore, to analyze the commutativity of a reducer in a sequential composition of map-reduce jobs, we may need model both mappers and reducers and do a backward analysis.

The contribution of the paper is twofold. We propose a verifiable programming language for reducers. Although it is still far away from a practical programming language, we believe that some ideas behind our language (e.g., the separation of control variables and data variables) would be valuable for the design of a practical reducer language. On the other hand, we propose the model of streaming numerical transducers, a transducer model over infinite alphabets. To our best knowledge, this is the first decidable automata model over infinite alphabets that allows linear arithmetics over the input values and the integer variables. Although we required that the transition graphs of SNTs are generalized flat,  SNTs with such kind of transition graphs turn out to be quite powerful, since they are capable of simulating reducer programs without nested loops, which is a typical scenario of reducer programs in practice.


\subsubsection{Acknowledgements.} Yu-Fang Chen is partially supported by the MOST project No. 103-2221-E-001-019-MY3. Zhilin Wu is partially supported by the NSFC grants No.\ 61100062, 61272135, 61472474, and 61572478.

% , parameterized counter automata, integer VASS

%restate the contribution of our work. (1) hint for verifiable reducer/list manipulating program language (2) SNT is a first autoamta with infinite alphabet supporting presburgh arithemetic over input and variables.


\bibliographystyle{abbrv}
\bibliography{data}

\newpage

%!TEX root = main-cav.tex

\begin{appendix}

\section{Formal Semantics of the Programming Language}
\begin{figure}
	\hspace{-0.4cm}
	\scalebox{0.9}{
		\begin{tabular}{|l|l|}
			\hline
			Transitions&
			Side Condition\\
			\hline
			$(y := e;p, w, \rho) \longrightarrow (p, w, \rho')$&
			$\rho'=\rho[\eval{e}{\rho}/y]$\\
			
%			$\rho'(z) =\rho(z)$ for $z\neq y$, $\rho'(y) = \eval{\rho}{e}$\\
			
			$(y \addeq e;p, w, \rho) \longrightarrow (p, w, \rho')$&
			$\rho'=\rho[\eval{y+e}{\rho}/y]$\\
%			$\rho'(z) =\rho(z)$ for $z\neq y$, $\rho'(y) = \eval{y+e}{\rho}$\\
			
			$(\ite{g}{s_1}{s_2};p, w, \rho) \longrightarrow (s_1;p, w, \rho)$&
			$\rho \models g$\\
			
			$(\ite{g}{s_1}{s_2};p, w, \rho) \longrightarrow (s_2;p, w, \rho)$& $\rho \not \models g$\\
			
			$(\nnext;p, w, \rho) \longrightarrow (p, \tail(w), \rho')$&
			$\rho'=\rho[\head(w)/\cur]$ if $\rho(\cur) \neq \bot$ \\
			
			$(x':=x;p, w, \rho) \longrightarrow (p, w, \rho')$&
			$\rho'  =\rho[\rho(x)/x']$\\
			
			$(\loopL{s};\mbox{ret }r, w, \rho) \longrightarrow (s;\loopL{s};\mbox{ret }r, w, \rho)$& \\
			
			$(\loopL{s};\mbox{ret }r, \epsilon, \rho) \longrightarrow (\mbox{ret }r,  \epsilon, \rho)$& 	\\	
			\hline
			
		\end{tabular}
	}
	\caption{The Semantics of the Programming Language}
	\label{fig:semantics}
\end{figure}


Formally, the semantics of a program $p$ in the programming language is defined as a transition system in Fig.~\ref{fig:semantics}. Let $p$ be a reducer program and $w$ be an input data word.  Each configuration of the transition system is a triple $(p', w', \rho)$, where $p'$ is a program, $w'$ is a suffix of $w$, and $\rho$ is an valuation over $X^+\cup Y$ such that $\rho(\cur)=\head(w')$. 
Let $\rho_w$ be an assignment such that $\rho_w(\cur)=\head(w)$ and $\rho_w(z)=\bot$ for $z \in X \cup Y$.
The initial configuration is $(p, \tail{w}, \rho_w)$.
We use $p(w)$ to denote the \emph{output} of $p$ on $w$. Then $p(w) =d$ if there exists a path from the initial configuration $(p, w, \rho_w)$ to some return configuration $(\mbox{ret }r,  \epsilon, \rho_r)$ such that $
\eval{r}{\rho_r}=d$. Otherwise, $p(w)=\bot$. Since the program is deterministic, i.e., each input data word has at most one output, the semantics of $p$ is well-defined.

\section{Proofs in Section~\ref{sec:def-snt}}




\newcommand\assume{\mathsf{assume}}

\newcommand\loc{\mathfrak{l}}

\noindent {\bf Proposition~\ref{prop-mrprog-to-snt}}.
{\it 
For each reducer program $p$, an equivalent SNT $\Ss$ can be constructed.
}

\smallskip

\begin{proof}
We introduce a few notations first.

Let $s$ be a loop-free program. An \emph{execution path} $\pi$ of $s$ is a maximal path in the control flow graph of $s$ (here we use the standard definition of control flow graphs). Each execution path $\pi$ corresponds to a program $s_\pi$ obtained by sequentially composing the statements in $\pi$, where the statements $\assume(g)$ are used to represent the guards $g$. Then $s$ can be seen as a union of $s_\pi$, where $\pi$ ranges over the execution paths of $s$. 

Let $p$ be a reducer program of the form $s_1; \nnext; \loopL{s_2;\nnext}$; ret $r$.  In the following, we show how to construct an SNT $\Ss_p$ to simulate $p$.

The loop body $s_2;\nnext$ can be seen as a union of programs $p_\pi$ for execution paths $\pi$. We assume that no two distinct programs $p_\pi$ share locations. We first transform the loop into a collection of state-disjoint cycles $C_\pi$, one for each program $p_\pi$.  Let us focus on a program $p_\pi$. The set of states in $P_\pi$ comprises the location $\loc_0$ which is the entry point of the loop, and the locations succeeding each $\nnext$ statement in $p_\pi$. Moreover, we identify the location succeeding the last $\nnext$ statement and the entry point. The effect of the subprogram $s'$ between two successive $\nnext$ statements in the locations $\loc_1,\loc_2$ can be summarized into a transition $(\loc_0, g', \eta', \loc_1)$ resp. $(\loc_1, g', \eta', \loc_2)$ of $p_\pi$. This is possible due to the following two constraints: 1) the conditions $g$ in the statements $\ite{g}{s'_1}{s'_2}$  of $p_\pi$ are the conjunctions of $\cur \odot c$ and $\cur \odot x$, 2) the assignments to the control variables are of the form $x:=x'$ for $x' \in X^+$, and the assignments to the data variables are of the form $y:=e$ and $y {+=} e$, where $e$ contains only control variables or $\cur$. As a result of the two constraints, we can trace the evolvement of the values of the control variables and simulate all the statements $\assume(g)$ occurring in $s'$ by a guard $g'$ (obtained from these guards $g$ by some variable substitutions),  moreover, the effects of all the assignments therein can be summarized into an assignment function $\eta'$.  Similarly, we can do the same for the subprogram between the entry point and the first $\nnext$ statement of $p_\pi$.

In addition, each execution path of $s_1;\nnext$ can be simulated by a simple path of transitions of $\Ss_p$, which ends in the state $\loc_0$, the entry point of the loop.

The output function $O_p$ of $\Ss_p$ is defined as follow: $O_p(\loc_0) = r$ and $O(\loc)$ is undefined for each other state $\loc$.\qed
\end{proof}

\hide{
\begin{algorithm}[H]
	%  \SetAlgoLine
	\KwData{A reducer program $p$}
	$Q=\{q_0\}, \delta=\emptyset, O=\emptyset$, $\mathsf{toState}(p) =q_0$, $\mathsf{toVisit}=\{(\mathsf{toState}(p),p,\ltrue,\emptyset)\}$\;
	\While{$\mathsf{toVisit}\neq \emptyset$}{
		remove $(q,p,g,\eta)$ from $\mathsf{toVisit}$\;
		\Switch{$p$}{
			\lCase{$y := e;p'$,$y \addeq e;p'$,$x'=x;p'$: }{add $(q,p',g,\eta[e/y])$, $(q,p',g,\eta[(y+e)/y])$, $(q,p',g,\eta[x'/x])$ to $\mathsf{toVisit}$, respectively}
			\lCase{$\ite{g'}{s_1}{s_2};p'$: }{add both $(q,s_1;p',g\wedge g',\eta)$ and $(q,s_2;p',g\wedge \neg g',\eta)$ to $\mathsf{toVisit}$}
			\lCase{$\loopL{s;}\mbox{ret }r$: }{add both $(q,s;\loopL{s;}\mbox{ret }r,g,\eta)$ and $(q,\mbox{ret }r, g,\eta)$ to $\mathsf{toVisit}$}
			\lCase{$\nnext;p'$: }{\label{alg:next}
				\uIf{$\mathsf{toState}(p') \not\in Q$}{add $(\mathsf{toState}(p'),p',\ltrue,\emptyset)$ to $\mathsf{toVisit}$ and add $\mathsf{toState}(p')$ to $Q$}
				add $(q, \mathsf{toState}(p'),g,\eta)$ to $\delta$
			}
			\lCase{$\mbox{ret }r: $}{\label{alg:output}
				add a fresh state $q_r$ to $Q$, 
				add $(q, q_r,g,\eta)$ to $\delta$, and $O:=O[r/q_r]$}
		}
	}
	\Return $(Q,X,Y,\delta, \mathsf{toState}(p),O)$\;
	
	\caption{Translate a Reducer Program to a SNT}
	\label{fig:reducer2SNT}
\end{algorithm}
We use a tuple $(q,p,g,\eta)$ to store intermediate results of the translation, where $q$ is the source SNT state, $p$ is a reducer program, $g$ is a guard, and $\eta$ is an assignment.
The algorithm begins with the tuple $(p,p,\ltrue,\emptyset)$. The algorithm add a transition to SNT only when a $\nnext$ statement is encountered (line~\ref{alg:next}). When a $\mbox{ret }r$ statement is encountered, the algorithm adds a fresh state $q_r$ to the SNT and extends the output function to $O[r/q_r]$ (line~\ref{alg:output}).

The SNT returned from Algorithm~\ref{fig:reducer2SNT} is not yet generalized flat. It might have cycles sharing more than one states. All the cycles coming from the loop and branches inside the loop. There must be at least one state $s$ shared by all cycles. Therefore, we can make it generalized flat by duplicating all shared stated other than $s$ so all cycles will have their own copy of the shared states other than $s$.  
}

\vspace{4mm}

\noindent {\bf Proposition~\ref{prop-snt-cmm-to-eqv}}. 
\emph{The commutativity problem of SNTs is reduced to the equivalence problem of SNTs in exponential time}.

\begin{proof}
Suppose that $\Ss=(Q, X, Y, \delta, q_0, O)$ is an SNT such that $X=\{x_1,\dots,x_k\}$ and $Y=\{y_1,\dots,y_l\}$. Without loss of generality, we assume that the output of $\Ss$ is defined only for data words of length at least two. We will construct two SNTs $\Ss_1$ and $\Ss_2$ so that $\Ss$ is commutative iff $\Ss$ is equivalent to both $\Ss_1$ and $\Ss_2$.
\begin{itemize}
\item The intuition of $\Ss_1$ is that over a data word $w=d_1 d_2 d_3 \dots d_n$ with $n\ge 2$, $\Ss_1$ simulates the run of $\Ss$ over $d_2 d_1 d_3 \dots d_n$, that is, the data word obtained from $w$ by swapping the first two data values.
%
\item The intuition of $\Ss_2$ is that over a data word $w=d_1 d_2 d_3 \dots d_n$ with $n\ge 2$, $\Ss_1$ simulates the run of $\Ss$ over $d_2 d_3 \dots d_n d_1$, that is, the data word obtained from $w$ by moving the first data value to the end. 
\end{itemize}
The correctness of this reduction follows from Proposition 1 in \cite{CHSW15}.

\smallskip

\noindent {\it The construction of $\Ss_1$}.

Intuitively, over a data word $w=d_1d_2 d_3 \dots d_n$, we introduce an additional control variable $x'$ to store $d_1$, then simulates the run of $\Ss$ over $d_2 d_1 d_3 \dots d_n$ as follows: When reading $d_2$ in $w$, the data variables are updated properly by letting $x'$ to represent $d_1$ and $\cur$ to represent $d_2$.

Without loss of generality, we assume that for each pair of transitions $q_0 \xrightarrow{(g_1,\eta_1)} q_1 \xrightarrow{(g_2,\eta_2)} q_2$ starting from the initial state $q_0$ in $\Ss$, the following constraints are satisfied,
\begin{itemize}
\item $g_1$ does not contain any variable from $X$ (otherwise, $g_1$ would be evaluated to $\lfalse$),
%
\item for each variable $x \in X$ such that $x$ occurs in $g_2$, it holds that $x \in \dom(\eta_1)$,
%
\item after these two transitions, the values of all the variables from $\dom(\eta_1) \cup \dom(\eta_2)$ are defined, more specifically, for each $y \in Y \cap \dom(\eta_2)$ and each $z \in \vars(\eta_2(y))$, it holds that $z \in \dom(\eta_1)$.
\end{itemize}

Let $q'_{0},q'_{1} \not \in Q$ and $x' \not \in X$. Then $\Ss_1 = (Q \cup \{q'_{0},q'_1\}, X \cup \{x'\}, Y, \delta_1, q'_{0}, O_1)$ such that 
\begin{itemize}
\item $O_1(q'_0)$ and $O_1(q'_1)$ are undefined, and for each $q \in Q$, $O_1(q)=O(q)$,
%
\item $\delta_1$ is constructed from $\delta$ as follows,
\begin{itemize}
\item each element of $\delta$ is an element of $\delta_1$,
%
\item for each pair of transitions $q_0 \xrightarrow{(g_1,\eta_1)} q_1 \xrightarrow{(g_2,\eta_2)} q_2$ in $\Ss$, we add the transitions $(q_0, \ltrue, \eta'_1, q'_1)$ and $(q'_1, g', \eta'_2, q_2)$ into $\delta_1$, where $\eta'_1,g',\eta'_2$ are defined in the following. Suppose for each $y_j \in Y \cap \dom(\eta_1)$, $\eta_1(y_j)=a_{j} + b_{j}\cur$, and for each $y_j \in Y \cap \dom(\eta_2)$, 
\[\eta_2(y_j)= y_j + a'_{j} + b'_{j,0}  \cur + \sum\limits_{x_{j'} \in \dom(\eta_1)} b'_{j,j'} x_{j'},\] 
or 
\[
\eta_2(y_j) = a'_{j} + b'_{j,0} \cur + \sum \limits_{x_{j'} \in \dom(\eta_1)} b'_{j,j'} x_{j'} .
\]
Then $\eta'_1, g', \eta'_2$ are defined as follows.
\begin{itemize}
\item $\eta'_1(x')=\cur$, for each $x \in X \cap \dom(\eta_2)$, $\eta'_1(x)=\cur$, and for all the other variables $z$ from $X \cup Y$, $\eta'_1(z)$ is undefined.
%
\item $g' = g_1 \wedge g'_2$, where $g'_2$ is obtained from $g_2$ by replacing $\cur$ with $x'$, and each $x \in X$ with $\cur$.
%
\item For each $x \in X$, if $x \in \dom(\eta_2)$, then $\eta'_2(x)$ is undefined, otherwise, if $x \in \dom(\eta_1)$, then $\eta'_2(x)=\cur$, otherwise, $\eta'_2(x)$ is undefined.
%
\item For each $y_j \in Y$, if $y_j \in \dom(\eta_2)$, then 
\[
\begin{array}{l c l}
\eta'_2(y_j) & = & (a_{j} + b_{j}\cur) + a'_{j} + b'_{j,0} x' + \sum \limits_{x_{j'} \in \dom(\eta_1)} b'_{j,j'} \cur \\
& = & (a_{j} + a'_{j}) + b'_{j,0} x' + (b_{j}  + \sum \limits_{x_{j'} \in \dom(\eta_1)} b'_{j,j'} )\cur,
\end{array}
\]
or 
\[
\begin{array}{l c l}
\eta'_2(y_j) & = & a'_{j} + b'_{j,0} x' + \sum \limits_{x_{j'} \in \dom(\eta_1)} b'_{j,j'} \cur  \\
& = & a'_{j} + b_{j,0} x' + (\sum \limits_{x_{j'} \in \dom(\eta_1)} b'_{j,j'}  )\cur.
\end{array}
\]
%
Otherwise, if $y_j \in \dom(\eta_1)$, then $\eta'_2(y_j)= a_{j} + b_{j} \cur$. Otherwise, $\eta'_2(y_j)$ is undefined.
\end{itemize}
\end{itemize}
\end{itemize}
It is easy to see that the size of $\Ss_1$ is polynomial with respect to the size of $\Ss$.

\smallskip

\noindent {\it The construction of $\Ss_2$}.

Intuitively, over a data word $w=d_1\dots d_n$, we introduce an additional control variable $x'$ to store $d_1$, then simulates the run of $\Ss$ over $d_2\dots d_n d_1$: When reaching the end of $w$, $\Ss_2$ outputs immediately by using $x'$ to represent $d_1$ and simulating the last transition of $\Ss$ over $d_2 \dots d_n d_1$. In order to simulate \emph{deterministically} the last transition of $\Ss$ over $d_2 \dots d_n d_1$ when reading the end of $w$ (since SNTs are required to be deterministic), we need record in the states of $\Ss_2$ the relationship between $d_1$ and all the values stored in the control variables. This implies an exponential blow-up of the size of $\Ss_2$ with respect to $\Ss$.

Let $c_{max}$ and $c_{min}$ denote the maximum resp. minimum constant occurring the guards of the transitions of $\Ss$. As a convention, let $c_{max}=c_{min}=0$ if no constants occur in $\Ss$.

Suppose $q'_{0} \not \in Q$ and $x' \not \in X$. Then $\Ss_2 = (Q', Y, \delta_2, q'_{0}, O_2)$, where $O',\delta_2,O_2$ are defined as follows. 
\begin{itemize}
\item $Q' = \{q'_0\} \cup \left(Q \times \left([c_{\min}, c_{\max}] \cup \{-\infty,+\infty\}\right) \times X^{\{=, <, >,\bot\}} \right)$, where in a state $(q,(c, o)) \in Q'$, the third component $o$ denotes the relationship between $x'$ and $x$, e.g. $o(x)=<$ means that $x' < x$.
%
\item $\delta_2$ is defined as follows, 
\begin{itemize}
\item for each $c \in [c_{min}, c_{max}] \cup \{-\infty,+\infty\}$, $\delta_2$ contains $(q'_0,\ltrue,\eta, (q_0,(c, o_0)))$, where $\dom(\eta)=\{x'\}$, $\eta(x')=\cur$, and $o_0(x) = \bot$ for each $x \in X$,
%
\item for each $(q,g,\eta,q') \in \delta$ and $(q,(c,o)) \in Q'$ such that $g \wedge \bigwedge \limits_{x \in X, o(x) \neq \bot} x'\ o(x)\ x$ is satisfiable, $\delta_2$ contains the following three transitions, 
$(q,(c,o)) \xrightarrow{(g \wedge \cur = x', \eta)} (q',(c,o'_1))$, 
$(q,(c,o)) \xrightarrow{(g \wedge \cur< x', \eta)}  (q',(c,o'_2))$,
and  $(q,(c,o)) \xrightarrow{(g \wedge \cur > x', \eta)} (q',(c,o'_3))$, where 
for each $x \in X$, if $x \in \dom(\eta)$, then $o'_1(x) := \ =$, $o'_2(x):=\ >$, and $o'_3(x) :=\ <$, otherwise, $o'_1(x) = o'_2(x) = o'_3(x) := o(x)$.
\end{itemize}
%
\item $O_2$ is defined as follows: Let $(q,(c,o)) \in Q'$  such that there is $(q,g,\eta,q') \in \delta$ satisfying that $\left(g_c \wedge \bigwedge \limits_{x \in X, o(x) \neq \bot} \cur\ o(x)\ x \right) \models g$, and $O(q')$ is defined, where $g_c := \cur = c$ if $c \in [c_{min}, c_{max}]$, $g_c:=\cur < c_{min}$ if $c=-\infty$, and $g_c:=\cur > c_{max}$ otherwise. Suppose 
\[O(q')=a_0 + a_1 x_1 + \dots + a_k x_k + b_1 y_1 + \dots + b_l y_l.\]
Then let
\[O_2((q,(c,o)))=a_0 + a_1 \eta'(x_1) + \dots + a_k \eta'(x_k) + b_1 \eta'(y_1) + \dots + b_l \eta'(y_l),\]
where for each $z \in \dom(\eta)$, $\eta'(z)=\eta(z)$, and for all the other variables $z' \in X \cup Y$, $\eta'(z')=z'$.  \\
We would like to remark that $O_2$ is well-defined since for each $(q,(c,o)) \in Q'$, there is a unique $(q,g,\eta,q') \in \delta$ satisfying the aforementioned constraint, as a result of the determinism of $\Ss$.
\end{itemize}
%
Note that $\Ss_1$ and $\Ss_2$ constructed above preserve the generalized flatness of $\Ss$.
\qed
\end{proof}


\noindent {\bf Proposition \ref{prop-snt-eqv-to-nzero}}.
\emph{From SNT $\Ss_1$ and $\Ss_2$, a SNT $\Ss_3$ can be constructed in polynomial time such that $\Ss_1$ and $\Ss_2$ are  inequivalent iff there is a data word $w$ such that the output of $\Ss_3$ over $w$ is nonzero.}

\begin{proof}
Let $\Ss_1 = (Q_1,X_1,Y_1,\delta_1,q_{1,0}, O_1)$ and  $\Ss_2 = (Q_2,X_2,Y_2,\delta_2,q_{2,0}, O_2)$ be two SNTs. Without loss of generality, we assume that $Q_1 \cap Q_2 = \emptyset$, $X_1 \cap X_2 = \emptyset$, and $Y_1 \cap Y_2 = \emptyset$. 

Intuitively, we construct $\Ss$ as the product of $\Ss_1$ and $\Ss_2$. Specifically, $\Ss=(Q_1 \times Q_2, X_1 \cup X_2, Y_1 \cup Y_2, \delta, (q_{1,0},q_{2,0}), O)$, where
\begin{itemize}
\item $\delta$ comprises $((q_1,q_2), g_1 \wedge g_2, \eta_1 \cup \eta_2, (q'_1,q'_2))$ such that $(q_1,g_1,\eta_1,q'_1) \in \delta_1$ and $(q_2,g_2,\eta_2,q'_2) \in \delta_2$,
%
\item for each $(q_1,q_2) \in Q_1 \times Q_2$, 
\begin{itemize}
\item if $O_1(q_1)$ is defined and $O_2(q_2)$ is undefined or vice versa, then $O((q_1,q_2))=1$, 
%
\item otherwise, if both $O_1(q_1)$ and $O_2(q_2)$ are defined, then $O((q_1,q_2))=O_1(q_1) - O_2(q_2)$, 
%
\item otherwise (both $O_1(q_1)$ and $O_2(q_2)$ are undefined), $O((q_1,q_2))$ is undefined. 
\end{itemize}
\end{itemize}
From the aforementioned construction and the assumption that $\Ss$ is well-defined, it is easy to see that $\Ss_1$ and $\Ss_2$ are  inequivalent iff there is a data word $w$ such that the output of $\Ss$ over $w$ is non-zero.\qed
\end{proof}


\vspace{4mm}

\noindent {\bf Proposition~\ref{prop-snt-norm}}.
{\it From each SNT, an equivalent normalized SNT can be constructed in exponential time.} 


\newcommand{\tog}[1]{\mathsf{toGuard(#1)}}
\newcommand{\toec}[1]{\mathsf{toEqClass(#1)}}
\begin{proof}
Given an SNT $\Ss=(Q, X, Y, \delta, q_0, O)$, we show that an equivalent normalized SNT ${\Ss}'=(Q', X, Y, \delta', q'_0, O')$  can be constructed.

Before presenting the construction, we introduce some notations first.

Let $\sntcset_\Ss=\{\sntc^{<}_{1},\dots, \sntc^{<}_{k}\} \cup [c_{min}, c_{max}] \cup \{\sntc^{>}_{1},\dots, \sntc^{>}_{k}\}$ and $\cabs_\Ss$ denote the set of partial functions from some $X$ to $\sntcset_\Ss$. Intuitively, $\cabs_\Ss$ is the set of abstractions of the control variables, where $\sntc^{<}_{1},\dots, \sntc^{<}_{k}$ (resp. $\sntc^{>}_{1},\dots, \sntc^{>}_{k}$) are the $k$ colors to denote the control variables whose values are less than $c_{min}$ (resp. greater than $c_{max}$). For $f \in \cabs_\Ss$ and $x, x' \in X$, $x'$ is said to be a \emph{successor} of $x$ wrt. $f$ if one of the following holds: 
\begin{itemize}
\item either $f(x)=c$ and $f(x')=c+1$ for $c \in \intnum$ such that $c, c+1 \in [c_{min},c_{max}]$, or

\item $f(x) = \sntc^{<}_i$ and $f(x') = \sntc^{<}_{j}$ for $i,j: 1 \le i < j \le k$, and the range of $f$ does not contain any color from $\{\sntc^{<}_{i+1},\dots, \sntc^{<}_{j-1}\}$, or

\item $f(x) = \sntc^{>}_i$ and $f(x') = \sntc^{>}_{j}$ for some $i,j: 1 \le i < j \le k$, and and the range of $f$ does not contain any color from $\{\sntc^{>}_{i+1},\dots, \sntc^{>}_{j-1}\}$. 
\end{itemize}
For $x \in X$, $x$ is said to be the \emph{maximum} (resp. \emph{minimum}) control variable wrt. $f$ if $x \in \dom(f)$ and there is no $x' \in X$ such that $f(x')$ is a successor of $f(x)$ (resp. $f(x)$ is a successor of $f(x')$). Let $\sntcset^{<}_\Ss$ denote $\{\sntc^{<}_{1},\dots, \sntc^{<}_{k}\}$, similarly, let $\sntcset^{>}_\Ss$ denote $\{\sntc^{>}_{1},\dots, \sntc^{>}_{k}\}$. Two colors from $\sntcset^{<}_\Ss$ are said to be \emph{adjacent} if they are $\sntc^{<}_i$ and $\sntc^<_{i+1}$ for some $i: 1 \le i < k$. Similarly for two colors from $\sntcset^>_\Ss$. A linear order can be defined on $\sntcset^<_\Ss$ as follows: A color $\sntc^<_i$ is said to be less than $\sntc^<_j$ if $i<j$. A similar order relation can be defined over $\sntcset^>_\Ss$. Moreover, these two linear orders can be extended to a linear order on $\sntcset_\Ss$ in a natural way.

For $f \in \cabs_\Ss$, let $\varphi_f$ denote the constraint over control variables represented by $f$.
\[
\begin{array}{l c l }
\varphi_f  &:=& \bigwedge \limits_{f(x_i)=c,  c_{min} \le c \le c_{max}} x_i = c \ \wedge \\
&  & \bigwedge \limits_{f(x_i)=\sntc^{<}_{i'}, f(x_j)=\sntc^{<}_{j'}, i' < j' } (x_i < x_j \wedge x_j < c_{min})\  \wedge \\
& & \bigwedge \limits_{f(x_i)=\sntc^{<}_{i'}= f(x_j)}(x_i = x_j \wedge x_j < c_{min})\ \wedge \\
& & \bigwedge \limits_{f(x_i)=\sntc^{>}_{i'}, f(x_j)=\sntc^{>}_{j'}, i' < j' } (x_i < x_j \wedge x_i > c_{max})\ \wedge \\
& & \bigwedge \limits_{f(x_i)=\sntc^{>}_{i'}= f(x_j)} (x_i = x_j  \wedge x_i > c_{max}).
\end{array}
\]

Then $Q'= Q \times 2^Y \times \cabs_\Ss$, and $q'_0=(q_0, \emptyset, f_0)$ such that $\dom(f_0)=\emptyset$. Moreover, $O'$ is defined as follows: For each $(q, Z, f) \in Q'$, if $O(q)$  is defined and $\vars(O(q)) \subseteq \dom(f) \cup Z$, then $O'((q, Z, f))=O(q)$, otherwise, $O((q, Z, f))$ is undefined. It remains to define $\delta'$.

The transition set $\delta'$ is defined by the following rule: 
For each $(q, g, \eta, q') \in \delta$, $\delta'$ includes all the transitions $(q, Z, f) \xrightarrow{(g',\eta')} (q', Z', f')$ satisfying the following constraints. 
\begin{itemize}
\item For each $x \in \dom(f) \cap \dom(\eta)$, it holds that $\eta(x) \in \{\cur\} \cup \dom(f)$.  Intuitively, this means that if the original value of $x$ is defined  and $x$ is updated by $\eta$, then the value of $x$ after the update should be defined as well.

\item  For each $y \in Z \cap \dom(\eta)$, it holds that $\vars(\eta(y)) \subseteq \{\cur\} \cup  \dom(f) \cup Z$.  Intuitively, this means that if the original value of $y$ is defined and $y$ is updated by $\eta$, then the value of $y$ after the update should be defined as well.

\item $Z'$ is the union of $Z$ and the set of $y \in Y \cap \dom(\eta)$ such that $\vars(\eta(y)) \subseteq \{\cur\} \cup  \dom(f) \cup Z$.

\item $g',\eta', f'$ satisfy one of the following constraints.
\begin{itemize}
\item The guard $g' := g \wedge \cur = c$ such that $g' \wedge \varphi_f$ is satisfiable. The assignment $\eta'$ is the restriction of  $\eta$ to $Y$, that is, $\dom(\eta') = \dom(\eta) \cap Y$ and for each $y \in \dom(\eta')$, $\eta'(y)=\eta(y)$. The abstraction function $f'$ is defined as follows: For each $x \in X$ such that $\eta(x)=\cur$,  let $f'(x) = c$. Moreover, for each $x \in X$ such that $\eta(x) = x'$, let $f'(x)=f(x')$. For each $x \in X \setminus \dom(\eta)$, let $f'(x)=f(x)$.
 
\item The guard $g' := g \wedge \cur < c_{min} \wedge \cur = x$ such that $g' \wedge \varphi_f$ is satisfiable and $f(x)=\sntc^{<}_i$ for some $i$.  The assignment $\eta'$ is the restriction of  $\eta$ to $Y$. The abstraction function $f'$ is defined as follows: For each $x' \in X$ such that $\eta(x')=\cur$, let $f'(x')=f(x)$. Moreover, for each $x' \in X$ such that $\eta(x')=x''$, let $f'(x')=f(x'')$. For each $x \in X \setminus \dom(\eta)$, let $f'(x)=f(x)$.

\item The guard $g': = g \wedge \cur < c_{min} \wedge  x_i < \cur \wedge \cur < x_j$ such that $g' \wedge \varphi_f$ is satisfiable and $x_j$ is a successor of $x_i$ wrt. $f$. The assignment $\eta'$ is the same as $\eta$, except that for each $x \in X$ such that $\eta(x)=x'$, let $\eta'(x)$ undefined.  The abstraction function $f'$ is defined as follows: Let $f(x_i)=\sntc^{<}_{i'}$ and $f(x_j)=\sntc^{<}_{j'}$. 
\begin{itemize}
\item For each $x \in X$ such that $\eta(x)=x'$, let $f'(x)=f(x')$.
%
\item If there is $x \in X$ such that $\eta(x)=\cur$, we do the following: If $i'+1 < j'$, then let $f'(x)= \sntc^{<}_{i'+1}$. Otherwise, since $\rng(f) \setminus \{f(x) \mid \eta(x)=\cur\}$ contains at most $k-1$ colors from $\sntcset^<_\Ss$,  we can remove $\{x\mid \eta(x)=\cur\}$ from the domain of $f$ and adjust $f$ a bit so that $f(x_i)$ and $f(x_j)$ become non-adjacent, while preserving the order relation on $\sntcset^<_\Ss$. Let $f''$ denote the resulting function. Suppose $f''(x_i)=\sntc^<_{i''}$. Then let $f'(x)=\sntc^<_{i''+1}$  for each $x \in X$ such that $\eta(x)=\cur$.
%
\item For each $x \in X \setminus \dom(\eta)$, let $f'(x)=f(x)$. 
\end{itemize}

\item The guard $g' := g \wedge \cur < c_{min} \wedge  \cur < x_i$ such that $g' \wedge \varphi_f$ is satisfiable and $x_i$ is the minimum control variable wrt. $f$. The assignment $\eta'$ is the same as $\eta$, except that for each $x \in X$ such that $\eta(x)=x'$, let $\eta'(x)$ undefined.  The abstraction function $f'$ is defined as follows: Let $f(x_i)=\sntc^<_{i'}$.
\begin{itemize}
\item For each $x \in X$ such that $\eta(x)=x'$, let $f'(x)=f(x')$.
%
\item If there is $x \in X$ such that $\eta(x)=\cur$, we do the following: If $i' > 1$, then let $f'(x)= \sntc^{<}_{i'-1}$. Otherwise, since $\rng(f) \setminus \{f(x) \mid \eta(x)=\cur\}$ contains at most $k-1$ colors from $\sntcset^<_\Ss$,  we can remove $\{x \mid \eta(x)=\cur\}$ from the domain of $f$ and adjust $f$ a bit so that $f(x_i)$ become different from $1$, while preserving the order relation on $\sntcset^<_\Ss$. Let $f''$ denote the resulting function. Suppose $f''(x_i)=\sntc^<_{i''}$. Then let $f'(x)=\sntc^<_{i''-1}$ for each $x \in X$ such that $\eta(x)=\cur$. 

\item For each $x \in X \setminus \dom(\eta)$, let $f'(x)=f(x)$. 
\end{itemize}
%
\item The guard $g' := g \wedge \cur > c_{max} \wedge \cur = x$ for some $x$ such that $g' \wedge \varphi_f$ is satisfiable and $f(x)=\sntc^{>}_i$ for some $i$.  The assignment $\eta'$ is the restriction of  $\eta$ to $Y$. The abstraction function $f'$ is defined as follows: For each $x' \in X$ such that $\eta(x')=\cur$, let $f'(x')=f(x)$. Moreover, for each $x' \in X$ such that $\eta(x')=x''$, let $f'(x')=f(x'')$. 
%
\item The guard $g' := g \wedge \cur > c_{max} \wedge  x_i < \cur \wedge \cur < x_j$ such that $x_j$ is the successor of $x_i$ wrt. $f$ and $g' \wedge \varphi_f$ is satisfiable. The assignment $\eta'$ is the same as $\eta$, except that for each $x \in X$ such that $\eta(x)=x'$, let $\eta'(x)$ undefined.  The abstraction function $f'$ is defined as follows: Let $f(x_i)=\sntc^{>}_{i'}$ and $f(x_j)=\sntc^{>}_{j'}$. 
\begin{itemize}
\item For each $x \in X$ such that $\eta(x)=x'$, let $f'(x)=f(x')$.
%
\item If there is $x \in X$ such that $\eta(x)=\cur$, we do the following: If $i'+1 < j'$, then let $f'(x)= \sntc^{>}_{i'+1}$. Otherwise, since $\rng(f) \setminus \{f(x) \mid \eta(x)=\cur\}$ contains at most $k-1$ colors from $\sntcset^>_\Ss$,  we can remove $\{x \mid \eta(x)=\cur\}$ from the domain of $f$ and adjust $f$ a bit so that $f(x_i)$ and $f(x_j)$ become non-adjacent, while preserving the order relation on $\sntcset^>_\Ss$. Let $f''$ denote the resulting function. Suppose $f''(x_i)=\sntc^>_{i''}$. Then let $f'(x)=\sntc^>_{i''+1}$ for each $x \in X$ such that $\eta(x)=\cur$.

\item For each $x \in X \setminus \dom(\eta)$, let $f'(x)=f(x)$.  
\end{itemize}
%
\item The guard $g' := g \wedge \cur > c_{max} \wedge  \cur > x_i$ such that $g' \wedge \varphi_f$ is satisfiable and $x_i$ is the maximum control variable wrt. $f$. The assignment $\eta'$ is the same as $\eta$, except that for each $x \in X$ such that $\eta(x)=x'$, let $\eta'(x)$ undefined.  The abstraction function $f'$ is defined as follows: Let $f(x_i)=\sntc^>_{i'}$.
\begin{itemize}
\item For each $x \in X$ such that $\eta(x)=x'$, let $f'(x)=f(x')$.
%
\item If there is $x \in X$ such that $\eta(x)=\cur$, we do the following: If $i' < k$, then let $f'(x)= \sntc^{>}_{i'+1}$. Otherwise, since $\rng(f) \setminus \{f(x) \mid \eta(x)=\cur\}$ contains at most $k-1$ colors from $\sntcset^>_\Ss$,  we can remove $\{x \mid \eta(x)=\cur\}$ from the domain of $f$ and adjust $f$ a bit so that $f(x_i)$ become different from $k$, while preserving the order relation on $\sntcset^>_\Ss$. Let $f''$ denote the resulting function. Suppose $f''(x_i)=\sntc^>_{i''}$. Then let $f'(x)=\sntc^>_{i''+1}$ for each $x \in X$ such that $\eta(x)=\cur$. 

\item For each $x \in X \setminus \dom(\eta)$, let $f'(x)=f(x)$. 
\end{itemize}
\end{itemize}
\end{itemize}
\qed
\end{proof}

\section{Proofs in Section~\ref{sec-sum}}


\noindent {\bf Proposition~\ref{prop-sum-path}}.
{\it Suppose that $P$ is a path and the initial values of $X \cup Y$ are represented by a symbolic valuation $\initval$. Then the values of $X \cup Y$ after traversing the path $P$ are specified by a symbolic valuation $\sumf^{(P,\initval)}$ satisfying the following conditions.
\begin{itemize}
\item The set of indices of $X$, i.e., $[k]$, is partitioned into $I^{\circled{P}}_{pe}$ and $I^{\circled{P}}_{tr}$, the indices of \emph{persistent} and \emph{transient} control variables, respectively. A control variable is persistent if its value has not been changed while traversing $P$, otherwise, it is transient.
\item For each $x_j\in X$ such that $j\in I^{\circled{P}}_{pe}$, $\sumf^{(P,\initval)}(x_j)=\sval(x_j)$.
%
\item  For each $x_j\in X$ such that $j\in I^{\circled{P}}_{tr}$,
$\sumf^{(P,\initval)}(x_j)=\vard^{\circled{P}}_{\pi^{\circled{P}}(j)}$, where $\pi^{\circled{P}}: I^{\circled{P}}_{tr} \rightarrow [r^{\circled{P}}]$ is an injective mapping from the index of a transient control variable to the index of the data value assigned to it.
% 
\item For each $y_j \in Y$, 
$
 \sumf^{(P,\initval)}(y_j)  =
 \cste^{\circled{P}}_{j} + 
 \cstl^{\circled{P}}_j \initval(y_j)  + 
  \sum\limits_{j'\in [k]}\csta^{\circled{P}}_{j,j'}\initval(x_{j'}) +
  \sum\limits_{j''\in [r^{\circled{P}}]}\cstb^{\circled{P}}_{j,j''} \vard^{\circled{P}}_{j''}$,
\hide{
\item For each $y_j \in Y$, 
\[
\small
\begin{array}{l}
\smallskip
\sumf^{(P,\initval)}(y_j)  = \\
\hspace{2mm} \cste^{\circled{P}}_{j} + \cstl^{\circled{P}}_j \initval(y_j)  + \csta^{\circled{P}}_{j,1} \initval(x_1) + \dots + \csta^{\circled{P}}_{j,k} \initval(x_k) +  \cstb^{\circled{P}}_{j,1} \vard^{\circled{P}}_1 +\dots + \cstb^{\circled{P}}_{j,r^{\circled{P}}} \vard^{\circled{P}}_{r^{\circled{P}}},
\end{array}
\]} 
where $\cste^{\circled{P}}_j,\cstl^{\circled{P}}_j, \csta^{\circled{P}}_{j,1},\dots,\csta^{\circled{P}}_{j,k}, \cstb^{\circled{P}}_{j,1},\dots,\cstb^{\circled{P}}_{j,r^{\circled{P}}}$ are integer constants such that $\cstl^{\circled{P}}_{j} \in \{0,1\}$ (as a result of the ``independently evolving and copyless'' constraint).  It can happen that $\cstl^{\circled{P}}_j =0$,  which means that $\initval(y_j)$ is irrelevant to $\sumf^{(P,\initval)}(y_j)$. Similarly for $\csta^{\circled{P}}_{j,1}=0$, and so on.
\end{itemize}
}

\begin{proof}
Suppose that $\Ss=(Q,X,Y, \delta,q_0,O)$ is an (normalized) SNT. Suppose that $P=p_0 \xrightarrow{(g_1,\eta_1)} p_1 \dots p_{n-1} \xrightarrow{(g_n,\eta_n)} p_{n}$ is a path of $\Ss$ and $\initval$ is a symbolic valuation representing the  initial values of the control and data variables.  When $P$ is traversed in a run of $\Ss$ over a data word $w$,  the data value in a position of $w$ may have to be equal or unequal to the initial value of some control variable or some other data value in $w$ that have been met before (enforced by the guards and assignments in $P$). Let $\sim$ denote the equivalence relation on $[n]$ induced by $P$ such that $i \sim j$ iff the guards and assignments on $P$ enforce that the data value in the $i$-th position of $w$ must equal to that in the $j$-th position of $w$. Assuming that there are $r^{\circled{P}}$ equivalence classes of $\sim$, we use the variables $\vard^{\circled{P}}_1,\vard^{\circled{P}}_2,\dots, \vard^{\circled{P}}_{r^{\circled{P}}}$ to denote the data values met when traversing $P$, one for each equivalence class. 

We show by an induction that for each $i: 1 \le i \le n$, a symbolic valuation $\sumf_i$ over $X^+ \cup Y$ can be constructed  to describe the value of $x_j$ (resp. $y_j$) after going through the first $i$ transitions of $P$. Moreover, an index set $I_i \subseteq [k]$ is computed as well.
%
\begin{itemize}
\item Let $\sumf_0=\initval[\vard^{\circled{P}}_1/\cur]$ and $I_0 = \emptyset$.
%
%\item For each $x_j \in X$, if $x_j \in \dom(\eta_1)$, then $e_{1,x_j}=d^{(1)}_1$, otherwise, $e_{1,x_j}=d^{(0)}_j$. For each $y_j \in Y$, if $y_j \in \dom(\eta_1)$, then $e_{1,y_j} = \theta_0(\eta_{1}(y_j))$,
%otherwise, $e_{1,y_j}=o_j$. 
%
\item Let $i: 1 \le i \le n$. 
\begin{itemize}
\item For each $x_j \in X$, if $x_j \in \dom(\eta_i)$, then $\sumf_i(x_j)=\sumf_{i-1}(\cur)$ and $I_i= I_{i-1} \cup \{j\}$, otherwise, $\sumf_i(x_j) = \sumf_{i-1}(x_j)$ and and $I_i= I_{i-1}$. If $i < n$, suppose  the data value in the $(i+1)$-th position is represented by $\vard^{\circled{P}}_s$ for $1 \le s \le r^{\circled{P}}$, then let $\sumf_i(\cur)=\vard^{\circled{P}}_s$. Otherwise, $\sumf_i(\cur)=\bot$.
%
\item For each $y_j \in Y$, if $y_j \in \dom(\eta_i)$, then $\sumf_i(y_j) = \eval{\eta_i(y_j)}{\sumf_{i-1}}$, otherwise, $\sumf_i(y_j) =\sumf_{i-1}(y_j)$.
%
%\item For each $x_{j} \in X$ (resp. $y_j \in Y$), $\theta_i(x_{j})=e_{i,x_{j}}$ (resp. $\theta_i(y_{j})=e_{i, y_{j}}$). If $i < n$, then $\theta_i(\cur)=d^{(1)}_{s}$, where $1\le s \le r$ and $k+i + 1 \in I_s$, otherwise, $\theta_i(\cur)=\bot$.
\end{itemize} 
\end{itemize}
Then let $I^{\circled{P}}_{tr}:=I_n$ and $I^{\circled{P}}_{pe}:=[k] \setminus I^{\circled{P}}_{tr}$. The injective mapping $\pi^{\circled{P}}$ is defined as follows: For each $j \in I^{\circled{P}}_{tr}$, there is $s \in [r^{\circled{P}}]$ such that $\sumf_n(x_j)=\vard^{\circled{P}}_{s}$, let $\pi^{\circled{P}}(j)=s$. The symbolic valuation $\sumf^{(P,\initval)}$ can be defined as the restriction of $\sumf_n$ to $X \cup Y$. Since for each assignment $\eta_i$ and $y_j \in Y$, $\eta_i(y_j) = e$ or $\eta_i(y_j) = y_j +e$ for $e \in \Ee_{X^+}$, it follows that $\sumf^{(P,\initval)}(y_j)$ is of the form required by the proposition.
\qed
\end{proof}


\noindent {\bf Proposition~\ref{prop-sum-cycle}}.
{\it 
Suppose that $C$ is a cycle and $P=C^{\ell}$ such that $\ell \ge 2$. Then the symbolic valuation $\sumf^{(C^\ell,\initval)}$ to summarize the computation of $\Ss$ on $P$ is as follows,\medskip\\
\resizebox{\hsize}{!}{
$\begin{array}{l c l}
\sumf^{(C^\ell,\initval)}(y_j)  & = & 
\left(1 + \cstl^{\circled{C}}_{j} + \dots +(\cstl^{\circled{C}}_{j})^{\ell - 1} \right)\cste^{\circled{C}}_{j} + (\cstl^{\circled{C}}_{j})^\ell \initval(y_j) + \smallskip\\
%
& & \sum \limits_{j' \in I^{\circled{C}}_{pe}} \left(1+\cstl^{\circled{C}}_{j} + \dots +(\cstl^{\circled{C}}_{j})^{\ell - 1} \right) \csta^{\circled{C}}_{j,j'}\initval(x_{j'}) +  \sum \limits_{j' \in I^{\circled{C}}_{tr}}  (\cstl^{\circled{C}}_{j})^{\ell - 1} \csta^{\circled{C}}_{j,j'} \initval(x_{j'}) +  \\
%
& & \sum \limits_{j' \in \rng(\pi^{\circled{C}})} \sum \limits_{m\in[\ell -1]}
\left(  \csta^{\circled{C}}_{j, (\pi^{\circled{C}})^{-1}(j')} +(\cstl^{\circled{C}}_{j})\cstb^{\circled{C}}_{j,j'} \right)
(\cstl^{\circled{C}}_{j})^{\ell-m-1}
\vard^{\circled{C , m}}_{j'} +\\
%
& & \sum \limits_{j' \in [r^{\circled{C}}] \setminus \rng(\pi^{\circled{C}})}\sum \limits_{m\in[\ell -1]} \left((\cstl^{\circled{C}}_{j})^{\ell - m} \cstb^{\circled{C}}_{j,j'} \right) \vard^{\circled{C , m}}_{j'} + 
\sum \limits_{j' \in [r^{\circled{C}}] }  
 \cstb^{\circled{C}}_{j, j'} \vard^{\circled{C , \ell}}_{j'},
\end{array} 
$}\medskip\\
where the variables $\vard^{\circled{C , m}}_{1},\dots, \vard^{\circled{C , m}}_{r^{\circled{C}}}$ for $m\in [\ell-1]$
 represent the data values introduced when traversing $C$ for the $m$-th time.
}

%
\begin{proof}
We prove by an induction on $\ell$ that $\sumf^{(C^\ell,\initval)}(y_j)$ is of the desired form required by the proposition.

\noindent The induction base: $\ell=2$.

\smallskip

Let $\vard^{(\circled{C, 2})}_{1}, \dots, \vard^{(\circled{C, 2})}_{r^{\circled{C}}}$ be the data values introduced when traversing the cycle for the second time. Then from Corollary~\ref{cor-comp-two-paths}, we know that $\sumf^{(C^{2},\initval)} = \sumf^{(C,\sumf^{(C,\initval)})}$ is defined as follows: For each $y_j \in Y$,

\[
\begin{array}{rl}
	\medskip
	\sumf^{(C^{2},\initval)}(y_j) = & 
	\left(\cste^{\circled{C}}_{j}+
	\cstl^{\circled{C}}_{j} \cste^{\circled{C}}_{j}\right)+ \left(\cstl^{\circled{C}}_{j}\right)^2 \initval(y_j)+ \sum \limits_{j' \in I^{\circled{C}}_{pe}} 
	\left(1+\cstl^{\circled{C}}_{j}\right)  \csta^{\circled{C}}_{j,j'} \initval(x_{j'}) +\\
	\medskip
	& 
	\sum \limits_{j' \in  I^{\circled{C}}_{tr}} 
	 \cstl^{\circled{C}}_{j} \csta^{\circled{C}}_{j,j'}  \initval(x_{j'}) +
	\sum \limits_{j' \in \rng(\pi^{\circled{C}})} \left( \csta^{\circled{C}}_{j,(\pi^{\circled{C}})^{-1}(j')}+\cstl^{\circled{C}}_{j} \cstb^{\circled{C}}_{j,j'} \right) \vard^{\circled{C,1}}_{j'} + 
	 \\
	%
	\smallskip
	& 
	\sum \limits_{j' \in [r^{\circled{C}}]\setminus \rng(\pi^{\circled{C}})} \left( \cstl^{\circled{C}}_{j} \cstb^{\circled{C}}_{j,j'} \right) \vard^{\circled{C,1}}_{j'} +
	
	\sum \limits_{j'\in[r^{\circled{C}}]} \cstb^{\circled{C}}_{j,j'} \vard^{\circled{C,2}}_{j'}.
\end{array}
\]

\noindent Induction step: Let $\ell \ge 3$.

From the induction hypothesis, we know that for each $y_j \in Y$, $\sumf^{(C^{\ell-1},\initval)}(y_j)$ is of the desired form.

From Corollary~\ref{cor-comp-two-paths}, $\sumf^{(C^\ell,\initval)} = \sumf^{(C, \sumf^{(C^{\ell-1},\initval)})}$. Then for each $y_j \in Y$, by unfolding the expressions $\sumf^{(C^{\ell-1},\initval)}(x_{j'})$ for $j' \in [k]$ and $\sumf^{(C^{\ell-1},\initval)}(y_{j''})$ for $j'' \in [l]$ in $\sumf^{(C, \sumf^{(C^{\ell-1},\initval)})}(y_j)$, we can observe that $\sumf^{(C^\ell,\initval)}(y_j)$ is of the desired form.
\qed
\end{proof}

\section{Proofs in Section~\ref{sec-glasso}}



\noindent {\bf Lemma~\ref{prop-cycle-schm}}.
{\it 
Suppose $\schm=C_{i_1}^{\ell_1} C_{i_2}^{\ell_2} \dots C_{i_t}^{\ell_t}$ is a cycle scheme, and $\initval$ is a symbolic valuation representing the initial values of the control and data variables. 
For all $j' \in  I^{\circled{C_{i_{1}}}}_{pe}$, let $r_{j'}$ be the largest number $r \in [t]$ such that $j'\in\bigcap_{s\in[r]} I^{\circled{C_{i_{s}}}}_{pe}$, i.e., $x_{j'}$ remains persistent when traversing $C_{i_1}^{\ell_1} C_{i_2}^{\ell_2} \dots C_{i_{r_{j'}}}^{\ell_{r_{j'}}}$.
Then for each $j\in [l]$ and $j' \in  I^{\circled{C_{i_{1}}}}_{pe}$, the coefficient of the $\initval(x_{j'})$-atom in $\sumf^{(\schm,\initval)}(y_j)$ is 
\begin{center}
\resizebox{0.8\hsize}{!}{
$e+\sum\limits_{s_1\in[t]}  
\left(1+\lambda^{\circled{C_{i_{s_1}}}}_{j} + \dots + (\lambda^{\circled{C_{i_{s_1}}}}_{j})^{\ell_{s_1}-1} \right) \csta^{\circled{C_{i_{s_1}}}}_{j,j'}\prod\limits_{{s_2}\in[{s_1}+1,t]}\left(\lambda^{\circled{C_{i_{s_2}}}}_{j}\right)^{\ell_{s_2}}$},
\end{center}
where (1) $e\!=\!0$ when $r_{j'}\!=\!t$ and (2) $e=(\lambda^{\circled{C_{i_s}}}_{j})^{\ell_s-1} \csta^{\circled{C_{i_{s}}}}_{j,j'}$ with $s=r_{j'}+1$ when $r_{j'}<t$.\\
The constant atom of $\sumf^{(\schm,\initval)}(y_j)$ is 
\begin{center}
\resizebox{0.7\hsize}{!}{$
\sum\limits_{{s_1}\in[t]}
\left(1+\lambda^{\circled{C_{i_{s_1}}}}_{j} + \dots + (\lambda^{\circled{C_{i_{s_1}}}}_{j})^{\ell_{s_1}-1} \right)
\cste^{\circled{C_{i_{s_1}}}}_{j} 
\prod\limits_{{s_2}\in[{s_1}+1,t]}\left(\lambda^{\circled{C_{i_{s_2}}}}_{j}\right)^{\ell_{s_2}}$}
\end{center}
Moreover, for all $j \in [l]$, in $\sumf^{(\schm,\initval)}(y_j)$, only the constant atom and the coefficients of the $\initval(x_{j'})$-atoms with $j' \in  I^{\circled{C_{i_{1}}}}_{pe}$ contain a subexpression of the form $ \mu_\schm \ell_1$ for some $\mu_\schm\in \intnum$.
}

\begin{proof}
The lemma can be shown by applying Proposition~\ref{prop-sum-cycle}, Corollary~\ref{cor-comp-two-paths}, and an induction on the length $t$ of the cycle schemes.\qed
\end{proof}


\smallskip


\noindent {\bf Lemma~\ref{prop-bnd-domain-2}}.
{\it 
Suppose that the decision procedure has not returned yet after Step II. 
For all cycle scheme $\schm$ and $y_j \in Y$, the constant atom and the coefficients of all non-constant atoms in ${\sumf^{(\schm, \sumf^{(H,\initval_\bot)})}}^-(y_j)$ are from a finite set $U \subset \intnum$ comprises\\ 
(1)
the constant atom and the coefficients of the non-constant atoms in the expression ${\sumf^{(C^{\ell_i}_{i}, \sumf^{(H,\initval_\bot)})}}^-(y_j)$ for $i\in [n]$ and $\ell_i \in \{1,2\}$,\smallskip\\
(2) the numbers $\csta^{\circled{C_{s_2}}}_{j,j'} + \cstb^{\circled{C_{s_1}}}_{j,\pi^{\circled{C_{s_1}}}(j')}$ and $\csta^{\circled{C_{s_1}}}_{j, j''} + \csta^{\circled{C_{s_2}}}_{j,j''}$, where  $s_1,s_2 \in [n], j\in[l],j' \in I^{\circled{C_{s_1}}}_{tr} \cap I^{\circled{C_{s_2}}}_{tr},  j'' \in [k]$.
}

\begin{proof}
For each cycle scheme $\schm=C^{\ell_1}_{i_1} \dots C^{\ell_t}_{i_t}$ and each $y_j \in Y$, suppose for each $s\in [t]$, the data values introduced when traversing $C_{i_s}^{\ell_s}$ in $\schm$ are represented by the variables $\vard^{\circled{C_{i_s} , 1}}_{s,1}$, $\dots$, $\vard^{\circled{C_{i_s} , 1}}_{s,r^{\circled{C_{i_s}}}}$, $\dots$, $\vard^{\circled{C_{i_s} , \ell_s}}_{s,1}$, $\dots$, $\vard^{\circled{C_{i_s} , \ell_s}}_{s,r^{\circled{C_{i_s}}}}$. Then for each $y_j \in Y$,
 ${\sumf^{(\schm,\sumf^{(H,\initval_\bot)})}}^-(y_j)$ is a linear combination of $\vard^{\circled{H}}_1$, $\dots$, $\vard^{\circled{H}}_{r^{\circled{H}}}$, $\vard^{\circled{C_{i_1} , 1}}_{1,1}$, $\dots$, $\vard^{\circled{C_{i_1} , \ell_1}}_{1,r^{\circled{C_{i_1}}}}$, $\dots$, $\vard^{\circled{C_{i_t} , 1}}_{t,1}$, $\dots$, $\vard^{\circled{C_{i_t} , \ell_t}}_{t, r^{\circled{C_{i_t}}}}$. 

Suppose for each $y_j \in Y$,
\[
\begin{array}{l cl }
{\sumf^{(\schm,\sumf^{(H,\initval_\bot)})}}^-(y_j) &:= & (\cste^{\circled{\schm}}_{j})'  + (\csta^{\circled{\schm}}_{j,1})' \vard^{\circled{H}}_1 + \dots + (\csta^{\circled{\schm}}_{j,r^{\circled{H}}})' \vard^{\circled{H}}_{r^{\circled{H}}} + \\
& & (\cstb^{\circled{\schm,1}}_{1,1})' \vard^{\circled{C_{i_1},1}}_{1,1}  + \dots + (\cstb^{\circled{\schm,\ell_1}}_{1,r^{\circled{C_{i_1}}}})' \vard^{\circled{C_{i_1},\ell_1}}_{1,r^{\circled{C_{i_1}}}}  +  \\
& & \dots + \\
& & (\cstb^{\circled{\schm,1}}_{t,1})' \vard^{\circled{C_{i_{t}},1}}_{t,1} + \dots + (\cstb^{\circled{\schm,\ell_{t}}}_{t,r^{\circled{C_{i_{t}}}}})' \vard^{\circled{C_{i_{t}},\ell_{t}}}_{t, r^{\circled{C_{i_{t}}}}}.
\end{array}
\]

In the following, we show by induction on $t$ that for each cycle scheme $\schm=C^{\ell_1}_{i_1} \dots C^{\ell_t}_{i_t}$ and $y_j \in Y$, the following results hold.
\begin{enumerate}
\item In ${\sumf^{(\schm,\sumf^{(H,\initval_\bot)})}}^-(y_j)$, the constant atom and all the coefficients of the non-constant atoms are from $U$.
%
\item For each $\vard^{\circled{H}}_{j'}$ such that ${\sumf^{(\schm,\sumf^{(H,\initval_\bot)})}}^-(x_{j''})=\vard^{\circled{H}}_{j'}$ for some $j''  \in [k]$, the following fact holds: if there is $s \in [t]$ such that $\cstl^{\circled{C_{i_s}}}_j =0$, let $s_0$ be the maximum $s$ satisfying the constraint, then $(\csta^{\circled{\schm}}_{j,j'})'=\csta^{\circled{C_{i_{s_0}}}}_{j, j''}$, otherwise, $(\csta^{\circled{\schm}}_{j,j'})'= \beta^{\circled{H}}_{j,j'}$.
%
\item For each $s \in [t]$, $i \in [\ell_s]$, and $j' \in [r^{\circled{C_{i_s}}}]$ such that ${\sumf^{(\schm,\sumf^{(H,\initval_\bot)})}}^-(x_{j''})=\vard^{\circled{\schm, i}}_{s, j'}$ for some $j''  \in [k]$, it holds that $i = \ell_s$, $j'' \in I^{\circled{C_{i_s}}}_{tr}$, $j'' \in I^{\circled{C_{i_{s'}}}}_{pe}$ for each $s': s < s' \le t$, and the following fact holds: if there is $s': s < s' \le t$ such that $\cstl^{\circled{C_{i_{s'}}}}_j =0$, let $s'_0$ be the maximum $s'$ satisfying the constraint, then $(\cstb^{\circled{\schm, i}}_{j,j'})'=\csta^{\circled{C_{i_{s'_0}}}}_{j, j''}$, otherwise, $(\cstb^{\circled{\schm, i}}_{j,j'})'= \cstb^{\circled{C_{i_s}}}_{j,j'}$. 
\end{enumerate}

Induction base: $t=1$. 
\begin{itemize}
\item The first result: Follow from the definition of $U$. 

\item The second result: If $\cstl^{\circled{C_{i_1}}}_j = 0$, then $(\csta^{\circled{\schm}}_{j,j'})'=\csta^{\circled{C_{i_{1}}}}_{j, j''}$,  otherwise, $(\csta^{\circled{\schm}}_{j,j'})'=0$, since the expression $\csta^{\circled{C_{i_1}}}_{j,j''} \ell_1$ is removed from the coefficient of the $\sumf^{(H,\initval_\bot)}(x_{j''})$-atom in $\sumf^{(C^{\ell_1}_{i_1},\sumf^{(H,\initval_\bot)})}(y_j)$, i.e. the $\vard^{\circled{H}}_{j'}$-atom in $\sumf^{(C^{\ell_1}_{i_1},\sumf^{(H,\initval_\bot)})}(y_j)$. 

\item The third result: Suppose that $(\sumf^{(C^{\ell_1}_{i_1},\sumf^{(H,\initval_\bot)})})'(x_{j''})=\vard^{\circled{C_{i_1}, i}}_{1, j'}$ for $i \in [\ell_1]$, $j' \in [r^{\circled{C_{i_1}}}]$, and $j'' \in [k]$. Then $j'' \in I^{\circled{C_{i_1}}}_{tr}$, otherwise, $\vard^{\circled{C_{i_1}, i}}_{1, j'}$ would not be assigned to $x_{j''}$. From this, we deduce that $i = \ell_s$. Moreover, $(\csta^{\circled{C^{\ell_1}_{i_1}}}_{j,j'})'= \cstb^{\circled{C_{i_1}}}_{j,j'}$. 
\end{itemize}

Induction step: Suppose $t \ge 2$ and $\schm=C^{\ell_1}_{i_1} \dots C^{\ell_t}_{i_t}$.

Let $\schm_1= C^{\ell_1}_{i_1} \dots C^{\ell_{t-1}}_{i_{t-1}}$.  Then for each $y_j \in Y$, 
\[{\sumf^{(\schm,\sumf^{(H,\initval_\bot)})}}^-(y_j)=
{\sumf^{(C^{\ell_t}_{i_t},\ {\sumf^{(\schm_1, \sumf^{(H,\initval_\bot)})}}^-)}}^-(y_j).\] 

By the induction hypothesis, the three results hold for ${\sumf^{(\schm_1, \sumf^{(H,\initval_\bot)})}}^-$.

We illustrate the arguments for the case $\cstl^{\circled{C_{i_t}}}_{j} = 1$. The case $\cstl^{\circled{C_{i_t}}}_{j} = 0$ is simpler and can be discussed similarly. Suppose $y_j \in Y$.  In the following, we check that the constant atom and the coefficients of all the non-constant atoms of ${\sumf^{(\schm,\sumf^{(H,\initval_\bot)})}}^-(y_j)$ belong to $U$.  
%For brevity, we abbreviate an(resp. the) atom of ${\sumf^{(\schm,\sumf^{(H,\initval_\bot)})}}^-(y_j)$ as an (resp. the) atom below.
\begin{itemize}
	\item $(\cste^{\circled{\schm}}_{j})' = 0 + (\cstl^{\circled{C_{i_t}}}_{j})^{\ell_t} (\cste^{\circled{\schm_1}}_j)' = (\cste^{\circled{\schm_1}}_j)' \in U$ (here $\cste^{\circled{C_{i_t}}}_{j} \ell_t$ is removed).
	%
	\item For each $j' \in [r^{\circled{H}}]$ s.t. there exists no $j'' \in [k]$ satisfying that ${\sumf^{(\schm_1,\sumf^{(H,\initval_\bot)})}}^-(x_{j''})=\vard^{\circled{H}}_{j'}$, $(\csta^{\circled{\schm}}_{j, j'})' = (\cstl^{\circled{C_{i_t}}}_{j})^{\ell_t} (\csta^{\circled{\schm_1}}_{j, j'})' = (\csta^{\circled{\schm_1}}_{j, j'})' \in U$.
	%
	\item For each $j' \in [r^{\circled{H}}]$ such that ${\sumf^{(\schm_1,\sumf^{(H,\initval_\bot)})}}^-(x_{j''})=\vard^{\circled{H}}_{j'}$ for some $j''  \in I^{\circled{C_{i_t}}}_{pe}$, $(\csta^{\circled{\schm}}_{j, j'})' = 0 + (\cstl^{\circled{C_{i_t}}}_{j})^{\ell_t} (\csta^{\circled{\schm_1}}_{j, j'})'= (\csta^{\circled{\schm_1}}_{j, j'})' \in U$ (here $\csta^{\circled{C_{i_t}}}_{j,j'} \ell_t$ is removed). In this case, we have ${\sumf^{(\schm,\sumf^{(H,\initval_\bot)})}}^-(x_{j''})=\vard^{\circled{H}}_{j'}$. From the induction hypothesis, it is easy to see that the second result holds for $\schm$, $y_j$.
	%
	\item For each $j' \in [r^{\circled{H}}]$ such that ${\sumf^{(\schm_1,\sumf^{(H,\initval_\bot)})}}^-(x_{j''})=\vard^{\circled{H}}_{j'}$ for some $j''  \in I^{\circled{C_{i_t}}}_{tr}$, $(\csta^{\circled{\schm}}_{j, j'})' =(\cstl^{\circled{C_{i_t}}}_{j})^{\ell_t-1} \csta^{\circled{C_{i_t}}}_{j,j''} + (\cstl^{\circled{C_{i_t}}}_{j})^{\ell_t} (\csta^{\circled{\schm_1}}_{j, j'})' = \csta^{\circled{C_{i_t}}}_{j,j''} + (\csta^{\circled{\schm_1}}_{j, j'})' $. From the induction hypothesis, we know that either $(\csta^{\circled{\schm_1}}_{j, j'})' = \csta^{\circled{C_{i_{s_0}}}}_{j,j''}$ if there is $s$ such that $\cstl^{\circled{C_{i_s}}}_j =0$,  or otherwise, $(\csta^{\circled{\schm_1}}_{j, j'})'=\cstb^{\circled{H}}_{j, j'}$. Therefore, $(\csta^{\circled{\schm}}_{j, j'})' =\csta^{\circled{C_{i_t}}}_{j,j''} +  \csta^{\circled{C_{i_{s_0}}}}_{j,j''}$ or $\csta^{\circled{C_{i_t}}}_{j,j''}+ \cstb^{\circled{H}}_{j, j'}$. We conclude that $(\csta^{\circled{\schm}}_{j, j'})' \in U$.
	%
	\item For each $s \in [t-1]$, $i \in [\ell_s]$, and $j' \in [r^{\circled{C_{i_s}}}]$ such that there does not exist $j'' \in [k]$ satisfying that ${\sumf^{(\schm_1,\sumf^{(H,\initval_\bot)})}}^-(x_{j''}) = \vard^{\circled{C_{i_s}, i}}_{s, j'}$, it holds that $(\cstb^{\circled{\schm,i}}_{s,j'})' = (\cstl^{\circled{C_{i_t}}}_{j})^{\ell_t}  (\cstb^{\circled{\schm_1, i}}_{s, j'})'  =  (\cstb^{\circled{\schm_1, i}}_{s, j'})'  \in U$. 
	%
	\item For each $s \in [t-1]$, $i \in [\ell_s]$, and $j' \in [r^{\circled{C_{i_s}}}]$ such that ${\sumf^{(\schm_1,\sumf^{(H,\initval_\bot)})}}^-(x_{j''}) = \vard^{\circled{C_{i_s}, i}}_{s, j'}$ for some $j'' \in I^{\circled{C_{i_t}}}_{pe}$, $(\cstb^{\circled{\schm,i}}_{s,j'})' =(\cstl^{\circled{C_{i_t}}}_{j})^{\ell_t}  (\cstb^{\circled{\schm_1, i}}_{s, j'})'+ 0  =  (\cstb^{\circled{\schm_1, i}}_{s, j'})' \in U$ (here $\csta^{\circled{C_{i_t}}}_{j, j''} \ell_t$ is removed). Moreover, from the induction hypothesis, the third condition holds for $\schm$ and $y_j$.
	%
	\item For each $s \in [t-1]$, $i \in [\ell_s]$, and $j' \in [r^{\circled{C_{i_s}}}]$ such that ${\sumf^{(\schm_1,\sumf^{(H,\initval_\bot)})}}^-(x_{j''}) = \vard^{\circled{C_{i_s}, i}}_{s, j'}$ for some $j'' \in I^{\circled{C_{i_t}}}_{tr}$, it holds that $i = \ell_s$, $j'' \in I^{\circled{C_{i_s}}}_{tr}$, and for each $s': s < s' \le t$, $j'' \in I^{\circled{C_{i_{s'}}}}_{pe}$. Then $(\cstb^{\circled{\schm,i}}_{s,j'})' =(\cstl^{\circled{C_{i_t}}}_{j})^{\ell_t}  (\cstb^{\circled{\schm_1, i}}_{s, j'})' + \csta^{\circled{C_{i_t}}}_{j, j''}  =  (\cstb^{\circled{\schm_1, i}}_{s, j'})' + \csta^{\circled{C_{i_t}}}_{j, j''}$. From the induction hypothesis,  if there is $s': s < s' \le t-1$ such that $\cstl^{\circled{C_{i_{s'}}}}_j =0$, let $s'_0$ be the maximum $s'$ satisfying the constraint, then $(\cstb^{\circled{\schm_1, i}}_{j,j'})'=\csta^{\circled{C_{i_{s'_0}}}}_{j, j''}$, otherwise, $(\cstb^{\circled{\schm_1, i}}_{j,j'})'= \cstb^{\circled{C_{i_s}}}_{j,j'}$. Therefore, $(\cstb^{\circled{\schm,i}}_{s,j'})' = \csta^{\circled{C_{i_{s'_0}}}}_{j, j''} + \csta^{\circled{C_{i_t}}}_{j, j''}$ or $\cstb^{\circled{C_{i_s}}}_{j,j'} + \csta^{\circled{C_{i_t}}}_{j, j''}$, thus belongs to $U$. In this case, ${\sumf^{(\schm,\sumf^{(H,\initval_\bot)})}}^-(x_{j''})=\vard^{\circled{C_{i_t},\ell_t}}_{t, \pi^{\circled{C_{i_t}}}(j'')}$ and $(\cstb^{\circled{\schm,\ell_t}}_{t, \pi^{\circled{C_{i_t}}}(j'')})' = \cstb^{\circled{C_{i_t}}}_{j, \pi^{\circled{C_{i_t}}}(j'')}$. So the third condition holds for $\schm$ and $y_j$.
	%
	\item The coefficients of all the other atoms are those of  the $\vard^{\circled{C_{i_t}, i}}_{j, j'}$-atoms for $i \in [\ell_t]$ in ${\sumf^{(C^{\ell_t}_{i_t},\sumf^{(H,\initval_\bot)})}}^-(y_j)$.
\end{itemize} \qed

\end{proof}
%%%%%%%%%%%%%%%%%%%%%%%%%%%%%%%%%%%%%%%%%%%%%%%%%%%%%%%
%%%%%%%%%%%%%%%%%%the proof of the claim%%%%%%%%%%%%%%%%%%%%%%%%
%%%%%%%%%%%%%%%%%%%%%%%%%%%%%%%%%%%%%%%%%%%%%%%%%%%%%%%


\section{Extension of the decision procedure: Constants in guards}

We now consider the situation that the guards of the transitions in $\Ss$ may contain constants, that is, the atomic formulae of the form $cur \odot c$ for integer constants $c$. 

We illustrate the arguments for generalized lassos and adapt the decision procedure Step I-III to Step I$''$-III$''$ below. The arguments for the SNTs whose transition graphs are not necessarily generalized lassos are similar.

Suppose $H (C_1, \dots, C_n)$ is a generalized lasso, $H=q_0 \dots q_m$, and $O(q_m)=a_0 + a_1 x_1 + \dots + a_k x_k + b_1 y_1 + \dots + b_l y_l$.

From the guards and assignments of the transitions in $H$, we know that some of $\vard^{\circled{H}}_1,\dots, \vard^{\circled{H}}_{r^{\circled{H}}}$ have to be integer constants in $[c_{min}, c_{max}]$. Let $J^{\circled{H}} \subseteq [r^{\circled{H}}]$ denote the set of indices $j \in [r^{\circled{H}}]$ such that $\vard^{\circled{H}}_j$ has to be in $[c_{min}, c_{max}]$. Let $c^{\circled{H}}_j \in [c_{min}, c_{max}]$ denote this constant corresponding to $\vard^{\circled{H}}_j$. Since $\Ss$ is assumed to be normalized, we know that for each $j \not \in J^{\circled{H}}$,  $\vard^{\circled{H}}_j < c_{min}$ or $\vard^{\circled{H}}_j > c_{max}$.
%
%
%
%
\smallskip\\
\framebox[\textwidth]{
\begin{minipage}{0.95\textwidth}
\noindent {\bf Step I$''$}. For each $j \in [r^{\circled{H}}] \setminus J^{\circled{H}}$, decide whether the coefficient of the $\vard^{\circled{H}}_j$-atom in $\eval{O(q_m)}{\sumf^{(H,\initval_\bot)}}$ is nonzero. If the answer is yes, then return $\ltrue$. If the decision procedure has not returned yet, substitute each data variable $\vard^{\circled{H}}_j$ for $j \in J^{\circled{H}}$ with $c^{\circled{H}}_j$ in $\eval{O(q_m)}{\sumf^{(H,\initval_\bot)}}$. Then $\eval{O(q_m)}{\sumf^{(H,\initval_\bot)}}$ becomes an integer constant $c^{\circled{H}}$. If $c^{\circled{H}} \neq 0$, then return $\ltrue$.  Otherwise, go to Step II$''$.
\end{minipage}
}\bigskip

For each $i_1 \in [n]$, let $J^{\circled{q_m}}$ denote the set of indices $j' \in [k]$ such that $\initval(x_{j'})$ has to be in $[c_{min}, c_{max}]$, which is enforced by the state $q_m$. Recall that the SNT $\Ss$ is state-dominating. Therefore, from the state $q_m$, we know which control variable should have a value in $[c_{min},c_{max}]$. For each $j' \in J^{\circled{q_m}}$, let $c^{\circled{q_m}}_{j'}$ denote this constant corresponding to $j'$. 

Let $\schm=C_{i_1}^{\ell_1} C^{\ell_2}_{i_2} \dots C^{\ell_t}_{i_t}$ be a cycle scheme. 
For each $i_1 \in [n]$, let $J^{\circled{C_{i_1}}}$ denote the set of indices $j'' \in [r^{\circled{C_{i_1}}}]$ such that there is $c^{\circled{C_{i_1}}}_{j''} \in [c_{min}, c_{max}]$ satisfying that for each $i \in [\ell_1]$, $\vard^{\circled{C_{i_1}, i}}_{1, j''}$ has to be equal to $c^{\circled{C_{i_1}}}_{j''}$, as a result of the guards and assignments on $C_{i_1}$. (Recall that the SNT $\Ss$ is required to be normalized). Note that the definition of $J^{\circled{C_{i_1}}}$ is independent of the choices of $\schm$.

Suppose $\schm=C_{i_1}^{\ell_1} C^{\ell_2}_{i_2} \dots C^{\ell_t}_{i_t}$ is a cycle scheme. Then for each $j' \in  I^{\circled{C_{i_1}}}_{pe} \setminus J^{\circled{q_m}}$, we can obtain $\mu_{\schm, (i_1,j')} \ell_1 + \nu_{\schm, (i_1,j')}$ as in Step II of Section~\ref{sec-glasso}. 

We then define $\mu'_{\schm,(i_1,0)} \ell_1 + \nu'_{\schm,(i_1,0)}$ as follows. Because $\initval(x_{j'})$ is a constant for each $j' \in J^{\circled{q_m}}$, and for each $j \in [r^{\circled{C_{i_1}}}]$ and $i \in [\ell_1]$, $\vard^{\circled{C_{i_1}, i}}_{1, j}$ is a constant, we should take as a group the constant atom of  $\eval{O(q_m)}{\sumf^{(\schm,\initval)}}$, the $\initval(x_{j'})$-atoms for $j' \in J^{\circled{q_m}}$, and the $\vard^{\circled{C_{i_1}, i}}_{1, j}$-atoms for $j \in [r^{\circled{C_{i_1}}}]$ and $i \in [\ell_1]$.
%
Recall that the constant atom of $\eval{O(q_m)}{\sumf^{(\schm,\initval)}}$ contains the following  subexpression,
\begin{center}
	\vspace{-0.2cm}
	\resizebox{0.8\hsize}{!}{$
e_0:= \sum \limits_{1 \le j \le l} b_j
\begin{array}{l}
 \left((\lambda^{\circled{C_{i_2}}}_{j})^{\ell_2} \dots (\lambda^{\circled{C_{i_t}}}_{j})^{\ell_t}\right)
\left(1+\lambda^{\circled{C_{i_1}}}_{j} + \dots + (\lambda^{\circled{C_{i_1}}}_{j})^{\ell_1-1} \right) \cste^{\circled{C_{i_1}}}_{j}. 
\end{array}
$}
\end{center}
Moreover, for each $j' \in  J^{\circled{q_m}} \cap I^{\circled{C_{i_1}}}_{pe}$, the coefficient of the $\initval(x_{j'})$-atom of $\eval{O(q_m)}{\sumf^{(\schm,\initval)}}$ contains the following subexpression, 
\begin{center}
	\resizebox{0.8\hsize}{!}{$
e_{j'} := \sum \limits_{1 \le j \le l} 
%\begin{array}{l}
b_j \left((\cstl^{\circled{C_{i_2}}}_{j})^{\ell_2} \dots (\cstl^{\circled{C_{i_t}}}_{j})^{\ell_t}\right) 
\left(1+\cstl^{\circled{C_{i_1}}}_{j} + \dots + (\cstl^{\circled{C_{i_1}}}_{j})^{\ell_1-1} \right) \csta^{\circled{C_{i_1}}}_{j,j'}.
%\end{array} \hspace{1cm}  
$}
\end{center}

For each $j'' \in J^{\circled{C_{i_1}}} \cap \rng(\pi^{\circled{C_{i_1}}})$ and $i \in [\ell_1]$, the coefficient of the $\vard^{\circled{C_{i_1}, i}}_{1, j''}$-atom of $\eval{O(q_m)}{\sumf^{(\schm,\initval)}}$ contains the following subexpression, 
\begin{center}
	\resizebox{0.95\hsize}{!}{$
f_{j'',i} := \sum \limits_{1 \le j \le l} 
%\begin{array}{l}
 b_j \left((\cstl^{\circled{C_{i_2}}}_{j})^{\ell_2} \dots (\cstl^{\circled{C_{i_t}}}_{j})^{\ell_t}\right) 
\left((\cstl^{\circled{C_{i_1}}}_{j})^{\ell_1-i-1} \csta^{\circled{C_{i_1}}}_{j, (\pi^{\circled{C_{i_1}}})^{-1}(j'') } + (\cstl^{\circled{C_{i_1}}}_{j})^{\ell_1-i} \cstb^{\circled{C_{i_1}}}_{j,j'}\right).
%\end{array}
$}
\end{center}

For each $j'' \in J^{\circled{C_{i_1}}} \setminus \rng(\pi^{\circled{C_{i_1}}})$ and $i \in [\ell_1]$, the coefficient of the $\vard^{\circled{C_{i_1}, i}}_{1, j''}$-atom of $\eval{O(q_m)}{\sumf^{(\schm,\initval)}}$ contains the following subexpression, 
\begin{center}
	\resizebox{0.75\hsize}{!}{$
f_{j'',i} := \sum \limits_{1 \le j \le l} 
%\begin{array}{l}
 b_j \left((\cstl^{\circled{C_{i_2}}}_{j})^{\ell_2} \dots (\cstl^{\circled{C_{i_t}}}_{j})^{\ell_t}\right) 
\left( (\cstl^{\circled{C_{i_1}}}_{j})^{\ell_1-i} \cstb^{\circled{C_{i_1}}}_{j,j'}\right).
%\end{array}
$}
\end{center}

Then we consider the expression 
\[e_0 + \sum \limits_{j' \in J^{\circled{q_m}} \cap I^{\circled{C_{i_1}}}_{pe}} (e_{j'} c^{\circled{q_m}}_{j'}) + \sum \limits_{j'' \in J^{\circled{C_{i_1}}} } \sum \limits_{i \in [\ell_1]} (f_{j'',i}\ c^{\circled{C_{i_1}}}_{j''}),\] 
which can be rewritten as $\mu'_{\schm,(i_1,0)} \ell_1 + \nu'_{\schm,(i_1,0)}$ for some integer constants $\mu'_{\schm, (i_1,0)}$ and $\nu'_{\schm, (i_1,0)}$. 
%If $\mu'_{\schm,(i_1,0)}=\mu_{\schm,(i_1,j')}=0$ for all $j' \in J^{\circled{q_m}} \cap I^{\circled{C_{i_1}}}_{pe}$, then we can ignore all subexpressions containing the cycle counter variable $\ell_1$ in   $\eval{O(q_m)}{\sumf^{(\schm,\initval)}}$, i.e., the subexpressions $\mu'_{\schm,(i_1,0)}\ell_1$ and $\mu_{\schm,(i_1,j')}\ell_1$ for all $j' \in J^{\circled{q_m}} \cap  I^{\circled{C_{i_1}}}_{pe}$.
%
\medskip\\
\framebox[\textwidth]{
	\begin{minipage}{0.95\textwidth}
		\noindent {\bf Step II$''$}. For each $i_1 \in [n]$, check all cycle scheme $\schm=C_{i_1}^{\ell_1} C_{i_2} \dots C_{i_t}$ such that $i_2,\dots,i_t$ are mutually distinct. There are only finitely many this kind of cycle schemes. If 
		one of the following constraints is satisfied, then return $\ltrue$. (1) There is $j' \in  I^{\circled{C_{i_1}}}_{pe} \setminus J^{\circled{q_m}}$ such that $\mu_{\schm,(i_1,j')} \neq 0$. (2) $\mu'_{\schm,(i_1,0)} \neq 0$.
		%
		If the decision procedure has not returned yet, then go to Step III$''$.
	\end{minipage}
}\smallskip\\

If after Step II$''$, the algorithm has not return yet, then similarly to Section~\ref{sec-glasso}, we can construct a finite set $U'' \subset \intnum$, which acts as a bounded domain for the constant atoms and the coefficients of non-constant atoms after removing all the subexpressions related to the cycle counting variables $\ell_1,\dots,\ell_t$. Moreover, we can construct the set $\mathscr{A''}$ of abstractions of cycle schemes.
%
\medskip\\
\framebox[\textwidth]{
	\begin{minipage}{0.95\textwidth}
		\noindent {\bf Step III$''$}. Similar to Step III, with $U,\mathscr{A}$ replaced by $U''$ and $\mathscr{A''}$.
	\end{minipage}
}\smallskip\\

\end{appendix}

%%%%%%%%%%%%%%%%%%%%%%%%%%%%%%%%%%%%%%%%%%%%%%%%%%%%%
%%%%%%%%%%%%%%%%%%%%%%%%%%%%%%%%%%%%%%%%%%%%%%%%%%%%%%%%%%%





\end{document}