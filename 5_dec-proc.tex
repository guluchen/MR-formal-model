%!TEX root = main-cav.tex

\section{Decision procedure for the non-zero output problem}\label{sec:dec-snt}
%
We prove our main result, Theorem~\ref{thm:correctness}, by presenting a decision procedure for the non-zero output problem of SNTs. We fix a normalized SNT $\Ss = (Q,X,Y,\delta,q_0,O)$ such that $X=\{ x_1,\dots, x_k\}$ and $Y = \{y_1,\dots,y_l\}$. Due to space constraint, we only present a simplified version where the transition guards are constant-free and leave the procedure for the general case in the appendix.
We define summaries of the computations of $\Ss$ on paths and cycles in Section~\ref{sec-sum}. We then present a decision procedure for the case that the transition graph of $\Ss$ is a \emph{generalized lasso} in Section~\ref{sec-glasso}. The transition graph of $\Ss$ is said to be a generalized lasso if it comprises a handle $H=q_0 q_1 \dots q_m$ and a collection of simple cycles $C_1,\dots,C_n$ such that $q_m$ is the unique state shared by each pair of distinct cycles from $\{C_1,\dots,C_n\}$. We will generalize the procedure to full SNTs in Section~\ref{sec-gflat}.

\begin{theorem}\label{thm:correctness}
The non-zero output problem of SNTs is decidable.
\end{theorem}


\subsection{Summarization of the computations on paths and cycles}\label{sec-sum}

Suppose $P=p_0 \xrightarrow{(g_1,\eta_1)} p_1 \dots p_{n-1} \xrightarrow{(g_n,\eta_n)} p_{n}$ is a path of $\Ss$. We assume that the initial values of the control and data variables are represented by a symbolic valuation $\sval$ over $X \cup Y$. When $P$ is traversed in a run of $\Ss$ over a data word $w$,  the data value in a position of $w$ may have to be (un)equal to the initial value of some control variable or some other data value in $w$ that have been met before (enforced by the guards and assignments in $P$). Let $\sim$ denote the equivalence relation on $[n]$ induced by $P$ such that $i \sim j$ iff the guards and assignments on $P$ enforce that the data value in the $i$-th position of $w$ must be equal to that in the $j$-th position of $w$. Assuming that there are $r^{\circled{P}}$ equivalence classes of $\sim$, we use the variables $\vard^{\circled{P}}_1,\vard^{\circled{P}}_2,\dots, \vard^{\circled{P}}_{r^{\circled{P}}}$ to denote the data values met when traversing $P$, one for each equivalence class. Note here we use the superscript ${\circled{P}}$ to denote the fact that $r^{\circled{P}}$ (resp. $\vard^{\circled{P}}_1$, $\dots$) is associated with the path $P$.

\begin{proposition}\label{prop-sum-path}
Suppose that $P$ is a path and the initial values of $X \cup Y$ are represented by a symbolic valuation $\initval$. Then the values of $X \cup Y$ after traversing the path $P$ are specified by a symbolic valuation $\sumf^{(P,\initval)}$ satisfying the following conditions.
\begin{itemize}
\item The set of indices of $X$, i.e., $[k]$, is partitioned into $I^{\circled{P}}_{pe}$ and $I^{\circled{P}}_{tr}$, the indices of \emph{persistent} and \emph{transient} control variables, respectively. A control variable is persistent if its value has not been changed while traversing $P$, otherwise, it is transient.
\item For each $x_j\in X$ such that $j\in I^{\circled{P}}_{pe}$, $\sumf^{(P,\initval)}(x_j)=\sval(x_j)$.
%
\item  For each $x_j\in X$ such that $j\in I^{\circled{P}}_{tr}$,
$\sumf^{(P,\initval)}(x_j)=\vard^{\circled{P}}_{\pi^{\circled{P}}(j)}$, where $\pi^{\circled{P}}: I^{\circled{P}}_{tr} \rightarrow [r^{\circled{P}}]$ is an injective mapping from the index of a transient control variable to the index of the data value assigned to it.
% 
\item For each $y_j \in Y$, 
$
 \sumf^{(P,\initval)}(y_j)  =
 \cste^{\circled{P}}_{j} + 
 \cstl^{\circled{P}}_j \initval(y_j)  + 
  \sum\limits_{j'\in [k]}\csta^{\circled{P}}_{j,j'}\initval(x_{j'}) +
  \sum\limits_{j''\in [r^{\circled{P}}]}\cstb^{\circled{P}}_{j,j''} \vard^{\circled{P}}_{j''}$,
\hide{
\item For each $y_j \in Y$, 
\[
\small
\begin{array}{l}
\smallskip
\sumf^{(P,\initval)}(y_j)  = \\
\hspace{2mm} \cste^{\circled{P}}_{j} + \cstl^{\circled{P}}_j \initval(y_j)  + \csta^{\circled{P}}_{j,1} \initval(x_1) + \dots + \csta^{\circled{P}}_{j,k} \initval(x_k) +  \cstb^{\circled{P}}_{j,1} \vard^{\circled{P}}_1 +\dots + \cstb^{\circled{P}}_{j,r^{\circled{P}}} \vard^{\circled{P}}_{r^{\circled{P}}},
\end{array}
\]} 
where $\cste^{\circled{P}}_j,\cstl^{\circled{P}}_j, \csta^{\circled{P}}_{j,1},\dots,\csta^{\circled{P}}_{j,k}, \cstb^{\circled{P}}_{j,1},\dots,\cstb^{\circled{P}}_{j,r^{\circled{P}}}$ are integer constants such that $\cstl^{\circled{P}}_{j} \in \{0,1\}$ (as a result of the ``independently evolving and copyless'' constraint).  It can happen that $\cstl^{\circled{P}}_j =0$,  which means that $\initval(y_j)$ is irrelevant to $\sumf^{(P,\initval)}(y_j)$. Similarly for $\csta^{\circled{P}}_{j,1}=0$, and so on.
\end{itemize}
\end{proposition}
In Proposition~\ref{prop-sum-path}, the sets $I^{\circled{P}}_{pe}$, $I^{\circled{P}}_{tr}$, the mapping $\pi^{\circled{P}}$, and the constants $\cste^{\circled{P}}_j,\cstl^{\circled{P}}_j, \dots, \cstb^{\circled{P}}_{j,r^{\circled{P}}}$ only depend on $P$ and are independent of $\initval$. Due to the uniquely-valued constraint of normalized SNTs, $\pi^{\circled{P}}$ is injective, and the inverse function of $\pi^{\circled{P}}$, denoted $(\pi^{\circled{P}})^{-1}$, exists.

As a corollary of Proposition~\ref{prop-sum-path}, the following result demonstrates how to summarize the computations of $\Ss$ on the composition of two paths.

\begin{corollary}\label{cor-comp-two-paths}
Suppose that $P_1$ and $P_2$ are two paths in $\Ss$ such that the last state of $P_1$ is the first state of $P_2$. Moreover, let $\sumf^{(P_1, \initval)}$ (resp. $\sumf^{(P_2, \initval)}$) be the symbolic valuation summarizing the computation of $\Ss$ on $P_1$ (resp. $P_2$). Then the symbolic valuation summarizing the computation of $\Ss$ on $P_1 P_2$ is $\sumf^{(P_2,\ \sumf^{(P_1,\initval)})}$.
\end{corollary}

In order to get a better understanding of the relation between $\sumf^{(P_2,\ \sumf^{(P_1,\initval)})}$ and $(\sumf^{(P_1, \initval)},\sumf^{(P_2, \initval)})$, in the following, for each $y_j \in Y$, we obtain a more explicit form of the expression $\sumf^{(P_2,\ \sumf^{(P_1,\initval)})}(y_j)$, by unfolding therein the expression $\sumf^{(P_1,\initval)}$\medskip.
\resizebox{\hsize}{!}{
	$\begin{array}{rl}
	\medskip
	\sumf^{(P_2,\ \sumf^{(P_1,\initval)})}(y_j) = & 
	\left(\cste^{\circled{P_2}}_{j}+
	\cstl^{\circled{P_2}}_{j} \cste^{\circled{P_1}}_{j}\right)+ \left(\cstl^{\circled{P_2}}_{j} \cstl^{\circled{P_1}}_{j} \right) \initval(y_j)+ \sum \limits_{j' \in I^{\circled{P_1}}_{pe}} 
	\left(\csta^{\circled{P_2}}_{j,j'} +\cstl^{\circled{P_2}}_{j} \csta^{\circled{P_1}}_{j,j'}\right) \initval(x_{j'}) +\\
	\medskip
	& 
	\sum \limits_{j' \in  I^{\circled{P_1}}_{tr}} 
	\left(\cstl^{\circled{P_2}}_{j} \csta^{\circled{P_1}}_{j,j'} \right) \initval(x_{j'}) +
	\sum \limits_{j' \in \rng(\pi^{\circled{P_1}})} \left( \csta^{\circled{P_2}}_{j,(\pi^{\circled{P_1}})^{-1}(j')}+\cstl^{\circled{P_2}}_{j} \cstb^{\circled{P_1}}_{j,j'} \right) \vard^{\circled{P_1}}_{j'} + 
	 \\
	%
	\smallskip
	& 
	\sum \limits_{j' \in [r^{\circled{P_1}}]\setminus \rng(\pi^{\circled{P_1}})} \left( \cstl^{\circled{P_2}}_{j} \cstb^{\circled{P_1}}_{j,j'} \right) \vard^{\circled{P_1}}_{j'} +
	
	\sum \limits_{j'\in[r^{\circled{P_2}}]} \cstb^{\circled{P_2}}_{j,j'} \vard^{\circled{P_2}}_{j'}.
	\end{array}$
}\medskip\\
In the equation, $j'\in  I^{\circled{P_1}}_{pe}$ implies that $x_{j'}$ remains unchanged when traversing $P_1$, which means the initial value of $x_{j'}$ before traversing $P_2$ is still $\initval(x_{j'})$ and therefore we have the item $ (\csta^{\circled{P_2}}_{j,j'}) \initval(x_{j'})$. When $j' \in \rng(\pi^{\circled{P_1}})$, the initial value of $x_{(\pi^{\circled{P_1}})^{-1}(j')}$ before traversing $P_2$ is $\vard^{\circled{P_1}}_{j'}$ and therefore we have the item $( \csta^{\circled{P_2}}_{j,(\pi^{\circled{P_1}})^{-1}(j')}) \vard^{\circled{P_1}}_{j'}$.
For all $j'\in [k] = I^{\circled{P_1}}_{pe} \cup I^{\circled{P_1}}_{tr}$, we have the item $(\cstl^{\circled{P_2}}_{j} \csta^{\circled{P_1}}_{j,j'}) \initval(x_{j'})$, i.e. the coefficient of $\initval(x_{j'})$ in $\sumf^{(P_1, \initval)}$ multiplied by $\cstl^{\circled{P_2}}_{j}$. Moreover, for all $j'\in [r^{\circled{P_1}}] = \rng(\pi^{\circled{P_1}}) \cup ([r^{\circled{P_1}}] \setminus \rng(\pi^{\circled{P_1}}))$, we have 
the item $( \cstl^{\circled{P_2}}_{j} \cstb^{\circled{P_1}}_{j,j'}) \vard^{\circled{P_1}}_{j'}$, i.e. the coefficient of $\vard^{\circled{P_1}}_{j'}$ in $\sumf^{(P_1, \initval)}$ multiplied by $\cstl^{\circled{P_2}}_{j}$.

In the following, by utilizing Proposition~\ref{prop-sum-path} and Corollary~\ref{cor-comp-two-paths}, for each path $C^{\ell}$ which is obtained by iterating a cycle $C$ for $\ell$ times, we illustrate how $\sumf^{(C^\ell,\initval)}$ is related to $\sumf^{(C, \initval)}$ and $\ell$. For convenience, we call $\ell$ a \emph{loop counter variable}.

\begin{proposition}\label{prop-sum-cycle}
Suppose that $C$ is a cycle and $P=C^{\ell}$ such that $\ell \ge 2$. Then the symbolic valuation $\sumf^{(C^\ell,\initval)}$ to summarize the computation of $\Ss$ on $P$ is as follows,\medskip\\
\resizebox{\hsize}{!}{
$\begin{array}{l c l}
\sumf^{(C^\ell,\initval)}(y_j)  & = & 
\left(1 + \cstl^{\circled{C}}_{j} + \dots +(\cstl^{\circled{C}}_{j})^{\ell - 1} \right)\cste^{\circled{C}}_{j} + (\cstl^{\circled{C}}_{j})^\ell \initval(y_j) + \smallskip\\
%
& & \sum \limits_{j' \in I^{\circled{C}}_{pe}} \left(1+\cstl^{\circled{C}}_{j} + \dots +(\cstl^{\circled{C}}_{j})^{\ell - 1} \right) \csta^{\circled{C}}_{j,j'}\initval(x_{j'}) +  \sum \limits_{j' \in I^{\circled{C}}_{tr}}  (\cstl^{\circled{C}}_{j})^{\ell - 1} \csta^{\circled{C}}_{j,j'} \initval(x_{j'}) +  \\
%
& & \sum \limits_{j' \in \rng(\pi^{\circled{C}})} \sum \limits_{s\in[\ell -1]}
\left(  \csta^{\circled{C}}_{j, (\pi^{\circled{C}})^{-1}(j')} +(\cstl^{\circled{C}}_{j})\cstb^{\circled{C}}_{j,j'} \right)
(\cstl^{\circled{C}}_{j})^{\ell-s-1}
\vard^{\circled{C , s}}_{j'} +\\
%
& & \sum \limits_{j' \in [r^{\circled{C}}] \setminus \rng(\pi^{\circled{C}})}\sum \limits_{s\in[\ell -1]} \left((\cstl^{\circled{C}}_{j})^{\ell - s} \cstb^{\circled{C}}_{j,j'} \right) \vard^{\circled{C , s}}_{j'} + 
\sum \limits_{j' \in [r^{\circled{C}}] }  
 \cstb^{\circled{C}}_{j, j'} \vard^{\circled{C , \ell}}_{j'},
\end{array} 
$}\medskip\\
where the variables $\vard^{\circled{C , s}}_{1},\dots, \vard^{\circled{C ,s}}_{r^{\circled{C}}}$ for $s\in [\ell]$
 represent the data values introduced when traversing $C$ for the $s$-th time.
\end{proposition}

From Proposition~\ref{prop-sum-cycle} and the fact that $\cstl_{j} \in \{0, 1\}$, we have the following observation.
\begin{itemize}
\item If $\cstl^{\circled{C}}_{j}=0$, then\medskip\\
\resizebox{0.9\hsize}{!}{$
\sumf^{(C^\ell,\initval)}(y_j)   =  \cste^{\circled{C}}_{j} +  \sum \limits_{j' \in I^{\circled{C}}_{pe}} \csta^{\circled{C}}_{j,j'} \initval(x_{j'}) +
\sum \limits_{j'  \in \rng(\pi^{\circled{C}})} \csta^{\circled{C}}_{j, (\pi^{\circled{C}})^{-1}(j')}\ \vard^{\circled{C , \ell  -  1}}_{j'} + \sum \limits_{j' \in [r^{\circled{C}}] }  
\cstb^{\circled{C}}_{j, j'} \vard^{\circled{C , \ell}}_{j'}.$}


\item If $\cstl^{\circled{C}}_{j}=1$, then\medskip\\
\resizebox{0.95\hsize}{!}{$
\begin{array}{l }
\sumf^{(C^\ell,\initval)}(y_j)  =    \ell \cste^{\circled{C}}_{j}  + \initval(y_j) +   \sum  \limits_{j' \in I^{\circled{C}}_{pe}} \ell \csta^{\circled{C}}_{j,j'}  \initval(x_{j'}) + 
\sum \limits_{j' \in I^{\circled{C}}_{tr}} \csta^{\circled{C}}_{j,j'} \initval(x_{j'}) +  \smallskip\\
\sum \limits_{j' \in \rng(\pi^{\circled{C}})} \sum \limits_{s\in[\ell -1]}
\left(\csta^{\circled{C}}_{j, (\pi^{\circled{C}})^{-1}(j')} + \cstb^{\circled{C}}_{j,j'} \right) \vard^{\circled{C , s}}_{j'} + \sum \limits_{j' \in [r^{\circled{C}}]  \setminus \rng(\pi^{\circled{C}}) }\sum \limits_{s\in[\ell -1]} 
\cstb^{\circled{C}}_{j,j'} \vard^{\circled{C , s}}_{j'} + \sum \limits_{j' \in [r^{\circled{C}}] }  
\cstb^{\circled{C}}_{j, j'} \vard^{\circled{C , \ell}}_{j'}.
\end{array}
$}
%
\hide{
\item If $\alpha^{\circled{C}}_{j,1}=-1$ and $\ell$ is even, then
\[
\begin{array}{l c l}
\chi^{\circled{C}}_{\ell}(y_j)  & = &  o_j + \sum \limits_{j'\le k, \pi_C(j') \neq j'} (-\beta^{\circled{C}}_{j,j'}) d^{(0)}_{j'} +  \\
%
& & \sum \limits_{j' \le r_C ,  j'+k \in \rng(\pi_C)} ( \beta^{\circled{C}}_{j, \pi_C^{-1}(j'+k)} - \gamma^{\circled{C}}_{j,j'}) d^{\circled{C , 1}}_{j'} + \\
%
& & \sum \limits_{j' \le r_C ,   j'+k \not \in \rng(\pi_C)} (-\gamma^{\circled{C}}_{j,j'}) d^{\circled{C , 1}}_{j'} + \dots + \\
%
& & \sum \limits_{j' \le r_C ,  j'+k \in \rng(\pi_C)} (\beta^{\circled{C}}_{j, \pi_C^{-1}(j'+k)}-\gamma^{\circled{C}}_{j,j'}) d^{\circled{C , \ell  -  1}}_{j'} + \\
%
& & \sum \limits_{j' \le r_C ,   j'+k \not \in \rng(\pi_C)} (-\gamma^{\circled{C}}_{j,j'}) d^{\circled{C , \ell  -  1}}_{j'} + \gamma^{\circled{C}}_{j, 1} d^{\circled{C , \ell}}_{1} + \dots + \gamma^{\circled{C}}_{j,r_C} d^{\circled{C , \ell}}_{r_C}.
\end{array} 
\]
\item If $\alpha_{j,1}=-1$ and $\ell$ is odd, then
\[
\begin{array}{l c l}
\smallskip
\chi^{\circled{C}}_{\ell}(y_j)  & = &  \alpha^{\circled{C}}_{j,0} - o_j + \sum \limits_{j' \le k, \pi_C(j')=j'} \beta^{\circled{C}}_{j,j'} d^{(0)}_{j'} +  \sum \limits_{j'\le k, \pi_C(j') \neq j'}  \beta^{\circled{C}}_{j,j'} d^{(0)}_{j'} +  \\
%
& & \sum \limits_{j' \le r_C ,  j'+k \in \rng(\pi_C)} ( -\beta^{\circled{C}}_{j, \pi_C^{-1}(j'+k)} +\gamma^{\circled{C}}_{j,j'}) d^{\circled{C , 1}}_{j'} + \\
%
& & \sum \limits_{j' \le r_C ,   j'+k \not \in \rng(\pi_C)} \gamma^{\circled{C}}_{j,j'} d^{\circled{C , 1}}_{j'} + \dots + \\
%
& & \sum \limits_{j' \le r_C ,  j'+k \in \rng(\pi_C)} (\beta^{\circled{C}}_{j, \pi_C^{-1}(j'+k)}-\gamma^{\circled{C}}_{j,j'}) d^{\circled{C , \ell  -  1}}_{j'} + \\
%
& & \sum \limits_{j' \le r_C ,   j'+k \not \in \rng(\pi_C)} (-\gamma^{\circled{C}}_{j,j'}) d^{\circled{C , \ell  -  1}}_{j'} + \gamma^{\circled{C}}_{j, 1} d^{\circled{C , \ell}}_{1} + \dots + \gamma^{\circled{C}}_{j,r_C} d^{\circled{C , \ell}}_{r_C}.
\end{array} 
\]
}
\end{itemize}
%
%From the analysis above, we observe that in $\chi^{\circled{C}}_\ell(y_j)$, 
%\begin{itemize}
%\item the constant coefficient is either $\alpha^{\circled{C}}_{j,0}$, or $\alpha^{\circled{C}}_{j,0} \ell$, 
%
%\item the coefficient of $o_j$ is $0$, or $1$, 
%
%\item for each data value $d^{(0)}_{j'}$, the coefficient of $d^{(0)}_{j'}$ is either $\beta^{\circled{C}}_{j,j'}$, or $0$, or $\beta^{\circled{C}}_{j,j'} \ell$,
%
%\item for each data value $d^{(C , i)}_{j'}$ with $i \ge 1$, the coefficient of $d^{(C , i)}_{j'}$ is either $0$, or $\beta^{\circled{C}}_{j, \pi_C^{-1}(j'+k)}$, or $\beta^{\circled{C}}_{j, \pi_C^{-1}(j'+k)}+\gamma^{\circled{C}}_{j,j'}$, or $\gamma^{\circled{C}}_{j,j'}$.
%\end{itemize}


\subsection{Decision procedure for generalized lassos}\label{sec-glasso}
%
Before presenting the decision procedure for generalized lassos, we introduce some notations.
Let $e$ be an expression consists of symbolic values $\initval(z)$ for $z\in X\cup Y$ and data variables $\vard_1, \dots, \vard_{s_2}$. More specifically, let $e:=\mu_0 + \mu_1 \initval(z_1) +\dots + \mu_{s_1} \initval(z_{s_1}) + \xi_1 \vard_1 + \dots + \xi_{s_2} \vard_{s_2}$,
such that $\mu_0,\mu_1,\dots,\mu_{s_1}, \xi_1,\dots,\xi_{s_2}$ are expressions containing only constants and loop counter variables.
Then we call $\mu_0$ as the \emph{constant atom}, $\mu_i \initval(z_i)$ the $\initval(z_i)$-atom for $i\in[s_1]$, and $\xi_j \vard_j$ the $\vard_j$-atom for $j\in[s_2]$ of the expression $e$. Moreover, $\mu_1, \dots, \mu_{s_1}, \xi_1,\dots, \xi_{s_2}$ are called the \emph{coefficients} of these atoms. A non-constant atom is said to be \emph{nontrivial} if its coefficient is \emph{not} identical to zero.

In the rest of this subsection, we assume that the transition graph of $\Ss$ comprises a handle $H=q_0 q_1 \dots q_m$ and a collection of simple cycles $C_1,\dots,C_n$ such that $q_m$ is the unique state shared by each pair of distinct cycles from $\{C_1,\dots,C_n\}$. Moreover, without loss of generality, we assume that $O(q_m) = a_0 + a_1 x_1 + \dots + a_k x_k + b_1 y_1 + \dots + b_l y_l$, and $O(q)$ is undefined for all the other states $q$.

A \emph{cycle scheme} $\schm$ is a path $C_{i_1}^{\ell_1} C_{i_2}^{\ell_2} \dots C_{i_t}^{\ell_t}$ such that $i_1,\dots,i_t \in [n]$, $\ell_1,\dots, \ell_t \ge 1$, and for each $j\in [t-1]$, $i_j \neq i_{j+1}$. Intuitively, $\schm$ is a path obtained by iterating $C_{i_1}$ for $\ell_1$ times, $C_{i_2}$ for $\ell_2$ times, and so on. From Proposition~\ref{prop-sum-cycle} and Corollary~\ref{cor-comp-two-paths}, a symbolic valuation $\sumf^{(\schm,\initval)}$ can be constructed 
to summarize the computation of $\Ss$ on $\schm$. 


\begin{lemma}\label{prop-cycle-schm}
Suppose $\schm=C_{i_1}^{\ell_1} C_{i_2}^{\ell_2} \dots C_{i_t}^{\ell_t}$ is a cycle scheme, and $\initval$ is a symbolic valuation representing the initial values of the control and data variables. 
For all $j' \in  I^{\circled{C_{i_{1}}}}_{pe}$, let $r_{j'}$ be the largest number $r \in [t]$ such that $j'\in\bigcap_{s\in[r]} I^{\circled{C_{i_{s}}}}_{pe}$, i.e., $x_{j'}$ remains persistent when traversing $C_{i_1}^{\ell_1} C_{i_2}^{\ell_2} \dots C_{i_{r_{j'}}}^{\ell_{r_{j'}}}$.
Then for each $j\in [l]$ and $j' \in  I^{\circled{C_{i_{1}}}}_{pe}$, the coefficient of the $\initval(x_{j'})$-atom in $\sumf^{(\schm,\initval)}(y_j)$ is 
\begin{center}
\resizebox{0.8\hsize}{!}{
$e+\sum\limits_{s_1\in[t]}  
\left(1+\lambda^{\circled{C_{i_{s_1}}}}_{j} + \dots + (\lambda^{\circled{C_{i_{s_1}}}}_{j})^{\ell_{s_1}-1} \right) \csta^{\circled{C_{i_{s_1}}}}_{j,j'}\prod\limits_{{s_2}\in[{s_1}+1,t]}\left(\lambda^{\circled{C_{i_{s_2}}}}_{j}\right)^{\ell_{s_2}}$},
\end{center}
where (1) $e\!=\!0$ when $r_{j'}\!=\!t$ and (2) $e=(\lambda^{\circled{C_{i_s}}}_{j})^{\ell_s-1} \csta^{\circled{C_{i_{s}}}}_{j,j'}$ with $s=r_{j'}+1$ when $r_{j'}<t$.\\
The constant atom of $\sumf^{(\schm,\initval)}(y_j)$ is 
\begin{center}
\resizebox{0.7\hsize}{!}{$
\sum\limits_{{s_1}\in[t]}
\left(1+\lambda^{\circled{C_{i_{s_1}}}}_{j} + \dots + (\lambda^{\circled{C_{i_{s_1}}}}_{j})^{\ell_{s_1}-1} \right)
\cste^{\circled{C_{i_{s_1}}}}_{j} 
\prod\limits_{{s_2}\in[{s_1}+1,t]}\left(\lambda^{\circled{C_{i_{s_2}}}}_{j}\right)^{\ell_{s_2}}$}
\end{center}
Moreover, for all $j \in [l]$, in $\sumf^{(\schm,\initval)}(y_j)$, only the constant atom and the coefficients of the $\initval(x_{j'})$-atoms with $j' \in  I^{\circled{C_{i_{1}}}}_{pe}$ contain a subexpression of the form $ \mu_\schm \ell_1$ for some $\mu_\schm\in \intnum$.
\end{lemma}
Notice that above, $\lambda^{\circled{C_{i_{s_1}}}}_j\in\{0,1\}$ for $j\in[l]$ and $s_1\in [t]$. Hence the value of $(1+\lambda^{\circled{C_{i_{s_1}}}}_{j} + \dots + (\lambda^{\circled{C_{i_{s_1}}}}_{j})^{\ell_{s_1}-1} )$ can only be $1$ or $\ell_{s_1}$ and $\left(\lambda^{\circled{C_{i_{s_2}}}}_{j}\right)^{\ell_{s_2}}\in\{0,1\}$.
Hence both the constant atom and the coefficient of the $\initval(x_{j'})$-atom with $j'\in I^{\circled{C_{i_{1}}}}_{pe}$ can be rewritten to the form of $c_0+c_1\ell_1+c_2\ell_2+\dots+c_t\ell_t$ for $c_0\ldots c_t\in \intnum$. Note that some of $c_0\ldots c_t$ might be zero.




We are ready to present the decision procedure. By the ``well-defined'' and ``uniquely-valued'' constraints of a normalized SNT, without loss of generality, we assume that $I^{\circled{H}}_{tr}=[k]$, that is, after traversing $H$, the values of all control variables become defined.
Under the assumption, for all $j' \in [k]$, the symbolic value of $\sumf^{(H,\sval_\bot)}(x_{j'})=\vard^{\circled{H}}_{\pi^{\circled{H}}(j')}$\smallskip\\
\framebox[\textwidth]{
\begin{minipage}{0.95\textwidth}
\noindent {\bf Step I}. Decide whether $\eval{O(q_m)}{\sumf^{(H,\sval_\bot)}}$ is not identical to zero.
This can be done by checking if the constant-atom or the coefficient of some non-constant atom of $\eval{O(q_m)}{\sumf^{(H,\sval_\bot)}}$ is not identical to zero.
If the answer is yes, then the decision procedure terminates and returns the answer $\ltrue$. Otherwise, go to Step II.
\end{minipage}
}\bigskip

The goal of Step II is either showing that $f=\eval{O(q_m)}{\sumf^{(\schm,\sumf^{(H,\sval_\bot)})}}$ all subexpressions containing cycle counter variables are identical to zero and hence can be ignored or showing that $f$ is not identical to zero. Let $\schm=C_{i_1}^{\ell_1} C_{i_2}^{\ell_2} \dots C_{i_t}^{\ell_t}$ be a cycle scheme. From Lemma~\ref{prop-cycle-schm}, for each $j'\in I^{\circled{C_{i_1}}}_{pe}$ and symbolic valuation $\sval$, the only subexpression containing $\ell_1$ in the coefficient of $\initval(x_{j'})$-atom of $\eval{O(q_m)}{\sumf^{(\schm,\initval)}}$ is
\begin{center}
	\resizebox{0.7\hsize}{!}{$
\sum \limits_{1 \le j \le l} 
b_j \left((\cstl^{\circled{C_{i_2}}}_{j})^{\ell_2} \dots (\cstl^{\circled{C_{i_t}}}_{j})^{\ell_t}\right) 
\left(1+\cstl^{\circled{C_{i_1}}}_{j} + \dots + (\cstl^{\circled{C_{i_1}}}_{j})^{\ell_1-1} \right) \csta^{\circled{C_{i_1}}}_{j,j'}.
\hspace{4mm} (\ast)
$}
\end{center}
Since $\cstl^{\circled{C_{i_1}}}_{j}, \cstl^{\circled{C_{i_2}}}_{j}, \dots, \cstl^{\circled{C_{i_t}}}_{j} \in \{0, 1\}$, the expression $(\ast)$  can be rewritten as  
 $\mu_{\schm, (i_1,j')} \ell_1 + \nu_{\schm, (i_1,j')}$ for some integer constants $\mu_{\schm, (i_1,j')}$ and $\nu_{\schm, (i_1,j')}$. 
 
The only subexpression containing $\ell_1$ in the constant atom of  $\eval{O(q_m)}{\sumf^{(\schm,\initval)}}$ is
\begin{center}
	\resizebox{0.7\hsize}{!}{$
\sum \limits_{1 \le j \le l} b_j
\begin{array}{l}
 \left((\lambda^{\circled{C_{i_2}}}_{j})^{\ell_2} \dots (\lambda^{\circled{C_{i_t}}}_{j})^{\ell_t}\right)
\left(1+\lambda^{\circled{C_{i_1}}}_{j} + \dots + (\lambda^{\circled{C_{i_1}}}_{j})^{\ell_1-1} \right) \cste^{\circled{C_{i_1}}}_{j}. \hspace{2mm} (\ast\ast)
\end{array}
$}
\end{center}
%
The expression $(\ast\ast)$ can be rewritten as $\mu_{\schm,(i_1,0)} \ell_1 + \nu_{\schm,(i_1,0)}$ for some integer constants $\mu_{\schm, (i_1,0)}$ and $\nu_{\schm, (i_1,0)}$. If $\mu_{\schm,(i_1,0)}=\mu_{\schm,(i_1,j')}=0$ for all $j' \in I^{\circled{C_{i_1}}}_{pe}$, then we can ignore all subexpressions containing the cycle counter variable $\ell_1$ in   $\eval{O(q_m)}{\sumf^{(\schm,\initval)}}$, i.e., the subexpressions $\mu_{\schm,(i_1,0)}\ell_1$ and $\mu_{\schm,(i_1,j')}\ell_1$ for all $j' \in I^{\circled{C_{i_1}}}_{pe}$.\smallskip\\
\framebox[\textwidth]{
	\begin{minipage}{0.95\textwidth}
		\noindent {\bf Step II}. For each $i_1 \in [n]$, check all cycle scheme $\schm=C_{i_1}^{\ell_1} C_{i_2} \dots C_{i_t}$ such that $i_2,\dots,i_t$ are mutually distinct. There are only finitely many this kind of cycle schemes. If 
		one of the following constraints is satisfied, then return $\ltrue$. \\(1) There is $j' \in  I^{\circled{C_{i_1}}}_{pe}$ such that $\mu_{\schm,(i_1,j')} \neq 0$. (2) $\mu_{\schm,(i_1,0)} \neq 0$.
		%
		If the decision procedure has not returned yet, then go to Step III.
	\end{minipage}
}\smallskip\\
If there exists $j' \in I^{\circled{C_{i_1}}}_{pe}$ such that $\mu_{\schm,(i_1,j')} \neq 0$, then we let $\vard^{\circled{H}}_{\pi^{\circled{H}}(j')} \neq 0$ and $\ell_1$ be arbitrarily large, so that the coefficient of the  $\vard^{\circled{H}}_{\pi^{\circled{H}}(j')}$-atom in $\eval{O(q_m)}{\sumf^{(\schm,\sumf^{(H,\sval_\bot)})}}$, which includes the expression $\mu_{\schm, (i_1,j')} \ell_1 + \nu_{\schm, (i_1,j')}$, dominates $\eval{O(q_m)}{\sumf^{(\schm,\sumf^{(H,\sval_\bot)})}}$. This is sufficient to make $\eval{O(q_m)}{\sumf^{(\schm,\sumf^{(H,\sval_\bot)})}}$ non-zero. Similarly, if $\mu_{\schm,(i_1,0)} \neq 0$, then we can let $\ell_1$ arbitrarily large to make the expression $\eval{O(q_m)}{\sumf^{(\schm,\sumf^{(H,\sval_\bot)})}}$ non-zero.
Similar argument can be applied for $\ell_2\dots\ell_n$.

If Step II does not return $\ltrue$, we show below that for all cycle scheme $\schm_1=C_{i_1}^{\ell_1} C_{i_2}^{\ell_2} \dots C_{i_{s_1}}^{\ell_{s_1}}$ with $i_1,i_2,\dots,i_{s_1} \in [n]$, all subexpressions containing cycle counter variables in $\eval{O(q_m)}{\sumf^{(\schm,\initval)}}$ are identical to zero and hence can be removed.
In Step II we already checked a cycle scheme $\schm_2=C_{i_1}^{\ell_1} C_{i'_2} \dots C_{i'_{s_2}}$, where the two sets of cycles in the tail 
$\{C_{i_2} ,\dots, C_{i_{s_1}}\}=\{C_{i'_2} \dots C_{i'_{s_2}}\}$. Step II guarantees all subexpressions containing $\ell_1$ in 
$\eval{O(q_m)}{\sumf^{(\schm_2,\initval)}}$ are identical to zero and hence can be removed.
Because for all $j\in[l]$, $\cstl^{^{\circled{C_1}}}_j, \dots, \cstl^{^{\circled{C_n}}}_j \in \{0,1\}$,   $(\lambda^{\circled{C_{i_2}}}_{j})^{\ell_2} \dots (\lambda^{\circled{C_{i_{s_1}}}}_{j})^{\ell_{s_1}} = \lambda^{\circled{C_{i'_2}}}_{j} \dots \lambda^{\circled{C_{i'_{s_2}}}}_{j}$. We proved that the $(\ast)$ and $(\ast\ast)$ style expressions are equivalent in both $\schm_1$ and $\schm_2$.
Hence we can also remove all subexpressions containing $\ell_1$ in  $\eval{O(q_m)}{\sumf^{(\schm_1,\initval)}}$.
Those subexpressions containing $\ell_2$ can also be removed by considering the cycle scheme $\schm_3=C_{i_2}^{\ell_2} C_{i''_3} \dots C_{i''_{s_3}}$ with $\{C_{i_3} ,\dots, C_{i_{s_1}}\}=\{C_{i''_3} \dots C_{i''_{s_3}}\}$ and applying a similar reasoning. The same applies to all other cycle counting variables $\ell_3,\dots,\ell_{s_1}$.
We use the notation ${\sumf^{(\schm,\initval)}}^-(y_j)$ to denote the expression obtained by removing from the constant atom and coefficients of the non-constant atoms of $\sumf^{(\schm,\initval)}(y_j)$ all subexpressions containing cycle counting variables, for all $y_j \in Y$. 

\begin{lemma}\label{prop-bnd-domain-1}
	Suppose that the decision procedure has not returned $\ltrue$ after Step~II. For each cycle scheme $\schm$, let $f=\eval{O(q_m)}{\sumf^{(\schm, \sumf^{(H,\sval_\bot)})}}$ and $f'=\eval{O(q_m)}{{\sumf^{(\schm, \sumf^{(H,\sval_\bot)})}}^-}$. For all valuation $\rho$, $\eval{f}{\rho}\neq 0$ iff $\eval{f'}{\rho} \neq 0$.
\end{lemma}





\begin{lemma}\label{prop-bnd-domain-2}
Suppose that the decision procedure has not returned yet after Step II. 
For all cycle scheme $\schm$ and $y_j \in Y$, the constant atom and the coefficients of all non-constant atoms in ${\sumf^{(\schm, \sumf^{(H,\initval_\bot)})}}^-(y_j)$ are from a finite set $U \subset \intnum$ comprises\\ (1)
the constant atom and the coefficients of the non-constant atoms in the expression ${\sumf^{(C^{\ell_i}_{i}, \sumf^{(H,\initval_\bot)})}}^-(y_j)$ for $i\in [n]$ and $\ell_i \in \{1,2\}$.\smallskip\\(2) the numbers $\csta^{\circled{C_{s_2}}}_{j,j'} + \cstb^{\circled{C_{s_1}}}_{j,\pi^{\circled{C_{s_1}}}(j')}$ and $\csta^{\circled{C_{s_1}}}_{j, j''} + \csta^{\circled{C_{s_2}}}_{j,j''}$, where  $s_1,s_2 \in [n], j\in[l],j' \in I^{\circled{C_{s_1}}}_{tr} \cap I^{\circled{C_{s_2}}}_{tr},  j'' \in [k]$. 

\end{lemma}

For each cycle scheme $\schm$, an abstraction of ${\sumf^{(\schm, \sumf^{(H,\initval_\bot)})}}^-$, denoted by $\abs(\schm)$,  is the union of the following three sets:
(1)~constant atom: $\{(0, ( {\cste^{(\schm)}_{1}}^-,\dots, {\cste^{(\schm)}_l}^-))\}$. (2)~control variable atom: $\{(j, (\cstg_{j,1},\dots, \cstg_{j,l})) \mid j \in [k]\}$, where $\cstg_{j, j'}$ is the coefficient of the $\sval(x_j)$-atom in ${\sumf^{(\schm,\sumf^{(H,\sval_\bot)})}}^-(y_{j'})$ for $j'\in[l]$. (3)~data variable atom: $\{(k+1, (c_1,\dots,c_l))\}$, where $(c_1,\dots,c_l) \in U^l$ is the coefficients of the $\vard'$-atom in ${(\sumf^{(\schm,\sumf^{(H,\sval_\bot)})}}^-(y_{j})$ for all $j\in [l]$ and $\vard'\not\in \{{\sumf^{(\schm,\sumf^{(H,\sval_\bot)})}}^-(x_{j'})\mid x_{j'}\in X\}$.
Let $\mathscr{A}=\bigcup \{\abs(\schm) \mid \schm \mbox{ a cycle scheme}\}$. Then $\mathscr{A}$ can be constructed as follows. We first compute $\abs(HC_1), \ldots \abs(HC_n)$ and then compute the next abstract elements from them w.r.t. $C_1\ldots C_n$ until reached a fixed point.\medskip\\
\framebox[\textwidth]{
	\begin{minipage}{0.95\textwidth}
		\noindent {\bf Step III} We first construct the set $\mathscr{A}$ and then. 
		\begin{enumerate}
			\item Check whether there is $(0,(c_{0,1},\dots,c_{0,l})) \in \mathscr{A}$ such that $a_0+b_1 c_{0,1}+\dots + b_l c_{0,l} \neq 0$. If the answer is yes, then return $\ltrue$.
			%
			\item Check whether there are $j \in [k]$ and $(j, (c_{j,1},\dots,c_{j,l})) \in \mathscr{A}$ such that $a_j + b_1 c_{j,1} + \dots + b_l c_{j,l} \neq 0$. If the answer is yes, then return $\ltrue$. 
			%
			\item Check whether there is $(k+1,(c_1,\dots,c_n)) \in \mathscr{A}$ such that $b_1 c_1 + \dots + b_l c_l \neq 0$. If the answer is yes, then return $\ltrue$. 
		\end{enumerate}
		If the decision procedure has not returned yet, return $\lfalse$.
	\end{minipage}
}\smallskip\\


\subsection{Decision procedure for SNTs}\label{sec-gflat}

We generalize the decision procedure for the case that the transition graphs of the SNTs are generalized lassos to the full class of SNTs.
We first define a \emph{generalized multi-lasso} as a sequence $\gmlasso= H_1 (C_{1,1},\dots,C_{1,n_1}) H_2 (C_{2,1},\dots,C_{2,n_2}) \dots H_r (C_{r,1},\dots, C_{r, n_r})$ s.t. (1) for each $s\in[r]$, $H_s = q_{s,1} \dots q_{s, m_s}$ and $H_s (C_{s,1},\dots,C_{s, n_s})$ is a generalized lasso, (2) for $1 \leq s< s' \leq r$, $H_s (C_{s,1},\dots,C_{s, n_s})$ and $H_{s'} (C_{s', 1},\dots,C_{s', n_{s'}})$ are state-disjoint, except the case that when $s'=s+1$, $q_{s, m_s}=q_{s',1}$, and (3) $q_{1,1}=q_0$.

Since the transition graph of $\Ss$ can be seen as a collection of generalized multi-lassos, in the following, we shall present the decision procedure by showing how to decide the non-zero output problem for generalized multi-lassos. 

We fix a generalized multi-lasso

\smallskip
\hspace{8mm} $\gmlasso= H_1 (C_{1,1},\dots,C_{1,n_1}) H_2 (C_{2,1},\dots,C_{2,n_2}) \dots H_r (C_{r,1},\dots, C_{r, n_r})$.

\smallskip
\noindent Without loss of generality, we assume that $O(q_{r,m_r})=a_0+a_1 x_1 + \dots + a_k x_k + b_1 y_1  + \dots + b_l y_l$ and $O(q')$ is undefined for every other state $q'$ in $\gmlasso$.\smallskip\\
\framebox[\textwidth]{
	\begin{minipage}{0.95\textwidth}
		\noindent {\bf Step I$'$}. We do the same analysis as in Step I for the path $H_1\dots H_r$.
	\end{minipage}
}\smallskip

Let $s\in [2,r]$. In order to analyze the set of cycles $\Cc=\{C_{s-1,1},\dots,C_{s-1,n_{s-1}}\}$, below we show how to summarize effect of the path $H_s\dots H_r$ to $O(q_{s-1, m_{s-1}})$, which is shared by all those cycles in $\Cc$.
Suppose that $\eval{O(q_{r,m_r})}{\sumf^{(H_s\dots H_{r}, \initval)}}=a_{s, 0}+a_{s, 1} \initval(x_1)+ \dots + a_{s, k} \initval(x_k) + b_{s,1} \initval(y_1) + \dots + b_{s, l} \initval(y_l)+e$, where $e$ is a linear combination of the data variables that represent the data values introduced when traversing $H_s\dots H_r$. 
%the constant atom is $a_{s,0}$, the coefficient of the $\initval(x_j)$-atom is $a_{s, j}$ for each $j \in [k]$, and the coefficient of the $\initval(y_{j'})$-atom is $b_{s, j'}$ for each $j' \in [l]$. 
Then we change the output function and let
$O(q_{s-1, m_{s-1}}):=a_{s, 0}+a_{s, 1} x_1 + \dots + a_{s, k} x_k + b_{s,1} y_1 + \dots + b_{s, l} y_l$.\smallskip\\
\framebox[\textwidth]{
	\begin{minipage}{0.95\textwidth}
\noindent {\bf Step II$'$}.  For each $s\in [r]$ and $s'\in [n_s]$, we check each cycle scheme $\schm = C^{\ell_1}_{s,s'} C_{i_{2}} \dots C_{i_{t}}$ such that $C_{i_{2}} \dots C_{i_{t}}\in \{C_{s, 1}, \dots, C_{s,n_s},\dots, C_{r,1}, \dots, C_{r,n_r}\}$ and $C_{i_{2}} \dots C_{i_{t}}$ are mutually distinct by performing an analysis of the expression $\eval{ O(q_{s, m_{s}})} {\sumf^{(\schm,\sumf^{(H_1 \dots H_{s}, \initval)} ) } }$, in a way similar to Step II. If the decision procedure does not return during the analysis, then go to Step III$'$.
	\end{minipage}
}\smallskip

Intuitively, in Step II$'$, during the analysis of $\eval{ O(q_{s, m_{s}})} {\sumf^{(\schm,\sumf^{(H_1 \dots H_{s}, \initval)} ) } }$, the effect of the paths $H_{s+1},  \dots,  H_r$ and the cycles $C_{i_{2}}, \dots, C_{i_{t}}$ is described by the expressions $\cstl^{\circled{H_{s+1}}}_j \dots \cstl^{\circled{H_{r}}}_j  \cstl^{\circled{C_{i_{2}}}}_j \dots \cstl^{\circled{C_{i_{t}}}}_j $ for $j \in [l]$. Since $O(q_{s, m_{s}})$ has already taken into consideration the expressions $\cstl^{\circled{H_{s+1}}}_j \dots \cstl^{\circled{H_{r}}}_j$ for $j \in [l]$, conceptually, in Step II$'$, we can do the analysis as if we have a generalized lasso where the handle is $H_1\dots H_s$ and the collection of cycles is $\{C_{s,1},\dots, C_{s,n_s}$, $\dots$, $C_{r,1},\dots, C_{r,n_r}\}$. 
%$\schm=C^{\ell_{s, 1}}_{i_{s,1}} \dots C^{\ell_{s, t_s}}_{i_{s, t_s}} C^{\ell_{s+1, 1}}_{i_{s+1, 1} } \dots C^{\ell_{s+1, t_{s+1}}}_{i_{s+1, t_{s+1}} } \dots C^{\ell_{r, 1}}_{i_{r,1}} \dots C^{\ell_{r, t_r}}_{i_{r, t_r}} $, where for each $s': s \le s' \le r$, $i_{s',1},\dots, i_{s', t_{s'}} \in [n_{s'}]$, 
%%%%%%%%%%%%%%%%%%%%%%%%%%%%%%%%%%%%%%%%%%%%%%%%%%%
%%%%%%%%%%%%%%%%%%%%%%%%%%%%%%%%%%%%%%%%%%%%%%%%%%%
%%%%%%%%%%%%%%%%%%%%%%%%%%%%%%%%%%%%%%%%%%%%%%%%%%%
\hide
{
At first, by using $O(q_m)$, we do the following computation, similarly to Step II: For each $i_1: 1 \le i_1 \le n$, if there are a cycle scheme $\schm$  
$HC_{i_1}^{\ell_1} C_{i_2}^{\ell_2} \dots C_{i_t}^{\ell_t}
$
or 
$HC_{i_1}^{\ell_1} C_{i_2}^{\ell_2} \dots C_{i_t}^{\ell_t} (C'_{i'_1})^{\ell'_1} (C'_{i'_2})^{\ell'_2} \dots (C'_{i'_{t'}})^{\ell'_{t'}}$,
and $j' \le k$ such that 
\begin{itemize}
\item $i_2,\dots,i_t \le n$ are mutually distinct, $\ell_2 = \dots = \ell_t = 1$, 
%
\item $i'_1,\dots,i'_{t'} \le n'$ are mutually distinct, $\ell'_2 = \dots = \ell'_{t'} = 1$, 
%
\item $\pi_{C_{i_1}}(j')=j'$, and $\mu_{\schm,(i_1,j')} \neq 0$ (recall that $\mu_{\schm,(i_1,j')}$ is obtained from the coefficient of $d^{(0)}_{\pi_H(j')-k}$ in  $\chi_\schm(O(q_m))$), 
\end{itemize}
then return $\ltrue$. 

Then by using $O(q'_{m'})$, we do the following: For each $i'_1: 1 \le i'_1 \le n'$, if there are a cycle scheme $\schm' =(C'_{i'_1})^{\ell'_1} (C'_{i'_2})^{\ell'_2} \dots (C'_{i'_{t'}})^{\ell'_{t'}}$, and $j' \le k$ such that
\begin{itemize}
\item $i'_2,\dots,i'_t \le n'$ are mutually distinct, $\ell'_2 = \dots = \ell'_t = 1$, 
%
\item $\pi_{C'_{i_1}}(j')=j'$, and $\mu_{\schm',(i'_1,j')} \neq 0$ (here $\mu_{\schm',(i_1,j')}$ is obtained from the coefficient of $d''_{j'}$ in  $\chi_{\schm'}(O(q'_{m'}))$, where $d''_1,\dots,d''_k$ denote the initial data values of $x_1,\dots,x_k$ respectively),
\end{itemize}
then return $\ltrue$. 

Similarly, we can apply an analysis for the constant coefficient to $\chi_\schm(O(q_m))$. 


If the decision procedure has not return yet, then go to Step III$'$. \qed
}
%%%%%%%%%%%%%%%%%%%%%%%%%%%%%%%%%%%%%%%%%%%%%%%%%%%
%%%%%%%%%%%%%%%%%%%%%%%%%%%%%%%%%%%%%%%%%%%%%%%%%%%
%%%%%%%%%%%%%%%%%%%%%%%%%%%%%%%%%%%%%%%%%%%%%%%%%%%
After Step II$'$, if the decision procedure has not returned yet, then similar to Lemma~\ref{prop-bnd-domain-2}, the following hold.
\begin{itemize}
\item For each $s \in [r]$ and each path $\schm=H_1 \schm_1 H_2 \dots H_s \schm_s$ such that for each $s'\in [s]$, $\schm_{s'}$ is a cycle scheme over the collection of cycles $\{C_{s',1},\dots,C_{s',n_{s'}}\}$, it holds that the constant atom and all the coefficients of the non-constant atoms in ${\sumf^{(\schm,\sval_\bot)}}^-(y_j)$ are from a bounded domain $U$.
%
\item Moreover,  an abstraction of $\schm$, denoted by $\abs(\schm)$, can be defined, so that $\mathscr{A}$, which contains the set of $\abs(\schm)$ for the paths $\schm=H_1 \schm_1 H_2 \dots H_s \schm_s$ (where $s \in [r]$), can be computed effectively from 
$H_1, C_{1,1}, \dots, C_{1,n_1},H_2,\dots, H_r,C_{r,1},\dots, C_{r,n_r}$.
\end{itemize}
%Similarly to the generalized lassos, we can construct a finite state automaton $\Aa'$ from $\chi_H,\chi_{C_1},\dots,\chi_{C_n},\chi_{H'}, \chi_{C'_1},\dots,\chi_{C'_{n'}}$ to record the coefficients in the states and simulate the evolvement of these coefficients. The final states of $\Aa$ represent the coefficients obtained when reaching the state $q'_{m'}$ in $\Ss$. 
\framebox[\textwidth]{
	\begin{minipage}{0.95\textwidth}
\noindent {\bf Step III$'$}. We apply the same analysis to $\mathscr{A}$ as in Step III. If the procedure does not return during the analysis, return $\lfalse$.
	\end{minipage}
}

