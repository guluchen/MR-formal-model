%!TEX root = main-cav.tex

\section{Decision procedure for the non-zero output problem}\label{sec:dec-snt}
Here we present a decision procedure for the non-zero output problem of normalized SNTs. We fix a normalized SNT $\Ss = (Q,X,Y,\delta,q_0,O)$ such that $X=\{ x_1,\dots, x_k\}$ and $Y = \{y_1,\dots,y_l\}$. Due to space constraint, we only present a simplified version where the transition guards are constant-free in the main text and leave the procedure for the general class in the appendix.

We define summaries of the computations of $\Ss$ on paths and cycles in Section~\ref{sec-sum}. We present a decision procedure for the case that the transition graph of $\Ss$ is a \emph{generalized lasso} in Section~\ref{sec-glasso}. The transition graph of $\Ss$ is said to be a generalized lasso if it comprises a handle $H=q_0 q_1 \dots q_m$ and a collection of simple cycles $C_1,\dots,C_n$ such that $q_m$ is the unique state shared by $C_1,\dots,C_n$. We generalize the procedure for a generalized lasso to full SNT in Section~\ref{sec-gflat}.

%Finally, we will show how to adapt the algorithm for the situation that the guards of the transitions may contain the comparisons with integer constants.%%%%%%%%%%%%%%%%%%%%%%%%%%%%%%%%%%%%%%%%%%%%%%%


\subsection{Summarization of the computations on paths and cycles}\label{sec-sum}

Suppose that $P=p_0 \xrightarrow{(g_1,\eta_1)} p_1 \dots p_{n-1} \xrightarrow{(g_n,\eta_n)} p_{n}$ is a path of $\Ss$. We assume that the initial values of the control and data variables are represented by a symbolic valuation $\sval$ over $X \cup Y$. When $P$ is traversed in a run of $\Ss$ over a data word $w$,  the data value in a position of $w$ may have to be equal or unequal to the initial value of some control variable or some other data value in $w$ that have been met before (enforced by the guards and assignments in $P$). Let $\sim$ denote the equivalence relation on $[n]$ induced by $P$ such that $i \sim j$ iff the guards and assignments on $P$ enforce that the data value in the $i$-th position of $w$ must equal to that in the $j$-th position of $w$. Assuming that there are $r^{\circled{P}}$ equivalence classes of $\sim$, we use the variables $\vard^{\circled{P}}_1,\vard^{\circled{P}}_2,\dots, \vard^{\circled{P}}_{r^{\circled{P}}}$ to denote the data values met when traversing $P$, one for each equivalence class. Note that here we use the superscript ${\circled{P}}$ to denote the fact that $r^{\circled{P}}$ is associated with the path $P$.

%Moreover, suppose that the $r$ fresh data values that are introduced when traversing the path are denoted by the variables $d^{(1)}_{1},\dots,d^{(1)}_{r}$, with one data value for each of $I_{1},\dots,I_{r}$. 
%In addition, suppose that the initial values of $y_1,\dots, y_l$ are denoted by the variables $o_1,\dots,o_l$ respectively. 

\begin{proposition}\label{prop-sum-path}
Suppose that $P$ is a path and the initial values of control and data variables are represented by a symbolic valuation $\initval$. Then the values of the control and data variables after traversing the path $P$ are specified by a symbolic valuation $\sumf^{(P,\initval)}$ of $X \cup Y$ satisfying the following conditions.
\begin{itemize}
\item The indices of $X$, i.e., $[k]$, is partitioned into $I^{\circled{P}}_{pe}$ and $I^{\circled{P}}_{tr}$, the indices of \emph{persistent} and \emph{transient} control variables, respectively. A control variable is persistent if its value has not been changed while traversing $P$, otherwise, it is transient.
\item For each $x_j\in X$ and $j\in I^{\circled{P}}_{pe}$, $\sumf^{(P,\initval)}(x_j)=\sval(x_j)$.
%
\item  For each $x_j\in X$ and $j\in I^{\circled{P}}_{tr}$,
$\sumf^{(P,\initval)}(x_j)=\vard^{\circled{P}}_{\pi^{\circled{P}}(j)}$, where $\pi^{\circled{P}}: I^{\circled{P}}_{tr} \rightarrow [r^{\circled{P}}]$ is an injective mapping from the index of a transient control variable to the index of the data value assigned to it.
% 
\item For each $y_j \in Y$, 
$
 \sumf^{(P,\initval)}(y_j)  =
 \cste^{\circled{P}}_{j} + 
 \cstl^{\circled{P}}_j \initval(y_j)  + 
  \sum\limits_{j'\in [k]}\csta^{\circled{P}}_{j,j'}\initval(x_{j'}) +
  \sum\limits_{j'\in [r^{\circled{P}}]}\cstb^{\circled{P}}_{j,j'} \vard^{\circled{P}}_{j'}$,
\hide{
\item For each $y_j \in Y$, 
\[
\small
\begin{array}{l}
\smallskip
\sumf^{(P,\initval)}(y_j)  = \\
\hspace{2mm} \cste^{\circled{P}}_{j} + \cstl^{\circled{P}}_j \initval(y_j)  + \csta^{\circled{P}}_{j,1} \initval(x_1) + \dots + \csta^{\circled{P}}_{j,k} \initval(x_k) +  \cstb^{\circled{P}}_{j,1} \vard^{\circled{P}}_1 +\dots + \cstb^{\circled{P}}_{j,r^{\circled{P}}} \vard^{\circled{P}}_{r^{\circled{P}}},
\end{array}
\]} 
where $\cste^{\circled{P}}_j,\cstl^{\circled{P}}_j, \csta^{\circled{P}}_{j,1},\dots,\csta^{\circled{P}}_{j,k}, \cstb^{\circled{P}}_{j,1},\dots,\cstb^{\circled{P}}_{j,r^{\circled{P}}}$ are integer constants such that $\cstl^{\circled{P}}_{j} \in \{0,1\}$ (as a result of the ``independently evolving and copyless'' constraint).  It can happen that $\cstl^{\circled{P}}_j =0$,  which means that $\initval(y_j)$ is irrelevant to $\sumf^{(P,\initval)}(y_j)$. Similarly for $\csta^{\circled{P}}_{j,1}=0$, and so on.
\end{itemize}
\end{proposition}
Note that in Proposition~\ref{prop-sum-path}, the two sets $I^{\circled{P}}_{pe}$ and $I^{\circled{P}}_{tr}$, the mapping $\pi^{\circled{P}}$, and the constants $\cste^{\circled{P}}_j,\cstl^{\circled{P}}_j, \dots, \cstb^{\circled{P}}_{j,r^{\circled{P}}}$ only depend on $P$ and are independent of $\initval$. Due to the unique-valued constraint of a normalized SNT, the inverse function of $\pi^{\circled{P}}$, denoted $(\pi^{\circled{P}})^{-1}$, is well-defined.

As a corollary of Proposition~\ref{prop-sum-path}, the following result demonstrates how to summarize the computations of $\Ss$ on the composition of two paths.

\begin{corollary}\label{cor-comp-two-paths}
Suppose that $P_1$ and $P_2$ are two paths in $\Ss$ such that the last state of $P_1$ is the first state of $P_2$. Moreover, let $\sumf^{(P_1, \initval)}$ (resp. $\sumf^{(P_2, \initval)}$) be the symbolic valuation summarizing the computation of $\Ss$ on $P_1$ (resp. $P_2$). Then the symbolic valuation summarizing the computation of $\Ss$ on $P_1 P_2$ is $\sumf^{(P_2,\ \sumf^{(P_1,\initval)})}$.
\end{corollary}

In order to get a better understanding of the relation between $\sumf^{(P_2,\ \sumf^{(P_1,\initval)})}$ and $(\sumf^{(P_1, \initval)},\sumf^{(P_2, \initval)})$, in the following, for each $y_j \in Y$, we obtain a more explicit form of the expression $\sumf^{(P_2,\ \sumf^{(P_1,\initval)})}(y_j)$, by unfolding therein the expression $\sumf^{(P_1,\initval)}$\medskip.
\hide{
$\begin{array}{rl}
\medskip
\sumf^{(P_2,\ \sumf^{(P_1,\initval)})}(y_j) = & 
\left(\cste^{\circled{P_2}}_{j}+\cstl^{\circled{P_2}}_{j} \cste^{\circled{P_1}}_{j}\right)+ \left(\cstl^{\circled{P_2}}_{j} \cstl^{\circled{P_1}}_{j} \right) \initval(y_j)+ \sum \limits_{j' \in I^{\circled{P_1}}_{pe}} \left(\csta^{\circled{P_2}}_{j,j'}\right) \initval(x_{j'}) + \sum \limits_{j' \in [k]} \left(\cstl^{\circled{P_2}}_{j} \csta^{\circled{P_1}}_{j,j'} \right) \initval(x_{j'}) + \\
%
\smallskip
& \sum \limits_{j' \in \rng(\pi^{\circled{P_1}})} \left( \csta^{\circled{P_2}}_{j,(\pi^{\circled{P_1}})^{-1}(j')}\right) \vard^{\circled{P_1}}_{j'} + \sum \limits_{j' \in [r^{\circled{P_1}}]} \left( \cstl^{\circled{P_2}}_{j} \cstb^{\circled{P_1}}_{j,j'} \right) \vard^{\circled{P_1}}_{j'} +

\sum \limits_{j'\in[r^{\circled{P_2}}]} \cstb^{\circled{P_2}}_{j,j'} \vard^{\circled{P_2}}_{j'}.


\end{array}$
}
\resizebox{\hsize}{!}{
	$\begin{array}{rl}
	\medskip
	\sumf^{(P_2,\ \sumf^{(P_1,\initval)})}(y_j) = & 
	\left(\cste^{\circled{P_2}}_{j}+
	\cstl^{\circled{P_2}}_{j} \cste^{\circled{P_1}}_{j}\right)+ \left(\cstl^{\circled{P_2}}_{j} \cstl^{\circled{P_1}}_{j} \right) \initval(y_j)+ \sum \limits_{j' \in I^{\circled{P_1}}_{pe}} 
	\left(\csta^{\circled{P_2}}_{j,j'} +\cstl^{\circled{P_2}}_{j} \csta^{\circled{P_1}}_{j,j'}\right) \initval(x_{j'}) +\\
	\medskip
	& 
	\sum \limits_{j' \in  I^{\circled{P_1}}_{tr}} 
	\left(\cstl^{\circled{P_2}}_{j} \csta^{\circled{P_1}}_{j,j'} \right) \initval(x_{j'}) +
	\sum \limits_{j' \in \rng(\pi^{\circled{P_1}})} \left( \csta^{\circled{P_2}}_{j,(\pi^{\circled{P_1}})^{-1}(j')}+\cstl^{\circled{P_2}}_{j} \cstb^{\circled{P_1}}_{j,j'} \right) \vard^{\circled{P_1}}_{j'} + 
	 \\
	%
	\smallskip
	& 
	\sum \limits_{j' \in [r^{\circled{P_1}}]\setminus \rng(\pi^{\circled{P_1}})} \left( \cstl^{\circled{P_2}}_{j} \cstb^{\circled{P_1}}_{j,j'} \right) \vard^{\circled{P_1}}_{j'} +
	
	\sum \limits_{j'\in[r^{\circled{P_2}}]} \cstb^{\circled{P_2}}_{j,j'} \vard^{\circled{P_2}}_{j'}.
	\end{array}$
}\medskip\\
In the equation, $j'\in  I^{\circled{P_1}}_{pe}$ implies that $x_{j'}$ remains unchanged when traversing $P_1$, which means the initial value of $x_{j'}$ before traversing $P_2$ is still $\initval(x_{j'})$ and therefore we have the item $ (\csta^{\circled{P_2}}_{j,j'}) \initval(x_{j'})$. When $j' \in \rng(\pi^{\circled{P_1}})$, the initial value of $x_{(\pi^{\circled{P_1}})^{-1}(j')}$ before traversing $P_2$ is $\vard^{\circled{P_1}}_{j'}$ and therefore we have the item $( \csta^{\circled{P_2}}_{j,(\pi^{\circled{P_1}})^{-1}(j')}) \vard^{\circled{P_1}}_{j'}$.
For all $j'\in [k] = I^{\circled{P_1}}_{pe} \cup I^{\circled{P_1}}_{tr}$, we have the item $(\cstl^{\circled{P_2}}_{j} \csta^{\circled{P_1}}_{j,j'}) \initval(x_{j'})$, i.e. the coefficient of $\initval(x_{j'})$ in $\sumf^{(P_1, \initval)}$ multiplied by $\cstl^{\circled{P_2}}_{j}$. Moreover, for all $j'\in [r^{\circled{P_1}}] = \rng(\pi^{\circled{P_1}}) \cup ([r^{\circled{P_1}}]\!\setminus\!\rng(\pi^{\circled{P_1}}))$, we have 
the item $( \cstl^{\circled{P_2}}_{j} \cstb^{\circled{P_1}}_{j,j'}) \vard^{\circled{P_1}}_{j'}$, i.e. the coefficient of $\vard^{\circled{P_1}}_{j'}$ in $\sumf^{(P_1, \initval)}$ multiplied by $\cstl^{\circled{P_2}}_{j}$.


\hide{
Below, we explain more intuitively how the coefficients in $\sumf^{(P_2,\ \sumf^{(P_1,\initval)})}(y_j)$ are obtained from $\sumf^{(P_1, \initval)}(y_j)$ and $\sumf^{(P_2, \initval)}(y_j)$.
\begin{itemize}
\item The constant coefficient is the sum of $\cste^{\circled{P_2}}_{j}$, i.e. the constant coefficient of $\sumf^{(P_2, \initval)}$,  and $\cstl^{\circled{P_2}}_{j} \cste^{\circled{P_1}}_{j}$, i.e. the constant coefficient of  $\sumf^{(P_1, \initval)}$ multiplied by $\cstl^{\circled{P_2}}_{j} $, that is, the coefficient of $\initval(y_j)$ in $\sumf^{(P_2, \initval)}$.
%
\item The coefficient of $\initval(y_j)$ is the product of the two coefficients of $\initval(y_j)$ in $\sumf^{(P_1, \initval)}$ and $\sumf^{(P_2, \initval)}$ respectively.
%
\item No matter $j'\in  I^{\circled{P_1}}_{pe}$ (the value of $x_{j'}$ remains unchanged when traversing $P_1$) or $j'\in  I^{\circled{P_1}}_{tr}$ (the value of $x_{j'}$ is changed to $\vard^{\circled{P_1}}_{\pi^{\circled{P_1}}(j')}$ after traversing $P_1$), $\initval(x_{j'})$ 
%
\item If $j' \in I^{\circled{P_1}}_{pe}$, then the value of $x_{j'}$ is unchanged when traversing $P_1$. This means that the initial value of $x_{j'}$ before traversing $P_2$ is still $\initval(x_{j'})$. Thus the coefficient of $\initval(x_{j'})$ is the sum of $\csta^{\circled{P_2}}_{j,j'}$, i.e. the coefficient of $\initval(x_{j'})$ in $\sumf^{(P_2, \initval)}$, and $\cstl^{\circled{P_2}}_{j} \csta^{\circled{P_1}}_{j,j'}$, i.e. the coefficient of $\initval(x_{j'})$ in $\sumf^{(P_1, \initval)}$ multiplied by $\cstl^{\circled{P_2}}_{j}$.
%
\item If $j' \in \rng(\pi^{\circled{P_1}})$, let $\pi^{\circled{P_1}}(j'')=j'$. Then the initial value of $x_{j''}$ before traversing $P_2$ is $\vard^{\circled{P_1}}_{j'}$. Therefore, the coefficient of $\vard^{\circled{P_1}}_{j'}$ is the sum of $\csta^{\circled{P_2}}_{j,j''}$, i.e. the coefficient of $\initval(x_{j''})$ in $\sumf^{(P_2, \initval)}$, and $\cstl^{\circled{P_2}}_{j} \cstb^{\circled{P_1}}_{j,j'}$, i.e. the coefficient of $\vard^{\circled{P_1}}_{j'}$ in $\sumf^{(P_1, \initval)}$ multiplied by $\cstl^{\circled{P_2}}_{j}$.
%
\item Similarly for the other coefficients.
\end{itemize}
}

In the following, by utilizing Proposition~\ref{prop-sum-path} and Corollary~\ref{cor-comp-two-paths}, for each path $C^{\ell}$ which is obtained by iterating a cycle $C$ for $\ell$ times, we illustrate how $\sumf^{(C^\ell,\initval)}$ is related to $\sumf^{(C, \initval)}$ and $\ell$.




%Suppose that $C$ is a cycle in $\Ss$, that is, a path $q_0 \xrightarrow{(g_1,\eta_1)} q_1 \dots q_{n-1} \xrightarrow{(g_n, \eta_n)} q_n$ such that $q_n = q_0$.  

%Suppose the initial values of the $k$ control variables are $d^{(0)}_1,\dots,d^{(0)}_k$. Moreover, suppose that the $r_C$ data values $d^{\circled{C\!,\!1}}_{1},\dots,d^{\circled{C\!,\!1}}_{r_C}$ are introduced when traversing the cycle for the first time. 
%In addition, suppose that the initial values of $y_1,\dots, y_l$ are $o_1,\dots,o_l$. 
%
%From Proposition~\ref{prop-sum-path}, we know that a function $\chi_C$ can be constructed to summarize the computation of $\Ss$ on $C$.
%\begin{itemize}
%\item There is an injective mapping $\pi_C: \{1,\dots,k\} \rightarrow \{1,\dots, k+r\}$ such that for each $x_j \in X$, if $\pi_C(j) \le k$, then $\pi_C(j)=j$ and $\chi_C(x_j)=d^{(0)}_{j}$, otherwise, $\chi_C(x_j)=d^{\circled{C\!,\!1}}_{\pi_C(j)-k}$.
% 
%\item For each $y_j \in Y$, $\chi_C(y_j) = \alpha^{\circled{C}}_{j,0} + \alpha^{\circled{C}}_{j,1} o_j + \beta^{\circled{C}}_{j,1} d^{(0)}_1 + \dots + \beta^{\circled{C}}_{j,k} d^{(0)}_k + \gamma^{\circled{C}}_{j,1} d^{\circled{C\!,\!1}}_1 +\dots + \gamma^{\circled{C}}_{j,r_C} d^{\circled{C\!,\!1}}_{r_C}$ such that $\alpha^{\circled{C}}_{j,1} \in \{0,+1\}$.
%\end{itemize}

\begin{proposition}\label{prop-sum-cycle}
Suppose $C$ is a cycle and $P=C^{\ell}$ such that $\ell \ge 2$. Then the symbolic valuation $\sumf^{(C^\ell,\initval)}$ to summarize the computation of $\Ss$ on $P$ is as follows,\medskip\\
\resizebox{\hsize}{!}{
$\begin{array}{l c l}
\sumf^{(C^\ell,\initval)}(y_j)  & = & 
\left(1 + \cstl^{\circled{C}}_{j} + \dots +(\cstl^{\circled{C}}_{j})^{\ell\!-\!1} \right)\cste^{\circled{C}}_{j} + (\cstl^{\circled{C}}_{j})^\ell \initval(y_j) + \smallskip\\
%
& & \sum \limits_{j' \in I^{\circled{C}}_{pe}} \left(1+\cstl^{\circled{C}}_{j} + \dots +(\cstl^{\circled{C}}_{j})^{\ell\!-\!1} \right) \csta^{\circled{C}}_{j,j'}\initval(x_{j'}) +  \sum \limits_{j' \in I^{\circled{C}}_{tr}}  (\cstl^{\circled{C}}_{j})^{\ell\!-\!1} \csta^{\circled{C}}_{j,j'} \initval(x_{j'}) +  \\
%
& & \sum \limits_{j' \in \rng(\pi^{\circled{C}})} \sum \limits_{m\in[\ell -1]}
\left(  \csta^{\circled{C}}_{j, (\pi^{\circled{C}})^{-1}(j')} +(\cstl^{\circled{C}}_{j})\cstb^{\circled{C}}_{j,j'} \right)
(\cstl^{\circled{C}}_{j})^{\ell-m-1}
\vard^{\circled{C\!,\!m}}_{j'} +\\
%
& & \sum \limits_{j' \in [r^{\circled{C}}] \setminus \rng(\pi^{\circled{C}})}\sum \limits_{m\in[\ell -1]} \left((\cstl^{\circled{C}}_{j})^{\ell\!-\!m} \cstb^{\circled{C}}_{j,j'} \right) \vard^{\circled{C\!,\!m}}_{j'} + 
\sum \limits_{j' \in [r^{\circled{C}}] }  
 \cstb^{\circled{C}}_{j, j'} \vard^{\circled{C\!,\!\ell}}_{j'},
\end{array} 
$}\medskip\\
where the variables $\vard^{\circled{C\!,\!1}}_{m},\dots, \vard^{\circled{C\!,\!m}}_{r^{\circled{C}}}$ for $m\in [\ell-1]$
 represent the data values introduced when traversing $C$ for the $m$-th time.
\end{proposition}

From Proposition~\ref{prop-sum-cycle} and the fact that $\cstl_{j} \in \{0, 1\}$, we have the following observation.
\begin{itemize}
\item If $\cstl^{\circled{C}}_{j}=0$, then\medskip\\
\resizebox{0.95\hsize}{!}{$
\sumf^{(C^\ell,\initval)}(y_j)   =  \cste^{\circled{C}}_{j} +  \sum \limits_{j' \in I^{\circled{C}}_{pe}} \csta^{\circled{C}}_{j,j'} \initval(x_{j'}) +
\sum \limits_{j'  \in \rng(\pi^{\circled{C}})} \csta^{\circled{C}}_{j, (\pi^{\circled{C}})^{-1}(j')}\ \vard^{\circled{C\!,\!\ell\!\!-\!\!1}}_{j'} + \sum \limits_{j' \in [r^{\circled{C}}] }  
\cstb^{\circled{C}}_{j, j'} \vard^{\circled{C\!,\!\ell}}_{j'}.$}


\item If $\cstl^{\circled{C}}_{j}=1$, then\medskip\\
\resizebox{0.78\hsize}{!}{$
\begin{array}{l c l}
\sumf^{(C^\ell,\initval)}(y_j)  & = &   \ell \cste^{\circled{C}}_{j}  + \initval(y_j) +   \sum  \limits_{j' \in I^{\circled{C}}_{pe}} \ell \csta^{\circled{C}}_{j,j'}  \initval(x_{j'}) + 
\sum \limits_{j' \in I^{\circled{C}}_{tr}} \csta^{\circled{C}}_{j,j'} \initval(x_{j'}) +  \smallskip\\
& & \sum \limits_{j' \in \rng(\pi^{\circled{C}})} \sum \limits_{m\in[\ell -1]}
\left(\csta^{\circled{C}}_{j, (\pi^{\circled{C}})^{-1}(j')} + \cstb^{\circled{C}}_{j,j'} \right) \vard^{\circled{C\!,\!m}}_{j'} +\smallskip\\
%
& & \sum \limits_{j' \in [r^{\circled{C}}]  \setminus \rng(\pi^{\circled{C}}) }\sum \limits_{m\in[\ell -1]} 
\cstb^{\circled{C}}_{j,j'} \vard^{\circled{C\!,\!m}}_{j'} + \sum \limits_{j' \in [r^{\circled{C}}] }  
\cstb^{\circled{C}}_{j, j'} \vard^{\circled{C\!,\!\ell}}_{j'}.
\end{array}
$}
%
\hide{
\item If $\alpha^{\circled{C}}_{j,1}=-1$ and $\ell$ is even, then
\[
\begin{array}{l c l}
\chi^{\circled{C}}_{\ell}(y_j)  & = &  o_j + \sum \limits_{j'\le k, \pi_C(j') \neq j'} (-\beta^{\circled{C}}_{j,j'}) d^{(0)}_{j'} +  \\
%
& & \sum \limits_{j' \le r_C\!,\! j'+k \in \rng(\pi_C)} ( \beta^{\circled{C}}_{j, \pi_C^{-1}(j'+k)} - \gamma^{\circled{C}}_{j,j'}) d^{\circled{C\!,\!1}}_{j'} + \\
%
& & \sum \limits_{j' \le r_C\!,\!  j'+k \not \in \rng(\pi_C)} (-\gamma^{\circled{C}}_{j,j'}) d^{\circled{C\!,\!1}}_{j'} + \dots + \\
%
& & \sum \limits_{j' \le r_C\!,\! j'+k \in \rng(\pi_C)} (\beta^{\circled{C}}_{j, \pi_C^{-1}(j'+k)}-\gamma^{\circled{C}}_{j,j'}) d^{\circled{C\!,\!\ell\!\!-\!\!1}}_{j'} + \\
%
& & \sum \limits_{j' \le r_C\!,\!  j'+k \not \in \rng(\pi_C)} (-\gamma^{\circled{C}}_{j,j'}) d^{\circled{C\!,\!\ell\!\!-\!\!1}}_{j'} + \gamma^{\circled{C}}_{j, 1} d^{\circled{C\!,\!\ell}}_{1} + \dots + \gamma^{\circled{C}}_{j,r_C} d^{\circled{C\!,\!\ell}}_{r_C}.
\end{array} 
\]
\item If $\alpha_{j,1}=-1$ and $\ell$ is odd, then
\[
\begin{array}{l c l}
\smallskip
\chi^{\circled{C}}_{\ell}(y_j)  & = &  \alpha^{\circled{C}}_{j,0} - o_j + \sum \limits_{j' \le k, \pi_C(j')=j'} \beta^{\circled{C}}_{j,j'} d^{(0)}_{j'} +  \sum \limits_{j'\le k, \pi_C(j') \neq j'}  \beta^{\circled{C}}_{j,j'} d^{(0)}_{j'} +  \\
%
& & \sum \limits_{j' \le r_C\!,\! j'+k \in \rng(\pi_C)} ( -\beta^{\circled{C}}_{j, \pi_C^{-1}(j'+k)} +\gamma^{\circled{C}}_{j,j'}) d^{\circled{C\!,\!1}}_{j'} + \\
%
& & \sum \limits_{j' \le r_C\!,\!  j'+k \not \in \rng(\pi_C)} \gamma^{\circled{C}}_{j,j'} d^{\circled{C\!,\!1}}_{j'} + \dots + \\
%
& & \sum \limits_{j' \le r_C\!,\! j'+k \in \rng(\pi_C)} (\beta^{\circled{C}}_{j, \pi_C^{-1}(j'+k)}-\gamma^{\circled{C}}_{j,j'}) d^{\circled{C\!,\!\ell\!\!-\!\!1}}_{j'} + \\
%
& & \sum \limits_{j' \le r_C\!,\!  j'+k \not \in \rng(\pi_C)} (-\gamma^{\circled{C}}_{j,j'}) d^{\circled{C\!,\!\ell\!\!-\!\!1}}_{j'} + \gamma^{\circled{C}}_{j, 1} d^{\circled{C\!,\!\ell}}_{1} + \dots + \gamma^{\circled{C}}_{j,r_C} d^{\circled{C\!,\!\ell}}_{r_C}.
\end{array} 
\]
}
\end{itemize}
%
%From the analysis above, we observe that in $\chi^{\circled{C}}_\ell(y_j)$, 
%\begin{itemize}
%\item the constant coefficient is either $\alpha^{\circled{C}}_{j,0}$, or $\alpha^{\circled{C}}_{j,0} \ell$, 
%
%\item the coefficient of $o_j$ is $0$, or $1$, 
%
%\item for each data value $d^{(0)}_{j'}$, the coefficient of $d^{(0)}_{j'}$ is either $\beta^{\circled{C}}_{j,j'}$, or $0$, or $\beta^{\circled{C}}_{j,j'} \ell$,
%
%\item for each data value $d^{(C\!,\!i)}_{j'}$ with $i \ge 1$, the coefficient of $d^{(C\!,\!i)}_{j'}$ is either $0$, or $\beta^{\circled{C}}_{j, \pi_C^{-1}(j'+k)}$, or $\beta^{\circled{C}}_{j, \pi_C^{-1}(j'+k)}+\gamma^{\circled{C}}_{j,j'}$, or $\gamma^{\circled{C}}_{j,j'}$.
%\end{itemize}


\subsection{Decision procedure for generalized lassos}\label{sec-glasso}
Before presenting the decision procedure for generalized lassos, we introduce some notations.
Suppose $P$ is a path in $\Ss$, $\initval$ is a symbolic valuation representing the initial values of the control and data variables, and $r^{\circled{P}}$ data values are introduced when traversing $P$. Moreover, let $e$ be an expression that is a linear combination of $\initval(x_1)$, $\dots$, $\initval(x_k)$, $\vard^{\circled{P}}_1$, $\dots$, and $\vard^{\circled{P}}_{r^{\circled{P}}}$, more precisely, 
\[e:=\mu_0 + \mu_1 \initval(x_1) +\dots + \mu_k \initval(x_k) + \nu_1 \initval(y_1)+\dots + \nu_l \initval(y_l)+ \xi_1 \vard^{\circled{P}}_{1} + \dots + \xi_{r^{\circled{P}}} \vard^{\circled{P}}_{r^{\circled{P}}},\]
 such that $\mu_0,\mu_1,\dots,\mu_k,\nu_1,\dots,\nu_{l}, \xi_1,\dots,\xi_{r^{\circled{P}}}$ are the expressions that contain none of the variables from $\initval(x_1)$, $\dots$, $\initval(x_k)$, $\initval(y_1)$, $\dots$, $\initval(y_l)$, $\vard^{\circled{P}}_1$, $\dots$, and $\vard^{\circled{P}}_{r^{\circled{P}}}$ (although they may contain some other variables). Then we call $\mu_0$ as the \emph{constant atom} of $e$ (with respect to $P$), and the subexpressions $\mu_1 \initval(x_1)$, $\dots$, $\mu_k \initval(x_k)$, $\nu_1 \initval(y_1)$, $\dots$, $\nu_l \initval(y_l)$, $\xi_1 \vard^{\circled{P}}_{1}$ , $\dots$,  $\xi_{r^{\circled{P}}} \vard^{\circled{P}}_{r^{\circled{P}}}$ as the $(\initval(x_1))$-atom, $\dots$, $(\initval(x_k))$-atom, $(\initval(y_1))$-atom, $\dots$, $(\initval(y_l))$-atom, $\vard^{\circled{P}}_{1}$-atom, $\dots$, and $(\vard^{\circled{P}}_{r^{\circled{P}}})$-atom of $e$ respectively. Moreover, $\mu_1, \dots, \mu_k, \nu_1,\dots, \nu_{l}, \xi_1,\dots, \xi_{r^{\circled{P}}}$ are called the \emph{coefficients} of these atoms. A non-constant atom is said to be \emph{nontrivial} if its coefficient is \emph{not} an expression that is identical to zero.

In the rest of this subsection, we assume that 
\begin{quote}
\it the transition graph of $\Ss$ comprises a handle $H=q_0 q_1 \dots q_m$ and a collection of simple cycles $C_1,\dots,C_n$ such that $q_m$ is the unique state shared by different cycles from $C_1,\dots,C_n$. Moreover, without loss of generality, we assume that $O(q_m) = a_0 + a_1 x_1 + \dots + a_k x_k + b_1 y_1 + \dots + b_l y_l$, and $O(q)$ is undefined for all the other states $q$.
\end{quote}

%%%%%%%%%%%%%%%%%%%%%%%%%%%%%%%%%%%%%%%%%%
%%%%%%%%%%%%%%%%%%%%%%%%%%%%%%%%%%%%%%%%%%
\hide{
From Proposition~\ref{prop-sum-path}, we know that a function $\chi_H$ can be constructed to summarize the computation of $\Ss$ on the handle. Let $d^{(0)}_{1}, \dots, d^{(0)}_{r_H}$ denote the $r_H$ data values introduced in the handle. Because all the values of the control and data variables are $\bot$ (undefined) in the initial configuration,  we can concretize $\chi_H$ as follows. 
\begin{itemize}
\item There is an injective mapping $\pi_H: \{1,\dots,k\} \rightarrow \{1,\dots, k+r\}$ such that for each $x_j \in X$, if $\pi_H(j) \le k$, then $\pi_H(j)=j$ and $\chi_H(x_j)=\bot$, otherwise, $\chi_H(x_j)=d^{(0)}_{\pi_H(j)-k}$. Without loss of generality, we assume $\pi_H(j) > k$ for each $j: 1\le j \le k$. Intuitively, this means that after traversing the handle $H$, the values of all control variables are defined (i.e. not $\bot$).
% 
\item For each $y_j \in Y$, $\chi_H(y_j) = \alpha^{\circled{H}}_{j,0} + \alpha^{\circled{H}}_{j,1} o_j + \gamma^{\circled{H}}_{j,1} d^{(0)}_1 +\dots + \gamma^{\circled{H}}_{j,r_H} d^{(0)}_{r_H}$ such that $\alpha^{\circled{H}}_{j,1} \in \{0,+1,-1\}$. Since the initial value of the control variable $y_j$ is $o_j=\bot$, we assume that $\alpha^{\circled{H}}_{j,1}=0$, without loss of generality.
\end{itemize}

Moreover, from Proposition~\ref{prop-sum-cycle}, we know that for each cycle $C_i$ ($1 \le i \le n$), a function $\chi^{(C_i)}_{\ell}$ can be constructed to summarize the computation of $\Ss$ on $C^{\ell}_i$ (the iteration of $C_i$ for $\ell$ times).
}
%%%%%%%%%%%%%%%%%%%%%%%%%%%%%%%%%%%%%%%%%%
%%%%%%%%%%%%%%%%%%%%%%%%%%%%%%%%%%%%%%%%%%
%%%%%%%%%%%%%%%%%%%%%%%%%%%%%%%%%%%%%%%%%%

A \emph{cycle scheme} $\schm$ is a path $C_{i_1}^{\ell_1} C_{i_2}^{\ell_2} \dots C_{i_t}^{\ell_t}$ such that $i_1,\dots,i_t \in \{1,\dots,n\}$, $\ell_1,\dots, \ell_t \ge 1$, and for each $j: 1 \le j < t$, $i_j \neq i_{j+1}$. Intuitively, $\schm$ is a path obtained by iterating $C_{i_1}$ for $\ell_1$ times, $C_{i_2}$ for $\ell_2$ times, and so on. From Proposition~\ref{prop-sum-cycle} and Corollary~\ref{cor-comp-two-paths}, we know that a symbolic valuation $\sumf^{(\schm,\initval)}$ can be constructed 
%by composing $\sumf^{((C_{i_1})^{\ell_1},\initval)}$, $\dots$, and $\sumf^{((C_{i_t})^{\ell_t},\initval)}$, 
to summarize the computation of $\Ss$ on $\schm$. Suppose for each $s: 1 \le s \le t$, the data values introduced when traversing $C_{i_s}^{\ell_s}$ in $\schm$ are represented by the variables $\vard^{\circled{C_{i_s}\!,\!1}}_{s,1}$, $\dots$, $\vard^{\circled{C_{i_s}\!,\!1}}_{s,r^{\circled{C_{i_s}}}}$, $\dots$, $\vard^{\circled{C_{i_s}\!,\!\ell_s}}_{s,1}$, $\dots$, $\vard^{\circled{C_{i_s}\!,\!\ell_s}}_{s,r^{\circled{C_{i_s}}}}$. Then for each $y_j \in Y$, $\sumf^{(\schm,\initval)}(y_j)$ is a linear combination of $\initval(x_1)$, $\dots$, $\initval(x_k)$, $\initval(y_j)$, $\vard^{\circled{C_{i_1}\!,\!1}}_{1,1}$, $\dots$, $\vard^{\circled{C_{i_1}\!,\!\ell_1}}_{1,r^{\circled{C_{i_1}}}}$, $\dots$, $\vard^{\circled{C_{i_t}\!,\!1}}_{t,1}$, $\dots$, $\vard^{\circled{C_{i_t}\!,\!\ell_t}}_{t, r^{\circled{C_{i_t}}}}$. 


\begin{proposition}\label{prop-cycle-schm}
Suppose $\schm=C_{i_1}^{\ell_1} C_{i_2}^{\ell_2} \dots C_{i_t}^{\ell_t}$ is a cycle scheme, and $\initval$ is a symbolic valuation representing the initial values of the control and data variables. Then for each $j: 1 \le j \le l$ and each $j' \in I^{\circled{C_{i_1}}}_{pe}$, the coefficient of the $(\initval(x_{j'}))$-atom of $\sumf^{(\schm,\initval)}(y_j)$ contains the following subexpression,
\[\left((\lambda^{\circled{C_{i_2}}}_{j})^{\ell_2} \dots (\lambda^{\circled{C_{i_t}}}_{j})^{\ell_t}\right) \\
\left(1+\lambda^{\circled{C_{i_1}}}_{j} + \dots + (\lambda^{\circled{C_{i_1}}}_{j})^{\ell_1-1} \right) \csta^{\circled{C_{i_1}}}_{j,j'},\]
and the constant atom of $\sumf^{(\schm,\initval)}(y_j)$ is
\[
\begin{array}{l c l}
\left((\lambda^{\circled{C_{i_2}}}_{j})^{\ell_2} \dots (\lambda^{\circled{C_{i_t}}}_{j})^{\ell_t}\right)
\left(1+\lambda^{\circled{C_{i_1}}}_{j} + \dots + (\lambda^{\circled{C_{i_1}}}_{j})^{\ell_1-1} \right) \cste^{\circled{C_{i_1}}}_{j} + \\
\left((\lambda^{\circled{C_{i_3}}}_{j})^{\ell_3} \dots (\lambda^{\circled{C_{i_t}}}_{j})^{\ell_t}\right)
\left(1+\lambda^{\circled{C_{i_2}}}_{j} + \dots + (\lambda^{\circled{C_{i_2}}}_{j})^{\ell_2-1} \right) \cste^{\circled{C_{i_2}}}_{j} +\\
 \dots + 
\left(1+\lambda^{\circled{C_{i_t}}}_{j} + \dots + (\lambda^{\circled{C_{i_t}}}_{j})^{\ell_t-1} \right) \cste^{\circled{C_{i_t}}}_{j}.
\end{array}
\]
Moreover, for each $j: 1 \le j \le l$, it holds that each \emph{nontrivial} non-constant atom of $\sumf^{(\schm,\initval)}(y_j)$ such that its coefficient contains a subexpression  of the form $c\ell_1$ for some $c \in \intnum$, is in fact an $(\initval(x_{j'}))$-atom of $\sumf^{(\schm,\initval)}(y_j)$ with $j' \in I^{\circled{C_{i_1}}}_{pe}$.
\end{proposition}


\begin{proposition}\label{test}
Suppose $\schm=C_{i_1}^{\ell_1} C_{i_2}^{\ell_2} \dots C_{i_t}^{\ell_t}$ is a cycle scheme, and $\initval$ is a symbolic valuation representing the initial values of the control and data variables. Let $B^i(j)$ be the largest number that $j\in\bigcap\limits_{n\in[i,B^i(j)]} I^{\circled{C_{i_{n}}}}_{pe}$.
Then for each $j: 1 \le j \le l$, the coefficient of the $(\initval(x_{j}))$-atom of $\sumf^{(\schm,\initval)}(y_j)$ is 
\[\sum\limits_{n\in[B^1(j)]} 
\prod\limits_{m\in[n+1,t]}\left(\lambda^{\circled{C_{i_m}}}_{j}\right)^{\ell_m} 
\left(1+\lambda^{\circled{C_{i_n}}}_{j} + \dots + (\lambda^{\circled{C_{i_n}}}_{j})^{\ell_n-1} \right) \csta^{\circled{C_{i_n}}}_{j,j'}\] plus $(\lambda^{\circled{C_{i_n}}}_{j})^{\ell_{B^1(j)+1}-1} \csta^{\circled{C_{i_n}}}_{j,j'}$ when $B^1(j)\neq t$,

contains the following subexpression,
	\[\left((\lambda^{\circled{C_{i_2}}}_{j})^{\ell_2} \dots (\lambda^{\circled{C_{i_t}}}_{j})^{\ell_t}\right) \\
	\left(1+\lambda^{\circled{C_{i_1}}}_{j} + \dots + (\lambda^{\circled{C_{i_1}}}_{j})^{\ell_1-1} \right) \csta^{\circled{C_{i_1}}}_{j,j'},\]
	and the constant atom of $\sumf^{(\schm,\initval)}(y_j)$ is
	\[
	\begin{array}{l c l}
	\left((\lambda^{\circled{C_{i_2}}}_{j})^{\ell_2} \dots (\lambda^{\circled{C_{i_t}}}_{j})^{\ell_t}\right)
	\left(1+\lambda^{\circled{C_{i_1}}}_{j} + \dots + (\lambda^{\circled{C_{i_1}}}_{j})^{\ell_1-1} \right) \cste^{\circled{C_{i_1}}}_{j} + \\
	\left((\lambda^{\circled{C_{i_3}}}_{j})^{\ell_3} \dots (\lambda^{\circled{C_{i_t}}}_{j})^{\ell_t}\right)
	\left(1+\lambda^{\circled{C_{i_2}}}_{j} + \dots + (\lambda^{\circled{C_{i_2}}}_{j})^{\ell_2-1} \right) \cste^{\circled{C_{i_2}}}_{j} +\\
	\dots + 
	\left(1+\lambda^{\circled{C_{i_t}}}_{j} + \dots + (\lambda^{\circled{C_{i_t}}}_{j})^{\ell_t-1} \right) \cste^{\circled{C_{i_t}}}_{j}.
	\end{array}
	\]
	Moreover, for each $j: 1 \le j \le l$, it holds that each \emph{nontrivial} non-constant atom of $\sumf^{(\schm,\initval)}(y_j)$ such that its coefficient contains a subexpression  of the form $c\ell_1$ for some $c \in \intnum$, is in fact an $(\initval(x_{j'}))$-atom of $\sumf^{(\schm,\initval)}(y_j)$ with $j' \in I^{\circled{C_{i_1}}}_{pe}$.
\end{proposition}


We are ready to present the decision procedure. Without loss of generality, we assume that $I^{\circled{H}}_{tr}=[k]$, that is, after traversing $H$, the values of all the control variables become defined.

\smallskip

\noindent {\bf Step I}. Decide whether $\eval{O(q_m)}{\sumf^{(H,\initval_0)}}$ is not identical to zero, where $\initval_0$ is the special symbolic valuation such that $\initval_0(x_j)=\bot$ for each $x_j \in X$ and $\initval_0(y_j)=\bot$ for each $y_j \in Y$. If the answer is yes, then the decision procedure terminates and returns the answer $\ltrue$. Otherwise, go to Step II. \qed

\medskip


From Proposition~\ref{prop-sum-path}, we know that for each $j' \in [k]$ (recall that we assume $I^{\circled{H}}_{tr}=[k]$), $\sumf^{(H,\initval_0)}(x_{j'})=\vard^{\circled{H}}_{\pi^{\circled{H}}(j')}$.  Let $\schm=C_{i_1}^{\ell_1} C_{i_2}^{\ell_2} \dots C_{i_t}^{\ell_t}$ be a cycle scheme. Then from Proposition~\ref{prop-cycle-schm}, for each $j' \in  I^{\circled{C_{i_1}}}_{pe}$, the coefficient of the $\vard^{\circled{H}}_{\pi^{\circled{H}}(j')}$-atom of $\eval{O(q_m)}{\sumf^{(\schm,\sumf^{(H,\initval_0)})}}$ contains the following subexpression, 
\[
\sum \limits_{1 \le j \le l} 
%\begin{array}{l}
b_j \left((\cstl^{\circled{C_{i_2}}}_{j})^{\ell_2} \dots (\cstl^{\circled{C_{i_t}}}_{j})^{\ell_t}\right) 
\left(1+\cstl^{\circled{C_{i_1}}}_{j} + \dots + (\cstl^{\circled{C_{i_1}}}_{j})^{\ell_1-1} \right) \csta^{\circled{C_{i_1}}}_{j,j'}.
%\end{array} \hspace{1cm}  
\hspace{4mm} (\ast)
\]
Since $\cstl^{\circled{C_{i_1}}}_{j}, \cstl^{\circled{C_{i_2}}}_{j}, \dots, \cstl^{\circled{C_{i_t}}}_{j} \in \{0, 1\}$, the expression $(\ast)$  can be rewritten as  
 $\mu_{\schm, (i_1,j')} \ell_1 + \nu_{\schm, (i_1,j')}$ for some integer constants $\mu_{\schm, (i_1,j')}$ and $\nu_{\schm, (i_1,j')}$. 
 
Similarly, from Proposition~\ref{prop-cycle-schm},  the constant atom of  $\eval{O(q_m)}{\sumf^{(\schm,\sumf^{(H,\initval_0)})}}$ contains the following  subexpression,
\[
\sum \limits_{1 \le j \le l} b_j
\begin{array}{l}
 \left((\lambda^{\circled{C_{i_2}}}_{j})^{\ell_2} \dots (\lambda^{\circled{C_{i_t}}}_{j})^{\ell_t}\right)
\left(1+\lambda^{\circled{C_{i_1}}}_{j} + \dots + (\lambda^{\circled{C_{i_1}}}_{j})^{\ell_1-1} \right) \cste^{\circled{C_{i_1}}}_{j}. \hspace{2mm} (\ast\ast)
\end{array}
\]
%
The expression $(\ast\ast)$ can be rewritten as $\mu_{\schm,(i_1,0)} \ell_1 + \nu_{\schm,(i_1,0)}$ for some integer constants $\mu_{\schm, (i_1,0)}$ and $\nu_{\schm, (i_1,0)}$.


 
 \smallskip

\noindent {\bf Step II}. For each $i_1: 1 \le i_1 \le n$, if there is a cycle scheme $\schm=C_{i_1}^{\ell_1} C_{i_2} \dots C_{i_t}$ such that $i_2,\dots,i_t$ are mutually distinct and 
one of the following constraints is satisfied, then return $\ltrue$.
\begin{itemize}
\item There is $j' \in  I^{\circled{C_{i_1}}}_{pe}$ such that $\mu_{\schm,(i_1,j')} \neq 0$.
%
\item $\mu_{\schm,(i_1,0)} \neq 0$.
%
\end{itemize}
%
If the decision procedure has not returned yet, then go to Step III. \qed

\medskip

Let $\schm=C_{i_1}^{\ell_1} C_{i_2}^{\ell_2} \dots C_{i_t}^{\ell_t}$ be a cycle scheme. If there is $j' \in I^{\circled{C_{i_1}}}_{pe}$ such that $\mu_{\schm,(i_1,j')} \neq 0$, then we let $\vard^{\circled{H}}_{\pi^{\circled{H}}(j')} \neq 0$ and $\ell_1$ arbitrarily large, so that the $\vard^{\circled{H}}_{\pi^{\circled{H}}(j')}$-atom of $\eval{O(q_m)}{\sumf^{(\schm,\sumf^{(H,\initval_0)})}}$, whose coefficient includes the expression $\mu_{\schm, (i_1,j')} \ell_1 + \nu_{\schm, (i_1,j')}$, dominates $\eval{O(q_m)}{\sumf^{(\schm,\sumf^{(H,\initval_0)})}}$. This is sufficient to make $\eval{O(q_m)}{\sumf^{(\schm,\sumf^{(H,\initval_0)})}}$ become non-zero. Similarly, if $\mu_{\schm,(i_1,0)} \neq 0$, then we can let $\ell_1$ arbitrarily large to make $\eval{O(q_m)}{\sumf^{(\schm,\sumf^{(H,\initval_0)})}}$ become non-zero.

\smallskip

We would like to remark that in Step II, for a given index $i_1$, we restrict the cycle schemes to those of the form $C_{i_1}^{\ell_1} C_{i_2} \dots C_{i_t}$.This is justified by the following argument: For each $j: 1 \le j \le l$, $\cstl^{^{\circled{C_1}}}_j, \dots, \cstl^{^{\circled{C_n}}}_j \in \{0,1\}$. Therefore, for each cycle scheme $\schm=C_{i_1}^{\ell_1} C_{i_2}^{\ell_2} \dots C_{i_t}^{\ell_t}$ and each $j: 1 \le j \le l$,  we have $(\lambda^{\circled{C_{i_2}}}_{j})^{\ell_2} \dots (\lambda^{\circled{C_{i_t}}}_{j})^{\ell_t} = \lambda^{\circled{C_{i_2}}}_{j} \dots \lambda^{\circled{C_{i_t}}}_{j}$.


\begin{proposition}\label{prop-bnd-domain}
Suppose that the decision procedure has not returned yet after Step II. For each cycle scheme $\schm=C_{i_1}^{\ell_1} C_{i_2}^{\ell_2} \dots C_{i_t}^{\ell_t}$ and $y_j \in Y$, let $(\sumf^{(\schm,\initval)})'(y_j)$ denote the expression obtained by removing from the coefficients of the atoms of $\sumf^{(\schm,\initval)}(y_j)$ all the expressions of the form $c\ \ell_1$, $\dots$, or $c\ \ell_t$ (where $c$ is an integer constant).  Then the following three facts hold.
\begin{enumerate}
\item Let $f=\eval{O(q_m)}{\sumf^{(\schm, \sumf^{(H,\initval_0)})}}$ and $f'=\eval{O(q_m)}{(\sumf^{(\schm, \sumf^{(H,\initval_0)})})'}$.
There is a valuation $\rho$ such that $\eval{f}{\rho}\neq 0$ iff there is a valuation $\rho$ such that $\eval{f}{\rho} \neq 0$.
%
\item There is a finite subset of $\intnum$, say $K$, such that for every cycle scheme $\schm$ and $y_j \in Y$, the constant atom and all the coefficients of atoms in $(\sumf^{(\schm, \sumf^{(H,\initval_0)})})'(y_j)$ are from $K$. 
%
\item For each cycle scheme $\schm$, an abstraction of $(\sumf^{(\schm, \sumf^{(H,\initval_0)})})'$, denoted by $\abs(\schm)$, can be defined such that $\abs(\schm) \subseteq \{0,\dots, k+1\} \times K^l$ and for each $j: 0 \le j \le k$,  $\abs(\schm) \cap (\{j\} \times K^l)$ is a singleton. Let $\Kk=\{\abs(\schm) \mid \schm \mbox{ a cycle scheme}\}$. Then $\Kk$ can be constructed effectively from $\sumf^{(H,\initval)}$, $\sumf^{(C_1,\initval)}$, $\dots$, and $\sumf^{(C_n,\initval)}$.
\end{enumerate}
\end{proposition}


\noindent {\bf Step III}. For each $U \in \Kk$, do the following.
\begin{enumerate}
\item Check whether there is $(0,(c_{0,1},\dots,c_{0,l})) \in U$ such that $a_0+b_1 c_{0,1}+\dots + b_l c_{0,l} \neq 0$. If the answer is yes, then return $\ltrue$.
%
\item Check whether there are $j \in [k]$ and $(j, (c_{j,1},\dots,c_{j,l})) \in U$ such that $a_j + b_1 c_{j,1} + \dots + b_l c_{j,l} \neq 0$. If the answer is yes, then return $\ltrue$. 
%
\item Check whether there is $(k+1,(c_1,\dots,c_n)) \in U$ such that $b_1 c_1 + \dots + b_l c_l \neq 0$. If the answer is yes, then return $\ltrue$. 
\end{enumerate}
If the decision procedure has not returned yet, return $\lfalse$.


%From the claim, it follows that the updates of the values of control and data variables along $\schm$, as well as the update of the value of the output expression $O(q_m)$, can be obtained by composing $\chi_H$, $(\chi^{(\Cc_{i_1})}_{\ell_1})'$, $\dots$, and $\chi^{(\Cc_{i_t})}_{\ell_t})'$, which can further be simulated by a $\intnum$-VASS $\Aa$, that is, an integer vector addition system with states (cf. \cite{HH14}). Then the non-zero output reachability of $\Ss$ is reduced to the non-zero reachability of $\Aa$, that is, given an index $i$, decide whether $\Aa$ can reach a vector $\vec{z}$ where $z_i \neq 0$. Finally, we decide the non-zero reachability of $\Aa$. If the answer is ``yes'', then return $\ltrue$, otherwise, return $\lfalse$.

%\begin{proposition}
%The non-zero reachability problem of $\intnum$-VASS is in NP.
%\end{proposition}
%
%\begin{proof}
%The non-zero reachability problem of $\intnum$-VASS can be reduced to the coverability problem of $\intnum$-VASS in polynomial time as follows: Given a $\intnum$-VASS $\Aa$ of dimension $m$ (that is, $m$ is the length of the integer vectors) and an index $i: 1 \le i \le m$, we construct a $\intnum$-VASS $\Bb$ as follows: From each final state of $\Aa$, nondeterministically we choose a subset $Idx \evalseteq \{1,\dots,m\}$, and for each $i' \in Idx$, replace $z_{i'}$ by $-z_{i'}$. It is easy to see that $\Aa$ can reach a vector $\vec{z}$ where $z_i \neq 0$ iff $\Bb$ can reach a vector covering $(0,\dots,1,\dots,0)$, that is, the vector whose $i$-th component is $1$ and all the other components are zero.
%
% From the fact that the coverability of $\intnum$-VASS is NP-complete (\cite{HH14}), we know that the non-zero reachability problem of $\intnum$-VASS is in NP. \qed
%\end{proof}
%
%
%Therefore, in this case, the non-zero output reachability problem is reduced to the non-zero reachability problem of $\intnum$-VAS, 

\subsection{Decision procedure for SNTs}\label{sec-gflat}

In this subsection, we generalize the decision procedure in Section~\ref{sec-glasso} for the special case that the transition graphs of the SNTs are generalized lassos to SNTs.

We first introduce a notation. A \emph{generalized multi-lasso} of the transition graph of $\Ss$ is a sequence 
\[
\gmlasso= H_1 (C_{1,1},\dots,C_{1,n_1}) H_2 (C_{2,1},\dots,C_{2,n_2}) \dots H_r (C_{r,1},\dots, C_{r, n_r}),
\]
such that 
\begin{itemize}
\item for each $s: 1 \le s \le r$, $H_s = q_{s,1} \dots q_{s, m_s}$ and $H_s (C_{s,1},\dots,C_{s, n_s})$ is a generalized lasso, 

\item for each $s, s': 1 \le s < s' \le r$, $H_s (C_{s,1},\dots,C_{s, n_s})$ and $H_{s'} (C_{s', 1},\dots,C_{s', n_{s'}})$ are state-disjoint,

\item $q_{1,1}=q_0$, and for each $s: 1 \le s < r$, $q_{s, m_s}=q_{s+1,1}$.
\end{itemize}

Since the transition graph of $\Ss$ can be seen as a collection of generalized multi-lassos, in the following, we shall present the decision procedure by showing how to decide the non-zero output problem for generalized multi-lassos. 

In the rest of this section, we fix a generalized multi-lasso
\[
\gmlasso= H_1 (C_{1,1},\dots,C_{1,n_1}) H_2 (C_{2,1},\dots,C_{2,n_2}) \dots H_r (C_{r,1},\dots, C_{r, n_r}).
\]
For each $s: 1 \le s \le r$, suppose $H_s=q_{s,1} \dots q_{s, m_s}$. Without loss of generality, we assume that $O(q_{r,m_r})=a_0+a_1 x_1 + \dots + a_k x_k + b_1 y_1  + \dots + b_l y_l$ and $O(q')$ is undefined for every other state $q'$ in $\gmlasso$.

%We shall illustrate the argument by the following situation, the transition graph comprises a handle $H=q_0\dots q_{m}$,  a collection of cycles $(C_1,\dots,C_n)$ such that $q_m$ is the unique state shared by each pair of them, another handle $H'=q'_0 \dots q'_{m'}$ such that $q'_0=q_m$, and another collection of cycles $(C'_1,\dots,C'_{n'})$ such that $q'_{m'}$ is the unique state shared by each pair of them. Moreover, $O(q'_{m'})$ is defined and $O(q)$ is undefined for all the other states $q$. Suppose $O(q'_{m'}) = a_0 + a_1 x_1 + \dots + a_k x_k + b_1 y_1 + \dots + b_l y_l$.


%%%%%%%%%%%%%%%%%%%%%%%%%%%%%%%%%%%%%%%%%%%%%%%%%%%%%%%%%%%%%%
%%%%%%%%%%%%%%%%%%%%%%%%%%%%%%%%%%%%%%%%%%%%%%%%%%%%%%%%%%%%%%
%%%%%%%%%%%%%%%%%%%%%%%%%%%%%%%%%%%%%%%%%%%%%%%%%%%%%%%%%%%%%%
\hide{
From Proposition~\ref{prop-sum-path}, we know that 
for the handle $H = q_0 \dots q_m$, a function $\chi_H$ can be constructed to summarize the computation on the handle. Suppose for each $j: 1 \le j \le l$,  $\chi_H(y_j) = \alpha^{\circled{H}}_{j,0} + \gamma^{\circled{H}}_{j,1} d^{(H,0)}_1 + \dots + \gamma^{\circled{H}}_{j,r_H} d^{(H,0)}_{r_H}$, where $d^{(H,0)}_1,\dots, d^{(H,0)}_{r_H}$ represent the $r_H$ data values introduced in the handle. Moreover, there is an injective mapping $\pi_H$ from $\{1,\dots,k\}$ to $\{1,\dots,k+r_H\}$ such that for each $j: 1 \le j \le k$, $\chi_H(x_j) = d^{(H,0)}_{\pi_H(j)-k}$.

Similarly, for the handle $H' = q'_0 \dots q'_{m'}$, a function $\chi_{H'}$ can be constructed to summarize the computation on the handle. Let $d'_1,\dots,d'_k$ and $o'_1,\dots,o'_k$ denote the initial values of the control and data variables respectively. Then for each $j: 1 \le j \le l$,  $\chi_{H'}(y_j) = \alpha^{\circled{H'}}_{j,0} + \alpha^{\circled{H'}}_{j,1} o'_j + \beta^{\circled{H'}}_{j,1} d'_1 + \dots + \beta^{\circled{H'}}_{j,k} d'_{k}+ \gamma^{\circled{H'}}_{j,1} d^{(H',0)}_1 + \dots + \gamma^{\circled{H'}}_{j,r_{H'}} d^{(H',0)}_{r_{H'}}$, where $d^{(H',0)}_1,\dots, d^{(H',0)}_{r_{H'}}$ represent the $r_{H'}$ data values introduced in the handle. Moreover, there is an injective mapping $\pi_{H'}$ from $\{1,\dots,k\}$ to $\{1,\dots,k+r_{H'}\}$ such that for each $j: 1 \le j \le k$, if $\pi_{H'}(j)=j$, then $\chi_{H'}(x_j) = d'_j$, otherwise, $\chi_{H'}(x_j) = d^{(H',0)}_{\pi_{H'}(j)-k}$.

From Proposition~\ref{prop-sum-cycle}, we know that for $\ell \ge 1$, the functions $\chi^{\circled{C_1}}_{\ell}$,$\dots$, $\chi^{\circled{C_n}}_{\ell}$, $\chi^{(C'_1)}_{\ell}$, $\dots$, and $\chi^{(C'_{n'})}_{\ell}$ can be defined to summarize to computation on $C^{\ell}_1$, $\dots$, $C^{\ell}_n$, $(C'_1)^{\ell}$, $\dots$, and $(C'_n)^{\ell}$ respectively. 

Let us check the expression $\chi_{H'}(O(q'_{m'}))$ defined as follows,
\[
\begin{array}{l c l}
\smallskip
\chi_{H'}(O(q'_{m'})) & = & a_0 + a_1 \chi_{H'}(x_1) + \dots a_k \chi_{H'}(x_k) + \\
\smallskip
& & b_1 \chi_{H'}(y_1) + \dots + b_l \chi_{H'}(y_l) \\
\smallskip
&  = &  \left(a_0+\sum \limits_{1 \le j \le l} b_j \alpha^{\circled{H'}}_{j,0} \right) + \sum \limits_{j \le k, \pi_{H'}(j)=j} \left(a_j + \sum \limits_{1 \le j' \le l} b_{j'} \beta^{\circled{H'}}_{j,j'} \right) d'_j  + \\
%
& & \sum \limits_{j \le k, \pi_{H'}(j) \neq j} \left(\sum \limits_{1 \le j' \le l} (b_{j'} \beta^{\circled{H'}}_{j,j'})\right) d'_j + \sum \limits_{1\le j \le l} (b_{j} \alpha^{\circled{H'}}_{j,0}) o'_{j} + \\
\smallskip
%
& & \sum \limits_{1 \le j \le r_{H'}, j+k \in \rng(\pi_{H'})} \left(a_{(\pi_{H'})^{-1}(j+k)}+\sum \limits_{1 \le j' \le l} (b_{j'} \gamma^{\circled{H'}}_{j,j'})\right) d^{(H',0)}_j +\\ & & \dots  + \sum \limits_{1 \le j \le r_{H'}, j \not \in \rng(\pi_{H'})} \left(\sum \limits_{1 \le j' \le l} (b_{j'} \gamma^{\circled{H'}}_{j,j'})\right) d^{(H',0)}_j.
\end{array}
\] 
Let $a'_0,a'_1,\dots,a'_k,b'_1,\dots,b'_l$ denote the constant coefficient, the coefficients of $d'_1,\dots,d'_k$, and the coefficients of $o'_1,\dots,o'_l$ in $\chi_{H'}(O(q'_{m'}))$ respectively. 

Then we can adapt the output function $O$ and set $O(q_m) = a'_0 + a'_1 x_1 + \dots a'_k x_k + b'_1 y_1 + \dots b'_l y_l$.
}
%%%%%%%%%%%%%%%%%%%%%%%%%%%%%%%%%%%%%%%%%%%%%%%%%%%%%%%%%%%%%%
%%%%%%%%%%%%%%%%%%%%%%%%%%%%%%%%%%%%%%%%%%%%%%%%%%%%%%%%%%%%%%
%%%%%%%%%%%%%%%%%%%%%%%%%%%%%%%%%%%%%%%%%%%%%%%%%%%%%%%%%%%%%%


\smallskip

\noindent {\bf Step I$'$}. We do the same analysis for the simple path $H_1\dots H_r$ as for the handle $H$ in Step I.

\smallskip

Suppose $s: 2 \le s \le r$, and  in $\eval{O(q_{r,m_r})}{\sumf^{(H_s\dots H_{r}, \initval)}}$, the constant atom is $a'_{s,0}$, the coefficient of the $\initval(x_j)$-atom is $a'_{s, j}$ for each $j \in [k]$, moreover, the coefficient of the $\initval(y_{j'})$-atom is $b'_{s, j'}$ for each $j' \in [l]$. Then we change the output function and let
\[O(q_{s-1, m_{s-1}}):=a'_{s, 0}+a'_{s, 1} x_1 + \dots + a'_{s, k} x_k + b'_{s,1} y_1 + \dots + b'_{s, l} y_l.\]

\smallskip 
\noindent {\bf Step II$'$}.  For each $1 \le s \le r$, each $i_{s,1}\in [n_s]$, and each cycle scheme $\schm = C^{\ell_1}_{i_{s,1}} C_{i_{s,2}} \dots C_{i_{s, t}}$ over the collection of cycles $\{C_{s, 1}, \dots, C_{s,n_s},\dots, C_{r,1}, \dots, C_{r,n_r}\}$ such that $i_{s,2},\dots, i_{s,t}$ are mutually distinct, we apply an analysis to the expression $\eval{ O(q_{s, m_{s}})} {\sumf^{(\schm,\sumf^{(H_1 \dots H_{s}, \initval_0)} ) } }$, similar to Step II. If the decision procedure does not return during the analysis, then go to Step III$'$.

\smallskip

Intuitively, in Step II$'$, during the analysis of $\eval{ O(q_{s, m_{s}})} {\sumf^{(\schm,\sumf^{(H_1 \dots H_{s}, \initval_0)} ) } }$, the effect of the paths $H_{s+1},  \dots,  H_r$ and the cycles $C_{i_{s,2}}, \dots, C_{i_{s, t}}$ is described by the expressions $\cstl^{\circled{H_{s+1}}}_j \dots \cstl^{\circled{H_{r}}}_j  \cstl^{\circled{C_{i_{s,2}}}}_j \dots \cstl^{\circled{C_{i_{s,t}}}}_j $ for $j \in [l]$. Since $O(q_{s, m_{s}})$ has already taken into consideration the expressions $\cstl^{\circled{H_{s+1}}}_j \dots \cstl^{\circled{H_{r}}}_j$ for $j \in [l]$, conceptually, in Step II$'$, we can do the analysis as if we have a generalized lasso where the handle is $H_1\dots H_s$ and the collection of cycles is $\{C_{s,1},\dots, C_{s,n_s}$, $\dots$, $C_{r,1},\dots, C_{r,n_r}\}$. 

%$\schm=C^{\ell_{s, 1}}_{i_{s,1}} \dots C^{\ell_{s, t_s}}_{i_{s, t_s}} C^{\ell_{s+1, 1}}_{i_{s+1, 1} } \dots C^{\ell_{s+1, t_{s+1}}}_{i_{s+1, t_{s+1}} } \dots C^{\ell_{r, 1}}_{i_{r,1}} \dots C^{\ell_{r, t_r}}_{i_{r, t_r}} $, where for each $s': s \le s' \le r$, $i_{s',1},\dots, i_{s', t_{s'}} \in [n_{s'}]$, 


%%%%%%%%%%%%%%%%%%%%%%%%%%%%%%%%%%%%%%%%%%%%%%%%%%%
%%%%%%%%%%%%%%%%%%%%%%%%%%%%%%%%%%%%%%%%%%%%%%%%%%%
%%%%%%%%%%%%%%%%%%%%%%%%%%%%%%%%%%%%%%%%%%%%%%%%%%%
\hide
{
At first, by using $O(q_m)$, we do the following computation, similarly to Step II: For each $i_1: 1 \le i_1 \le n$, if there are a cycle scheme $\schm$  
$HC_{i_1}^{\ell_1} C_{i_2}^{\ell_2} \dots C_{i_t}^{\ell_t}
$
or 
$HC_{i_1}^{\ell_1} C_{i_2}^{\ell_2} \dots C_{i_t}^{\ell_t} (C'_{i'_1})^{\ell'_1} (C'_{i'_2})^{\ell'_2} \dots (C'_{i'_{t'}})^{\ell'_{t'}}$,
and $j' \le k$ such that 
\begin{itemize}
\item $i_2,\dots,i_t \le n$ are mutually distinct, $\ell_2 = \dots = \ell_t = 1$, 
%
\item $i'_1,\dots,i'_{t'} \le n'$ are mutually distinct, $\ell'_2 = \dots = \ell'_{t'} = 1$, 
%
\item $\pi_{C_{i_1}}(j')=j'$, and $\mu_{\schm,(i_1,j')} \neq 0$ (recall that $\mu_{\schm,(i_1,j')}$ is obtained from the coefficient of $d^{(0)}_{\pi_H(j')-k}$ in  $\chi_\schm(O(q_m))$), 
\end{itemize}
then return $\ltrue$. 

Then by using $O(q'_{m'})$, we do the following: For each $i'_1: 1 \le i'_1 \le n'$, if there are a cycle scheme $\schm' =(C'_{i'_1})^{\ell'_1} (C'_{i'_2})^{\ell'_2} \dots (C'_{i'_{t'}})^{\ell'_{t'}}$, and $j' \le k$ such that
\begin{itemize}
\item $i'_2,\dots,i'_t \le n'$ are mutually distinct, $\ell'_2 = \dots = \ell'_t = 1$, 
%
\item $\pi_{C'_{i_1}}(j')=j'$, and $\mu_{\schm',(i'_1,j')} \neq 0$ (here $\mu_{\schm',(i_1,j')}$ is obtained from the coefficient of $d''_{j'}$ in  $\chi_{\schm'}(O(q'_{m'}))$, where $d''_1,\dots,d''_k$ denote the initial data values of $x_1,\dots,x_k$ respectively),
\end{itemize}
then return $\ltrue$. 

Similarly, we can apply an analysis for the constant coefficient to $\chi_\schm(O(q_m))$. 


If the decision procedure has not return yet, then go to Step III$'$. \qed
}
%%%%%%%%%%%%%%%%%%%%%%%%%%%%%%%%%%%%%%%%%%%%%%%%%%%
%%%%%%%%%%%%%%%%%%%%%%%%%%%%%%%%%%%%%%%%%%%%%%%%%%%
%%%%%%%%%%%%%%%%%%%%%%%%%%%%%%%%%%%%%%%%%%%%%%%%%%%

\smallskip

After Step II$'$, if the decision procedure has not returned yet, then similar to Proposition~\ref{prop-bnd-domain}, the following results hold.
\begin{itemize}
\item For each $s \in [r]$ and each path $\schm=H_1 \schm_1 H_2 \dots H_s \schm_s$ such that for each $s'\in [s]$, $\schm_{s'}$ is a cycle scheme over the collection of cycles $\{C_{s',1},\dots,C_{s',n_{s'}}\}$, it holds that the constant atom and all the coefficients of the non-constant atoms in $(\sumf^{(\schm,\initval_0)})'(y_j)$ are from a bounded domain $K$.
%
\item Moreover,  an abstraction of $\schm$, denoted by $\abs(\schm)$, can be defined, so that $\Kk$, which is the set of $\abs(\schm)$ for the paths $\schm=H_1 \schm_1 H_2 \dots H_s \schm_s$ (where $s \in [r]$), can be computed effectively from 
\[\sumf^{(H_1,\initval)}, \sumf^{(C_{1,1},\initval)}, \dots, \sumf^{(C_{1,n_1},\initval)},\dots, \sumf^{(H_r,\initval)}, \sumf^{(C_{r,1},\initval)}, \dots, \sumf^{(C_{r,n_r},\initval)}.\]
\end{itemize}

%Similarly to the generalized lassos, we can construct a finite state automaton $\Aa'$ from $\chi_H,\chi_{C_1},\dots,\chi_{C_n},\chi_{H'}, \chi_{C'_1},\dots,\chi_{C'_{n'}}$ to record the coefficients in the states and simulate the evolvement of these coefficients. The final states of $\Aa$ represent the coefficients obtained when reaching the state $q'_{m'}$ in $\Ss$. 

\smallskip 

\noindent {\bf Step III$'$}. We apply the same analysis to $\Kk$ as in Step III. Finally, if the procedure does not return during the analysis, return $\lfalse$.

